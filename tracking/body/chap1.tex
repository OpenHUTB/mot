% !Mode:: "TeX:UTF-8"

\chapter{绪论}

\section{研究背景及意义}
% nai
近年来,人工智能技术发展突飞猛进,并以不可阻挡之势影响着社会和人们的生活。
许多国家和地区都将人工智能的发展作为提高国家实力、增强核心竞争力和建设现代化强国的核心所在。
例如,2020 年 3 月,中共中央政治局常委会提出将人工智能作为新型基础设施建设七大板块中的重要一项,以便更好地推动中国经济的转型和升级。
2021 年 3 月,美国国家人工智能安全委员会发布了《人工智能国家安全委员会最终报告》,提出推动人工智能以及相关技术的研究和应用,以解决国家安全和国防需求~\cite{schmidt2021national}。
而计算机视觉作为人工智能最重要的领域之一,正在成为科学研究和产业落地的重要方向。

% 博士论文
动态开放场景下的视频多目标跟踪方法是计算机视觉方向基础且重要的研究领域,同时也是全球大学、研究所和公司所亟待解决的核心问题。
多目标跟踪的任务是确定视频中每一帧中所有目标的位置和身份,并得到每个目标的运动轨迹。
解决该问题不仅可以得到目标的时空位置和身份信息,同时也为情景分析等更高层的视觉任务提供支撑。
比如跟踪轨迹的变化包含了所跟踪目标的速度和方向等信息,跟踪开始和结束时刻是目标进入和离开相机视野的时间,跟踪的结果还隐含了所跟踪对象的动作信息,
而且分析多个被跟踪目标的跟踪结果可以得到许多更加综合的信息,比如累计特定的时间段中某个相机视野中出现行人的数目、各个对象之间的社交距离、场景拥挤状况等信息,
这些是更高层分析和理解的基础。
% 可以为动作的检测和识别、行为分析和预测等更高级计算机视觉任务提供重要的信息。
因此,动态开放场景下的多目标跟踪方法在很多领域都有重要的巨大的实用价值,主要包括以下几个的实际场景:

(1)  智能监控。
近年来发生了许多突发性公共安全事件,如 2014 年 12 月 31 日上海外滩的踩踏事件;
2015 年 9 月 24日,沙特阿拉伯麦加发生朝圣者严重踩踏事故;
2016 年 5 月 11 日美国马萨诸塞州发生持刀伤人事件;
2021 年 4 月 30 日,以色列发生踩踏事件;
2021 年 11 月 21 日,美国威斯康星州发生汽车撞人事件;
2022 年 1 月 1日,印控克什米尔地区查谟附近寺庙发生踩踏事件等。
构建安全的社会环境、实时查看环境动态和预防公共安全事件已成为各国政府必须履行的职责,而智能视频监控则成为实现这个目的一种有效安全的方法。
根据全球视频监控市场显示,北美是全球最大的市场,其中美国单独占全球市场份额的比例约为 $ 26\%$,中东、欧洲和非洲市场规模相对较小。
2015 年,国家发改委等部委发布《关于加强公共安全视频监控建设联网应用工作的若干意见》要求,重点公共区域监控摄像头的完好率要达到 $ 98\% $,普通公共区域摄像头完好率要达到 $ 95\% $,同时公安部通过互联平台对各个省的安全城市状况进行监督,近年来我国各级政府对视频监控的重视程度也越来越高,视频监控市场得到了飞速发展。
但是,随着视频监控部署扩大,系统的维护成本也在逐年增加,对面海量的视频数据,仅仅依靠人力很难做到实时处理,后期视频的调取分析也面临着巨大挑战,而且人工处理时常会出现漏检的问题。
无论是从人力成本的角度还是问题的复杂性的角度,自动化和智能化的视频监控分析技术必然是未来的主流。
所以由此而来的是在视频监控系统中增加自动化和智能化视频处理,借助于计算机强大的运算和存储能力,利用各种图像处理和计算机视觉算法,在海量的监控视频数据中智能识别出人们最想要的目标和相关信息,极大的解决了人类处理视频数据的局限性,高效实时地根据得到的监控数据判断公共场合中出现的突发状况,从而能够快速的定位并赶到现场,极大地减少事故所带来的损失和危害,甚至能够通过综合各种信息,一定程度上做到事故的预警,达到提前预警、实时监控、事后取证等目的。
实现该系统需要一系列图像处理和计算机视觉算法,包括检测、分割、跟踪、情景分析等。
而其中跟踪算法,特别是多目标跟踪算法是智能视频监控系统的核心所在。
% 
例如,根据行人的跟踪轨迹可以判断出朝向车辆运动方向走等危险的行为动作,为智能驾驶提供警告信息,从而避免事故的发生;
根据车辆的运动分析,可以实时检测出危险的运动方向,比如冲向密集的人群或者障碍物等,从而达到危险的实时监控。
在集会、商场等人流量特别大的区域,利用智能视频监控可以统计并分析出是否出现人流量过大或者预测可能发生的踩踏等情况,做到事前预警、实时处理危险情况。
同时对已有的海量视频数据和多目标跟踪数据的挖掘,可以找出人流和车流密集的地方,通过调节红绿灯来实现通行优先级的调整或者交警人为干预的方式,提前疏导和控制,为城市的安全稳定运行提供支撑和保证。
还可以通过分析多目标跟踪的结果,研究群体的出行规律、社交距离等,为新冠疫情的防控提供可靠且有效的信息来源。
%另外,多目标跟踪算法不一定非要实时进行。
%通过离线多目标跟踪算法对大量视频中的目标进行跟踪,得到大量的轨迹,分析这些轨迹有助于发现人们经常聚集在一起的区域,这可以为规划建筑物的逃生路线提供帮助。
%轨迹分析也可以用来了解人们在大型商场或购物街内的移动方式。除了针对公共安全场合,智能视频监控技术也被广泛应用于道路交通管理。
%例如通过跟踪道路上行驶的车辆可以调整交通信号灯,从而减缓交通堵塞现象。

(2) 智能驾驶。
智能驾驶一般是表示使用车上的智能驾驶系统代替人部分或者全部的驾驶功能~\cite{bergmann2019tracking}。
在减少驾驶员介入的情况下,智能车辆能自主实现启动、巡航和停车等功能。
目前,百度阿波罗、特斯拉和 Mobileye 等国内外著名的科技公司,都已实现了智能驾驶系统的测试和验证,并取得了一定的商业价值。
多目标跟踪技术作为智能驾驶落地和产业化最重要也是最基本的功能之一,成为智能驾驶大规模商业化必须考虑的因素。
%智能驾驶汽车要想实现真正的普及,目标跟踪是必须考虑的最基本也是最重要的功能之一。
由于减少了驾驶员的操作和干预,智能汽车需要根据摄像头获取的视频数据和多目标跟踪算法来感知环境信息,为安全自主的驾驶行为决策提供支撑。
比如智能汽车可以利用多目标跟踪算法来判断周围车辆或行人的距离、速度和加速度等信息,并在此基础上进行行为动作的预测来规避危险。
%为了确保车辆遵守交通规则,智能驾驶汽车需要利用目标跟踪算法获取相应的信息来对行车场景做出正确的感知。

(3)  智能机器人。
目前机器人的研究和应用己经从实验室走向生活,从自动化逐步变为智能化。
虽然智能机器人当前还不能拥有和人类一样的自主意识,但却在特定功能方面已经拥有和人类相媲美的能力甚至突破人类的某些局限。
随着智能机器人算法和技术的逐步完善和成熟,一些类似于扫地机器人的产品已经开始进入人们的日常生活当中。
在完全不变和静止的环境中,利用基本的计算机视觉目标检测算法就可以进行障碍物的识别和路径规划。
但是当机器人在动态开放的环境当中进行导航和规划时,需要跟踪运动的物体并预测是否会与机器人本身的运动产生冲突或碰撞。
另外,随着人工智能的发展,特别是计算机视觉算法的进步,传统人利用键盘和鼠标等机械的方式实现与机器的交互出现了许多更加智能化人性化的交互方式,比如动作、手势等。
当在动态开放场景存在多个目标时,多目标跟踪作为视觉人机交互信息和目标与目标之间交汇信息抽取的基础。
所以动态开放环境下的多目标跟踪算法对于智能机器人的实现非常基础和重要。

%(4)  体育比赛分析。
%目视频多目标跟踪还可以应用到体育视频分析中,例如通过跟踪足球或篮球比赛中的运动员可以得到运动员们的运动轨迹信息,跟据这些轨迹信息可以客观地衡量每个球员的身体表现,还可以分析比赛中的失误,为教练训练运动员、制定更好的比赛策略等提供科学依据。

除了介绍的三个方向之外,多目标跟踪算法在国防军事、体育赛事分析等许多方向同样得到了及其广泛的使用。
并且多目标跟踪方法在研究领域也有重要的意义。
% bergmann2019tracking,chen2019aggregate,chu2019famnet,
许多人工智能顶级期刊和顶级会议每年都有大量的多目标跟踪有关的学术研究成果~\cite{sun2021deep,xu2020how,young-chul2019online,zhang2019robust}。
同时,多目标跟踪挑战赛(MOT Challenge)每年会吸引世界各地的研究所、公司、和高校的研究者参加比赛,将所提出的多目标跟踪算法和其他人的方法进行同台较量,以促进优秀思想和算法的实现和交流。
所以,多目标跟踪无论是在工业界还是学术界都有着重大而实际的意义。


\section{多目标跟踪的研究现状}
由于多目标跟踪任务有着极其重要的产业价值和研究价值,
许多国内外的公司和大学都将多目标跟踪任务作为研究重点。
各种高性能跟踪方法被陆续提出,促进了多目标跟踪技术的进步~\cite{xu2020how,chen2019aggregate}。
% ref: https://zhuanlan.zhihu.com/p/97449724?from_voters_page=true
一般完整的多目标跟踪算法包括以下几个流程:目标检测、表征获取、相似度评估以及目标关联。
由于目标检测通常作为一个单独的计算机视觉任务进行研究和处理,视频多目标跟踪主要关注后面三个步骤,
根据方法的类别,多目标跟踪方法可以被分为特征提取算法、数据关联算法以及深度学习算法。

% ① 检测 ②特征提取、运动预测 ③相似度计算 ④数据关联。
% 关联代价 -> 特征提取
\subsection{特征提取算法}
目标的特征提取是多目标跟踪算法的基础。
如果能够提取到十分鲁棒的目标特征,即使利用最基本的关联方法也会获得较好的效果。
一般的目标特征提取包括外观特征、交互特征、运动特征等。
%
% 外观特征
%对于数据关联多目标跟踪算法的特征提取来说,外观是一个重要的特征。
目标的外观特征对于多目标跟踪方法特别重要。
从具体实现上看,目标的外观特征建模包含外观特征抽取以及相似性计算。
外观特征使用一些表征来代表被跟踪目标的外观特性,而相似性计算是衡量不同目标之间的相似度。
%,而经过简单的推算可以将相似度变为关联匹配代价。

外观特征使用各种不同的特征来表征所跟踪的目标,这些特征分为局部特征和全局特征两大类。
% 局部特征
假设把图像中的像素视为最理想的局部区域,那么就可把光流特征看成是一种局部的表征。
一些多目标跟踪方法先将检测结果在时间维度上串联成轨迹段,然后使用光流特征进行目标关联~\cite{using2009,zhao2012tracking}。
因为光流特征和目标的运动相关,所以一般使用光流对运动进行编码~\cite{optic_flow,choi2015near-online}。
考虑到在被跟踪目标非常拥挤的视频中视觉表征效果不佳,光流特征被用于提取跟踪目标的运动模式~\cite{ali2008floor}。
KLT 算法~\cite{lucas1981an} 将视频帧的局部匹配从滑动窗口转换为偏移量的求解,该算法已成功在单目标跟踪~\cite{shi1994good} 和多目标跟踪任务中得到使用。
当抽取到适用于跟踪的表征时,便可使用它来生成短的跟踪轨迹段~\cite{using2009,zhao2012tracking}。
% 非局部特征 -> 全局
与局部特征相比,全局特征是从更广的区域进行特征的提取。
% 原始全局特征
全局特征一般分为原始全局特征、梯度全局特征以及协方差全局特征。
它们之间的差别在于计算视频帧外观特征时像素值之间做差分的次数。
原始全局特征表示不对视频帧中的相邻像素值进行差分操作,梯度全局特征表示进行一次视频帧中像素值的差分。
其中的原始全局特征是多目标跟踪中经常使用的一种外观特征,包括原始像素值模板~\cite{yamaguchi2011who}、颜色直方图~\cite{mitzel2010multi-person} 等。
梯度全局特征是根据梯度来进行特征的表征,一般利用的是方向梯度直方图。
%(Histogram of Oriented Gradient,HOG)。
协方差全局特征使用图像像素的全局协方差矩阵~\cite{tuzel2006region},该特征在研究中得到了广泛的使用~\cite{kuo2010multi-target,hu2012single,henriques2011globally}。
通常梯度全局特征是经典的相似性衡量手段,但是它忽视了视频帧中全局的位置状况。
虽然局部特征是一种可行的策略,可是这种特征对于旋转和目标遮挡等情况处理不佳。
而方向梯度直方图可以很好地表征被跟踪目标的形状特征,且对于光照改变等挑战拥有较好的鲁棒性,但是它却不能很好地应对形变和遮挡的情况。
因为协方差全局特征以更多的计算为代价来考虑更加广泛的信息,所以它相比于前两种特征更加鲁棒。


运动模型建模了目标的动作,用来预测被跟踪目标在后续视频图像中的空间位置信息,以达到缩小搜索空间的目的~\cite{motion_doc}。
一般可以认为被跟踪目标不会有大的加速或减速,而是进行平稳的运动。
经典的建模方法包括非线性以及线性运动模型。
% ,yu2007multiple
最直观的建模方法就是线性运动模型~\cite{breitenstein2009robust,shafique2008a},该方法假定所跟踪的目标运动都是匀速的~\cite{breitenstein2009robust},
所以有位置平滑和速度平滑等方法来建模目标的运动~\cite{kuo2011how}。
% xing2009multi-object
位置平滑是指控制观测坐标与预测坐标的差值~\cite{yang2012an};
速度平滑是指利用被跟踪目标的速度在连续视频帧中是平滑变化的假定来进行建模~\cite{qin2012improving};
线性运动模型一般用来表示被跟踪目标的运动,但大部分被跟踪目标实际的运动并不是线性的。
%有些情况下线性运动模型无法处理。
因此在此基础上所提出的非线性运动方法能够处理被跟踪目标随意运动的情况,并且可对被跟踪目标进行更加精细的运动建模~\cite{yang2012multi-target}。



相似性计算旨在衡量两个视觉特征的相似性。
相似性计算可以是基于一种特征,也可以依赖多种特征。
前者仅仅使用一种特征来表示外观,或者利用目标之间的空间距离作为相似度进行关联。
%又或者计算亲和力。
% ali2008floor
原始全局特征中的像素值模板外观特征一般利用归一化互相关来衡量两者之间的相似度~\cite{yamaguchi2011who,wu2015global}。
% zhang2008global,using2009,choi2010multiple
原始全局特征中的颜色直方图特征通常先利用巴塔恰里雅距离来度量两个颜色直方图之间的可分离性,接着把可分离性转为相似性~\cite{kratz2010tracking,qin2012improving}。
把类似距离的不相似性转为相似性也可用于协方差矩阵的视频帧特征~\cite{henriques2011globally}。
%除了以上经典的方法之外,
%一般利用词袋法处理点的外观特征。
而利用多个特征融合不同种类信息可以让目标的视觉特征对动态开放的场景拥有更强的适应性。
%
为了达到上述目的,可以使用连接策略来融合不同方向的特征,进而获得更加鲁棒的视觉特征。
%但如何融合来自不同方向的多个线索同样是一个具有挑战的任务。
%常见的基于多线索的外观模型的融合方法包括串联、集成、连接、合并和相乘。
%具体而言,串联表示利用不同种类的图像特征的串联以实现更加精细化的外观建模~\cite{rodriguez2009tracking}。
%集成一般是利用集成算法按照顺序从特征候选区中选择一部分特征,
%% ,li2009learning
%比如从各阶特征中利用一系列集成学习的方法~\cite{kuo2010multi-target,block_mot} 来选取最鲁棒的的表征。
%连接是指连接各种不同信息源的表征,比如连接颜色直方图、方向梯度直方图等特征来抽取目标的视觉特征~\cite{brendel2011multiobject,mot_doc};
%% mitzel2010multi-person,
%合并方法是从不同特征中获得相似性度量,并把得到的值进行加权合并~\cite{liu2012automatic};
%相乘与合并策略相似,不同的相似度根据相乘产生一个更加综合的相似度值~\cite{yang2009detection,song2010a,berclaz2006robust}。
%需要注意的是通常在应用此策略时要进行独立性假设;



\subsection{数据关联算法}
视频多目标跟踪方法通常被分为在线方法和离线方法两种。
在线多目标跟踪方法只能利用当前帧和历史帧的数据,不能利用当前帧以后的任何数据,而离线方法可以利用整个视频的数据。
但是相比于离线方法,在线方法更利于在实际场景中进行应用。
在线目标关联方法不能使用将来视频的任何数据,而是应该输出当前时刻图像的候选检测和目标历史跟踪轨迹段的关联。

% 在线
常见的在线目标关联方法包括全局最近邻数据关联、联合概率数据关联等。
% 全局最近邻
全局最近邻方法是一种经典的在线数据关联方法,该方法建立当前时刻视频帧的候选检测与历史轨迹之间的关联关系,然后利用求解二分图匹配问题,在所有候选检测结果中匹配最大概率关联。
该方法最关键的步骤是将多目标跟踪问题建模成二分图匹配问题~\cite{bertsekas1994linear,veenman2001resolving},以便能够有效地解决它。
% JVC 算法~\cite{okuma2004a,bertsekas1992a}~和
该研究方向已存在不少经典方法用于求解这个最优化模型,比如拍卖算法~\cite{bertsekas1988the}、匈牙利算法~\cite{unkres1957algorithms}~等。
全局最近邻算法因其算法流程的简洁以及容易改进的特点得到了广泛的应用。
并且二分图匹配任务在学术界己被深入研究,因此该方法能够很方便地在实际工程中进行测试和部署。
由于全局最近邻方法对关联步骤所做的分配是确定性的,并未将关联正确或者错误的概率加以考虑,所以必须提取一个鲁棒的特征。
当提取的特征不佳时,全局最近邻方法通常只能在跟踪目标图像特征差异很大或者空间距离非常大时才能够很好地工作,同时检测器必须获得准确的结果以保证该算法达到良好的性能。
% 联合概率
为了解决全局最近邻方法只能根据最好的关联候选进行历史轨迹的更新且并未将概率或者是疑似的目标关联加以考虑的问题,
联合概率目标关联方法~\cite{bar-shalom1987tracking} 被提出来建模这种不确定性,进而将所有疑似的候选检测都进行建模。
% 的加权组合
对于联合概率目标关联,历史轨迹段参考当前所有可能的检测结果进行更新。

% 离线
%将在线多目标跟踪目标关联方法用于视频数据的在线处理,每当新的视频帧到达时就要执行关联算法来解决关联匹配问题。
与在线方法每次处理当前时刻的图像并进行目标关联不同,离线的多目标跟踪数据关联方法可以把将来的视频图像数据加以利用以进行多目标跟踪。
当考虑某一帧和其他所有帧的关联关系时,为了缓解关联关系随着跟踪过程呈指数级增长的问题,批处理窗口方法被用来进行目标的关联。
离线的多目标跟踪数据关联方法的好处是既可以利用关联度很高的数据来处理关联不确定的问题,又能保证算法的时间复杂度不是特别高。
常见的离线多目标跟踪目标关联方法有多假设多目标跟踪、子图分解、网络流多目标跟踪目标关联和层次目标关联等。

多假设跟踪~\cite{reid1989an,multi_hypo} 采用推迟处理的策略,按照时间线对存在的关联进行假定,并利用后面得到的检测来处理当前时刻的不确定性关联。
与在每个时刻确定概率最高的检测和联合概率方法不同,为了更准确地预测目标关联可能性,多假设跟踪会传递当前时刻的假设到后面视频帧中,
并且提供了通用的步骤来处理每个目标的全部过程:目标轨迹的初始化、跟踪以及消亡。

%
% zhang2008global
网络流数据关联方法~\cite{wang2016joint,schulter2017deep,net_flow} 通过使用网络来表征跟踪目标的状态改变,将多目标跟踪中的目标关联任务建模成最小化网络流任务。
这个算法将视频时间维上所有的检测都考虑进来,所以直观上它比在线多目标跟踪算法中的全局最近邻标方法会有更优的效果。
%由于网络流数据关联法只是所有检测结果已知的情况下一系列假设轨迹集合的最大后验概率,由此来求解最优解,考虑到每个被跟踪目标只会有一个轨迹与它对应,所以可以裁剪一些假设,所以它并不是完全的贝叶斯目标关联算法。
与一般的任务不一样的是网络流算法可以在多项式时间复杂度范围内得到全局最优解。
可以将网络流算法应用于多目标跟踪任务当中~\cite{zhang2008global},并提出解决网络流目标关联方法~\cite{efficient1990}。
之后考虑利用连续最短路径实现在多项式时间复杂度范围内进行网络流任务的求解~\cite{2011globally-optimal}。
还可以在最小代价网络流中加入配对的代价~\cite{chari2015on}。
同时可以在网络流方法中加入跟踪目标的身份,网络中的节点表示把被跟踪目标身份信息赋给它的可能性,并求得满意的解~\cite{2015target}。
%在放松约束条件之后的每次迭代过程屮,被求解的问题被简化为分别为每个被跟踪目标确定最佳轨迹,并利用动态规划方法可在线性时间内找到满足条件的最优解。
还有一种端到端的方法抽取目标关联任务中特征的算法~\cite{schulter2017deep}。

子图分解法也是进行目标关联的策略,在该方法中目标关联问题被建模为无向图求解任务。
无向图的每个节点表示检测器输出,图中的边表示两个节点是同一身份的可能性。
% 其中的轨迹路径不会有合并或者分叉,
一般的网络流算法把目标关联表示成求解不相交的轨迹,因为目标在被跟踪的过程中不能分为两个目标。
这个模型的优点是很简洁且容易理解,缺点是没有考虑检测结果的不完美有时会为一个目标产生一系列类似结果的情况。
在不相交路径的网络流方法中,当面对许多可能的轨迹时,无法求解最优的轨迹。
可以利用最小成本算法,将多目标跟踪的目标关联看成是子图分解任务~\cite{tang2015subgraph}。
面对视频中跟踪目标数目的不确定,这种抽象的方法能很好地适应并可以利用最优化算法进行求解,使它在时空上合并同一跟踪目标的一系列检测,以进行更好的多目标跟踪。
基于前面工作的继续设计以同时考虑再识别特征和局部块的策略。
再识别特征为相隔较远的图像帧检测结果的关联贡献了有价值的特征,而局部块为相邻帧的检测贡献了有效的衡量方法。
在动态开放场景中,虽然目标之间相似性计算很精确,但由于存在外观上相似的目标,它们身份有可能根本不同,这导致在整个方法中加入视频帧跨度大的再识别特征依然是很困难的工作。
所以在大部分局部特征的基础上合并时间跨度大的特征是很有意义的。
根据这种经验,该团队算法继续提出了优化方案,将时间跨度大的目标再识别特征加入到多目标跟踪的数据关联当中。
%该工作在原始图里加入了规则边以及提升边。
%规则边说明了问题可行的解决方法,提升边则是将时间跨度大的目标重识别特征加入到模型当中,只有当规则边形成有效路径时,才会考虑时间间隔较远的外观相似目标。


层次关联算法先使用少量的视频帧产生小范围内可信的的跟踪轨迹段,再根据生成的跟踪轨迹段产生时间范围更大的跟踪轨迹段~\cite{mot_doc}。
该算法可以有效缓解离线批处理中所用时间会根据跟踪轨迹段变长而指数级增长的问题。
%,使用层次关联算法可以有效缓解这个问题。
一般来说,利用跟踪轨迹段的进行目标关联可以极大缩小任务规模,且能够较好地应对跟踪过程中出现的遮挡问题。
但是该算法的先决条件是先产生跟踪轨迹段,而跟踪轨迹段产生的细小误差都会传递到最后的多目标跟踪。
早些时候使用跟踪挂起关联策略~\cite{kaucic2005a},对于目标被其他物体部分遮挡的情况,挂起多目标跟踪过程,挂起的跟踪轨迹使用外观特征来进行关联。
随后使用分阶段的策略~\cite{wu2007detection} 和其他方法,利用二分图匹配中的局部特征来辨别所跟踪的目标。
%面对长时间遮挡的情况,通过对集覆盖问题的对数近似解得到局部轨迹段的关联结果。
使用层级架构来实现的目标匹配~\cite{huang2008robust} 关联邻近视频帧的重复的检测来获得跟踪短轨迹段。
随后根据视频帧的图像特征以及运动趋势,把生成的短跟踪轨迹段连接成更长的跟踪轨迹。
%然后进行更高层次的关联,使用交替优化方法生成最后的跟踪轨迹。
还可以使用集成学习等一系列机器学习方法~\cite{li2009learning} 以监督学习的方式而不是启发式定义的方式自动学习短轨迹段之间的相似性。
另外在线多目标跟踪中的条件随机场模型~\cite{condition_field,asrcf2} 着重应对短跟踪轨迹段关联时的不确定性问题。


% TODO
\subsection{深度学习算法}
% wang2016joint,b19,b23,b24,dual_matching,b3,kim2021discriminative
近年来,深度学习方法在视频多目标跟踪领域得到了广泛的使用~\cite{ke2021prototypical,baisa2021robust,hsu2022multi}。
利用深度学习算法进行被跟踪目标的特征提取成为一个重要策略,也就是抽取更鲁棒的特征来进行相似性度量。
在多目标跟踪中使用深度学习算法的应用一般包括深度表征使用、使用深度网络进行相似性计算、联合检测和跟踪的方法。

一种提高算法效果直接而有效的策略是依靠深度模型强大的学习能力,直接用目标识别任务中抽取到的外观表征来表示所跟踪目标的外观~\cite{kim2015multiple},也可以利用深度模型抽取到的类似于光流的运动表征来表示所跟踪对象的运动~\cite{b18}。

但是直接利用在其它计算机视觉问题中提取到的深度表征并未考虑视频多目标跟踪本身的性质。
一个改进策略是参考行人再识别的方法,直接学习被跟踪目标之间的外观相似性。
可以利用孪生网络模型对两个检测是否为相同身份进行预测,这个模型把被跟踪目标的外观特征和运动特征一起加以考虑~\cite{RN454,twin_net}。
还有研究利用孪生网络来对两个被跟踪对象是不是相同身份进行预测,它先离线训练孪生网络,然后在进行多目标跟踪时把训练完成的孪生网络和相似性计算执行联合学习~\cite{wang2016joint}。
因为直观上对于相同身份的检测,视频帧序列中相隔较近的检测结果比相隔较远的检测结果有更大可能的相似,所以可利用卷积网络来学习被跟踪目标之间的相似性~\cite{b19}。
另外可以把注意力机制加入到孪生神经网络当中,利用注意力确保模型能够只注意检测结果中的前景部分,进而应对检测结果不完美和目标被遮挡等挑战~\cite{gm_phd}。
%
还可以使用循环网络抽取时间序列特征来建模被跟踪目标之间的相似性。
为了融合目标的外观、交互以及运动信息,网络框架可以利用连接的方法,基于长短时记忆网络来提取历史轨迹段与候选检测之间的相似性特征~\cite{b23}。
双向长短时记忆网络利用时间注意力机制来缓解错误的跟踪轨迹信息对数据关联的所造成的问题~\cite{dual_matching}。

最近为了避免检测特征的重复抽取并加速多目标跟踪算法的运行和加快算法的产业化落地,将目标检测和目标跟踪两个计算机视觉任务合并在一个深度网络中进行处理已成为一种趋势~\cite{jde,voigtlaender2019mots,fairmot}。
TrackR-CCNN~\cite{voigtlaender2019mots} 基于Mask-RCNN~\cite{he2017mask} 添加了一个行人重识别分支来预测边界框和目标特征。
JDE~\cite{jde} 基于YOLOv3~\cite{redmon2018yolov3} 在测试时获得了接近实时的跟踪速度。
FairMOT~\cite{fairmot} 发现基于锚框的检测器预测出的目标边界框可能会和实际的目标中心没有对齐,提出方法以解决严重的歧义和身份切换导致跟踪精度降低的问题。
%双线性长短时记忆网络的隐含层特征与输入之间采用乘性耦合方式代替传统长短时记忆网络中的加性耦合,能够更好地建模跟踪目标的历史外观特征~\cite{b3}。


\section{相关的数据集}
为了让各个算法有一个公平的比较以及加速研究过程,现有各种多目标跟踪公开的数据集来进行方法效果的测试。
经常使用的基准数据集包括 KITTI140~\cite{autonomous_vechicle}、PETS-2009~\cite{ferryman2009pets2009}、AVG~\cite{benfold2009guiding}、TUD~\cite{andriluka2010monocular}、MOTChallenge~\cite{leal2015motchallenge,mot16}等。
其中,MOTChallenge 公开数据集已发展成当前该方向使用最多的标准,同时本文为了验证进行视觉跟踪时人脑的行为响应和深度神经网络模型之间的关系,引入了眼睛凝视数据集 StudyForrest~\cite{gaze_forrest}。
因此,本文重点介绍这两个公开数据集。

%\begin{figure*}[ht]
%	\centering
%	\includegraphics[width=0.98\textwidth]{./figures/C1Fig/otb2013.pdf}
%	\caption{MOT17 数据集所包含的视频示例}
%	\label{fig:otb2013}
%\end{figure*}

\subsection{MOTChallenge 基准数据集}
MOTChallenge 是一个使用广泛的视频多目标跟踪公开数据集,它从已有的公开数据集中收集了 14 个视频序列,包括 KITTI~\cite{autonomous_vechicle}、AVG~\cite{benfold2009guiding}、PETS-2009~\cite{ferryman2009pets2009}、TUD~\cite{andriluka2010monocular} 和 ETH~\cite{ess2007depth}等,同时把以上数据划分为训练集与测试集,但是仅仅发布了有真实标签的训练集。
需要在线上上传利用测试集视频产生的输出,以得到算法的各种性能评价指标。
此外在这个数据集中的 MOT17 还提供了使用 DPM~\cite{felzenszwalb2009object}、Faster-RCNN~\cite{b8} 和 SDP~\cite{sdp} 检测器获得所有视频帧的检测输出,以公平地衡量不同跟踪算法基于相同检测器的性能;
MOT20~\cite{dendorfer2020mot20} 包含了 8 非常具有挑战的视频序列,如图~\ref{fig:c1:mot} 所示展示了数据集中各种动态开放场景下的多目标跟踪环境。

\begin{figure*}[ht]
	\centering
	\includegraphics[width=0.98\textwidth]{./figures/C1Fig/mot.pdf}
	\caption{多目标跟踪数据集示例}
	\label{fig:c1:mot}
\end{figure*}

该数据集包含了动态开放场景下的各种条件:
(1)数据集的图像序列是相机在各种光照情况下采集的,比如白天、晚上等;
(2)同时包括固定相机平台采集的视频图像和移动相机采集的视频图像;
(3)包括各种相机拍摄的角度,比如平行拍摄、俯视拍摄、低角度拍摄等。



% MOT综述: /data2/whd/win10/doc/paper/doctor/doctor.Data/PDF/1948313292
从各个角度出发设计了许多评价指标来有效地衡量方法的好坏,通常不同的衡量标准反映多目标跟踪方法不同的特性,经典常见的评价标准包括以下几个:
% https://www.cnblogs.com/wemo/p/10628836.html
\begin{itemize}
	\item 多目标跟踪准确度(Multiple  Object Tracking Accuracy,MOTA)是通过误判率、缺失率和误配率得来的,记为:$1 - \frac{\sum_t fp_t + m_t + mme_t}{\sum_t g_t}$;
	
	\item 多目标跟踪精度(Multiple  Object Tracking Precision,MOTP)是位置误差的评判指标,记为:$\frac{\sum_{t,i} d_t^i}{\sum_i c_t}$;
	
	\item 身份切换数目(ID Switch,IDS)记为:$\sum_t{mme_t}$;
	
	\item 大于百分之八十跟踪成功的目标数(Mostly Tracked,MT);
%	超过80\%的真实标注轨迹被成功跟踪的目标数目
	
	\item 大于百分之八十跟踪丢失的目标数(Mostly Lost,ML);
%	超过80\%的真实标注轨迹丢失的目标数目
	
	\item 假阳性数目(False Positives,FP):预测框没有检测框和它匹配,$\sum_{t}{fp_t}$;
	
	\item 目标丢失数目(False Negatives,FN):检测框没有预测框和它匹配,$\sum_t{m_t}$;
	
	\item 碎片轨迹总数(Fragmentation,Frag):所有真实标注轨迹被中断的次数。
\end{itemize}
其中,$fp_t$ 是第 $t$ 帧多目标跟踪算法产生的跟踪轨迹中关联失败的计数,即假阳性的计数。
$m_t$ 是关联失败的真实跟踪轨迹的数目,即假阴性的计数。
$mme_t$ 是相比于前面一个视频图像的关联产生身份切换的计数。
$g_t$ 是第 $t$ 个视频帧中真实跟踪轨迹的计数,
$d_t^i$ 是第 $i$ 个成功关联对的代价,
$c_t$ 是关联成功的计数。



\subsection{StudyForrest 数据集}
在实验中,使用《阿甘正传》电影作为动态开放自然环境的一种近似,完整的实验细节参考了原始数据集 StudyForrest~\cite{gaze_forrest}。
原始的《阿甘正传》电影包含的实际的 7 个电影片段。
这 7 个部分连接起来并重新分割为 8 个视频刺激段,每个视频段对应一个核磁共振记录。
时间戳参照《阿甘正传》 2002 年发行的 DVD,PAL 制式,DE103519SV,每秒25帧,并按照“HH:MM:SS.FRAME”的格式给出。
该数据集包含 15 个受试,使用飞利浦 Achieva dStream 核磁共振扫描仪记录下他们在观看好莱坞电影《阿甘正传》时的脑部激活数据和眼睛凝视数据,另外 15 位不在核磁共振扫描仪中记录的受试仅仅记录下他们的凝视数据,这些不在核磁共振扫描仪中记录的数据仅仅用于提高平稳跟踪事件的检测精度。
使用液晶投影仪进行电影刺激的播放,受试在配有前反射镜的核磁共振扫描仪中进行电影的观看,并记录下观看电影时的脑部激活情况。
凝视数据使用高频眼球跟踪仪 EyeLink 1000 进行记录,在记录核磁共振扫描数据时,配有长焦镜头并以 1000 次每秒的速度进行采样。
%在每个会话开始阶段进行 13 点校正。
使用 7 特斯拉的高精度核磁共振扫描仪,重复时间为 2 秒,体素大小为 $3 \times 3 \times 3$ 立方毫米获取核磁共振记录。
此外,还获得了一套综合的辅助数据(弥散张量成像、磁敏感加权图像、血管造影)以及用于评估技术和生理噪声成分的测量数据。



%\begin{figure*}[ht]
%	\centering
%	\includegraphics[width=0.98\textwidth]{./figures/C1Fig/challenge.pdf}
%	\caption{OTB数据集所划分的11类挑战属性}
%	\label{fig:challenge}
%\end{figure*}

% 参考:https://blog.csdn.net/AMDS123/article/details/81184250
% 毕设:/data2/whd/win10/doc/paper/doctor/doctor.Data/PDF/2507414993
\section{多目标跟踪的主要问题和挑战}
虽然近些年来视频多目标跟踪已经取得了一些进步,但是在动态开放场景下依然是一个非常有挑战的任务。
在真实场中下,跟踪多个目标的过程中所存在单目标跟踪器效率低且可解释性不足、检测器的不精确和相互遮挡的动态开放场景等是制约多目标跟踪性能提升和算法工程化应用的重要原因。
因此,设计一个高效的多目标跟踪算法来应对现实场景中的这些挑战并成功地进行多目标跟踪具有重大的意义。
%如图~\ref{fig:challenge}~所示,OTB数据集将目标在动态开放场景下所经历的变化划分为11个挑战属性。
%而这种分类也得到了广大研究者的认可。
%事实上,这些挑战也是目标跟踪的主要难点所在。

(1)单目标跟踪算法扩展到多目标场景中存在的问题

%一种直观的解决多目标跟踪的思路是将单目标欧跟踪算法直接扩展到多目标场景中,即多每一个待跟踪目标分别进行单目标跟踪。
随着目标检测技术的发展,跨帧连接检测结果的数据关联算法已经成为多目标跟踪任务一种常见的解决方案。
然而,数据关联方法过分依赖于不完美的目标检测器。
如果目标检测结果不准确、遗漏或误检,则跟踪的目标很容易丢失。
可以通过使用最新精度较高的单目标跟踪器来缓解此类问题,
此类跟踪器使用第一帧中的检测结果并在线更新单目标跟踪模型,以预测后续帧中跟踪目标的空间位置和大小。
%在此,将单个对象跟踪器和数据关联的优点结合在一个统一的框架中来解决这个问题。
%在大多数帧中,使用单个对象跟踪器来跟踪每个目标。然后当跟踪分数低于阈值时应用数据关联。该方案表明被跟踪的目标可能会经历较大的外观变化或被其他物体遮挡。
%近年来,随着深度学习的发展,特征提取能力越来越强,单目标跟踪算法精度也越来越高。
然而,在多目标跟踪场景中,为每个被跟踪目标建立一个单目标跟踪器,这里主要存在以下两个问题:
\begin{itemize}
	
	\item 模型的复杂性和可理解性问题:基于深度学习的单目标跟踪算法在提高精度的同时带来的另一个问题是随着单目标跟踪模型的层数越来越多,导致跟踪模型运算量急剧增大,严重制约了这种方法的实际应用。
	并且深度神经网络模型越来越像一个“黑盒”,可理解性越来越差,偏离了人工智能为人服务的初衷,与人的能力差距越来越大,这为模型的优化和人类利用人工智能技术认识自身带来了巨大障碍。
	
	%\item 模型复杂性问题:跟踪效率问题是单目标跟踪算法扩展到多目标场景时新产生的问题。
	%由于对场景中的每一个新出现的目标都要建立一个相应的单目标跟踪模型,其跟踪效率随着待跟踪的目标数目的增加而大幅降低,
	%同时深度单目标跟踪模型的运算量随着模型的增大而急剧增大,这些因素极大地限制了这种方法在实际中的应用。
	
	% 基于预测的范式
	\item 跟踪漂移问题:跟踪漂移问题是单目标跟踪器也需要解决的问题,通常是当跟踪目标被遮挡时,使得单目标跟踪模型关注到遮挡物上,发生跟踪漂移的现象。
	把单目标跟踪模型直接迁移到视频多目标跟踪时,特别是目标密集的的情况下,多个目标之间会发生相互遮挡会更加频繁,跟踪漂移问题会变得更加突出。
	
\end{itemize}


(2)基于检测的多目标数跟踪目标关联方法的不足

相对于单目标跟踪,多目标跟踪还需要处理两个其他的挑战:随着时间的变化定位所有目标的位置以及保持每个目标的身份信息。
除了包括与单目标跟共有的不足和挑战外外,还要应对更具有挑战性的因素,比如跟踪轨迹的开始与结束、干扰物频繁的遮挡、相似的外观特征、多个跟踪目标之间的相互影响等。
进行多目标跟踪时必须将以下两个问题都考虑进去,其中一个是如何区分同一视频帧中各个目标之间的相似性,而另一个问题是如何利用所得到的相似性来判断视频帧之间的目标身份是否相同。
前者主要包括如何建模外观特征、运动特征、排斥、碰撞和空间交叉,后者主要是和时间上的数据关联有关。

目前根据检测器结果进行目标关联的方法一般会有下面两个问题:

\begin{itemize}
	% 非局部注意力、sture
	\item 建模时空特征的高度复杂性。
	基于数据关联的多目标跟踪算法在进行当前视频帧候选检测和历史轨迹的关联时,需要一个非常鲁棒的时空特征来保证相似度计算的准确性,然而存在历史轨迹特征建模的不充分、空间维度上的检测结果特征和时空维度的历史轨迹特征之间的特征不平衡等问题,这些不可靠的特征会严重影响在线目标关联多目标跟踪方法的性能。
	%基于检测的数据关联多目标跟踪算法的性能于所采用的目标检测器的性能密切相关,然而现实的空间维度上的目标检测器并不是完美的,这些不可靠的检测结果会传播到时间维度上的数据关联步骤,从来严重影响在线数据关联多目标跟踪算法的性能。
	
	\item 检测和跟踪相互独立问题。
	目前利用目标关联进行视频多目标跟踪的方法把空间维度上目标检测与时间维度上的数据关联建模成是两个分开处理的研究问题,然而检测与跟踪两大计算机视觉任务之间有着密不可分的关系,优秀的检测器可以有效地提高跟踪的效果,同时如果将多目标跟踪结果反馈给前一步的检测器可以进一步提升目标检测器的效果,所以将检测和跟踪合并进入同一个任务是更好地提升视频多目标跟踪精度和速度的核心,同时也是多目标跟踪方法将来研究的方向。
\end{itemize}

因此,单跟踪目标模型的可解释性和数据关联算法中时空特征建模的复杂性是两个重要的难题。
%如何在复杂的场景下成功地捕获目标巨大的外观变化和应对遮挡问题是跟踪算法取得良好跟踪效果的关键所在。
本工作主要根据这两大类挑战进行研究。

%为了充分说明多目标跟踪的困难性,本文先简单介绍单目标跟踪的挑战,主要集中在设计复杂的外观模型和运动模式,包括11个挑战属性。

%(1) 光照变化:光照变化导致目标的外观发生极大的变化,会严重影响目标的颜色分布甚至是梯度信息。由于颜色特征和梯度特征是目标跟踪算法最常使用的两种特征,因此光照变化会在一定程度上影响算法的跟踪性能。
%
%(2) 尺度变化:目标尺度在跟踪过程中经常会发生变化。如果跟踪算法不能准确的估计目标的尺度,那么训练得到的外观模型就不能很好的建模目标的外观。这可能导致错误在跟踪过程中不断地积累,进而造成跟踪失败的结果。
%
%(3)遮挡:遮挡是目标跟踪中非常难解决的难题之一。由于目标跟踪是一个在线任务,因此跟踪模型必须在跟踪过程中不断地更新来捕获目标最新的外观变化。如果跟踪算法不能准确地检测到遮挡的存在,那么模型更新操作会将遮挡考虑进去,进而造成模型被污染和模型漂移问题。
%
%(4)形变:形变主要针对非刚体目标(如人和动物等)的跟踪,也是目标跟踪的难题之一。当非刚体目标发生剧烈的形变时,目标的外观甚至是宽高比都会有明显的变化。跟踪模型很难准确地捕获到这种外观和宽高比的变化。
%
%(5) 运动模糊:运动模糊主要会在目标快速运动的时候发生。当出现运动模糊时,目标的颜色、纹理甚至梯度信息都会变得模糊和不明显。这会影响跟踪算法区分目标和背景的判别能力。
%
%(6) 快速运动:快速运动指相邻两帧中的目标位置发生了很大变化。由于目标跟踪是一个在线任务,因此跟踪算法通常在有限的搜索范围内来定位目标以保证算法的实时性。当目标发生快速运动时,跟踪算法不可能在搜索范围内准确地定位目标,进而会导致跟踪失败。
%
%(7) 平面内旋转:平面内旋转指目标在图像平面内发生旋转。由于目标跟踪所使用的特征大都不具备旋转不变性,因此当目标发生平面内旋转时,跟踪算法很难准确地定位目标。
%
%(8) 平面外旋转:平面外旋转指目标的旋转超出了图像平面范围。当目标发生平面外旋转时,目标原来的可见部分将会消失,而原来的不可见部分将会出现在图像平面。在这种情况下,跟踪算法很难有效地捕获到目标不可见部分的外观信息,从而导致跟踪失败。
%
%(9) 目标消失:目标消失是指目标从图像平面内暂短消失。如上所示,目标跟踪是一个在线任务,跟踪模型必须在线更新来适应目标的外观变化。如果跟踪算法不能成功地检测到目标消失,那么模型更新将会导致模型漂移问题发生。即使目标在后来的跟踪过程中重新出现,跟踪模型也不能再准确地跟踪到目标。
%
%(10) 背景嘈杂:背景嘈杂指图像平面内包含很多与目标外观相似的干扰项。由于这些干扰项具有与目标相似的外观,因此跟踪模型很容易将背景中的干扰项作为最终的跟踪结果。
%
%(11) 低分辨率:低分辨率主要是由拍摄视频的设备所造成。如果视频的分辨率很低,那么目标的各种特征也将不明显和不具有判别力。因此,学习得到的跟踪模型将不能很好的区别目标和背景。



% 解决思路:时空注意力
% 2: 自下而上+自上而下的注意力
% 3: 全局注意力(时空)
% 4:时空互表征学习(不对称)
% 5:(检测)空间 -> (数据关联)(时空):解决重复+端到端训练
\section{本文主要的贡献与创新}
从上面的分析可知在进行多目标跟踪过程中存在的跟踪模型可解释性和时空特征复杂性两个难题,也是影响多目标跟踪算法理解和部署的重要原因。
处理好上述两个挑战对提升多目标跟踪性能非常重要。
所以本研究利用神经解剖对齐、非局部注意力、时空互表征学习和端到端等理论和模型,设计了一系列方案来处理这些挑战来增强模型的可解释性和提高多目标跟踪的效果。
本研究主要的贡献点包括:

(1)提出一种神经解剖对齐的类脑单目标跟踪深度神经网络模型。
为了理解和解释日益加深的深度单目标跟踪模型,降低模型复杂性,解决跟踪精度和人脑本身处理平滑跟踪之间的矛盾,
首先,该方法尝试将深度神经网络的各个模块与人类大脑皮层平滑跟踪通路相关脑区的解剖结构进行对齐,产生更加符合皮层解剖结构的类脑跟踪网络。
其次,对人类进行视觉目标跟踪时的大脑皮层激活响应和眼动行为数据进行分析,找出和跟踪任务相关的皮层区域和相应的激活响应。
最后,为了合理地评价模型的类脑性能,设计了一个新颖的度量方法来衡量类脑深度神经网络与人类大脑之间皮层激活响应和人眼行为的相似性。
通过深度探索模型的跟踪性能和与皮层和行为的相似性,发现设计的模型和人类大脑处理视觉跟踪任务时的关联性,并从模型结构、激活响应、行为动作等方面进行解释。

(2)为了应对多目标跟踪时目标外观、姿势变化、频繁遮挡等问题,并解决进行传统数据关联时历史轨迹特征建模的不完整性和一般卷积操作中特征提取的局部性,设计了一种在跨时空范围的非局部注意力模型,以达到更好的关联效果。
首先将非局部注意力层嵌入到传统卷积神经网络中,来自适应地提取跨空间和时间区域而不是局部区域的全局特征,使用非局部注意力机制来抑制目标检测不准确和遮挡问题。
其次,该方法还提出了一个注意力关联网络来处理多目标跟踪中的序列相关性和遮挡。
在关联轨迹和检测结果时,所提出的网络不仅生成目标检测与历史轨迹中观测值之间的相似性,还生成与所有行人序列的一致性,以减轻轨迹中不可靠样本和单目标跟踪器的跟踪漂移问题。
最后,设计了一个目标关联的学习算法和对应的数据处理策略。
在执行训练过程之前,利用多目标跟踪数据集的检测结果来生成各种行人段,并以等概率随机采样轨迹段,以满足网络的输入大小要求。
同时,从一系列数据增强策略中制定了一个方案,以解决模型训练过程中数据不足和过拟合的问题。

(3)提出了一种时空互表征学习方法。
为了解决当前帧检测结果与历史轨迹序列特征差异的问题,并解决当前检测结果的时间特征被忽略和关联双方特征不平衡的问题,使得当前候选检测在对象关联上可以更好地与历史轨迹进行关联。
首先,提出了一种新颖的时空互表征学习架构来解决当前检测的空间特征与目标关联的历史序列的时空特征之间的特征差异问题。
其次,为了增强所提出方法的互学习和识别能力,提出了交叉损失、模态损失和相似性损失,这些精细设计的损失都有助于检测学习网络获得时间特征,缓解特征不平衡的问题。
最后提出了一个基于单目标跟踪预测的多目标跟踪方法,通过使用时空互表征学习的策略来缓解单目标跟踪的漂移问题,提高所提出方法的准确性和鲁棒性。

(4)
提出了一种端到端模型架构和训练方法来联合目标检测和多目标跟踪。
传统的多目标跟踪将目标检测作为一个独立的前置任务进行研究和处理,但在实际应用中目标检测和多目标跟踪两个任务进行了重复的特征提取,导致计算代价过高,难以满足实际应用实时性的要求。
该方法使用同一个深度网络同时处理目标检测和数据关联,并解决了检测和跟踪两个任务连接时存在的数据不一致问题,并达到实时的跟踪效果。
首先。设计了一个端到端架构来联合处理目标检测和在线多目标跟踪两个任务。
其次,为了解决目标检测子模块的输出与关联子模块的输入之间的边界框数目和大小不一致的问题,提出了联接子模块和合适的训练数据产生策略。
最后,设计了一个迭代训练方法来训练检测子模块和关联子模块,并完全以端到端模式执行在线多目标跟踪过程。
并且在基准数据集上做了一系列消去试验研究,表明所提出的算法获得了比许多在线多目标跟踪方法更好的跟踪性能。

% 说tu 
以上四个贡献点在逻辑上为依次递进的关系,如图~\ref{fig:c1:organization}~的中间所示,第二章的单目标类脑跟踪模型为第三章和第四章基于单目标跟踪预测的多目标跟踪范式提供基础,第三章利用非局部注意力机制以更好地提取数据关联中历史轨迹的特征,第四章利用时空互学习策略解决当前候选检测结果和历史轨迹的特征不平衡问题,第五章在第四章的基础之上将目标检测任务合并进多目标跟踪任务,最后将本文的贡献在一个智能驾驶系统中进行应用验证。

% https://texample.net/tikz/examples/feature/smartdiagram/
%\begin{figure*}[ht]
%	\centering
%	\includegraphics[width=0.98\textwidth]{./figures/C1Fig/organization.pdf}
%	\caption{本文的组织结构}
%	\label{fig:organization}
%\end{figure*}


% 锦上添花
% +绪论
% + 结论和展望
\begin{figure*}[ht]
	\centering
	\includegraphics[width=0.90\textwidth]{./figures/C1Fig/network.pdf}
	\caption{本文组织结构}
	\label{fig:c1:organization}
\end{figure*}

\section{本文组织结构}
本文研究和探索了在动态开放场景下的视频多目标跟踪问题,来尝试解决单目标跟踪器的可解释性和时空特征建模的复杂性问题。
本文总共分为七个章节,整体逻辑架构如图~\ref{fig:c1:organization}~所示。


第一章为绪论,概述了多目标跟踪问题的背景和意义、国内外研究现状以及相应的数据集等,着重介绍了多目标跟踪任务的问题和挑战以及本研究的主要创新点。

第二章介绍深度神经网络模型在处理视觉目标跟踪问题时候可解释性不足的的问题。
然后重点介绍所提出的神经解剖对齐的视觉目标跟踪模型的具体细节。
最后将所设计的算法在公共数据集上进行测试并和其他方法进行比较来论证所提出方法的有效性。

第三章首先分析基于多目标跟踪所存在的问题和传统方法的局限性。
然后介绍了设计的嵌入在卷积神经网络中的非局部注意力层来自适应地提取跨时空区域的非局部特征。
最后介绍了该算法在基准数据集上的测试性能。

第四章根据时空互学习的思想,先简述了目前存在的多目标跟踪数据关联方法所存在的问题。
随后进行了相关方法的简述。
再详细介绍了所设计的时空互表征学习算法并在多目标跟踪数据集上和其他方法进行比较。

第五章先概述了目前检测和跟踪分开研究所存在的问题,并且引出了端到端学习的必要性。
然后详细讨论了联合检测和关联的研究现状。
详细阐述了所设计的端到端检测和关联方法和该方法在多个多目标跟踪数据集上的实验结果。

然后,第六章将本文所设计的跟踪方法部署在一个实时多目标跟踪和分析系统中,论证了本文工作的的实用性和对现实社会的贡献。

结论与展望章节先简要总结了本文的所有研究点。
随后讨论了本文研究尚存在的缺点和不足,最后提出了将来可能的探索方向。



