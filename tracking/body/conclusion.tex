% !Mode:: "TeX:UTF-8"

\addcontentsline{toc}{chapter}{结论与展望} %添加到目录中\quad 
\chapter*{结论与展望}

在动态开放的多目标跟踪环境中,多目标跟踪模型的可解释性和关联特征建模的复杂性是制约跟踪算法取得好的跟踪效果的主要原因。
因此,本文根据跟踪模型在动态开放场景中的可解释性和关联特征建模复杂等挑战进行研究,设计了一系列模型来有针对性地处理这两个问题来提升多目标跟踪模型的效果。
下面简要说明本文所取得的成果以及将来的研究展望。

一、本文的主要成果

根据神经科学、深度学习、计算机视觉等领域的思想和方法来应对动态开放跟踪场景下多目标跟踪所存在的可解释性和关联特征问题,取得的主要成果包括:

(1)
提出了一种类脑跟踪模型,来解决传统深度跟踪模型进行视觉目标跟踪和人类大脑处理平滑跟踪机理之间的矛盾。
从网络结构、中间激活和行为输出三个角度来设计和衡量类脑模型的类脑相似性。
在网络结构方面,通过将平稳跟踪相关脑区映射到不同深度模型的模块,使深度网络结构与人类大脑结构一致,达到理解目前日益加深的深度跟踪模型的作用,同时能去除冗余的模块,加快跟踪模型的运行速度,以达到应用实时性的要求。
%首先,提出一种和人脑解剖结构对齐的深度跟踪神经网络模型,使其更加符合大脑皮层平稳跟踪的解剖通路。
在中间层激活方面,将大脑皮层中 MT/MST 的激活与类脑神经网络动态滤波网络层的激活进行对比,发现了它们之间激活的相关性,为模型的运动感知提供了具体的解释,激活的相似性同样为脑机接口的应用提供了可能。
在行为输出方面,主要考虑了人眼注意力范围和类脑跟踪网络输出的边界框之间的交并比。
并从中间层激活和行为输出方面定量化衡量了大脑皮层处理模式和类脑跟踪模型处理之间的相似性,发现类脑模型能较好地模拟大脑处理视觉目标跟踪的机理和行为。
所提出的类脑模型不仅能够很好地符合计算机视觉对目标跟踪的工程需求,还能对模型所具有的结构和输出行为做出合理地解释,拉近神经科学和计算机科学之间的距离,使大脑的理解和神经网络的设计得以相互促进。
%最后,设计了一种新颖的度量方法来衡量类脑跟踪模型与人脑皮层响应和人眼行为之间的相似性,以评估所提出模型的类脑性能。
%通过在公共的数据集上进行实验,深度分析了类脑模型的跟踪性能与大脑皮层响应和人眼行为之间的相似性,验证了所提出的类脑跟踪模型不仅能够取得比较好的跟踪效果,而且表明所设计的模型和人脑大脑处理平稳跟踪任务时的相关性,并从类脑模型结构、皮层激活响应和人眼行为等方面进行了解释。

(2)
提出了一种综合时空特征的非局部注意力模型。
%为了应对多目标跟踪环境种目标外观变化、频繁遮挡等问题,解决当前视频帧的检测结果与历史轨迹段进行数据关联时,当前检测结果缺乏时间特征和传统卷积操作提取特征的局部性,提出全局注意力模型,以达到更好的关联效果。
由于传统的卷积神经网络在提取视频序列的特征时不能很好地同时考虑时间和空间上的特征,无法处理动态开放场景下各种复杂的挑战,
本文提出了一种在卷积神经网络中加入全局注意力层来自适应地学习视频序列中的时空非局部特征,由于考虑了整个跟踪轨迹的全局信息,能够很好地抑制其中检测结果不准确等情况。
并使用了预测跟踪范式来很好地处理了检测器的检测结果不准确和没有利用目标的运动信息等问题,在单目标跟踪不可靠时使用注意力关联策略来对跟踪漂移进行纠正,较好地处理了目标遮挡和跟踪轨迹中存在的异常样本对跟踪性能的损害。
同时利用一系列数据增强方法和训练策略来进行模型的训练,经过各种消去研究和实验分析证明了非局部注意力机制和预测跟踪范式整体提高了动态开放场景下多目标跟踪的性能。
%,以抑制部分检测不准确和遮挡的挑战。
%其次,还提出了一种注意力关联方法来处理多目标跟踪过程中序列的相关性和遮挡,以缓解历史轨迹中不可靠样本对关联的影响。
%最后,提出一种关联训练方法和数据增强策略解决训练数据不足的问题。
%在基准数据集上进行一系列实验证明了所提出的非局部注意力机制和注意力关联网络可以取得较好的的跟踪结果。

(3)  
提出了一种时空互学习模型来处理多目标跟踪数据关联时的特征不平衡问题。
当进行数据关联时当前帧的检测结果缺少历史轨迹的时间信息,导致不能进行有效地数据关联,严重损害了多目标跟踪的性能。
%%为了解决当前检测结果缺少时间信息和历史轨迹序列特征之间的差异问题,使当前候选检测在进行数据关联时能够很好地进行关联。
而所提出的时空互表征学习方法将检测集的空间特征和序列集的时空特征置于同一时空互表征空间,并通过选择反向传播策略,使检测学习网络能够很好地学习到历史轨迹序列的时间特征,增强当前检测的表达能力,
并通过设计合理有效的损失函数来提升模型的辨别能力,使时空互表征学习能够更好地进行。
并利用单目标跟踪和数据关联方法解决了检测结果不完美和单目标跟踪器漂移的问题,
经过一系列的消去实验和性能测试表明所提出的时空互表征学习方法能很好地解决在线多目标跟踪数据关联中的特征不平衡问题,并提升了跟踪性能。
%首先,本文提出了一种时空互表征学习框架来解决当前检测结果仅包含的空间信息和历史轨迹序列包含时空信息之间的特征不平衡问题。
%其次,为了增强所提出方法的互学习和特征区分能力,设计了三种损失函数:交叉损失、模态损失和相似性损失,有助于检测学习网络获得历史序列中的时间特征。
%最后,设计了一种基于预测的多目标跟踪范式,通过使用时空互增强得到的特征来缓解单目标跟踪器的漂移现象。
%通过在多目标跟踪基准数据集上执行一系列消去试验,证明了该方法能够在复杂的跟踪场景下取得非常不错的跟踪性能。

(4)  
提出了一种联合目标检测和数据关联的在线多目标跟踪方法,解决了传统计算机视觉将检测和跟踪分开处理对目标特征重复提取的问题。
并利用检测子模块进行目标检测和特征的提取,连接子模块来进行前后帧目标特征的连接和融合,关联子模块来进行关联矩阵的预测,并进行在线多目标跟踪。
在此过程中,解决了进行联合训练时检测子模块的输出和关联子模块的输入在目标大小和位置上的不一致问题,利用传统关联方法生成伪标签的方法为端到端的模型训练提供合适的训练数据,并缓解了检测器误差会传播到数据关联步骤的问题,
经过一系列的性能实验和消去研究表明所提出的端到端方法极大提升了多目标跟踪算法的运行速度和跟踪效果,使多目标跟踪算法能在实际场景中达到实时的效果。
%端到端的模型和训练方法。
%为了解决检测和跟踪两个任务分开处理时对实时性造成的影响,本文利用端到端的模型处理了两个任务之间的矛盾,达到了实际应用实时性的要求。
%首先,设计了一种端到端的模型架构来联合处理目标检测和在线多目标跟踪任务。
%其次,为了端到端模型中目标检测子网络的输出和目标关联子模块的输入之间的不一致问题,提出了联合子模块和合适的伪标签生成方法。
%最后,设计了一种两阶段迭代训练方法来训练所提出的检测子模块和关联子模块,并以一种完全端到端的方式进行实时在线多目标跟踪。
%在公开的基准数据上进行了一系列消去研究,表明所提出的联合检测和关联的方法取得了相对于许多其他在线多目标跟踪方法有竞争力的跟踪精度和运行效率。

二、将来研究展望

本文设计的模型虽然获得了较好的测试效果,但依然存在不足之处。
比方说本文所设计的算法在类脑相似性精度和端到端模型的设计上仍有一定的改进空间。
为更好地提升跟踪模型的效果,将来会从以下三个方向进行研究:


(1)在类脑跟踪模型中融入大脑皮层腹侧流类脑结构的设计,并考虑其它模态的融合。
在本文的工作中,主要参照人类进行目标跟踪时背侧流的处理通路,然而腹侧流仅仅使用简单的卷积操作进行模拟,可以参照腹侧流通路的解剖结构,利用卷积和循环结构,构建大脑对齐的深度模型,可以设计相对应的类脑图像识别网络,并获得输入图像在深度网络中的激活响应,研究图像识别功能在大脑皮层和深度神经网络中表现的激活相似性,同时进行进行神经度量和行为度量,利用类脑识别分数进行深度神经网络和大脑的对比,并利用设计并训练好的类脑模型,确定识别的图像模式分别在深度网络和大脑皮层中的位置,寻找图像识别的类别信息在深度网络中的表征模式与大脑中的激活特征之间的映射关系。
%此外,由于人脑是一个多模态数据同时处理并相互影响的场所,如何加入其他模态的数据进行分析也是一个提高类脑模型精度的关键。

(2)
考虑大脑中其他模态的信息和视觉信息的相互作用,类比于人类感知中的“通感”。
在人所接收的所有感知信息中,听觉信息是仅次于视觉信息的第二大感知源,两者占人所有感知信息的百分之九十以上,所以在建模类脑的多模态感知时主要考虑人类的听觉。
听觉识别通路主要包括初级听觉皮层 A1、带状区 Belt、伞状区 PB、中颞/下颞区,并利用卷积神经网络和循环网络为基本结构,模拟人脑处理模式,构建大脑对齐的类脑模型,进行激活和行为之间的对比,包括大脑激活和神经网络激活之间的相似性、网络预测的音频类别和人类动作选择之间的相似性。
同时由于听觉识别的核心区域中颞/下颞区和第一阶段中的目标识别的下颞核心区有重叠部位,为视觉和听觉的融合处理提供了神经解剖基础,可以进一步在图像识别模型和音频识别模型的基础之上设计视觉/听觉融合模块,进行高层特征级别融合,有望进一步提高处理复杂环境输入的能力和提升类脑模型的预测能力。

(3)
提高基于预测跟踪范式的运行效率。
将单目标跟踪器应用到多目标跟踪的应用中,需要给每一个目标都初始化一个跟踪器,这给多目标跟踪带来了较大的运行开销。
所以未来可以考虑将多个单目标跟踪器进行合并,共享跟踪的基础模块,仅仅保存不同目标之间的差异信息,减少存储开销和运行代价,提高跟踪的速度,以更好的将基于数据关联的多目标跟踪算法应用到现实的实时系统中。
因此,这也是未来的一个重要的研究方向。

(4)
考虑设计完全端到端检测跟踪模型和训练算法,并实现像素级跟踪。
本文中提出的联合检测和数据关联的在线多目标跟踪方案虽然能把检测和关联放在一个深度模型中,但训练过程仍然是两阶段的,因此在未来的工作中可以考虑设计一种一阶段的训练方法,达到一次性训练完成整个模型。
同时,还可以将关联后的在线多目标跟踪的后处理步骤融入到端到端模型中,甚至将跟踪的精度作为损失函数加入到模型的训练过程中,进一步提高端到端检测跟踪模型的精度,
并考虑将基于锚框的跟踪升级为更加精细的像素级跟踪。
因此,提出一种能够完全端到端的跟踪模型对减少人工设计跟踪处理策略和应对复杂多变的场景具有十分重要的意义。



