% !Mode:: "TeX:UTF-8"

\chapter{ 基于非局部注意力机制的多目标跟踪数据关联策略} \label{chap:nonlocal}
% /data2/whd/win10/doc/paper/doctor/doctor.Data/PDF/0668102945

\section{引言}
% Introduction
视频多目标跟踪是计算机视觉方向最基本和最核心的科学问题之一,并且在智能监控、智能机器人、无人驾驶车和人机交互等方面得到了大规模的使用~\cite{autonomous_vechicle}。
视频多目标跟踪的目标是在图像帧序列中准确估计所有对象的状态(包括位置和身份),它旨在通过在整个视频帧中查找目标位置和维护目标身份来估计多个对象的轨迹。
%它是计算机视觉领域最基本和最重要的问题之一,并且在视频监控、智能机器人、无人驾驶车和人机交互等方面有着广泛的应用。
% 尽管多近年来多目标跟踪有了长足的发展,
近年来,由于深度学习~\cite{b4} 的进步,多目标跟踪取得了一定的进展,但是由于背景复杂和相互遮挡等挑战的存在使得动态开放场景下的多目标跟踪仍然是一个非常困难的问题。
一般来说,现有的视频多目标跟踪方法一般分为在线多目标跟踪算法与离线多目标跟踪算法。
离线多目标跟踪算法利用历史和将来的视频帧来产生跟踪轨迹,但是在线多目标跟踪算法仅使用当前时刻可以使用的数据。
尽管离线多目标跟踪算法可以应对一些不确定跟踪的情况,但是无法在实时场景下得到应用。

% 摘要
在线多目标跟踪作为视频分析和多媒体应用中的一个基本问题,
常用的基于检测跟踪框架的主要挑战是如何将候选检测结果与现有的轨迹段进行关联。
在这方面,本章提出了一种非局部注意关联方法,并将其应用于一个统一的在线多目标跟踪框架,该框架集成了单目标跟踪和数据关联方法各自的优点。
具体来说,该方法提出一种非局部注意关联网络(Non-local Attention Association Networks,NAAN)融合空间和时间特征来进行新目标的检测和历史轨迹的关联。
利用非局部注意力生成跨空间和时间的非局部注意力特征,使得跟踪模型能够关注整个轨迹的信息,而不是局部注意力特征,以克服噪声检测、遮挡和目标之间频繁交互等问题。


随着目标检测技术~\cite{b8} 的发展,跨帧链接检测结果的数据关联算法已经成为多目标跟踪的主流。
然而,这些方法严重依赖于不完美的检测器。
如果检测结果不准确、遗漏或错误,则跟踪对象容易丢失。
可以通过在多目标跟踪环境中使用最新的且精度较高的单目标跟踪器~\cite{b10} 来缓解这种问题。
单目标跟踪器使用第一帧中的检测结果并在线更新单目标跟踪模型,以确定后续帧中跟踪目标的位置和大小~\cite{weight_based}。
然而,当跟踪目标被遮挡~\cite{local_sparse} 时,这种方法容易发生漂移。
为此可以将单目标跟踪器和数据关联的优点结合在一个统一的框架中来解决这个问题。
在大多数视频帧中,使用单目标跟踪器来跟踪每个目标。
然后当跟踪分数低于阈值时应用数据关联方法解决漂移问题。
该方案表明被跟踪的目标可能会经历较大的外观变化或被其他物体遮挡。

通常,直接使用现有的单目标跟踪器进行多目标跟踪的主要挑战是处理跟踪目标和类内干扰物之间的频繁交互。
此外,单目标跟踪器在在线模型更新过程中通常会遇到正负样本之间数据不平衡的问题。
在单目标跟踪器的搜索区域中,被跟踪目标中心附近只有少数地方对应于正样本,而其他地方的所有样本都是负样本,所以背景区域的大多数位置将生成负样本。
这种情况可能会造成正负样本之间的不平衡,削弱单目标跟踪模型的判别能力。
如图~\ref{fig:nlaa_tracking_problem} 所示,这个问题在多目标跟踪任务的上下文中进一步加剧。
如果单目标跟踪模型被大量背景负样本淹没,那么当搜索区域中出现类似的干扰项时,跟踪器很容易发生漂移。
当跟踪过程变得不可靠时,需要使用数据关联方法将候选检测与历史行人序列联系起来。


\begin{figure*}[ht]
	\centering
	\includegraphics[width=0.8\textwidth]{figures/C3Fig/tracking_problem.pdf}
	\caption{数据关联的动机}
	\label{fig:nlaa_tracking_problem}
\end{figure*}

在执行数据关联时,需要将一系列先前跟踪的对象与当前帧检测进行比较。 
多目标跟踪任务中最常见的跟踪对象就是行人,其中数据关联问题也称为行人重新识别,该任务具有各种挑战性因素,包括相似的外观、姿势变化、频繁的遮挡等。
然而,传统的卷积操作只关注局部特征和检测区域。
在多目标跟踪的背景下,这些跟踪结果可能会带有一些未对齐错误或丢失跟踪目标部件的噪声。
因此,历史轨迹中的不准确和被遮挡的结果很可能会导致单目标跟踪模型的错误更新,从而导致轨迹特征提取模型的有效性降低。
为了解决上述问题,需要为数据关联设计更有效的轨迹特征提取方法。
本章提出了跨时空范围的非局部特征提取模型,来应对传统卷积操作中特征提取的局部性问题。
%这些因素促使为数据关联设计有效的轨迹特征提取模型,从而抑制上述问题。
%为了,确保所。
%
最后在公开的基准数据集上进行一系列研究实验说明了所提出的算法与各种基于身份保留的在线跟踪器相比表现良好。

本章的主要贡献如下:
\begin{itemize}
	\item  设计了一个嵌入在卷积神经网络中的非局部注意力层来自适应地提取跨空间和时间区域而不是局部区域的全局特征,并使用非局部注意力机制来抑制不准确和被遮挡的错误检测。
	\item  提出了一个注意力关联网络来处理多目标跟踪中的序列相关性和遮挡问题。在关联当前检测结果和历史轨迹时,所提出的网络不仅生成目标检测结果与历史轨迹之间的相似性,还生成所有行人序列的一致性,以减轻轨迹中不可靠样本的影响。
	\item  提出了一种数据关联的训练方法。在执行训练过程之前,利用多目标跟踪数据集的检测结果来生成各种行人段,并以等概率随机对轨迹段进行采样,以满足网络的输入大小要求。同时,从一系列数据增强策略中制定了一个方案,以解决模型训练过程中数据不足和模型欠拟合的问题。
	\item  通过在多目标跟踪基准数据集上进行大量的消去实验并与最先进的多目标跟踪方法进行比较来证明所提出算法的有效性。
\end{itemize}


\section{相关工作}
\subsection{多目标跟踪}
多目标跟踪任务的目标是解决数据关联问题,它通常采用检测跟踪范式。
依据多目标跟踪算法是否使用将来视频帧的信息,可以分为在线和离线跟踪方法。
离线多目标跟踪方法~\cite{b2,b17} 使用来自过去和未来帧的检测结果进行批处理。
此类方法具有利用所有视频帧全局信息的优势。
通常,离线多目标跟踪方法将多目标跟踪任务建模为各种形式的全局优化问题,例如网络流~\cite{b17} 和多割~\cite{b2}。
相比之下,在线多目标跟踪方法~\cite{b10, PHD_filter} 不能利用来自未来帧的检测结果和帧信息,并且在目标对象被严重遮挡或检测不准确时可能表现不佳。
因此,鲁棒的外观模型对于关联在线多目标跟踪的检测结果至关重要。
最近已经提出了一些使用深度学习模型的在线方法\cite{b10,b23,b24}。 
孪生网络~\cite{b1} 对来自 RGB 图像空间的外观信息和来自光流图的运动信息进行编码,然后通过基于线性规划的跟踪器处理获得的特征。
在 AMIR~\cite{b23} 中,LSTM 网络被用来对外观特征进行建模。
该方法通过逐步获取轨迹段中的图像来预测相似度分数。
在本章的工作中,引入了一种在线注意力关联多目标跟踪方案来处理不准确的检测和遮挡问题。
大量的实验表明该方法与最先进的在线多目标跟踪方法相比,所提出的在线算法可以实现良好的身份保持和跟踪性能。

\subsection{注意力机制}
许多视觉任务方法都采用了注意力机制,例如图像字幕~\cite{b25}、视觉问答~\cite{b27} 和图像分类~\cite{b29}等。
视觉注意力机制使模型能够专注于输入图像的最相关区域,以提取适合大量特定视觉任务的判别性特征。
非局部均值~\cite{b30} 是一类成熟的滤波方法,它计算视频图像内所有像素的加权平均。
该算法可以使远处的像素基于图像块的图像相似度对某个位置的响应做出贡献。
这种非局部滤波器的思想后来发展成为一种称为三维块匹配~\cite{b31} 的方案,它对一组相似但非局部的图像块进行滤波。
块匹配与神经网络一起用于图像去噪~\cite{b33}。
非局部注意力机制也成功应用于纹理合成\cite{b34}、超分辨率\cite{b35} 和修复\cite{b36} 等领域。
自注意力模块关注特征空间中的所有位置并取加权平均来进行某个位置响应的计算。
在本章的研究工作中,将利用非局部注意机制集成空间和时间特征到所提出的多目标跟踪算法中。


\section{非局部注意力关联算法}
本节中提出的解决前面提到问题的方法是利用单目标跟踪和非局部注意关联来维持多目标跟踪过程中目标的身份。
图~\ref{nonlocal_attention_network} 显示了所提出的在线多目标跟踪流程。
对每帧中的所有目标检测结果,先利用单目标跟踪器对每个检测目标进行正常的单目标跟踪过程和身份的维持。
%先是利用单目标跟踪器进行正常的目标跟踪和身份的维持。
目标刚出现时将跟踪目标的状态设置为跟踪,直到跟踪结果变得不可靠(例如,跟踪分数较低或跟踪结果与检测结果不一致),在这种情况下,跟踪目标的状态被视为漂移,
然后暂停单目标跟踪器并执行注意关联以计算历史跟踪轨迹与未被任何跟踪目标覆盖的当前检测结果之间的相似性。
一旦漂移目标通过注意力关联与检测结果相关联,则更新跟踪状态并恢复跟踪过程,
这个过程在章节~\ref{attention_association} 中进行详细的描述。
%当出现正当或相互干扰等情况让跟踪过程变得不可靠时,
%则使用注意力关联方法关联当前帧的候选检测结果与历史行人序列。



\subsection{方法框架}

\begin{figure*}[ht]
	\centering
	\includegraphics[width=0.85\textwidth]{figures/C3Fig/MOT_pipline.pdf}
	\caption{在线多目标跟踪流程}
	\label{fig:nlaa_MOT_pipline}
\end{figure*}


如图~\ref{fig:nlaa_MOT_pipline} 所示,所利用的在线多目标跟踪流程主要由三个子任务组成:单目标跟踪、检测和注意力关联。 
每个跟踪对象由五种状态组成:出生、激活、跟踪、漂移和死亡。
对于给定的行人图像序列,跟踪器的目标是使用深度卷积模块来提取行人的特征表示,从而在嵌入空间中实现基于轨迹的行人重新识别。 
学习图像序列代表性特征的关键因素是将轨迹的时空特征合并到特征中。 
为此,将非局部注意力层引入传统卷积神经网络以学习图像序列的时空依赖性。 
在章节~\ref{nonlocal_attention_network} 中提出了一个非局部注意关联网络 NAAN,以在不同的特征级别应用此操作。

\subsection{非局部注意力网络} \label{nonlocal_attention_network}

\begin{figure*}[ht]
	\centering
	\includegraphics[width=0.6\textwidth]{figures/C3Fig/association_network.pdf}
	\caption{NAAN 中的注意力关联模块概述}
	\label{fig:nlaa_association_network}
\end{figure*}

该方案如图~\ref{fig:nlaa_association_network} 所示,为了提取行人历史序列图像的特征表示,将通过统一采样策略选择的轨迹帧子集作为网络的输入。 
然后,结合非局部注意层和特征池化层的骨干卷积神经网络获得用于基于轨迹重新识别的特征,随后计算用于 $ M $ 张图像行人的合并特征与当前检测结果之间的相似性。
其中的非局部注意力网络如图~\ref{fig:nlaa_attention_network} 所示,用于提取历史行人序列的特征表示,
给定 $T$ 采样图像作为输入,并使用五个非局部注意力层和一系列 ResNet-50 网络提取行人图像序列的时空信息,然后在 3D 平均池化中将特征池化为用于注意力关联的一维向量。

\begin{figure*}[ht]
	\centering
	\includegraphics[width=0.6\textwidth]{figures/C3Fig/attention_network.pdf}
	\caption{非局部注意力网络的详细描述}
	\label{fig:nlaa_attention_network}
\end{figure*}


\begin{figure*}[ht]
	\centering
	\includegraphics[width=0.45\textwidth]{figures/C3Fig/attention_layer.pdf}
	\caption{非局部注意力层的细节}
	\label{fig:nlaa_attention_layer}
\end{figure*}

为了有效提取历史轨迹的时空特征,这里使用非局部注意机制并将非局部块~\cite{b37} 嵌入到主干卷积神经网络。
继非局部均值算法~\cite{b30} 之后,该研究工作在骨干卷积神经网络中定义了一个非局部操作,如图~\ref{fig:nlaa_attention_layer} 所示,
$\otimes$ 表示矩阵乘法,$\oplus$ 表示元素求和,并对每一行执行 softmax 操作。 
绿色框表示 $1 \times 1 \times1$ 卷积。 
在这里,采用了具有 $C$ 个通道瓶颈的嵌入式高斯版本。
\begin{equation}
y^i=\frac{1}{C\left( x \right)} \sum_{\forall j}  f{\left( x_i,x_j \right) g\left( x_j \right) }\mbox{,}
\label{nonlocal_operation}
\end{equation}
其中,$i$ 是要需要计算响应的时空输出位置下标,$j$ 为枚举历史轨迹中所有可能的时空位置下标,$x$ 为输入图像序列,并且 $y$ 是与 $ x $ 有相同尺寸的特征。 
两个输入变量的函数 $f$ 计算 $i$ 和 $j$ 之间的亲和度。
$g$ 表示处于位置 $j$ 上的输入特征。
响应由因子 $ C\left(x\right) $ 进行正则化,
公式中非局部层中的操作是一种子注意力机制,在非局部注意力网络~\cite{b37} 中也有提到,并设置:
\begin{equation}
C\left(x\right)=\sum_{\forall j}f\left(x_i,x_j\right)\mbox{,}
\end{equation}
在非局部操作中使用高斯函数的简单扩展来计算嵌入空间中的相似性。
此外在本研究中,使用点积相似度 $\theta \left(x_i\right)^T \phi \left(x_j\right)$,并将 $f$ 定义为:
\begin{equation}
f\left(x_i,x_j\right)=e^{ \theta \left(x_i\right)^T \phi \left(x_j\right) }\mbox{,}
\end{equation}
其中 $ \theta \left(x_i\right) = W_\theta x_i $ 和 $ \phi \left(x_j\right)=W_\phi x_j $ 是两个嵌入特征。
为简单起见,只考虑线性嵌入形式的 $g$,即 $g\left(x_j\right) = W_g x_j $,
其中 $W_\theta$、$W_\phi$ 和 $W_g $ 表示相应的需要要学习的权重矩阵。 
该块($\theta$、$\phi$ 和 $g$)在空间-时间中实现为 $1 \times 1 \times 1$ 卷积。

总之,从长度为 $T$ 行人图像的序列中获得输入特征张量 $ X\subseteq R^{C\times T\times H\times W} $,让 $ x_i \in R^C $ 从 $X$ 中采样,目的是从所有图像中聚合空间位置特征和时间特征。  
非局部注意力操作对应的输出 $y_i\in R^C$ 可以详细表述如下:
\begin{equation}
y^i=\frac{1}{\sum_{\forall j} e^{\theta\left(x_i\right)^T \phi \left(x_j\right)}} \sum_{\forall j} e^{\theta\left(x_i\right)^T \phi \left(x_j\right)} g\left(x_j\right)\mbox{,}
\end{equation}
其中 $i,j=\left[1,THW\right]$ 索引了二维空间特征图和所有时间序列帧中的位置。 
首先,通过使用 $ 1 \times 1 \times 1 $ 卷积核实现的线性变换函数 $\theta$、$\phi$ 和 $g$。 
随后利用嵌入的高斯实例化,通过所有坐标的加权平均 $x_j$ 来表示每个坐标处的响应 $x_i$。

最终整个非局部层最终被形式化为:
\begin{equation}
Z=W_{Z}Y+X\mbox{。}
\end{equation}
如公式~\ref{nonlocal_operation} 中所定义,非局部操作 $Y$ 的输出被添加到原始特征张量 $X$ 中,并进行了 $1 \times 1 \times 1 $ 卷积核的变换 $W_Z$,故 $Y$ 被映射到原始特征空间 $R^{C}$ 中。
直觉上可以将非局部操作归因于在给定时间内提取特定位置的特征,其中网络应通过非局部上下文来考虑序列内的时空依赖性。
如图~\ref{fig:nlaa_attention_network} 所示,在本章中的行人非局部注意力关联方案中,将五个非局部注意力层加入到骨干卷积神经网络 ResNet-50 里,以理解轨迹段中呈现的语义关系。
此外,与连续叠加卷积和循环神经网络算子相比,非局部操作能直接计算轨迹的时空位置之间的关系,达到快速捕获远程和全局依赖关系的目的。


\subsubsection{特征池化层}
如图~\ref{fig:nlaa_attention_network} 所示,将轨迹段的图像序列输入给具有非局部注意力层的骨干卷积神经网络后,使用特征池化层获得注意力关联的最终特征。
随后沿时空维度使用三维平均池化,将序列图像的输出特征聚合成一个有代表性的特征向量,然后进行批量正则化。

\subsection{注意力关联}
\label{attention_association}
在跟踪过程中,一旦单目标跟踪过程变得不可靠,就暂停单目标跟踪器并将跟踪目标的状态设置为漂移。 
然后如~\ref{nonlocal_attention_network} 节中所讨论的,利用注意力关联方法,来确定是否应该将目标状态保持为漂移或将其更改为跟踪状态。 
通常,使用目标的单目标跟踪分数(即置信度图中的最高值)来衡量单目标跟踪的可靠性。 
然而,如果只依赖于跟踪分数,那么背景上的虚警检测很容易被高置信度地持续跟踪。 
考虑到一个被跟踪的目标在几帧中都没有得到任何检测,很可能是误报检测,利用跟踪器和检测器给出的边界框之间的重叠来过滤掉误报。 
因此,可以将跟踪目标的状态定义为:
\begin{equation}
s_{tracking}=\left\{
\begin{array}{rcl}
1 & {if \ s > \tau_s \ and \ o_{m} > \tau_o}\\
0 & {otherwise}\mbox{,} 
\end{array} \right.
\end{equation}
其中,$ s_{tracking} $表示跟踪状态,1 表示跟踪,0 表示漂移,$s$、$\tau_s$ 和 $\tau_o$ 分别是跟踪目标的得分、跟踪得分的阈值和重叠率。 
历史轨迹 $o_{m}$ 的平均重叠定义为:
\begin{equation}
\label{overlap_mean}
o_{m}=\frac{\sum_{1}^{L} o\left(t_l,D_L\right)}{L}\mbox{,}
\end{equation}
考虑 $\sum_{1}^{L} o\left(t_l,D_L\right) $ 在过去 $ L $ 跟踪帧的平均值 $ o_{m} $ 作为决定跟踪状态时的另一种度量。
在公式~\ref{overlap_mean} 中,跟踪目标和检测之间的重叠率定义为:
\begin{equation}
\label{overlap_target_detection}
o \left(t_l,D_L\right) =\left\{
\begin{array}{rcl}
1& {if \ max \left(IOU \left(t_l,D_l\right) \right) > \tau_o } \\
0& {otherwise}\mbox{,} \\
\end{array} \right.
\end{equation}
其中,$t_l$ 表示第 $ l $ 帧的检测结果,$ D_l $ 为第 $ l $ 帧全部检测结果,$ T_l $ 为历史全部跟踪轨迹,如果前一个跟踪目标 $ t_1 \in T_l $ 与全部检测结果 $ D_l $ 之间的最大交并比大于 $\tau_o$,$o \left(t_l,D_l\right) $ 设置为 1。
否则,$o \left(t_l,D_l\right) $ 设置为 0。

在计算注意力关联的外观相似度之前,可以先利用运动线索来选择候选检测。 
当跟踪目标发生漂移时,将边界框的尺度保持在最后一帧 $k-1$,并利用线性运动方法来推断目标在当前视频帧 $k$ 中的坐标。 
令 $ c_{k-1}=\left[x_{k-1},y_{k-1}\right] $ 表示目标在 $ k-1 $ 帧处的中心坐标。 
计算目标在 $ k-1 $ 帧处的速度 $ v_{k-1} $ 为:
\begin{equation}
v_{k-1}=\frac{c_{k-1}-c_{k-K}}{K}\mbox{,}
\end{equation}
其中 $ K $ 表示计算速度的帧间隔。 
那么当前帧 $k$ 中跟踪目标的坐标预测为:
\begin{equation}
c_k=c_{k-1}+v_{k-1}\mbox{。}
\end{equation}

给定目标的预测位置,将预测位置周围没有被任何跟踪目标覆盖的检测(检测与预测位置之间的距离小于 $\tau_d$)作为候选检测。
同时还计算了检测结果与目标轨迹中的观察值之间的外观相似度。 
然后选择相似度最高的检测结果并设置相似度阈值 $\tau_a$ 来决定是否将漂移目标与该检测结果相关联。


\subsubsection{相似性计算}
为了计算历史轨迹和当前帧候选检测结果之间的相似性,将历史轨迹的非局部注意力特征($1 \times 2048$)和当前帧候选检测结果的嵌入特征($1\times 2048$)合并到一个单独的特征并将其输入到一个输入大小为 4096 维的全连接模块,并输出轨迹段和候选检测的相似性概率。 
这个全连接模块有三层,分别有 4096、512 和 64 个隐藏单元。 
此外,每个全连接层都与批量正则化层和整流线性单元相结合。 
全连接层的最后一层预测历史行人轨迹与候选检测结果之间的相似性概率。
然后,对相似性预测执行具有交叉熵损失的二分类器。

最后,根据当前帧检测结果和历史轨迹成对的相似性分数在候选检测和漂移目标之间进行分配。


\subsubsection{训练策略}
利用真实检测结果和身份在 MOT16 和 MOT17 训练集中提供的信息来生成一些关联的候选检测图像和轨迹检测的身份信息,用于训练关联网络。
然而,训练数据只包含有限的身份,每个身份的序列由有限的样本组成。
因此,所提出的网络容易对训练集欠拟合。
为了缓解这个问题,在训练中采用了一些数据增强策略。
首先通过随机选择轨迹来训练 NAAN。
然后,随机生成对应于轨迹段的 $T$ 个检测结果。
此外,通过随机裁剪和重新缩放输入图像来增加训练集。
为了在实验中模拟嘈杂的多目标跟踪环境并防止在训练过程中欠拟合,通过用不同于真实轨迹身份的图像随机替换轨迹段中的一些图像,将噪声样本添加到训练轨迹段序列中。
由于训练集中的一些轨迹只包含几个样本,所以以相等的概率随机采样每个轨迹,以减轻类不平衡的影响并满足 NAAN 输入大小要求。
最后,通过优化交叉熵损失来训练所提出的 NAAN。


\subsubsection{轨迹的出现和消亡}
在跟踪过程中,使用多目标跟踪基准数据集~\cite{b42} 提供的检测结果来初始化对象跟踪过程并产生新的轨迹。
如果候选检测结果与任何轨迹边界框的的交并比小于阈值,则认为它是新出现的候选轨迹。
为了避免误报,只有当新出现的候选中边界框序列在 $L$ 连续帧期间都高于阈值 $\tau_i$ 时,才将其视为新轨迹。
关于轨迹消亡,当轨迹与任何检测结果都没有重叠时认为跟踪出现漂移并从跟踪结果中去除。
如果轨迹持续漂移超过 $\tau_t$ 帧或移出视野,则将结束轨迹。
但是,如果相同的跟踪对象再次出现,那么它将以与以前相同的身份被恢复。
此外,通过对收集到的观察样本进行统一采样以减少数据冗余,具体方法是将跟踪目标的 $M$ 个最近观测值和样本长度为 $T$ 的轨迹段用于注意关联,
该变量是以固定大小输入到非局部注意力关联网络中。



\section{实验结果与分析}
在本节中,将展示所提出的方法在公共基准数据集中的性能,并将其与现有的最新方法进行比较。 
首先简要介绍本研究中使用的数据集和评估指标,然后介绍所提出方法的实施细节和消去实验。 
在与其他方法进行比较之后,将介绍所使用方法的参数并对结果进行分析。

\subsection{基准数据集和评价指标}
在 MOT16 和 MOT17~\cite{b42} 基准数据集上评估了提出的在线多目标跟踪算法。 
MOT16 数据集包含 14 个视频序列(7 个用于训练,7 个用于测试),总共 11,235 帧和 292,733 个手动注释的真实边界框。
MOT17 基准数据集包含与 MOT16 数据集相同数量的视频序列,同时还提供三种不同的检测器结果(DPM \cite{dpm}、Faster-RCNN \cite{faster-rcnn} 和 SDP \cite{sdp})以获取更多信息以综合评估跟踪算法的性能。

本章使用多目标跟踪基准数据集~\cite{b42} 的多个评估指标进行性能比较,
除了标准的多目标跟踪精度(MOTA~\cite{b44})和多目标跟踪精度(MOTP~\cite{b4})之外,性能指标还包括 ID 召回率~\cite{b45}(IDR,正确识别的真实检测)、误报数(FP)、漏报数(FN)、ID 切换数(IDS)和片段数(Frag)。 
另外 IDR~\cite{b45} 最近已被添加到多目标跟踪基准测试中,以衡量跟踪器的身份保留能力。


\subsection{实验设置}
在线多目标跟踪流程第一步中使用单目标跟踪~\cite{b46},
%使用与  跟踪器相同的功能。
当跟踪对象漂移时使用注意力关联模块,并利用在 ImageNet 数据集上预训练的 ResNet-50 卷积块作为共享基础网络。 
轨迹段的长度设置为 $ T=8 $,轨迹中收集的最大样本数设置为 $M=100$。
所有跟踪对象的输入图像尺寸都调整为 $256 \times 128$。
学习率为 $10^{-4}$ 的 Adam 优化器用于训练非局部注意力网络,批量大小设置为 32。
在 NVIDIA GeForce RTX 2080Ti 上训练过程 1.5 小时,持续 40 个迭代周期。
考虑到多目标基准数据集规模不大,阈值参数的所有值都是根据 MOT16 和 MOT17 训练集上的 MOTA 性能设置的。
$ F $ 是视频的帧率,
计算跟踪目标速度的区间设置为 $K=0.3F$。
轨迹初始化阈值设置为 $\tau_i=0.2F$,而轨迹终止的阈值设置为 $\tau_t=2F$。
跟踪分数和外观相似度的阈值分别设置为 $\tau_s=0.2$ 和 $\tau_a=0.8$。
重叠和距离的阈值分别设置为 $\tau_o=0.5$ 和 $\tau_d=2$。
此外如图~\ref{fig:nlaa_grid_search} 所示,为了避免手动随意设置超参数的局限,本实验通过网格搜索的方法选择跟踪分数和外观的阈值。
所提出的跟踪方法使用 Python 的软件库 Pytorch 0.4.1~\cite{b49} 进行实现。


\subsection{消去实验}
如图~\ref{fig:nlaa_ablation} 所示,通过每次禁用一个基础模块来进行消去研究,以验证每个模块在所提出的方法中的贡献。
与在多目标跟踪测试数据集上的完整模型(44.5$\%$) 相比,都比每个基准方法的 MOTA 分数高。 
说明提出的组件都有助于多目标跟踪性能的提升。
当直接使用跟踪分数进行注意力关联时,MOTA 指标下降了 13.1$\%$,表明所提出的 NAAN 的优势。
B2 中性能的退化证明了添加到标准卷积神经网络中的非局部注意层的有效性。 
每个基线方法可以描述如下:

\begin{itemize}
	\item  B1 表示禁用所提出的 NAAN 并使用跟踪分数来关联历史轨迹和当前检测结果。 
	具体来说,将跟踪器的卷积滤波器应用于候选检测,并直接使用置信图中的最大跟踪分数作为注意力关联的外观相似度。
	\item  B2 表示禁用非局部注意力层,并使用标准的卷积神经网络架构提取历史轨迹段的特征,将其用于轨迹的身份验证。
\end{itemize}


\begin{figure*}[ht]
	\centering
	\includegraphics[width=0.8\textwidth]{figures/C3Fig/ablation.pdf}
	\caption{消去实验结果}
	\label{fig:nlaa_ablation}
\end{figure*}


\begin{figure*}[ht]
	\centering
	\includegraphics[width=0.8\textwidth]{figures/C3Fig/grid_search.pdf}
	\caption{网格搜索超参数}
	\label{fig:nlaa_grid_search}
\end{figure*}


\subsection{在多目标跟踪基准数据集上进行评估}
表~\ref{tab:nlaa_tracking_performance} 和表~\ref{tab:nlaa_performance_MOT17} 展示了所提出的方法在 MOT16 和 MOT17 数据集的定量性能,并在 MOT16 和 MOT17 基准测试集上和常见的方法进行比较。 
所提出的方法在 MOT16 和 MOT17 数据集上获得了较好的 MOTA 分数,并且在 MOTA、MOTP、IDR、FP、FN、IDS 和 Frag 指标方面优于常见的方法。 
在 MOT16 基准测试中,与第二好的在线多目标跟踪方法相比,所提出的方法在 MOTA 上有 0.6$\%$ 的性能提升,在 MOTP 上有 0.9$\%$ 的提升。 
特别是在 MOT17 数据集上,与第二好的在线多目标跟踪方法相比,所提出的方法在 MOTA 中获得了 5.7$\%$ 的性能提升,在 MOTP 中获得了 0.6$\%$ 的性能提升。 
此外,所提出的跟踪器在 MOT16 数据集上的所有在线跟踪器中实现了最佳 FP 和 Frag 值。
所提出的跟踪器在在线多目标跟踪方法中实现了 MOTA 和 MOTP 的最佳性能,证明了所提出的方法在保持身份和跟踪方面的优势。 


\vspace{1.0em}
\renewcommand\arraystretch{1.5}
\begin{table}[htbp]\wuhao
	\centering
	\caption{在 MOT16 数据集上跟踪结果的比较}
	\vspace{0.3em}
	\begin{tabular}{c|ccccccc}
%		{p{2.5cm}<{\centering} p{1.0cm}<{\centering} p{1.0cm}<{\centering} p{1.0cm}<{\centering}p{1.0cm}<{\centering}p{1.0cm}<{\centering}p{1.0cm}<{\centering}p{1.0cm}<{\centering}}
%		\toprule[1.5pt]
%		\hline
		\hline
		方法& MOTA$\uparrow$& MOTP$\uparrow$& IDR$\uparrow$& FP$ \downarrow $& FN$ \downarrow $& IDS$ \downarrow $& Frag$ \downarrow $ \\
		\hline
%		\midrule[1.0pt]
		VOBT\cite{b50}& 38.4& 75.4& 28.7& 11,517& 99,463& 1,321& 2,140 \\
		EAMTT\cite{b51}& 38.8& 75.1& 31.5& 8,114& 102,452& 965& 1,657 \\
		oICF\cite{b52}& 43.2& 74.3& {\textbf{37.2}}& 6,651& 96,515& {\textbf{381}}& 1,404 \\
		DDAL\cite{b24}& 43.9& 74.7& 34.1& 6,450& {\textbf{95,175}}& 676& 1,795 \\
		NAAN & {\textbf{44.5}}& {\textbf{75.6}}& 32.8& {\textbf{5,346}}& 98,740& 698& {\textbf{1,252}} \\
%		\hline
		\hline
%		\bottomrule[1.5pt]		
	\end{tabular}
	\label{tab:nlaa_tracking_performance}
\end{table}


\vspace{1.0em}
\renewcommand\arraystretch{1.5}
\begin{table}[htbp]\wuhao
	\centering
	\caption{在 MOT17 数据集上跟踪结果的比较}
	\vspace{0.3em}
	\begin{tabular} {c|ccccccc}
%		{p{2.5cm}<{\centering} p{1.0cm}<{\centering} p{1.0cm}<{\centering} p{1.0cm}<{\centering}p{1.0cm}<{\centering}p{1.0cm}<{\centering}p{1.0cm}<{\centering}p{1.0cm}<{\centering}}
%		\toprule[1.5pt]
%		\hline
		\hline
		方法& MOTA$ \uparrow $& MOTP$ \uparrow $& IDR$ \uparrow $& FP$ \uparrow $& FN$ \downarrow $& IDS$ \downarrow $& Frag$ \downarrow $ \\
		\hline
		GM\_PHD\cite{b50}& 36.4& 76.2& 24.7& 23,723& 330,767& 4,607& 11,317 \\
		GMPHD\_KCF\cite{b51}& 39.6& 74.5& 29.1& 50,903& 284,228& 5,811& 7,414 \\
		GNN\cite{b53}& 45.5& 76.3& \textbf{41.8}& 25,685& 277,663& 4,091& 5,579 \\
		E2EM\cite{b52}& 47.5& 76.5& 37.9& 20,655& 272,187& 3,632 & 12,712 \\
		NAAN & {\textbf{53.2}}& {\textbf{77.1}}& 39.6& {\textbf{15,093}}& \textbf{245,802}& {\textbf{3,012}}& {\textbf{932}} \\
%		\bottomrule[1.5pt]
%		\hline
		\hline		
	\end{tabular}
	\label{tab:nlaa_performance_MOT17}
\end{table}


\subsection{参数分析}
如图~\ref{fig:nlaa_parameter} 所示,本节利用几个实验来展示不同阈值的设置对跟踪性能的影响,包括轨迹初始化阈值、轨迹终止阈值、跟踪分数、外观相似度分数、重叠率和距离。 
$\tau_s$ 和 $\tau_a$ 分别是跟踪分数和外观相似度的阈值。
$\tau_t$ 和 $\tau_i$ 是轨迹终止和初始化的正则化阈值因子。
$\tau_o$ 和 $\tau_d$ 分别是重叠率和距离的正则化阈值。 
$\tau_t$ 和 $\tau_d$ 的值从 $ [0.5, 3] $ 映射到 $ [0, 1] $。
所有超参数都在这些不同的参数设置上进行分析。 
正则化的 MOTA 随 $\tau_s$ 和 $\tau_a$ 的设置变化很大。 
因此,为了避免手动设置超参数的缺点,并减少优化超参数的工作量,基于已训练好的 NAAN ,选择 $ \tau _s $ 和 $ \tau _a $ 两个超参数进行超参数的网格搜索,以确定当前环境下合适的超参数配置。
图~\ref{fig:nlaa_grid_search} 中考虑了不同超参数值的影响,
超参数 $\tau_s$ 和 $\tau_a$ 是通过网格搜索选择的,在 MOT16 训练数据集上对于不同的 $\tau_s$ 和 $\tau_a$ 设置取得不同的 MOTA,并将其正则化到 $\left[0,1\right]$ 范围之内。 
%当 $\tau_s=0.2$ 和 $\tau_a=0.8$ 时,得到最大化的正则化 MOTA。
当 $\tau_s=0.2$ 和 $\tau_a=0.8$ 时,得到最大化正则化 MOTA,
因此,按照以上方案设置超参数。

\begin{figure*}[ht]
	\centering
	\includegraphics[width=0.8\textwidth]{figures/C3Fig/parameter.pdf}
	\caption{每个超参数对实验性能的影响}
	\label{fig:nlaa_parameter}
\end{figure*}


\subsection{讨论}
本章提出了非局部注意力关联方案联合处理轨迹级运动关联和在线多目标跟踪的相关问题。 
联合任务是通过关联历史轨迹和当前帧候选检测结果来实现的。 
在图~\ref{fig:nlaa_tracking_result} 中展示了不同环境下的跟踪结果示例。
第一排为在繁忙路口的公交车上拍摄的视频片段 MOT16-13,第二排为在夜间步行街且为高架视点拍摄的视频片段 MOT16-04,最后一排是从低角度拍摄的步行街场景的视频示例视频序列片段 MOT16-09,三个都展示了当跟踪过程在倒数第二帧中发生漂移时,所提出的方法能够很好的关联行人,解决单目标跟踪器应用到多目标跟踪环境中的漂移问题。
通常,当行人快速移动或受到其他行人的影响时,单目标跟踪器可能会发生漂移。 
使用注意力关联方法能及时纠正漂移问题。 
此外,单目标跟踪器可以有效克服遮挡的缺陷。
\begin{figure*}[ht]
	\centering
	\includegraphics[width=1.0\textwidth]{figures/C3Fig/tracking_result.pdf}
	\caption{不同环境下的跟踪结果示例}
	\label{fig:nlaa_tracking_result}
\end{figure*}

当前的实验设置有两个限制,
首先,跟踪方案的最佳性能和几个超参数的选择有关。
如图~\ref{fig:nlaa_grid_search} 所示,目前一些超参数是通过网格搜索进行设置的,这在一定程度上能搜索到一个合适但不是最优的超参数配置。
如果提供足够数量的训练数据,这些超参数可以通过上述网格搜索方法进行学习或优化,减少手动搜索参数的工作量并提高跟踪模型的泛化能力。 
其次,考虑到多目标跟踪这个巨大且困难的任务无法保证该模型是最优解决方案,在目前的工作中仅找到多目标跟踪问题的可行解。 
对上述两个问题的贡献可能会进一步改进跟踪性能。


\section{本章小结}
在本章的研究工作中,将单目标跟踪和注意关联算法的优点整合到一个统一的在线多目标跟踪框架中。 
对于轨迹段的特征提取,使用非局部注意力机制来提取轨迹段的时空特征。
对于注意力关联,利用非局部注意力模块的特征来关联候选检测和历史轨迹以抑制噪声检测和遮挡。 
在公开的多目标跟踪基准上进行了一系列消去研究和性能测试,证明了所提出的非局部注意力关联方法能较好的应对动态开放场景下多目标跟踪所存在的挑战。 
后面可以考虑将最新的且精度高的单目标跟踪器纳入到所提出的跟踪架构,并解决历史轨迹和当前检测结果特征不平衡的问题,以进一步提高多目标跟踪的精度并为实际应用做出贡献。





