\special{dvipdfmx:config z 0} %取消PDF压缩,加快速度,最终版本生成的时候最好把这句话注释掉

% 调试哪个章节就加入哪个章节,其他的章节自动屏蔽
%\includeonly{chap1}

\documentclass[UTF8,10pt,aspectratio=43,mathserif,table]{beamer}
%设置为 Beamer 文档类型,设置字体为 10pt,长宽比为16:9,数学字体为 serif 风格
\batchmode

\usepackage{graphicx}
\usepackage{animate}
\usepackage{hyperref}

%导入一些用到的宏包
\usepackage{amsmath,bm,amsfonts,amssymb,enumerate,epsfig,bbm,calc,color,ifthen,capt-of,multimedia,hyperref}
\usepackage{xeCJK} %导入中文包
\setCJKmainfont{SimHei} %字体采用黑体  Microsoft YaHei

\usetheme{Berlin} % 主题
\usecolortheme{sustech} % 主题颜色

\usepackage[ruled,linesnumbered]{algorithm2e}

\usepackage{fancybox}
\usepackage{xcolor}
\usepackage{times}
\usepackage{listings}

\usepackage{booktabs}
\usepackage{colortbl}

% 插入中文日期
\usepackage{datetime} %日期
\renewcommand{\today}{\number\year 年 \number\month 月 \number\day 日}


\newcommand{\Console}{Console}
\lstset{ %
	backgroundcolor=\color{white},   % 选择背景颜色
	basicstyle=\footnotesize\rmfamily,     % size of fonts used for the code
	columns=fullflexible,
	breaklines=true,                 % automatic line breaking only at whitespace
	captionpos=b,                    % 设置说明文字在底部
	tabsize=4,
	commentstyle=\color{mygreen},    % comment style
	escapeinside={\%*}{*)},          % if you want to add LaTeX within your code
	keywordstyle=\color{blue},       % keyword style
	stringstyle=\color{mymauve}\ttfamily,     % string literal style
	numbers=left, 
	%	frame=single,
	rulesepcolor=\color{red!20!green!20!blue!20},
	% identifierstyle=\color{red},
	language=c
}

\setsansfont{Microsoft YaHei}
\setmainfont{Microsoft YaHei}

\definecolor{mygreen}{rgb}{0,0.6,0}
\definecolor{mymauve}{rgb}{0.58,0,0.82}
\definecolor{mygray}{gray}{.9}
\definecolor{mypink}{rgb}{.99,.91,.95}
\definecolor{mycyan}{cmyk}{.3,0,0,0}


% 参考文献
\usepackage[backend=bibtex,sorting=none]{biblatex}
\addbibresource{../reference.bib} %BibTeX数据文件及位置
% 脚注(https://blog.csdn.net/nima1994/article/details/80744545):
% \footfullcite{bib_item} %文献item
% 将参考文献图标改成标准格式,\begin{document} 之前添加如下
\setbeamertemplate{bibliography item}[text]


%题目,作者,学校,日期
\title{复杂场景下多目标跟踪方法研究}
%\title{多目标跟踪场景下的时空注意力方法研究}
%\title{复杂场景下多目标跟踪时空注意力方法研究}
%\subtitle{\fontsize{9pt}{14pt}\textbf{Research on Multi-Object Tracking Method in Complex Scene}}
%\newline
\author{\newline \newline \newline \newline \newline 答辩人: 王海东  \newline  指导老师: 李智勇教授}
\institute{\fontsize{8pt}{14pt}湖南大学信息科学与工程学院}
%\data{2022 年 6 月 25 日}
\date{\today}

%学校Logo
%\pgfdeclareimage[height=0.5cm]{sustech-logo}{sustech-logo.pdf}
%\logo{\pgfuseimage{sustech-logo}\hspace*{0.3cm}}

\AtBeginSection[]
{
	\begin{frame}<beamer>
	\frametitle{\textbf{目录}}
	\tableofcontents[currentsection]
\end{frame}
}
\beamerdefaultoverlayspecification{<+->}


% -----------------------------------------------------------------------------
\begin{document}
% -----------------------------------------------------------------------------

\frame{\titlepage}

\section[目录]{}   %目录
\begin{frame}{目录}
\tableofcontents
\end{frame}

% -----------------------------------------------------------------------------


% !Mode:: "TeX:UTF-8"

\chapter{绪论}

\section{研究背景及意义}
% nai
近年来,人工智能技术发展突飞猛进,并以不可阻挡之势影响着社会和人们的生活。
许多国家和地区都将人工智能的发展作为提高国家实力、增强核心竞争力和建设现代化强国的核心所在。
例如,2020 年 3 月,中共中央政治局常委会提出将人工智能作为新型基础设施建设七大板块中的重要一项,以便更好地推动中国经济的转型和升级。
2021 年 3 月,美国国家人工智能安全委员会发布了《人工智能国家安全委员会最终报告》,提出推动人工智能以及相关技术的研究和应用,以解决国家安全和国防需求~\cite{schmidt2021national}。
而计算机视觉作为人工智能最重要的领域之一,正在成为科学研究和产业落地的重要方向。

% 博士论文
动态开放场景下的视频多目标跟踪方法是计算机视觉方向基础且重要的研究领域,同时也是全球大学、研究所和公司所亟待解决的核心问题。
多目标跟踪的任务是确定视频中每一帧中所有目标的位置和身份,并得到每个目标的运动轨迹。
解决该问题不仅可以得到目标的时空位置和身份信息,同时也为情景分析等更高层的视觉任务提供支撑。
比如跟踪轨迹的变化包含了所跟踪目标的速度和方向等信息,跟踪开始和结束时刻是目标进入和离开相机视野的时间,跟踪的结果还隐含了所跟踪对象的动作信息,
而且分析多个被跟踪目标的跟踪结果可以得到许多更加综合的信息,比如累计特定的时间段中某个相机视野中出现行人的数目、各个对象之间的社交距离、场景拥挤状况等信息,
这些是更高层分析和理解的基础。
% 可以为动作的检测和识别、行为分析和预测等更高级计算机视觉任务提供重要的信息。
因此,动态开放场景下的多目标跟踪方法在很多领域都有重要的巨大的实用价值,主要包括以下几个的实际场景:

(1)  智能监控。
近年来发生了许多突发性公共安全事件,如 2014 年 12 月 31 日上海外滩的踩踏事件;
2015 年 9 月 24日,沙特阿拉伯麦加发生朝圣者严重踩踏事故;
2016 年 5 月 11 日美国马萨诸塞州发生持刀伤人事件;
2021 年 4 月 30 日,以色列发生踩踏事件;
2021 年 11 月 21 日,美国威斯康星州发生汽车撞人事件;
2022 年 1 月 1日,印控克什米尔地区查谟附近寺庙发生踩踏事件等。
构建安全的社会环境、实时查看环境动态和预防公共安全事件已成为各国政府必须履行的职责,而智能视频监控则成为实现这个目的一种有效安全的方法。
根据全球视频监控市场显示,北美是全球最大的市场,其中美国单独占全球市场份额的比例约为 $ 26\%$,中东、欧洲和非洲市场规模相对较小。
2015 年,国家发改委等部委发布《关于加强公共安全视频监控建设联网应用工作的若干意见》要求,重点公共区域监控摄像头的完好率要达到 $ 98\% $,普通公共区域摄像头完好率要达到 $ 95\% $,同时公安部通过互联平台对各个省的安全城市状况进行监督,近年来我国各级政府对视频监控的重视程度也越来越高,视频监控市场得到了飞速发展。
但是,随着视频监控部署扩大,系统的维护成本也在逐年增加,对面海量的视频数据,仅仅依靠人力很难做到实时处理,后期视频的调取分析也面临着巨大挑战,而且人工处理时常会出现漏检的问题。
无论是从人力成本的角度还是问题的复杂性的角度,自动化和智能化的视频监控分析技术必然是未来的主流。
所以由此而来的是在视频监控系统中增加自动化和智能化视频处理,借助于计算机强大的运算和存储能力,利用各种图像处理和计算机视觉算法,在海量的监控视频数据中智能识别出人们最想要的目标和相关信息,极大的解决了人类处理视频数据的局限性,高效实时地根据得到的监控数据判断公共场合中出现的突发状况,从而能够快速的定位并赶到现场,极大地减少事故所带来的损失和危害,甚至能够通过综合各种信息,一定程度上做到事故的预警,达到提前预警、实时监控、事后取证等目的。
实现该系统需要一系列图像处理和计算机视觉算法,包括检测、分割、跟踪、情景分析等。
而其中跟踪算法,特别是多目标跟踪算法是智能视频监控系统的核心所在。
% 
例如,根据行人的跟踪轨迹可以判断出朝向车辆运动方向走等危险的行为动作,为智能驾驶提供警告信息,从而避免事故的发生;
根据车辆的运动分析,可以实时检测出危险的运动方向,比如冲向密集的人群或者障碍物等,从而达到危险的实时监控。
在集会、商场等人流量特别大的区域,利用智能视频监控可以统计并分析出是否出现人流量过大或者预测可能发生的踩踏等情况,做到事前预警、实时处理危险情况。
同时对已有的海量视频数据和多目标跟踪数据的挖掘,可以找出人流和车流密集的地方,通过调节红绿灯来实现通行优先级的调整或者交警人为干预的方式,提前疏导和控制,为城市的安全稳定运行提供支撑和保证。
还可以通过分析多目标跟踪的结果,研究群体的出行规律、社交距离等,为新冠疫情的防控提供可靠且有效的信息来源。
%另外,多目标跟踪算法不一定非要实时进行。
%通过离线多目标跟踪算法对大量视频中的目标进行跟踪,得到大量的轨迹,分析这些轨迹有助于发现人们经常聚集在一起的区域,这可以为规划建筑物的逃生路线提供帮助。
%轨迹分析也可以用来了解人们在大型商场或购物街内的移动方式。除了针对公共安全场合,智能视频监控技术也被广泛应用于道路交通管理。
%例如通过跟踪道路上行驶的车辆可以调整交通信号灯,从而减缓交通堵塞现象。

(2) 智能驾驶。
智能驾驶一般是表示使用车上的智能驾驶系统代替人部分或者全部的驾驶功能~\cite{bergmann2019tracking}。
在减少驾驶员介入的情况下,智能车辆能自主实现启动、巡航和停车等功能。
目前,百度阿波罗、特斯拉和 Mobileye 等国内外著名的科技公司,都已实现了智能驾驶系统的测试和验证,并取得了一定的商业价值。
多目标跟踪技术作为智能驾驶落地和产业化最重要也是最基本的功能之一,成为智能驾驶大规模商业化必须考虑的因素。
%智能驾驶汽车要想实现真正的普及,目标跟踪是必须考虑的最基本也是最重要的功能之一。
由于减少了驾驶员的操作和干预,智能汽车需要根据摄像头获取的视频数据和多目标跟踪算法来感知环境信息,为安全自主的驾驶行为决策提供支撑。
比如智能汽车可以利用多目标跟踪算法来判断周围车辆或行人的距离、速度和加速度等信息,并在此基础上进行行为动作的预测来规避危险。
%为了确保车辆遵守交通规则,智能驾驶汽车需要利用目标跟踪算法获取相应的信息来对行车场景做出正确的感知。

(3)  智能机器人。
目前机器人的研究和应用己经从实验室走向生活,从自动化逐步变为智能化。
虽然智能机器人当前还不能拥有和人类一样的自主意识,但却在特定功能方面已经拥有和人类相媲美的能力甚至突破人类的某些局限。
随着智能机器人算法和技术的逐步完善和成熟,一些类似于扫地机器人的产品已经开始进入人们的日常生活当中。
在完全不变和静止的环境中,利用基本的计算机视觉目标检测算法就可以进行障碍物的识别和路径规划。
但是当机器人在动态开放的环境当中进行导航和规划时,需要跟踪运动的物体并预测是否会与机器人本身的运动产生冲突或碰撞。
另外,随着人工智能的发展,特别是计算机视觉算法的进步,传统人利用键盘和鼠标等机械的方式实现与机器的交互出现了许多更加智能化人性化的交互方式,比如动作、手势等。
当在动态开放场景存在多个目标时,多目标跟踪作为视觉人机交互信息和目标与目标之间交汇信息抽取的基础。
所以动态开放环境下的多目标跟踪算法对于智能机器人的实现非常基础和重要。

%(4)  体育比赛分析。
%目视频多目标跟踪还可以应用到体育视频分析中,例如通过跟踪足球或篮球比赛中的运动员可以得到运动员们的运动轨迹信息,跟据这些轨迹信息可以客观地衡量每个球员的身体表现,还可以分析比赛中的失误,为教练训练运动员、制定更好的比赛策略等提供科学依据。

除了介绍的三个方向之外,多目标跟踪算法在国防军事、体育赛事分析等许多方向同样得到了及其广泛的使用。
并且多目标跟踪方法在研究领域也有重要的意义。
% bergmann2019tracking,chen2019aggregate,chu2019famnet,
许多人工智能顶级期刊和顶级会议每年都有大量的多目标跟踪有关的学术研究成果~\cite{sun2021deep,xu2020how,young-chul2019online,zhang2019robust}。
同时,多目标跟踪挑战赛(MOT Challenge)每年会吸引世界各地的研究所、公司、和高校的研究者参加比赛,将所提出的多目标跟踪算法和其他人的方法进行同台较量,以促进优秀思想和算法的实现和交流。
所以,多目标跟踪无论是在工业界还是学术界都有着重大而实际的意义。


\section{多目标跟踪的研究现状}
由于多目标跟踪任务有着极其重要的产业价值和研究价值,
许多国内外的公司和大学都将多目标跟踪任务作为研究重点。
各种高性能跟踪方法被陆续提出,促进了多目标跟踪技术的进步~\cite{xu2020how,chen2019aggregate}。
% ref: https://zhuanlan.zhihu.com/p/97449724?from_voters_page=true
一般完整的多目标跟踪算法包括以下几个流程:目标检测、表征获取、相似度评估以及目标关联。
由于目标检测通常作为一个单独的计算机视觉任务进行研究和处理,视频多目标跟踪主要关注后面三个步骤,
根据方法的类别,多目标跟踪方法可以被分为特征提取算法、数据关联算法以及深度学习算法。

% ① 检测 ②特征提取、运动预测 ③相似度计算 ④数据关联。
% 关联代价 -> 特征提取
\subsection{特征提取算法}
目标的特征提取是多目标跟踪算法的基础。
如果能够提取到十分鲁棒的目标特征,即使利用最基本的关联方法也会获得较好的效果。
一般的目标特征提取包括外观特征、交互特征、运动特征等。
%
% 外观特征
%对于数据关联多目标跟踪算法的特征提取来说,外观是一个重要的特征。
目标的外观特征对于多目标跟踪方法特别重要。
从具体实现上看,目标的外观特征建模包含外观特征抽取以及相似性计算。
外观特征使用一些表征来代表被跟踪目标的外观特性,而相似性计算是衡量不同目标之间的相似度。
%,而经过简单的推算可以将相似度变为关联匹配代价。

外观特征使用各种不同的特征来表征所跟踪的目标,这些特征分为局部特征和全局特征两大类。
% 局部特征
假设把图像中的像素视为最理想的局部区域,那么就可把光流特征看成是一种局部的表征。
一些多目标跟踪方法先将检测结果在时间维度上串联成轨迹段,然后使用光流特征进行目标关联~\cite{using2009,zhao2012tracking}。
因为光流特征和目标的运动相关,所以一般使用光流对运动进行编码~\cite{optic_flow,choi2015near-online}。
考虑到在被跟踪目标非常拥挤的视频中视觉表征效果不佳,光流特征被用于提取跟踪目标的运动模式~\cite{ali2008floor}。
KLT 算法~\cite{lucas1981an} 将视频帧的局部匹配从滑动窗口转换为偏移量的求解,该算法已成功在单目标跟踪~\cite{shi1994good} 和多目标跟踪任务中得到使用。
当抽取到适用于跟踪的表征时,便可使用它来生成短的跟踪轨迹段~\cite{using2009,zhao2012tracking}。
% 非局部特征 -> 全局
与局部特征相比,全局特征是从更广的区域进行特征的提取。
% 原始全局特征
全局特征一般分为原始全局特征、梯度全局特征以及协方差全局特征。
它们之间的差别在于计算视频帧外观特征时像素值之间做差分的次数。
原始全局特征表示不对视频帧中的相邻像素值进行差分操作,梯度全局特征表示进行一次视频帧中像素值的差分。
其中的原始全局特征是多目标跟踪中经常使用的一种外观特征,包括原始像素值模板~\cite{yamaguchi2011who}、颜色直方图~\cite{mitzel2010multi-person} 等。
梯度全局特征是根据梯度来进行特征的表征,一般利用的是方向梯度直方图。
%(Histogram of Oriented Gradient,HOG)。
协方差全局特征使用图像像素的全局协方差矩阵~\cite{tuzel2006region},该特征在研究中得到了广泛的使用~\cite{kuo2010multi-target,hu2012single,henriques2011globally}。
通常梯度全局特征是经典的相似性衡量手段,但是它忽视了视频帧中全局的位置状况。
虽然局部特征是一种可行的策略,可是这种特征对于旋转和目标遮挡等情况处理不佳。
而方向梯度直方图可以很好地表征被跟踪目标的形状特征,且对于光照改变等挑战拥有较好的鲁棒性,但是它却不能很好地应对形变和遮挡的情况。
因为协方差全局特征以更多的计算为代价来考虑更加广泛的信息,所以它相比于前两种特征更加鲁棒。


运动模型建模了目标的动作,用来预测被跟踪目标在后续视频图像中的空间位置信息,以达到缩小搜索空间的目的~\cite{motion_doc}。
一般可以认为被跟踪目标不会有大的加速或减速,而是进行平稳的运动。
经典的建模方法包括非线性以及线性运动模型。
% ,yu2007multiple
最直观的建模方法就是线性运动模型~\cite{breitenstein2009robust,shafique2008a},该方法假定所跟踪的目标运动都是匀速的~\cite{breitenstein2009robust},
所以有位置平滑和速度平滑等方法来建模目标的运动~\cite{kuo2011how}。
% xing2009multi-object
位置平滑是指控制观测坐标与预测坐标的差值~\cite{yang2012an};
速度平滑是指利用被跟踪目标的速度在连续视频帧中是平滑变化的假定来进行建模~\cite{qin2012improving};
线性运动模型一般用来表示被跟踪目标的运动,但大部分被跟踪目标实际的运动并不是线性的。
%有些情况下线性运动模型无法处理。
因此在此基础上所提出的非线性运动方法能够处理被跟踪目标随意运动的情况,并且可对被跟踪目标进行更加精细的运动建模~\cite{yang2012multi-target}。



相似性计算旨在衡量两个视觉特征的相似性。
相似性计算可以是基于一种特征,也可以依赖多种特征。
前者仅仅使用一种特征来表示外观,或者利用目标之间的空间距离作为相似度进行关联。
%又或者计算亲和力。
% ali2008floor
原始全局特征中的像素值模板外观特征一般利用归一化互相关来衡量两者之间的相似度~\cite{yamaguchi2011who,wu2015global}。
% zhang2008global,using2009,choi2010multiple
原始全局特征中的颜色直方图特征通常先利用巴塔恰里雅距离来度量两个颜色直方图之间的可分离性,接着把可分离性转为相似性~\cite{kratz2010tracking,qin2012improving}。
把类似距离的不相似性转为相似性也可用于协方差矩阵的视频帧特征~\cite{henriques2011globally}。
%除了以上经典的方法之外,
%一般利用词袋法处理点的外观特征。
而利用多个特征融合不同种类信息可以让目标的视觉特征对动态开放的场景拥有更强的适应性。
%
为了达到上述目的,可以使用连接策略来融合不同方向的特征,进而获得更加鲁棒的视觉特征。
%但如何融合来自不同方向的多个线索同样是一个具有挑战的任务。
%常见的基于多线索的外观模型的融合方法包括串联、集成、连接、合并和相乘。
%具体而言,串联表示利用不同种类的图像特征的串联以实现更加精细化的外观建模~\cite{rodriguez2009tracking}。
%集成一般是利用集成算法按照顺序从特征候选区中选择一部分特征,
%% ,li2009learning
%比如从各阶特征中利用一系列集成学习的方法~\cite{kuo2010multi-target,block_mot} 来选取最鲁棒的的表征。
%连接是指连接各种不同信息源的表征,比如连接颜色直方图、方向梯度直方图等特征来抽取目标的视觉特征~\cite{brendel2011multiobject,mot_doc};
%% mitzel2010multi-person,
%合并方法是从不同特征中获得相似性度量,并把得到的值进行加权合并~\cite{liu2012automatic};
%相乘与合并策略相似,不同的相似度根据相乘产生一个更加综合的相似度值~\cite{yang2009detection,song2010a,berclaz2006robust}。
%需要注意的是通常在应用此策略时要进行独立性假设;



\subsection{数据关联算法}
视频多目标跟踪方法通常被分为在线方法和离线方法两种。
在线多目标跟踪方法只能利用当前帧和历史帧的数据,不能利用当前帧以后的任何数据,而离线方法可以利用整个视频的数据。
但是相比于离线方法,在线方法更利于在实际场景中进行应用。
在线目标关联方法不能使用将来视频的任何数据,而是应该输出当前时刻图像的候选检测和目标历史跟踪轨迹段的关联。

% 在线
常见的在线目标关联方法包括全局最近邻数据关联、联合概率数据关联等。
% 全局最近邻
全局最近邻方法是一种经典的在线数据关联方法,该方法建立当前时刻视频帧的候选检测与历史轨迹之间的关联关系,然后利用求解二分图匹配问题,在所有候选检测结果中匹配最大概率关联。
该方法最关键的步骤是将多目标跟踪问题建模成二分图匹配问题~\cite{bertsekas1994linear,veenman2001resolving},以便能够有效地解决它。
% JVC 算法~\cite{okuma2004a,bertsekas1992a}~和
该研究方向已存在不少经典方法用于求解这个最优化模型,比如拍卖算法~\cite{bertsekas1988the}、匈牙利算法~\cite{unkres1957algorithms}~等。
全局最近邻算法因其算法流程的简洁以及容易改进的特点得到了广泛的应用。
并且二分图匹配任务在学术界己被深入研究,因此该方法能够很方便地在实际工程中进行测试和部署。
由于全局最近邻方法对关联步骤所做的分配是确定性的,并未将关联正确或者错误的概率加以考虑,所以必须提取一个鲁棒的特征。
当提取的特征不佳时,全局最近邻方法通常只能在跟踪目标图像特征差异很大或者空间距离非常大时才能够很好地工作,同时检测器必须获得准确的结果以保证该算法达到良好的性能。
% 联合概率
为了解决全局最近邻方法只能根据最好的关联候选进行历史轨迹的更新且并未将概率或者是疑似的目标关联加以考虑的问题,
联合概率目标关联方法~\cite{bar-shalom1987tracking} 被提出来建模这种不确定性,进而将所有疑似的候选检测都进行建模。
% 的加权组合
对于联合概率目标关联,历史轨迹段参考当前所有可能的检测结果进行更新。

% 离线
%将在线多目标跟踪目标关联方法用于视频数据的在线处理,每当新的视频帧到达时就要执行关联算法来解决关联匹配问题。
与在线方法每次处理当前时刻的图像并进行目标关联不同,离线的多目标跟踪数据关联方法可以把将来的视频图像数据加以利用以进行多目标跟踪。
当考虑某一帧和其他所有帧的关联关系时,为了缓解关联关系随着跟踪过程呈指数级增长的问题,批处理窗口方法被用来进行目标的关联。
离线的多目标跟踪数据关联方法的好处是既可以利用关联度很高的数据来处理关联不确定的问题,又能保证算法的时间复杂度不是特别高。
常见的离线多目标跟踪目标关联方法有多假设多目标跟踪、子图分解、网络流多目标跟踪目标关联和层次目标关联等。

多假设跟踪~\cite{reid1989an,multi_hypo} 采用推迟处理的策略,按照时间线对存在的关联进行假定,并利用后面得到的检测来处理当前时刻的不确定性关联。
与在每个时刻确定概率最高的检测和联合概率方法不同,为了更准确地预测目标关联可能性,多假设跟踪会传递当前时刻的假设到后面视频帧中,
并且提供了通用的步骤来处理每个目标的全部过程:目标轨迹的初始化、跟踪以及消亡。

%
% zhang2008global
网络流数据关联方法~\cite{wang2016joint,schulter2017deep,net_flow} 通过使用网络来表征跟踪目标的状态改变,将多目标跟踪中的目标关联任务建模成最小化网络流任务。
这个算法将视频时间维上所有的检测都考虑进来,所以直观上它比在线多目标跟踪算法中的全局最近邻标方法会有更优的效果。
%由于网络流数据关联法只是所有检测结果已知的情况下一系列假设轨迹集合的最大后验概率,由此来求解最优解,考虑到每个被跟踪目标只会有一个轨迹与它对应,所以可以裁剪一些假设,所以它并不是完全的贝叶斯目标关联算法。
与一般的任务不一样的是网络流算法可以在多项式时间复杂度范围内得到全局最优解。
可以将网络流算法应用于多目标跟踪任务当中~\cite{zhang2008global},并提出解决网络流目标关联方法~\cite{efficient1990}。
之后考虑利用连续最短路径实现在多项式时间复杂度范围内进行网络流任务的求解~\cite{2011globally-optimal}。
还可以在最小代价网络流中加入配对的代价~\cite{chari2015on}。
同时可以在网络流方法中加入跟踪目标的身份,网络中的节点表示把被跟踪目标身份信息赋给它的可能性,并求得满意的解~\cite{2015target}。
%在放松约束条件之后的每次迭代过程屮,被求解的问题被简化为分别为每个被跟踪目标确定最佳轨迹,并利用动态规划方法可在线性时间内找到满足条件的最优解。
还有一种端到端的方法抽取目标关联任务中特征的算法~\cite{schulter2017deep}。

子图分解法也是进行目标关联的策略,在该方法中目标关联问题被建模为无向图求解任务。
无向图的每个节点表示检测器输出,图中的边表示两个节点是同一身份的可能性。
% 其中的轨迹路径不会有合并或者分叉,
一般的网络流算法把目标关联表示成求解不相交的轨迹,因为目标在被跟踪的过程中不能分为两个目标。
这个模型的优点是很简洁且容易理解,缺点是没有考虑检测结果的不完美有时会为一个目标产生一系列类似结果的情况。
在不相交路径的网络流方法中,当面对许多可能的轨迹时,无法求解最优的轨迹。
可以利用最小成本算法,将多目标跟踪的目标关联看成是子图分解任务~\cite{tang2015subgraph}。
面对视频中跟踪目标数目的不确定,这种抽象的方法能很好地适应并可以利用最优化算法进行求解,使它在时空上合并同一跟踪目标的一系列检测,以进行更好的多目标跟踪。
基于前面工作的继续设计以同时考虑再识别特征和局部块的策略。
再识别特征为相隔较远的图像帧检测结果的关联贡献了有价值的特征,而局部块为相邻帧的检测贡献了有效的衡量方法。
在动态开放场景中,虽然目标之间相似性计算很精确,但由于存在外观上相似的目标,它们身份有可能根本不同,这导致在整个方法中加入视频帧跨度大的再识别特征依然是很困难的工作。
所以在大部分局部特征的基础上合并时间跨度大的特征是很有意义的。
根据这种经验,该团队算法继续提出了优化方案,将时间跨度大的目标再识别特征加入到多目标跟踪的数据关联当中。
%该工作在原始图里加入了规则边以及提升边。
%规则边说明了问题可行的解决方法,提升边则是将时间跨度大的目标重识别特征加入到模型当中,只有当规则边形成有效路径时,才会考虑时间间隔较远的外观相似目标。


层次关联算法先使用少量的视频帧产生小范围内可信的的跟踪轨迹段,再根据生成的跟踪轨迹段产生时间范围更大的跟踪轨迹段~\cite{mot_doc}。
该算法可以有效缓解离线批处理中所用时间会根据跟踪轨迹段变长而指数级增长的问题。
%,使用层次关联算法可以有效缓解这个问题。
一般来说,利用跟踪轨迹段的进行目标关联可以极大缩小任务规模,且能够较好地应对跟踪过程中出现的遮挡问题。
但是该算法的先决条件是先产生跟踪轨迹段,而跟踪轨迹段产生的细小误差都会传递到最后的多目标跟踪。
早些时候使用跟踪挂起关联策略~\cite{kaucic2005a},对于目标被其他物体部分遮挡的情况,挂起多目标跟踪过程,挂起的跟踪轨迹使用外观特征来进行关联。
随后使用分阶段的策略~\cite{wu2007detection} 和其他方法,利用二分图匹配中的局部特征来辨别所跟踪的目标。
%面对长时间遮挡的情况,通过对集覆盖问题的对数近似解得到局部轨迹段的关联结果。
使用层级架构来实现的目标匹配~\cite{huang2008robust} 关联邻近视频帧的重复的检测来获得跟踪短轨迹段。
随后根据视频帧的图像特征以及运动趋势,把生成的短跟踪轨迹段连接成更长的跟踪轨迹。
%然后进行更高层次的关联,使用交替优化方法生成最后的跟踪轨迹。
还可以使用集成学习等一系列机器学习方法~\cite{li2009learning} 以监督学习的方式而不是启发式定义的方式自动学习短轨迹段之间的相似性。
另外在线多目标跟踪中的条件随机场模型~\cite{condition_field,asrcf2} 着重应对短跟踪轨迹段关联时的不确定性问题。


% TODO
\subsection{深度学习算法}
% wang2016joint,b19,b23,b24,dual_matching,b3,kim2021discriminative
近年来,深度学习方法在视频多目标跟踪领域得到了广泛的使用~\cite{ke2021prototypical,baisa2021robust,hsu2022multi}。
利用深度学习算法进行被跟踪目标的特征提取成为一个重要策略,也就是抽取更鲁棒的特征来进行相似性度量。
在多目标跟踪中使用深度学习算法的应用一般包括深度表征使用、使用深度网络进行相似性计算、联合检测和跟踪的方法。

一种提高算法效果直接而有效的策略是依靠深度模型强大的学习能力,直接用目标识别任务中抽取到的外观表征来表示所跟踪目标的外观~\cite{kim2015multiple},也可以利用深度模型抽取到的类似于光流的运动表征来表示所跟踪对象的运动~\cite{b18}。

但是直接利用在其它计算机视觉问题中提取到的深度表征并未考虑视频多目标跟踪本身的性质。
一个改进策略是参考行人再识别的方法,直接学习被跟踪目标之间的外观相似性。
可以利用孪生网络模型对两个检测是否为相同身份进行预测,这个模型把被跟踪目标的外观特征和运动特征一起加以考虑~\cite{RN454,twin_net}。
还有研究利用孪生网络来对两个被跟踪对象是不是相同身份进行预测,它先离线训练孪生网络,然后在进行多目标跟踪时把训练完成的孪生网络和相似性计算执行联合学习~\cite{wang2016joint}。
因为直观上对于相同身份的检测,视频帧序列中相隔较近的检测结果比相隔较远的检测结果有更大可能的相似,所以可利用卷积网络来学习被跟踪目标之间的相似性~\cite{b19}。
另外可以把注意力机制加入到孪生神经网络当中,利用注意力确保模型能够只注意检测结果中的前景部分,进而应对检测结果不完美和目标被遮挡等挑战~\cite{gm_phd}。
%
还可以使用循环网络抽取时间序列特征来建模被跟踪目标之间的相似性。
为了融合目标的外观、交互以及运动信息,网络框架可以利用连接的方法,基于长短时记忆网络来提取历史轨迹段与候选检测之间的相似性特征~\cite{b23}。
双向长短时记忆网络利用时间注意力机制来缓解错误的跟踪轨迹信息对数据关联的所造成的问题~\cite{dual_matching}。

最近为了避免检测特征的重复抽取并加速多目标跟踪算法的运行和加快算法的产业化落地,将目标检测和目标跟踪两个计算机视觉任务合并在一个深度网络中进行处理已成为一种趋势~\cite{jde,voigtlaender2019mots,fairmot}。
TrackR-CCNN~\cite{voigtlaender2019mots} 基于Mask-RCNN~\cite{he2017mask} 添加了一个行人重识别分支来预测边界框和目标特征。
JDE~\cite{jde} 基于YOLOv3~\cite{redmon2018yolov3} 在测试时获得了接近实时的跟踪速度。
FairMOT~\cite{fairmot} 发现基于锚框的检测器预测出的目标边界框可能会和实际的目标中心没有对齐,提出方法以解决严重的歧义和身份切换导致跟踪精度降低的问题。
%双线性长短时记忆网络的隐含层特征与输入之间采用乘性耦合方式代替传统长短时记忆网络中的加性耦合,能够更好地建模跟踪目标的历史外观特征~\cite{b3}。


\section{相关的数据集}
为了让各个算法有一个公平的比较以及加速研究过程,现有各种多目标跟踪公开的数据集来进行方法效果的测试。
经常使用的基准数据集包括 KITTI140~\cite{autonomous_vechicle}、PETS-2009~\cite{ferryman2009pets2009}、AVG~\cite{benfold2009guiding}、TUD~\cite{andriluka2010monocular}、MOTChallenge~\cite{leal2015motchallenge,mot16}等。
其中,MOTChallenge 公开数据集已发展成当前该方向使用最多的标准,同时本文为了验证进行视觉跟踪时人脑的行为响应和深度神经网络模型之间的关系,引入了眼睛凝视数据集 StudyForrest~\cite{gaze_forrest}。
因此,本文重点介绍这两个公开数据集。

%\begin{figure*}[ht]
%	\centering
%	\includegraphics[width=0.98\textwidth]{./figures/C1Fig/otb2013.pdf}
%	\caption{MOT17 数据集所包含的视频示例}
%	\label{fig:otb2013}
%\end{figure*}

\subsection{MOTChallenge 基准数据集}
MOTChallenge 是一个使用广泛的视频多目标跟踪公开数据集,它从已有的公开数据集中收集了 14 个视频序列,包括 KITTI~\cite{autonomous_vechicle}、AVG~\cite{benfold2009guiding}、PETS-2009~\cite{ferryman2009pets2009}、TUD~\cite{andriluka2010monocular} 和 ETH~\cite{ess2007depth}等,同时把以上数据划分为训练集与测试集,但是仅仅发布了有真实标签的训练集。
需要在线上上传利用测试集视频产生的输出,以得到算法的各种性能评价指标。
此外在这个数据集中的 MOT17 还提供了使用 DPM~\cite{felzenszwalb2009object}、Faster-RCNN~\cite{b8} 和 SDP~\cite{sdp} 检测器获得所有视频帧的检测输出,以公平地衡量不同跟踪算法基于相同检测器的性能;
MOT20~\cite{dendorfer2020mot20} 包含了 8 非常具有挑战的视频序列,如图~\ref{fig:c1:mot} 所示展示了数据集中各种动态开放场景下的多目标跟踪环境。

\begin{figure*}[ht]
	\centering
	\includegraphics[width=0.98\textwidth]{./figures/C1Fig/mot.pdf}
	\caption{多目标跟踪数据集示例}
	\label{fig:c1:mot}
\end{figure*}

该数据集包含了动态开放场景下的各种条件:
(1)数据集的图像序列是相机在各种光照情况下采集的,比如白天、晚上等;
(2)同时包括固定相机平台采集的视频图像和移动相机采集的视频图像;
(3)包括各种相机拍摄的角度,比如平行拍摄、俯视拍摄、低角度拍摄等。



% MOT综述: /data2/whd/win10/doc/paper/doctor/doctor.Data/PDF/1948313292
从各个角度出发设计了许多评价指标来有效地衡量方法的好坏,通常不同的衡量标准反映多目标跟踪方法不同的特性,经典常见的评价标准包括以下几个:
% https://www.cnblogs.com/wemo/p/10628836.html
\begin{itemize}
	\item 多目标跟踪准确度(Multiple  Object Tracking Accuracy,MOTA)是通过误判率、缺失率和误配率得来的,记为:$1 - \frac{\sum_t fp_t + m_t + mme_t}{\sum_t g_t}$;
	
	\item 多目标跟踪精度(Multiple  Object Tracking Precision,MOTP)是位置误差的评判指标,记为:$\frac{\sum_{t,i} d_t^i}{\sum_i c_t}$;
	
	\item 身份切换数目(ID Switch,IDS)记为:$\sum_t{mme_t}$;
	
	\item 大于百分之八十跟踪成功的目标数(Mostly Tracked,MT);
%	超过80\%的真实标注轨迹被成功跟踪的目标数目
	
	\item 大于百分之八十跟踪丢失的目标数(Mostly Lost,ML);
%	超过80\%的真实标注轨迹丢失的目标数目
	
	\item 假阳性数目(False Positives,FP):预测框没有检测框和它匹配,$\sum_{t}{fp_t}$;
	
	\item 目标丢失数目(False Negatives,FN):检测框没有预测框和它匹配,$\sum_t{m_t}$;
	
	\item 碎片轨迹总数(Fragmentation,Frag):所有真实标注轨迹被中断的次数。
\end{itemize}
其中,$fp_t$ 是第 $t$ 帧多目标跟踪算法产生的跟踪轨迹中关联失败的计数,即假阳性的计数。
$m_t$ 是关联失败的真实跟踪轨迹的数目,即假阴性的计数。
$mme_t$ 是相比于前面一个视频图像的关联产生身份切换的计数。
$g_t$ 是第 $t$ 个视频帧中真实跟踪轨迹的计数,
$d_t^i$ 是第 $i$ 个成功关联对的代价,
$c_t$ 是关联成功的计数。



\subsection{StudyForrest 数据集}
在实验中,使用《阿甘正传》电影作为动态开放自然环境的一种近似,完整的实验细节参考了原始数据集 StudyForrest~\cite{gaze_forrest}。
原始的《阿甘正传》电影包含的实际的 7 个电影片段。
这 7 个部分连接起来并重新分割为 8 个视频刺激段,每个视频段对应一个核磁共振记录。
时间戳参照《阿甘正传》 2002 年发行的 DVD,PAL 制式,DE103519SV,每秒25帧,并按照“HH:MM:SS.FRAME”的格式给出。
该数据集包含 15 个受试,使用飞利浦 Achieva dStream 核磁共振扫描仪记录下他们在观看好莱坞电影《阿甘正传》时的脑部激活数据和眼睛凝视数据,另外 15 位不在核磁共振扫描仪中记录的受试仅仅记录下他们的凝视数据,这些不在核磁共振扫描仪中记录的数据仅仅用于提高平稳跟踪事件的检测精度。
使用液晶投影仪进行电影刺激的播放,受试在配有前反射镜的核磁共振扫描仪中进行电影的观看,并记录下观看电影时的脑部激活情况。
凝视数据使用高频眼球跟踪仪 EyeLink 1000 进行记录,在记录核磁共振扫描数据时,配有长焦镜头并以 1000 次每秒的速度进行采样。
%在每个会话开始阶段进行 13 点校正。
使用 7 特斯拉的高精度核磁共振扫描仪,重复时间为 2 秒,体素大小为 $3 \times 3 \times 3$ 立方毫米获取核磁共振记录。
此外,还获得了一套综合的辅助数据(弥散张量成像、磁敏感加权图像、血管造影)以及用于评估技术和生理噪声成分的测量数据。



%\begin{figure*}[ht]
%	\centering
%	\includegraphics[width=0.98\textwidth]{./figures/C1Fig/challenge.pdf}
%	\caption{OTB数据集所划分的11类挑战属性}
%	\label{fig:challenge}
%\end{figure*}

% 参考:https://blog.csdn.net/AMDS123/article/details/81184250
% 毕设:/data2/whd/win10/doc/paper/doctor/doctor.Data/PDF/2507414993
\section{多目标跟踪的主要问题和挑战}
虽然近些年来视频多目标跟踪已经取得了一些进步,但是在动态开放场景下依然是一个非常有挑战的任务。
在真实场中下,跟踪多个目标的过程中所存在单目标跟踪器效率低且可解释性不足、检测器的不精确和相互遮挡的动态开放场景等是制约多目标跟踪性能提升和算法工程化应用的重要原因。
因此,设计一个高效的多目标跟踪算法来应对现实场景中的这些挑战并成功地进行多目标跟踪具有重大的意义。
%如图~\ref{fig:challenge}~所示,OTB数据集将目标在动态开放场景下所经历的变化划分为11个挑战属性。
%而这种分类也得到了广大研究者的认可。
%事实上,这些挑战也是目标跟踪的主要难点所在。

(1)单目标跟踪算法扩展到多目标场景中存在的问题

%一种直观的解决多目标跟踪的思路是将单目标欧跟踪算法直接扩展到多目标场景中,即多每一个待跟踪目标分别进行单目标跟踪。
随着目标检测技术的发展,跨帧连接检测结果的数据关联算法已经成为多目标跟踪任务一种常见的解决方案。
然而,数据关联方法过分依赖于不完美的目标检测器。
如果目标检测结果不准确、遗漏或误检,则跟踪的目标很容易丢失。
可以通过使用最新精度较高的单目标跟踪器来缓解此类问题,
此类跟踪器使用第一帧中的检测结果并在线更新单目标跟踪模型,以预测后续帧中跟踪目标的空间位置和大小。
%在此,将单个对象跟踪器和数据关联的优点结合在一个统一的框架中来解决这个问题。
%在大多数帧中,使用单个对象跟踪器来跟踪每个目标。然后当跟踪分数低于阈值时应用数据关联。该方案表明被跟踪的目标可能会经历较大的外观变化或被其他物体遮挡。
%近年来,随着深度学习的发展,特征提取能力越来越强,单目标跟踪算法精度也越来越高。
然而,在多目标跟踪场景中,为每个被跟踪目标建立一个单目标跟踪器,这里主要存在以下两个问题:
\begin{itemize}
	
	\item 模型的复杂性和可理解性问题:基于深度学习的单目标跟踪算法在提高精度的同时带来的另一个问题是随着单目标跟踪模型的层数越来越多,导致跟踪模型运算量急剧增大,严重制约了这种方法的实际应用。
	并且深度神经网络模型越来越像一个“黑盒”,可理解性越来越差,偏离了人工智能为人服务的初衷,与人的能力差距越来越大,这为模型的优化和人类利用人工智能技术认识自身带来了巨大障碍。
	
	%\item 模型复杂性问题:跟踪效率问题是单目标跟踪算法扩展到多目标场景时新产生的问题。
	%由于对场景中的每一个新出现的目标都要建立一个相应的单目标跟踪模型,其跟踪效率随着待跟踪的目标数目的增加而大幅降低,
	%同时深度单目标跟踪模型的运算量随着模型的增大而急剧增大,这些因素极大地限制了这种方法在实际中的应用。
	
	% 基于预测的范式
	\item 跟踪漂移问题:跟踪漂移问题是单目标跟踪器也需要解决的问题,通常是当跟踪目标被遮挡时,使得单目标跟踪模型关注到遮挡物上,发生跟踪漂移的现象。
	把单目标跟踪模型直接迁移到视频多目标跟踪时,特别是目标密集的的情况下,多个目标之间会发生相互遮挡会更加频繁,跟踪漂移问题会变得更加突出。
	
\end{itemize}


(2)基于检测的多目标数跟踪目标关联方法的不足

相对于单目标跟踪,多目标跟踪还需要处理两个其他的挑战:随着时间的变化定位所有目标的位置以及保持每个目标的身份信息。
除了包括与单目标跟共有的不足和挑战外外,还要应对更具有挑战性的因素,比如跟踪轨迹的开始与结束、干扰物频繁的遮挡、相似的外观特征、多个跟踪目标之间的相互影响等。
进行多目标跟踪时必须将以下两个问题都考虑进去,其中一个是如何区分同一视频帧中各个目标之间的相似性,而另一个问题是如何利用所得到的相似性来判断视频帧之间的目标身份是否相同。
前者主要包括如何建模外观特征、运动特征、排斥、碰撞和空间交叉,后者主要是和时间上的数据关联有关。

目前根据检测器结果进行目标关联的方法一般会有下面两个问题:

\begin{itemize}
	% 非局部注意力、sture
	\item 建模时空特征的高度复杂性。
	基于数据关联的多目标跟踪算法在进行当前视频帧候选检测和历史轨迹的关联时,需要一个非常鲁棒的时空特征来保证相似度计算的准确性,然而存在历史轨迹特征建模的不充分、空间维度上的检测结果特征和时空维度的历史轨迹特征之间的特征不平衡等问题,这些不可靠的特征会严重影响在线目标关联多目标跟踪方法的性能。
	%基于检测的数据关联多目标跟踪算法的性能于所采用的目标检测器的性能密切相关,然而现实的空间维度上的目标检测器并不是完美的,这些不可靠的检测结果会传播到时间维度上的数据关联步骤,从来严重影响在线数据关联多目标跟踪算法的性能。
	
	\item 检测和跟踪相互独立问题。
	目前利用目标关联进行视频多目标跟踪的方法把空间维度上目标检测与时间维度上的数据关联建模成是两个分开处理的研究问题,然而检测与跟踪两大计算机视觉任务之间有着密不可分的关系,优秀的检测器可以有效地提高跟踪的效果,同时如果将多目标跟踪结果反馈给前一步的检测器可以进一步提升目标检测器的效果,所以将检测和跟踪合并进入同一个任务是更好地提升视频多目标跟踪精度和速度的核心,同时也是多目标跟踪方法将来研究的方向。
\end{itemize}

因此,单跟踪目标模型的可解释性和数据关联算法中时空特征建模的复杂性是两个重要的难题。
%如何在复杂的场景下成功地捕获目标巨大的外观变化和应对遮挡问题是跟踪算法取得良好跟踪效果的关键所在。
本工作主要根据这两大类挑战进行研究。

%为了充分说明多目标跟踪的困难性,本文先简单介绍单目标跟踪的挑战,主要集中在设计复杂的外观模型和运动模式,包括11个挑战属性。

%(1) 光照变化:光照变化导致目标的外观发生极大的变化,会严重影响目标的颜色分布甚至是梯度信息。由于颜色特征和梯度特征是目标跟踪算法最常使用的两种特征,因此光照变化会在一定程度上影响算法的跟踪性能。
%
%(2) 尺度变化:目标尺度在跟踪过程中经常会发生变化。如果跟踪算法不能准确的估计目标的尺度,那么训练得到的外观模型就不能很好的建模目标的外观。这可能导致错误在跟踪过程中不断地积累,进而造成跟踪失败的结果。
%
%(3)遮挡:遮挡是目标跟踪中非常难解决的难题之一。由于目标跟踪是一个在线任务,因此跟踪模型必须在跟踪过程中不断地更新来捕获目标最新的外观变化。如果跟踪算法不能准确地检测到遮挡的存在,那么模型更新操作会将遮挡考虑进去,进而造成模型被污染和模型漂移问题。
%
%(4)形变:形变主要针对非刚体目标(如人和动物等)的跟踪,也是目标跟踪的难题之一。当非刚体目标发生剧烈的形变时,目标的外观甚至是宽高比都会有明显的变化。跟踪模型很难准确地捕获到这种外观和宽高比的变化。
%
%(5) 运动模糊:运动模糊主要会在目标快速运动的时候发生。当出现运动模糊时,目标的颜色、纹理甚至梯度信息都会变得模糊和不明显。这会影响跟踪算法区分目标和背景的判别能力。
%
%(6) 快速运动:快速运动指相邻两帧中的目标位置发生了很大变化。由于目标跟踪是一个在线任务,因此跟踪算法通常在有限的搜索范围内来定位目标以保证算法的实时性。当目标发生快速运动时,跟踪算法不可能在搜索范围内准确地定位目标,进而会导致跟踪失败。
%
%(7) 平面内旋转:平面内旋转指目标在图像平面内发生旋转。由于目标跟踪所使用的特征大都不具备旋转不变性,因此当目标发生平面内旋转时,跟踪算法很难准确地定位目标。
%
%(8) 平面外旋转:平面外旋转指目标的旋转超出了图像平面范围。当目标发生平面外旋转时,目标原来的可见部分将会消失,而原来的不可见部分将会出现在图像平面。在这种情况下,跟踪算法很难有效地捕获到目标不可见部分的外观信息,从而导致跟踪失败。
%
%(9) 目标消失:目标消失是指目标从图像平面内暂短消失。如上所示,目标跟踪是一个在线任务,跟踪模型必须在线更新来适应目标的外观变化。如果跟踪算法不能成功地检测到目标消失,那么模型更新将会导致模型漂移问题发生。即使目标在后来的跟踪过程中重新出现,跟踪模型也不能再准确地跟踪到目标。
%
%(10) 背景嘈杂:背景嘈杂指图像平面内包含很多与目标外观相似的干扰项。由于这些干扰项具有与目标相似的外观,因此跟踪模型很容易将背景中的干扰项作为最终的跟踪结果。
%
%(11) 低分辨率:低分辨率主要是由拍摄视频的设备所造成。如果视频的分辨率很低,那么目标的各种特征也将不明显和不具有判别力。因此,学习得到的跟踪模型将不能很好的区别目标和背景。



% 解决思路:时空注意力
% 2: 自下而上+自上而下的注意力
% 3: 全局注意力(时空)
% 4:时空互表征学习(不对称)
% 5:(检测)空间 -> (数据关联)(时空):解决重复+端到端训练
\section{本文主要的贡献与创新}
从上面的分析可知在进行多目标跟踪过程中存在的跟踪模型可解释性和时空特征复杂性两个难题,也是影响多目标跟踪算法理解和部署的重要原因。
处理好上述两个挑战对提升多目标跟踪性能非常重要。
所以本研究利用神经解剖对齐、非局部注意力、时空互表征学习和端到端等理论和模型,设计了一系列方案来处理这些挑战来增强模型的可解释性和提高多目标跟踪的效果。
本研究主要的贡献点包括:

(1)提出一种神经解剖对齐的类脑单目标跟踪深度神经网络模型。
为了理解和解释日益加深的深度单目标跟踪模型,降低模型复杂性,解决跟踪精度和人脑本身处理平滑跟踪之间的矛盾,
首先,该方法尝试将深度神经网络的各个模块与人类大脑皮层平滑跟踪通路相关脑区的解剖结构进行对齐,产生更加符合皮层解剖结构的类脑跟踪网络。
其次,对人类进行视觉目标跟踪时的大脑皮层激活响应和眼动行为数据进行分析,找出和跟踪任务相关的皮层区域和相应的激活响应。
最后,为了合理地评价模型的类脑性能,设计了一个新颖的度量方法来衡量类脑深度神经网络与人类大脑之间皮层激活响应和人眼行为的相似性。
通过深度探索模型的跟踪性能和与皮层和行为的相似性,发现设计的模型和人类大脑处理视觉跟踪任务时的关联性,并从模型结构、激活响应、行为动作等方面进行解释。

(2)为了应对多目标跟踪时目标外观、姿势变化、频繁遮挡等问题,并解决进行传统数据关联时历史轨迹特征建模的不完整性和一般卷积操作中特征提取的局部性,设计了一种在跨时空范围的非局部注意力模型,以达到更好的关联效果。
首先将非局部注意力层嵌入到传统卷积神经网络中,来自适应地提取跨空间和时间区域而不是局部区域的全局特征,使用非局部注意力机制来抑制目标检测不准确和遮挡问题。
其次,该方法还提出了一个注意力关联网络来处理多目标跟踪中的序列相关性和遮挡。
在关联轨迹和检测结果时,所提出的网络不仅生成目标检测与历史轨迹中观测值之间的相似性,还生成与所有行人序列的一致性,以减轻轨迹中不可靠样本和单目标跟踪器的跟踪漂移问题。
最后,设计了一个目标关联的学习算法和对应的数据处理策略。
在执行训练过程之前,利用多目标跟踪数据集的检测结果来生成各种行人段,并以等概率随机采样轨迹段,以满足网络的输入大小要求。
同时,从一系列数据增强策略中制定了一个方案,以解决模型训练过程中数据不足和过拟合的问题。

(3)提出了一种时空互表征学习方法。
为了解决当前帧检测结果与历史轨迹序列特征差异的问题,并解决当前检测结果的时间特征被忽略和关联双方特征不平衡的问题,使得当前候选检测在对象关联上可以更好地与历史轨迹进行关联。
首先,提出了一种新颖的时空互表征学习架构来解决当前检测的空间特征与目标关联的历史序列的时空特征之间的特征差异问题。
其次,为了增强所提出方法的互学习和识别能力,提出了交叉损失、模态损失和相似性损失,这些精细设计的损失都有助于检测学习网络获得时间特征,缓解特征不平衡的问题。
最后提出了一个基于单目标跟踪预测的多目标跟踪方法,通过使用时空互表征学习的策略来缓解单目标跟踪的漂移问题,提高所提出方法的准确性和鲁棒性。

(4)
提出了一种端到端模型架构和训练方法来联合目标检测和多目标跟踪。
传统的多目标跟踪将目标检测作为一个独立的前置任务进行研究和处理,但在实际应用中目标检测和多目标跟踪两个任务进行了重复的特征提取,导致计算代价过高,难以满足实际应用实时性的要求。
该方法使用同一个深度网络同时处理目标检测和数据关联,并解决了检测和跟踪两个任务连接时存在的数据不一致问题,并达到实时的跟踪效果。
首先。设计了一个端到端架构来联合处理目标检测和在线多目标跟踪两个任务。
其次,为了解决目标检测子模块的输出与关联子模块的输入之间的边界框数目和大小不一致的问题,提出了联接子模块和合适的训练数据产生策略。
最后,设计了一个迭代训练方法来训练检测子模块和关联子模块,并完全以端到端模式执行在线多目标跟踪过程。
并且在基准数据集上做了一系列消去试验研究,表明所提出的算法获得了比许多在线多目标跟踪方法更好的跟踪性能。

% 说tu 
以上四个贡献点在逻辑上为依次递进的关系,如图~\ref{fig:c1:organization}~的中间所示,第二章的单目标类脑跟踪模型为第三章和第四章基于单目标跟踪预测的多目标跟踪范式提供基础,第三章利用非局部注意力机制以更好地提取数据关联中历史轨迹的特征,第四章利用时空互学习策略解决当前候选检测结果和历史轨迹的特征不平衡问题,第五章在第四章的基础之上将目标检测任务合并进多目标跟踪任务,最后将本文的贡献在一个智能驾驶系统中进行应用验证。

% https://texample.net/tikz/examples/feature/smartdiagram/
%\begin{figure*}[ht]
%	\centering
%	\includegraphics[width=0.98\textwidth]{./figures/C1Fig/organization.pdf}
%	\caption{本文的组织结构}
%	\label{fig:organization}
%\end{figure*}


% 锦上添花
% +绪论
% + 结论和展望
\begin{figure*}[ht]
	\centering
	\includegraphics[width=0.90\textwidth]{./figures/C1Fig/network.pdf}
	\caption{本文组织结构}
	\label{fig:c1:organization}
\end{figure*}

\section{本文组织结构}
本文研究和探索了在动态开放场景下的视频多目标跟踪问题,来尝试解决单目标跟踪器的可解释性和时空特征建模的复杂性问题。
本文总共分为七个章节,整体逻辑架构如图~\ref{fig:c1:organization}~所示。


第一章为绪论,概述了多目标跟踪问题的背景和意义、国内外研究现状以及相应的数据集等,着重介绍了多目标跟踪任务的问题和挑战以及本研究的主要创新点。

第二章介绍深度神经网络模型在处理视觉目标跟踪问题时候可解释性不足的的问题。
然后重点介绍所提出的神经解剖对齐的视觉目标跟踪模型的具体细节。
最后将所设计的算法在公共数据集上进行测试并和其他方法进行比较来论证所提出方法的有效性。

第三章首先分析基于多目标跟踪所存在的问题和传统方法的局限性。
然后介绍了设计的嵌入在卷积神经网络中的非局部注意力层来自适应地提取跨时空区域的非局部特征。
最后介绍了该算法在基准数据集上的测试性能。

第四章根据时空互学习的思想,先简述了目前存在的多目标跟踪数据关联方法所存在的问题。
随后进行了相关方法的简述。
再详细介绍了所设计的时空互表征学习算法并在多目标跟踪数据集上和其他方法进行比较。

第五章先概述了目前检测和跟踪分开研究所存在的问题,并且引出了端到端学习的必要性。
然后详细讨论了联合检测和关联的研究现状。
详细阐述了所设计的端到端检测和关联方法和该方法在多个多目标跟踪数据集上的实验结果。

然后,第六章将本文所设计的跟踪方法部署在一个实时多目标跟踪和分析系统中,论证了本文工作的的实用性和对现实社会的贡献。

结论与展望章节先简要总结了本文的所有研究点。
随后讨论了本文研究尚存在的缺点和不足,最后提出了将来可能的探索方向。







\chapter{大脑皮层平滑跟踪通路解剖对齐的视觉跟踪模型}
\label{chap:btn}
% 更接近于人 vs 更接近于神
% regressor  回归变量
% STS: superior temporal sulcus (STS)
% SEF: supplementary eye field (SEF)

\section{引言}
% BTN第一章第一段
作为多目标跟踪的基础,基于深度神经网络的单目标跟踪模型已经取得了巨大的进步~\cite{ILSVRC15},然而随之而来的是越来越复杂的模型和可解释性的缺失,导致单目标跟踪算法难以扩展到多目标场景,并严重制约了模型的理解与实际应用。
尽管更深的深度神经网络确实能提高跟踪的精度,但这并不能提高神经网络模型和人脑的相似性~\cite{rajalingham2018large}。
深度学习发展初期部分人工神经网络模块可以映射到大脑皮层视觉通路的相应区域,然而随着模型的发展,比如 GoogleNet~\cite{szegedy2015going} 或 Inception-v4~\cite{szegedy2017inception}中,人工神经网络中众多复杂模块与大脑皮层中少量的视觉区域之间的关系越来越弱。
为了在模型中更准确地捕捉大脑皮层的处理模式,提高模型的可解释性和降低模型复杂性,仅仅基于传统视觉数据集进行模型架构搜索似乎不再是可行的解决方案。


%最终导致视觉任务中精度高的网络越来越深
在目标识别领域,深度模型在构建具有神经可解释的模型方面取得了一些成就~\cite{kubilius2019brain-like},
出现了一些设计类脑深度神经网络架构用于图像识别的工作~\cite{TangSchrimpfLotter2018Recurrent, kar2019evidence}。
%特别是是用于图像识别的深度神经网络~\cite{Deng2009ImageNet},
其中的神经元能部分解释为什么人类大脑视觉皮层中的神经细胞对图像有特定的激活~\cite{yamins2014performance,khaligh2014deep,gucclu2015deep,murugesan2017brain,cichy2016deep,yamins2016using}。
这些模型也部分预测了灵长类图像分类的行为和评估~\cite{rajalingham2018large,kubilius2016deep},提高模型可解释性的同时降低了模型复杂度。
同时这些优秀的类脑模型能预测在人类大脑皮层通路中产生的激活模式~\cite{bashivan2019neural},为脑机接口的实现提供了很好的机会。


% BTN第二章第三段
在本章的研究工作中,尝试将深度神经网络模块与人脑大脑皮层平滑跟踪通路的解剖结构进行对齐,这将产生层数更少、更易解释且更像人的类脑跟踪模型。
为了衡量模型的效果,设计了一个更符合神经解剖学对齐的深度神经网络,称之为类脑跟踪网络(Brain-like Tracking Network, BTN),在评估视觉通路的模型类脑相似性方面表现突出,同时在 StudyForrest 数据集中获得了良好的视觉跟踪效果~\cite{gaze_forrest}。
BTN 是大脑皮层视觉跟踪通路的浅层循环类脑架构,因此 BTN 具有和大脑更加相似的结构来预测眼球运动行为和神经激活,并构建了一种新颖的类脑跟踪分数(Brain-like Tracking Score, BTS)来衡量类脑相似性,在由行为和皮层记录组成的新基准数据集 Tracking-Gump 上取得了较好的类脑跟踪性能。
经过大量的实验发现类脑的网络架构是产生这些类脑的结果的原因,这和人类大脑皮层通路处理刺激输入时所获得的大脑激活的先验知识是一致的~\cite{TangSchrimpfLotter2018Recurrent, yin2020deep, kar2019evidence}。
最后,为了比较 BTN 中 DFN\textsubscript{MT/MST} 中的激活模式和人类视觉皮层的中颞区和上颞内侧区(Medial Temporal/Medial Superior Temporal,MT/MST)的响应,发现 BTN 可以较好地预测该皮层区域的激活响应,这也是第一个在神经记录上这样做比较的跟踪模型。

% Q:视觉信号的无限传输、视觉假体的兼容
% 2008年9个电极 -> 2020年16个阵列*64个电极=1024个电极
%预测视觉神经网络的响应,有望解决盲人的视觉问题(比方说盲人戴个装有微型摄像头的眼睛就可以实现光幻视,以达到直接产生视觉意识或者人造视觉意识),实现从计算机视觉到人工视觉的转变。
%借鉴图像识别中的生物机制解释,解决目标跟踪中的解释性问题。
%深度学习发展的两个方向:第一个方向是尽可能提高各个指标的性能,超越和延伸人类在特定方面的能力(强调能力)。
%第二个方向是表现地尽可能贴近人类,使机器更加人性化,同时尝试使用计算机理解人类大脑的意识形成机制,并解决人的一些视觉感知和视觉主观意识问题(强调感觉和意识),
%计算机视觉与人工视觉的对比关系:
%\begin{itemize}
%	\item 适应性:计算机视觉(机器视觉)适应性差,容易收到复杂背景及环境变化的影响;而人工视觉(人类视觉)适应性强,可在复杂环境中识别目标。
%	\item 智能化:机器视觉虽然可以利用人工智能及神经网络技术,但智能很差,不能很好的识别变化的目标;人类视觉具有高级功能,可运用逻辑分析及推理能力识别变化的目标,并能总结规律。
%	\item 彩色识别能力:机器视觉(计算机视觉)受硬件条件的限制,目前一般的图像采集系统对色彩的分辨能力较差;人类视觉(人工视觉)对色彩的分辨能力强,但容易受人的心理影响,不能量化。
%	\item 灰度分辨力:机器视觉分辨力强,目前一般使用256灰度级,采样系统可具有10bit、12bit、16bit等灰度级,例如维视图像MV-EM510M工业相机就常用于灰度图像处理;而人类视觉灰度分辨力差,一般只能分辨64个灰度级。
%	\item 空间分辨率:机器视觉在空间分辨率上具有各种分辨率的面阵摄像机和线阵摄像机,通过配置各种维视图像光学镜头,可以观察小到微米大到天体的目标;而人类视觉分辨率较差,不能观看微小的目标。
%	\item 机器视觉的快门时间可达到10微妙左右,高速像机帧率可达到1000以上,处理器的速度越来越快;人类视觉上,0.1秒的视觉暂留使人眼无法看清较快速运动的目标。
%	\item 感光范围:机器视觉能感应从紫外到红外的较宽光谱范围,另外有X光等特殊摄像机;人类视觉只可看见在400nm-750nm范围的可见光。
%	\item 环境要求:机器视觉对环境适应性强,另外可加防护装置;人类视觉对环境温度、湿度的适应性差,另外有许多场合对人有损害。
%	\item 观测精度:机器视觉观测精度高,可达到微米级,易量化;人类视觉观测精度低,无法量化。
%	\item 其它:机器视觉可连续工作;人类视觉易受心理影响,易疲劳。
%\end{itemize}
	



% 全面的抹除记忆->指定时间或者内容来进行记忆的擦除


%基于以上分析,本章基于相关滤波框架提出了一种新的集成相关跟踪算法来准确地跟踪目标。因为该方法在相关滤波的框架内利用集成学习进行跟踪,所以它被命名为集成相关跟踪方法。具体来说,该方法学习多个相关滤波器建模目标在跟踪过程中所出现的不同外观模式,并且执行集成操作以检测目标的准确位置。所学习的每个滤波器表示目标的一种特定外观模式。为了获得不同的滤波器来进行集成跟踪,本章设计了一种新的回溯算法学习多个相关滤波器以捕获目标不同的外观模式。该回溯算法利用了目标在跟踪过程中的时间上下文信息,能够有效地生成滤波器并保持较低的计算复杂度。此外,通过考虑目标历史的外观信息和最近的外观变化,本章提出了一种新的在线权值分配算法来为不同的滤波器分配合理的权值。由于利用了多个相关滤波器建模目标在跟踪过程中的不同外观模式并且使用了集成学习进行目标跟踪,本章所提出的集成相关跟踪方法能够捕获目标在复杂场景下的外观变化和应对遮挡等问题以取得更好的跟踪效果。本章的主要贡献如下:
%\begin{itemize}
%	\item 提出了一种基于相关滤波框架的集成相关跟踪方法。该方法学习多个相关滤波器建模目标不同的外观模式,并且执行集成跟踪以准确地检测目标。
%	\item  基于目标的时间上下文信息,提出了一种回溯算法来生成多个滤波器建模目标的外观模式以执行集成跟踪操作。
%	\item  通过考虑目标历史的外观信息和最近的外观变化,提出了一种在线权值分配算法来为不同的滤波器分配合理的权值。
%\end{itemize}

\section{相关工作}
本节将主要介绍与该研究相关的工作,包括人工智能和神经科学关系的背景、视觉目标跟踪中的深度学习网络和人类的平滑跟踪。

\subsection{人工智能和神经科学}
由于大脑是唯一已知的真正通用智能的样例,
人类在某些能力方面拥有卓越的性能,所以人们也想让人工设计的智能体也具备这样的能力。
神经科学领域的发现能够启发新的深度模型架构或损失函数设计。
神经科学对人工智能的影响,可以类比于飞鸟对制造飞机的启发。
%其中完全仿生学方法制造的扑翼飞机并没有理解了流体力学后制造的固定翼飞机那么成功。
%从宏观到微观依次为中枢神经系统(CNS)、系统、地图、网络、神经元、突触、细胞;
%或者社会学/人际交互、心理学、(认知)、高阶大脑区域/复杂行为、大脑区域/行为的开端(传感、运动输出)、神经元组(群)、单个神经元、细胞/分子。
反过来,由于人工智能的研究和发展不需要受限于生物大脑和社会伦理的限制,比如实验条件、结构约束等,可以仿照生物学规律进行模型的设计,实验得出的结论可以反过来促进人们对大脑未知领域的理解,
实现人工智能和神经科学发展的相互促进。


%\subsection{功能性核磁共振}
本章实验的数据是利用神经科学中常用的功能性核磁共振成像(functional Magnetic Resonance Imaging,fMRI)来收集人进行单目标跟踪时大脑皮层的活动。
其中核磁共振为磁矩不为零的原子核\footnote{原子核为带正电的粒子,不能自旋的原子核没有磁矩,能够自旋的原子核拥有电流,所以产生相应地磁场,并形成磁矩。},在外部磁场影响下自旋能级会产生塞曼分裂\footnote{即塞曼效应,表示原子的光谱在外磁场中出现分裂,产生的跃迁是原子核自旋在核塞曼能级上发生的跃迁。},共振吸收部分射频辐射的过程\footnote{其共振频率波段处于射频段,由于人体中氢原子丰度高,临床使用氢原子共振,共振频率是42.58MHz/T,其他磁场强度按比例折算。}。
% 射频是指频率范围从300KHz到300GHz之间,波长在1毫米到1米之间的电磁波。
在神经科学领域,核磁共振成像(Magnetic Resonance Imaging,MRI)提供内部结构的图片,扫描脑灰质、白质、脑脊液的形态结构,
而功能性核磁共振成像是根据大脑执行特定任务时局部脑区血氧水平产生的改变,来观察执行特定任务时“脑激活”的状况,可以显示活动水平非常细微变化的成像,时间分辨率为秒级。

\subsection{深度跟踪神经网络}
最近许多工作表明,显著目标是通过一系列中心凹视图提取得到的~\cite{mnih2014recurrent,draw}。 
BTN 中的目标提取模块是用一个在神经解剖学上可解释的二维高斯滤波器~\cite{ratm} 进行实现。 
在目标跟踪中关注特定的外观表示与关注空间特征一样重要。 
具有递归的方法可以自适应地改变卷积模块的滤波器,因此可以将其更改为图像帧上的表示并优化模型性能~\cite{stollenga2014deep}。 
动态滤波器网络(Dynamic Filter Network,DFN)~\cite{brabandere2016dynamic}的滤波器是根据输入特征在线获得的,使用 DFN 可以在不降低精度的情况下实现模型压缩。
依赖于输入的状态转换可以帮助学习马尔可夫决策模型~\cite{karl2017deep},
在这里不直接使用该结果,而是使用这个思想来进行显著特征的预测。

% glimpses 显著性目标
在视觉目标跟踪任务中,一般很适合使用循环结构和显著性目标,但是它们只能在简单背景的单色视频中表现很好~\cite{ratm}。
一些工作~\cite{ssrcnn}通过目标检测器的表示获得了显著目标的结果,并将目标检测器的结果作为循环神经网络(Recurrent Neural Network,RNN)的输入~\cite{su2020improved}。
但是,这个方法需要处理每个完整的视频帧。
最近的一项研究~\cite{gordon2017re3} 明确利用 RNN 来建模人类的注意力机制。
本文的研究类似于循环注意力模型~\cite{ratm},并利用长短时记忆网络(Long Short-Term Memory,LSTM)实现的人类注意力来更好地提取多帧中的运动特征。
此外,本章尝试设计一个类脑跟踪网络,该网络将获得更好的 BTS 并在 Tracking-Gump 数据集上超越现有的跟踪模型~\cite{gaze_forrest}。


% 和带有类脑跟踪分数的的
\begin{figure*}
	\centering
	\includegraphics[width=6.2in]{figures/C2Fig/introduction.pdf}
	\caption{
		神经解剖学对齐的深度神经网络协同设计
	}
	\label{fig:c2:introduction}
\end{figure*}

% Following Forrest Gump: Smooth pursuit related brain activation during free movie viewing
\subsection{平滑跟踪}
通常使用人工合成刺激来研究人眼凝视,比如研究眼跳采用改变目标位置时眼睛注视的形式,而研究平滑跟踪则使用线性或者正弦运动的点。
使用人工合成数据最大的优势是拥有定义良好的属性和明显特征,特定的特征简化了眼球跟踪和血氧水平依赖信号的分析的过程。
通过仅仅沿着一个独立的特征位置,可以进行理想的刺激调制,能更加精确地建立起特定脑区之间激活的联系。
但这些优势所牺牲的是人工合成数据并不能代表人类正常的视觉,实际有效的视觉输入会更加复杂,并且动态开放的环境中平滑跟踪不会单独出现,而是和眼跳和注视的序列混在一起。
因此,在单一背景上的人工合成刺激忽略了可能的背景信息、拥挤效应和全部眼睛运动规划流程的影响。
另外一个缺点是长时间的跟踪单一背景下的人工合成刺激会导致人注意力维持的降低。

特别是日常生活中,在动态背景上进行平滑跟踪,并行处理互相冲突的信息通常是必要的,这种情况下可能在不同方向上包含多个移动的目标。
值得注意的是从猴子的研究中得出结论:V5 不仅在平滑跟踪中,而且即使在视网膜上冲突时不同刺激之间的相互作用中也发挥重要的作用~\cite{neural_sp}。
然而,在自然条件下处理平滑跟踪眼睛运动所引起的动态视觉输入时必须非常小心。
然而即使是对于非常大且多样的数据集,最近计算机视觉领域所取得的进步已经开始能够自动理解动态开放的场景。
相比于人工合成的刺激,当使用不受限的动态自然刺激,在实验中如何利用运动眼睛的显示指令(比如:“跟踪这个点”、“当目标出现时进行眼跳”等),将凝视轨迹分成不同眼睛运动成分仍然很有挑战。
手动标注“真实数据”被认为是眼睛运动成分分类的黄金标准,对于每一秒的凝视信号,只有注视和眼跳能在 4 到 15 秒的时间内到达场景的任何地方~\cite{gold_standard}。
比如,StudyForrest 数据集中提供了大约 30 小时的 fMRI 和同步的凝视记录~\cite{gaze_forrest}。
因此,StudyForrest 数据集同时记录了视觉和非视觉线索相关的脑部激活,可以用来更好的进行眼睛运动成分的分类。

最近几年在眼睛运动数据,特别是平滑跟踪的自动识别分析中取得了很大的进展。
基于部分或全部真实值标注好的大规模数公开据集,提出了一些眼睛跟踪分类算法\cite{var_natural,dynamic_eye,gold_standard,auto_classification}。
因为之前大多数分类算法是基于静态刺激进行开发,所以仅仅标注了注视和眼跳。
因此这些算法会错误地将平滑跟踪分类成注视或者眼跳,阻止了与平滑跟踪相关激活的识别。
为了解决这个问题开发了自动多观察者平滑跟踪算法(Multi-Observer  Smooth Pursuit,MOSP),该算法能够以较高的分类精度区分三种主要的眼睛运动,分别是注视、眼跳和平滑跟踪~\cite{auto_classification,eye_movement_swap}。

在本章的研究中利用了最新提出的几种眼动识别方法。
使用最先进的计算机视觉和眼睛运动分类算法,从大规模的 StudyForrest 数据集中分析了凝视和 fMRI 记录,并能够在动态开放的自然场景(好莱坞电影)中识别与平滑跟踪和眼跳眼睛运动相关的大脑区域。



\section{类脑跟踪网络}

在本节中分三步介绍所提出的 BTN。
首先简要指出了设计类脑模型的动机和准则,
其次详细介绍了 BTN 模型中的每个模块,
最后介绍用于训练 BTN 的损失函数。


\subsection{设计准则}
该研究设计的目标是在视觉目标跟踪的深度神经网络(Deep Neural Networks,DNN)模型和大脑中的平滑跟踪通路之间获得高度的相似性,
同时遵循两个准则来设计 BTN~\cite{kubilius2018predict}:

(1) \textbf{结构}:在跟踪性能相差不多的模型架构中,一般更加倾向于使用类脑模型,因为它模型复杂度低、更容易理解并且可以满足大脑的解剖约束。
使用深度神经网络是因为它们的神经元是数据处理的基本单元,并且深度神经网络中的所有神经激活都可以清晰地映射到大脑皮层激活~\cite{yamins2016using}。
除此之外,由于视频序列中的时间属性,视觉目标跟踪自然会考虑使用循环连接。
同时背侧通路的激活也具有时间属性,因此也假定 BTN 会为时间序列生成激活响应。

(2) \textbf{功能}: 
%it is a mechanistic model of the brain. 
% 内部构件
在类脑模型中,满足神经解剖学约束的中间层激活响应和最终输出行为会有更准确的结果,
这使得设计的类脑模型能够更好地预测视觉皮层的激活和人眼的运动行为,
能同时满足计算机视觉目标跟踪预测和神经科学平滑跟踪预测的需求。

大脑中平滑跟踪运动的目的是转动眼球使移动目标的图像在人眼中央凹上的位置保持不变。
如图~\ref{fig:c2:introduction} 所示,在平滑跟踪的大脑通路中,初级视觉皮层 (V1) 是对输入信号进行第一阶段的预处理,
MT/MST 整合了空间中的运动信息,
而额叶视区(Frontal Eye Field,FEF)生成预测性眼球运动信号~\cite{b11,b13,b14}。

受大脑中平滑跟踪通路的启发,构建从大脑皮层区域(图~\ref{fig:c2:introduction} 下部)到深度神经网络层(图~\ref{fig:c2:introduction} 上部)的神经解剖映射。
通过量化的大脑相似性分数,并利用所获得的皮层解剖知识来启发 BTN 的设计。
BTN 包括映射到人脑的四个区域:初级视觉皮层 (V1) ,中颞区和上颞内侧区(MT/MST)、额叶视区(FEF)和脑干/小脑 。
CONV\textsubscript{V1} 是经典的卷积层,执行预处理以减少数据大小。
DFN\textsubscript{MT/MST} 是动态滤波器网络,RNN\textsubscript{FEF} 是循环神经网络。
对于最左边的输入刺激,右上角的两个图像分别表示大脑对齐模型中的深度神经网络的激活响应和边界框,
而右下图的两个图像分别表示跟踪时大脑皮层的激活和眼睛注视的位置,
同时 BTS 显示了计算机视觉跟踪性能与大脑跟踪响应之间的相似性关系。
为了比较模型,通过检查 DNN 中各层的激活来构建到皮层的映射,以便能够很好地理解特定大脑皮层区域中的激活,
理想情况下,这种大脑皮层激活不需要多余参数的类脑模型所预测得到,将会降低传统深度跟踪模型的冗余度。
因此如图~\ref{fig:c2:pipeline} 所示,BTN 由卷积层、DFN、LSTM 和 全连接(Fully Connected,FC)层四个神经网络模块组成,
它们类比于平滑跟踪通路中的 V1、MT/MST、FEF、脑干/小脑,其中脑干/小脑是运动预测器,将 FEF 的输出转换为相应的运动响应。
这种明确的大脑分区思想是设计类脑跟踪模型重要的一步,并且致力于寻找更通用的网络结构。
整个模型包括在皮层区域没有差异的神经网络,以及各种改进 BTN 的连接。
%我们现在介绍 BTN 的处理流程。





\subsection{BTN 架构}
% 处理流程
首先简要介绍 BTN 的处理流程,如图~\ref{fig:c2:pipeline} 所示,在平滑跟踪中发挥特定作用的大脑皮层区域用矩形框表示。
对于给定的输入帧 $F_t$ 和注意力参数 $a_t$,模仿空间注意力的中央凹从 $F_t$ 中选择包含目标的局部区域视图 $f_t$。
此外,在考虑外观特征 $\alpha_t$ 的基础上,使用包括初级视觉皮层 V1 的背侧和腹侧通路的外观选择器,获得跟踪目标更加精细的特征 $m_t$,并在 LSTM 中更新隐藏状态 $h_t$。
并将 FEF 的输出 $o_t$ 和背侧流输出 $d_t$ 合并输入到脑干/小脑模块中进行解码,
%然后使用脑干/小脑来解码 LSTM 的输出 $o_t$ 和背侧流的输出 $d_t$,
来预测下一帧的空间注意力 $a_{t+1}$ 、外观注意力 $\alpha_{t +1}$、眼位校正信号 $\Delta p$,甚至眼球运动信号 $\Delta p_t$。
% 外观注意力(Retina -> V1,VN -> MT/MST)
%具体而言,FEF 对于基于目标速度启动跟踪学习可能非常重要,
% PN: pontine nuclei脑桥核
% 预测中的残差信息(预测性编码假设)
%小脑和脑干一起(包括脑桥核(PN)、绒球、小脑蚓体和前庭核(VN))由全连接层(FC)建模,
%以产生注意力 $a_{t+1}$,外观 $\alpha_{t +1}$,以及眼位校正信号 $\Delta p$ 和眼球运动信号 ($\Delta p_t$)。
%甚至最后通过动眼神经生成眼球运动信号 ($\Delta p_t$) 。
所有黑色箭头表示信息在同一个时间步内流动,而灰色箭头代表不同时间步之间的连接。
下面详细介绍类脑跟踪架构的每个模块。

\begin{figure*}
	\centering
	\includegraphics[width=5.4in]{figures/C2Fig/pipeline.pdf}
	\caption{
		类脑视觉目标跟踪的网络架构
	}
	\label{fig:c2:pipeline}
\end{figure*}

\subsubsection{视网膜和背侧/腹侧通路}

视网膜上的空间选择器为当前帧 $F_t$ 选择中央凹视图 $f_t$,它的输出被输入到两条相互连接的腹侧/背侧通路。
腹侧通路识别被跟踪的目标,而背侧通路学习被跟踪目标的运动特征。
% 引入非局部注意力机制。(自上而下的注意力)
腹侧/背侧通路中的外观注意力由当前帧自下而上的中心凹视图信息 $f_t$ 和前一帧自上而下的外观注意力信息 $\alpha_{t+1}$ 所驱动。
然而,中央凹的空间注意力只依赖于空间注意力信息 $a_{t+1}$,
在这种情况下,自下而上的信息仅具有局部影响并依赖于某个位置的显著性输入,
但自上而下的信息将全局特征结合到局部分析中~\cite{attention}。
%


%背侧通路中的神经元将显著目标 $d_t$ 和中央凹视图中的背景进行分割。
%使用精调后的表征 $m_t$ 计算工作内存 $h_t$。


(1) \textbf{视网膜}: 
在类脑模型的处理流程中,根据空间注意力机制~\cite{ratm} 进行建模。
对于输入帧 $F_t \in R^{W \times H}$ 形成矩阵 $M_t^x \in R^{w \times W}$ 和 $M_t^y \in R^{h \times H}$,
矩阵中的每一行都包含一个高斯分布。
高斯分布的位置和宽度决定了输入帧的哪些部分被选为视网膜中的中央凹视图 $f_t$。
因此,中央凹视图 $f_t \in R^{h \times w}$ 可以表示为:
\begin{equation}
f_t = M_t^y F_t (M_t^x)^T \mbox{,}
\end{equation}
用连续矩阵行的分布中心、步长和方差来表示注意力。
与循环注意力跟踪类似~\cite{hart},只有步长和分布中心是通过 LSTM 预测的,而方差依赖于步长。
该操作在估计较小的方差时避免了过度的偏差,有助于提高学习的速度。
此外,中央凹 $f_t$ 的大小依赖于具体的实验分析。


(2) \textbf{腹侧通路}: 
%初级视觉皮层 V1 和腹侧通路学习外观表征 $v_t$,
腹侧通路(包括初级视觉皮层 V1)将中央凹视图 $f_t$ 转换为固定维度的外观表征向量 $v_t$,其中包括被跟踪目标的空间特征和外观特征,
具体地网络结构依赖于具体的实验分析。
在本章的腹侧通路架构中,使用卷积神经网络实现初级视觉皮层 V1,这是腹侧通路和背侧通路共享的模块。
然而,通常使用若干个卷积层和最大池化层来实现模仿人类初级视觉皮层 V1~\cite{theoretical_neuroscience}。
%这些层模仿人类的 V1~\cite{theoretical_neuroscience},并与背侧通路共享。
在这以后,处理流程分为背侧通路和腹侧通路。
最后,通过卷积神经网络提取被跟踪目标的特征 $v_t$,以实现腹侧通路功能。

(3) \textbf{背侧通路}: 
背侧通路(包括 V1 和 MT/MST)用于提取运动特征,将显著目标 $d_t$ 和中央凹视图中的背景进行分割。
%并计算中央凹视图的前景分割 $d_t$。
使用动态滤波网络~\cite{brabandere2016dynamic} 建模背侧通路(MT/MST),用于处理中央凹视图前景和背景之间的空间关系。
%背侧通路(包括 V1 和 MT/MST)中的神经元将显著目标 $d_t$ 和中央凹视图中的背景进行分割。
%使用精调后的表征 $m_t$ 计算工作内存 $h_t$。
FC$(\cdot)$ 表示全连接层,
所以 MT/MST 中基于外观表示 $\alpha_t$ 动态预测卷积滤波器 $\phi_t$ 为:
\begin{equation}
\left\{ \phi _t ^i \right\}_{i=1}^N = \text{FC}(\alpha_t) \mbox{,}
\end{equation}
V1 输出的的特征经过具有 $N$ 个卷积层的非线性滤波器,
然后,将其输出传递给卷积核大小为 $1 \times 1$ 的卷积和 Sigmoid 激活函数,将特征转换为二维掩码 $d_t$。
$d_t$ 中的每一点都表示被所跟踪目标占据的概率。

背侧通路的位置图是基于腹侧通路抽取的目标表示,这模拟了视觉系统中的噪声抑制机制。
因为在当前帧没有跟踪目标的表征时,LSTM 中的目标表征不会被覆盖,所以使用这种机制可以处理遮挡和漂移的情况。
因此,腹侧和背侧通路的输出可以被整合为:
\begin{equation}
m_t = \text{FC}(conc(v_t \odot d_t)) \mbox{,}
\end{equation}
其中 $\odot$ 表示通过特征掩码执行显著目标提取的哈达玛积,
$conc$ 表示将矩阵的所有行连接成向量的连接运算符。


\subsubsection{额叶视区}
所提出的类脑跟踪方法性能取决于估计下一帧中目标外观和位置的能力。
因此,它在很大程度上依赖于跟踪目标状态的预测。
而 LSTM 可以利用时空和外观特征,使所提出的模型能够处理遮挡和漂移的情况,例如跟踪目标被其他干扰物遮挡的情况。

如图~\ref{fig:c2:pipeline} 所示,将 LSTM 模块命名为 FEF,利用输出误差来预测眼球运动。
在平滑跟踪中,FEF 区域接受来自腹侧/背侧流的输出 $m_t$,
精调后的目标表征 $m_t$ 用于更新 FEF 中的隐藏状态 $h_t$,
并在 FEF 中存在类似的预测动作~\cite{b11,b13,b14}。
当模型训练过程中,被跟踪目标和眼球运动之间的延迟减少时,输出误差趋向于零。
因此,LSTM 需要即使在没有图像输入的情况下依然能够推断眼球的运动。
\begin{equation} \label{equ:c2:LSTM}
h_t, o_t = \text{LSTM}(h_{t-1}, m_t) \mbox{。}
\end{equation}


\subsubsection{脑干和小脑}
脑干和小脑一起(包括脑桥核、绒球、小脑蚓体和前庭核)使用全连接层进行建模,
%以产生注意力 $a_{t+1}$,外观 $\alpha_{t +1}$,以及眼位校正信号 $\Delta p$,
%甚至最后通过动眼神经生成眼球运动信号 ($\Delta p_t$) 。
然后利用背侧流的输出 $d_t$ 和额叶视区的输出 $o_t$ 来估计注意力 $a_{t+1}$ 和外观 $\alpha_{t+1}$,
最后通过动眼神经生成眼球运动信号 $\Delta p_t$。
\begin{equation} \label{equ:c2:FC}
\Delta p_t, \Delta a_{t+1}, \alpha_{t+1} = \text{FC}(conc(d_t), o_t) \mbox{,}
\end{equation}
\begin{equation} \label{equ:c2:attention}
a_{t+1} = a_t + tanh(c) \Delta a_{t+1} \mbox{,}
\end{equation}
其中 $c$ 是一个可训练参数,用较小的值进行初始化以限制训练期间模型更新的大小。
方程~\ref{equ:c2:LSTM} 到 ~\ref{equ:c2:attention} 描述信息更新的过程。
通过累积注意力变化来计算眼睛注视的位置,
如章节~\ref{sec:loss} 所示,学习空间注意力来估计下一帧中被跟踪目标的位置,并预测在时间 $t$ 相对于注意力的注视位置 $p_t$。
\begin{equation}
p_t = a_t + \Delta p_t \mbox{。}
\end{equation}

此外,参考脑干和小脑都可以利用反向动力学控制器进行建模的发现和想法~\cite{b9,purkinje_IDC},
基于动力学的平滑跟踪模型~\cite{b9},这里假设反向动力学控制器是完美的,因此眼球的运动信号可以表示为:
\begin{equation}
\Delta p_t = \Delta p \mbox{,}
\end{equation}
其中 $\Delta p$ 是眼球运动的低通滤波器。
基于这个假设,在该研究中不需要实现反向动力学控制器。



\subsection{损失函数} \label{sec:loss}
可以通过优化一组损失来训练所提出的 BTN,包括跟踪损失、背侧/腹侧流损失和辅助损失。
BTN 损失 $L_{b}$ 由下式给出:
\begin{equation}
L_{b} = L_t + L_d + L_v + L_a \mbox{。}
\end{equation}
BTN 损失的详细信息如下所述。

\subsubsection{跟踪损失}
为了达到视觉跟踪的目的,即定位目标在当前帧的位置,损失函数第一项采用预测的边界框和真实边界框之间的交并比。
由于交并比对目标和图片尺寸具有不变性,用它来衡量定位跟踪目标的准确性是比较适合的。
虽然交并比不对应任何概率分布并且不能被归一化,但它经常用于跟踪性能的评估~\cite{vot2016}。
同时,使用交并比的交叉熵损失函数代替传统的 L2 损失函数是由于目标定位使用 L2 损失的一个缺点是会使模型在训练过程中更偏向于尺寸更大的物体,因为大目标的 L2 损失更容易大于小目标。
所以在这里根据 UnitBox~\cite{unitbox} 将跟踪损失 $ L_t $ 表示成交并比的负对数:
\begin{equation}
L_t = -\log(\mbox{IoU} ( \frac{p_t \cap \hat{p}_t}{p_t \cup \hat{p}_t} ))\mbox{。}
\end{equation}

\subsubsection{背侧流损失}
使用背侧流中的空间注意力机制从视频帧中挑选出被跟踪的目标。
为了估计该模块的参数,跟踪系统必须预测目标的运动,而这个预测过程很难恢复原状。
使用两项背侧流损失来保证随着关注目标区域的减少,跟踪的精度会上升。
背侧流损失 $ L_d $ 的第一项限制了预测的注意力区域尽可能地覆盖目标的边界框,
而第二项防止注意力区域变得太大(最大为整个输入帧的大小),这里的对数操作是为了适当的裁剪以避免数值不稳定。
\begin{equation}
L_d = -\log (\frac{a_t \cap p_t}{p_t}) - \log (1 - \frac{a_t \cap F_t}{a_t \cup F_t})\mbox{。}
\end{equation}

\subsubsection{腹侧流损失}
当被跟踪目标(比如一个特定的行人)是动态开放环境中时,设置腹侧流中外观注意力的目的是为了抑制干扰。
为了达到这个目的,在外观注意力上设置一个损失函数以选出被跟踪的目标。
给定注意力区域 $a$ 和边界框 $p$,% (x,y,w,h)
记 $r(a_t, p_t): R^4 \times R^4 \rightarrow \left\{0,1\right\} ^{h_v \times w_v}$ 为目标函数,
输出一个和 V1 的输出一样大的二进制掩膜。
该掩膜就叫做显著目标区域 $g$,其中边界框和显著目标重叠的位置值为 1,其他地方为 0,也就是只保留下边界框中包含目标的像素。
%如果将交叉熵记作 $H(p, q) = -\sum_{z} p(z)\log q(z)$,那么
腹侧流损失函数可以表示为显著目标区域的预测值和前背景分割的真实值之间的交叉熵:
\begin{equation}
L_v = - r(a_t, p_t) \log(d_t)\mbox{。}
\end{equation}

%\subsubsection{正则化项}
%对模型的参数 $\theta$ 和动态参数的期望值 $\phi_t(\alpha_t)$ 应用 L2 正则化,表示为:
%\begin{equation}
%L_u = 
%\frac{1}{2} \left\vert \left\vert \theta \right\vert \right\vert _2 ^2 + 
%\frac{1}{2} \left\vert \left\vert \phi_t \right\vert \right\vert _2 ^2 .
%\end{equation}

\subsubsection{辅助损失}
对模型的参数 $\theta$ 和动态参数的期望值 $\phi_t(\alpha_t)$ 应用 L2 正则化。
同时为了避免超参数调优,参考多任务学习~\cite{multitask}来学习各个损失的权重 $\lambda$。
使用全为 1 的向量初始化权重后,给损失函数添加正则化项,则辅助损失 $L_a$ 可以表示为:
\begin{equation}
L_a = 
\frac{1}{2} \left\vert \left\vert \theta \right\vert \right\vert _2 ^2 + 
\frac{1}{2} \left\vert \left\vert \phi_t \right\vert \right\vert _2 ^2
- \sum_{i} \log (\lambda_i^{-1})\mbox{。}
\end{equation}




\section{大脑数据分析}
为了比较 DNN 的视觉目标跟踪和人脑平滑跟踪的相似性,首先需要从各种眼球运动中提取平滑跟踪,然后选择相应的大脑区域和激活用于相似性计算。

\subsection{视频刺激的运动估计}
由于平滑跟踪行为和视频中目标的移动高度相关,首先使用计算机视觉方法对所有视频帧的运动进行估计。
尽管目前已存在许多精度高的运动估计算法,但仍然会产生额外的噪声输出,所以这里使用两种不同的算法来增强运动估计的鲁棒性。
% 孔径问题:从小孔中观察一块移动的黑色幕布观察不到任何变化。但实际情况是幕布一直在移动中。(人类的视觉系统在局部观察时有孔径问题)
% 解释:https://blog.csdn.net/dengheCSDN/article/details/81747063
% 根据结构张量能区分图像的平坦区域、边缘区域与角点区域(https://www.cnblogs.com/revere7/p/9705956.html)
第一种算法是仅对孔径问题不敏感的点(比如角点)进行运动估计,以提供稀疏光流场并估计运动~\cite{structure_tensor}。
首先对输入视频做采样因子为 2 的空间二次抽样,然后创建一个拥有五个空间层和两个时间层时空高斯金字塔。
对于这些多尺度表征的每一层,计算每个像素的运动速度。
对速度的估计值相对于原始视频分辨率进行正则化,并以类似于金字塔合成的方式合并到处理流程当中~\cite{pyramid_synthesis},并对过高的速度值使用对应速度的百分之九十进行裁剪。
第二个算法使用基于光流计算的插值方法~\cite{epicflow}。
该算法首先使用边缘保持插值的稠密匹配,然后进行能量最小化操作。
%一个EpicFlow算法估计运动的例子如图~\ref{fig:gaze_traces}b所示。
最后使用这两种算法,计算出每个视频帧像素偏移的平均长度。

%\begin{figure*}[ht]
%	\centering
%	% 导入.pdf垂直显示?
%	\includegraphics[width=6in]{./figures/C2Fig/gaz_traces.jpg}
%	\vspace{0.2em}
%	\caption{平滑跟踪存在性演示和计算。
%		(a)对于受试位于核磁共振扫描仪中观看 StudyForrest 数据集,叠加凝视轨迹的样例帧(超过400ms的时段;每个颜色对应一个受试)。
%		通过被拉长的点云可以证明平滑跟踪是存在的。
%		(b)使用EpicFlow算法计算光流。
%		估计的运动(黑线)和视频中实际的运动对应得很好。
%		黑线表示“多观察者平滑跟踪(MOSP)”算法的输出(比如在400ms的时间窗口中自动检测平滑跟踪片段)。
%		} 
%	\label{fig:gaze_traces}
%\end{figure*}


\subsection{平稳跟踪样本的提取}
实验中每个凝视样本都包含一个四元组:时间、显示器坐标系下的横坐标 $x$ 和纵坐标 $y$、眼跟踪质量的置信度估计。
因为数据集使用的是单眼跟踪,所以置信度为 1 表示好的眼跟踪,而 0 表示跟踪丢失。
经数据检查后发现在所有受试中跟踪丢失的比例一般在 $1.2\%$ 和 $16.7\%$ 之间,而受试 5 和受试 20 跟踪丢失特别明显,丢失比例分别是 $86.7\%$ 和 $39.0\%$,所以将他们的数据从分析中剔除。
%
对余下的凝视样本利用一种比较有代表性的基于密度的聚类算法 MOSP~\cite{mosp,mosp_imp} 进行分类以得到平稳跟踪样本。
该算法在手动标注的公共数据集上和几个最新的眼动检测算法~\cite{fix_sp_det,remodnav} 相比有很好的分类性能。
MOSP 算法的另一个优势是在程序中使用了简单且容易理解的门限值,这样保证在使用过程中比较容易进行调整,而不像一些深度学习方法~\cite{auto_classification,gazenet} 的超参数调整困难。
这一步 MOSP 取得较高的平滑跟踪检测精度(比如较低的误检)对后续的分析非常重要。

MOSP 算法包括两个步骤,第一步负责眼跳检测,先将眼跳剔除,第二步负责区分注视和平滑跟踪。
眼跳检测算法~\cite{var_natural} 使用一个高速度门限值和一个低速度门限值。
眼跳检测器使用高速门限值进行初始化,然后往门限值两端进行扩展,直到小于低速门限值。
使用大于输入样本之间噪声的高速门限值使算法对噪声更加鲁棒。
MOSP 算法的第二步是进一步处理眼跳间的间隔,通过给几乎可以确定是注视的样本赋予一个注视的标签。
余下的样本就标记为平滑跟踪的候选结果,并在所有受试中进行处理。
然后,只有当凝视样本密度大于某个门限值时才使用基于密度的聚类算法 DBSCAN 进行聚类。
基于密度的聚类算法能很好地检测出平滑跟踪是因为平滑跟踪的的两个属性。
% 
首先,只有在给定时间内刺激出现运动时,平滑跟踪才会出现;并且视频刺激中每一帧移动的目标数目较少。
%
第二个特点是移动的目标能吸引受试的注意力(特别是在电影中),并且它们经常会被超过一个受试所跟踪。
%MOSP算法输出和聚类的示例如图~\ref{fig:gaze_traces}所示。
平滑跟踪的这两个属性使算法能够把真实的平滑跟踪从漂移和类平滑跟踪中区分开来,这对于 fMRI 实验中的平稳跟踪样本的提取特别重要。
%(因为他们比实验室记录有更高的噪声)。
即使一些凝视样本被错误的标记成候选平滑跟踪,只要在当前时间的相同区域其他受试没有相同的模式,那么它也不会被标记成平滑跟踪,通过这些机制提高算法的鲁棒性对识别与处理平滑跟踪相关的大脑区域特别重要。
%然而当没有足够受试跟踪一个目标时,可能会有丢失一些平滑跟踪的缺点(即降低对平滑跟踪的敏感性),增长的特异性对识别与处理平滑跟踪相关的脑区特别重要。

由于原始的 MOSP 算法是针对 GazeCom 数据集~\cite{var_natural} 进行设计和优化的,在本实验中对一些参数与进行了调整。
%(参见在线数据的完整详细信息)。
因为增加凝视轨迹的数目会提高平滑跟踪检测算法的精度,所以同时使用了不在核磁共振扫描仪里的记录(获取难度较小,且相比于扫描仪记录有更少的噪声)和在扫描仪里的记录。
% 观测角中每一度应该包含大约相同的像素
尽管刺激大小不同,对不在扫描仪里的记录使用单位观测角中所包含的像素数来进行按比例缩放;
所以使用两个数据集进行平滑跟踪片段检测会有更高的置信度。
%(2秒间隔下共有的平滑跟踪为0.84的$r^2$)。


\subsection{核磁共振数据分析}
本实验使用 SPM12 工具进行核磁共振数据的分析。
首先对每个 fMRI 记录利用标准的处理流程进行预处理~\cite{mri_analysis}。
包括对每个会话的平均图像进行功能性数据对齐(由于公开数据集已经做了时间校正,在这里就没有使用间时间校正),并将对齐后的数据与大脑解剖的 T1 扫描进行配准,
%\footnote{使用头动校正处理不同扫描之间体素对应关系的不一致,导致血液动力学响应被头动引起的信号所淹没。},
然后将其正则化到 MNI 模板,再重采样到 $3 \times 3 \times 3$立方毫米的体素中。
最终,在脉冲的半峰全宽上使用 8 毫米高斯核进行平滑。

在 StudyForrest 数据集的记录中,将整部电影刺激分割成 8 个不同的视频片段,每个大约 15 分钟,在扫描仪中分别展示给每个受试。
在第一阶段的个体分析中,
%(个体内的分析,8 个电影片段之间的分析),
为了建模整个《阿甘正传》电影,将 8 个记录会话组合到一个设计矩阵中,每一行表示一个电影片段。
在设计矩阵的每个会话中,从同一受试观察不同的电影片段的凝视中回归出一个平滑跟踪回归变量、一个眼跳回归变量和一个电影运动回归变量。
为了考虑受试之间血液动力学响应开始时刻和宽度的变化,沿着时间和离散度梯度方向使用标准血液动力学响应函数(Hemodynamic Response Function,HRF)。
除了以上三个回归变量,还使用了从预处理重对齐步骤中得到的六个头部运动分量\footnote{3个平移分量($x$、$y$、$z$)和3个旋转分量} 作为用于校正的干扰因素回归变量。

为了和扫描仪重复时间相一致,眼动和运动回归变量建模成有 2 秒间隔的事件序列,因此每个事件都被表示成两个扫描之间变化的回归变量。
% 幅值为跟踪点在fMRI上的影响范围?
% 幅值从0增加到1(相当于减了一个平均值?)
每个事件的幅值都根据 2 秒时间窗口内对应的眼动或运动数目进行调整,当和整体均值相同时对应的值为 0。
%,并线性增长到最大值1。
回归建模的详细描述在下一节~\ref{sec:regressor_modelling} 中给出。
%随着回归变量如何建模变得明显,
在回归变量建模中,如果没有创建两个之间的强烈的正相关或者负相关,就不可能同时建模注视和平滑跟踪。
为了使注视和平滑跟踪相互依赖更加清晰,可以考虑受试开始跟踪一个目标。
然后随着注视幅值的减少,平滑跟踪回归变量的幅值就会成比例增加。

对每个受试数据独立地应用广义线性模型后,使用每个回归变量血液动力学响应函数的幅值部分(为了计算和平滑跟踪相关的感兴趣部分进行对比,该处理过程横跨了对应 8 个电影片段的 8 个记录会话)。
这些对比包括眼动和运动主要响应的对比、平滑跟踪和眼跳之间的对比,以及眼动和运动的对比。
最后,在第二阶段受试之间核磁共振分析中,对 13 个有效受试的第一阶段对比结果进行单样本 $t$ 检验。
%聚类结果(使用$p$值小于0.001的初始门值校正情况下,$p$值小于0.05的多重比较谬误)覆盖在三维大脑模型上,并在实验和结果章节~\ref{sec:c2:results} 进行展示。

\subsection{回归变量建模} \label{sec:regressor_modelling}
如上节所述,回归变量并不是对每个眼动事件单独地进行建模,而是把它置于 2 秒时间间隔内,根据所在时间窗口中每种眼动的数目进行调制。
在该实验中将电影本身运动考虑进来,在连续 2 秒的时间窗口内通过电影运动的均值来进行调制。
%
具体来说,将眼动调制参数量级的计算加以考虑,它有三个主要影响因素:
% 不同刺激,特定功能(平滑跟踪)在同一受试之间激活的平均效果
% 为了得到实时的sp响应,不进行不同刺激的回归?
\begin{enumerate}
	\item 捕获不同刺激之间眼动的差异,并通过每个受试的每种眼动平均百分比表现出来,这等同于不同观察行为的平均。
	
	% 不同眼动类型 方差不同
	% (方差越大,眼动调制强度越小)
	\item 捕获分布上的差异和不同眼动之间的方差,这是一个常数,且眼动调制强度和方差成反比。
	为了保证有 $95\%$ 的调制值在 1 以下,调制因子从数据中进行选择。
%	(如图~\ref{fig:sp_ratio}所示)。
	% $m_{sac}$
	由于与眼跳相关的不同输入刺激拥有较小的方差,因此将眼跳的调制值设置为 1.5。
	% 拉近两者的分布
	% $m_{sp}$
	为了反映平滑跟踪时眼睛运动中较大的方差,将平滑跟踪的调制值设置为 5 ,当没有移动目标时候,平滑跟踪不会出现,而当存在一个显著移动的目标时,平滑跟踪能够执行较长一段的时间。
	
	%% 不同的受试(二层次分析)
	\item 捕获不同受试之间的变化。
	特定受试的基本要素是基于每个眼动的观察,如果受试之间眼动存在较大不同,可能会直接或间接导致所得到的大脑连接性的不同~\cite{mueller2013individual,vanderwal2017individual}。
\end{enumerate}
%(1)第一个因素是捕获不同自然刺激之间眼动的差异,并通过每个受试的每种眼动平均百分比表现出来,这等同于不同观察行为的平均。
%(2)第二个因素是捕获分布上的差异和不同眼动之间的方差,它们是一个常数,眼动调制强度和方差成反比(方差越大,眼动调制强度越小)。
%为了保证有$95\%$的调制值在1以下,因子从数据中进行选择(如图~\ref{fig:sp_ratio}所示)。
%由于与眼跳相关的不同输入刺激拥有较小的方差,因此眼跳的调制值$modulation_{sac}$设置为1.5。
%% 拉近两者的分布
%为了反映平滑跟踪眼睛运动中较大的方差,将平滑跟踪的调制值$modulation_{sp}$设置为5,当没有移动目标时候,平滑跟踪不会出现,但是当一个显著移动目标存在时,平滑跟踪能够执行较长一段的时间。
%% 不同的受试(二层次分析)
%(3)第三个因素是捕获不同受试之间的变化。
%特定受试的要素是基于每个眼动的普遍性观察,受者试之间的不同,可能会直接或间接导致大脑连接性的不同\cite{mueller2013individual,vanderwal2017individual}。
%在 StudyForrest 数据集的案例中,不同受试之间眼跳的变化从$5.8\%$到$12.4\%$,平滑跟踪的变化从$11.5\%$到$19.3\%$。
%因此,如果使用总平均值,一些受试的相关激活会被抑制,一些又会被放大。

在 StudyForrest 数据集中,不同受试之间眼跳所占的比例从 $5.8\%$ 到 $12.4\%$,平滑跟踪所占的比例从 $11.5\%$ 到 $19.3\%$。
因此,如果使用总平均值,一些受试的相关激活会被抑制,一些又会被放大。
%比如对于一个假想的受试,他在全部实验中,平滑跟踪所占的百分比为 $overall_{sp}=15\%$,并在给定的片段有 $clip_{sp}=10\%$ 不执行平滑跟踪。
%现在在一个2秒时间窗口中 $window_{sp}=85\%$ 的间隔被标记为平滑跟踪,
%% 实际执行平滑跟踪的比例为:85%-10%
%这时候调制强度
%\begin{equation} \label{equ:modulation_sp}
%modulation_{sp} = \frac{window_{sp}-clip_{sp}}{m_{sp} * overall_{sp}} = \frac{85\%-10\%}{5*15\%} = 1
%\end{equation}
%根据公式~\ref{equ:modulation_sp} 
在数据集中所有 2 秒间隔内计算调制参数后,发现平滑跟踪回归变量和眼跳回归变量之间的皮尔逊相关系数 $r=0.02$,这很好的表明了两者之间不相关且没有共享的可变性。

同理对于运动估计调制参数遵循和眼动参数建模相似的流程。
并且,这里对于每个刺激通过平均运动的内容来捕获他们的稳定状态。
在所有刺激片段中,结果值根据运动值的 90\% 进行正则化,并将结果的最大值限制到 1,以较少离群值的影响。


% 表明眼跳比平滑跟踪一些
%\begin{figure*}[ht]
%	\centering
%	\includegraphics[width=6in]{./figures/C2Fig/sp_ratio.jpg}
%	\vspace{0.2em}
%	\caption{当进行事件相关的个体层次分析时,在2秒时间窗口内检测出来的眼跳(左)比例的概率分布和平滑跟踪(右)比例的概率分布,正则化的目的是为了使每个受试的均值为1。
%	更陡峭的分布表示不同受试和时间之间有更高的可变性。
%	眼跳的比例(红色)有更低的可变性,并且是以1为中心。
%	平滑跟踪(蓝色)有更大的可变性,其峰值接近于0表示有较少的平滑跟踪(比如场景中没有平滑跟踪的目标)。
%	} 
%	\label{fig:sp_ratio}
%\end{figure*}


%\subsection{额外用于验证的回归变量}
除了平滑跟踪和眼跳的回归变量,本实验还使用了运动回归变量用于建模视频中的全局运动,
并进一步探索了两种变化,
第一个是运动建模的变化,通过凝视位置周围的窗口来获得运动的局部估计,
第二个是在相同的窗口为了和视网膜上运动相似,从平均上下文速度中抽取平滑跟踪的速度。
%由于局部运动的结果低于全局运动水平,所以我们没有在结果章节中说明,而是在后面讨论部分讨论它。
为了进一步理解是什么驱动了眼睛运动,也将场景动态开放性和边缘密度估计的模型作为额外的回归变量,这些被调制值等同于之前描述的运动回归变量。
使用标准显著性模型来计算场景动态开放性,作为每一帧的显著性熵~\cite{itti1998a}。
与图像显著性熵类似,边缘密度在拉普拉斯金字塔的第三层作为每一帧绝对像素值的熵,它表示每度大约 3-6 个循环空间频率范围中的边缘,接近人类对比敏感函数的峰值。
%同时这些结果在后面都会讨论。



% ref: Brain-Score: Which Artificial Neural Network for Object Recognition is most Brain-Like? -> Bran Benchmarks
\section{评价指标:类脑跟踪分数}
这一部分介绍了衡量深度神经网络模型与人类大脑皮层之间的相似性的类脑跟踪分数 BTS。
BTS 是在特定实验数据集上测试的指标,包括眼动行为预测性和皮层神经预测性两个指标。

为了获得关于大脑相似性的量化指标,参考开源平台 Brain-score~\cite{SchrimpfKubilius2018BrainScore} 并提出了 BTS,这是一个设计良好的用于评估跟踪模型类脑预测能力的指标:
(a) 对于 StudyForrest 数据集的每个输入视频帧,预测人眼跟踪目标时的平均人眼运动~\cite{gaze_forrest};
(b) 在 StudyForrest 数据集~\cite{gaze_forrest} 上预测人类视觉区域 MT/MST 中大脑皮层位置对每个输入视频帧的平均激活响应。
%所有眼动的娇娇
为了在统一指标上评估 BTN,该指标计算了眼动行为预测性指标和皮层神经皮层预测性指标的平均值。

\subsection{眼动行为预测性}
眼动行为度量的目的是衡量对于特定任务深度神经网络输出与人眼行为输出之间的相似性~\cite{rajalingham2018large}。
在人眼目标跟踪中,被试者的注意力是一个与人瞳孔有关的圆形范围。
因此,将模型呈现的行为模式(眼睛注视的位置和瞳孔的大小)建模为圆形,并且它不同于视觉目标跟踪中的长方形边界框。
同时,由于该工作不仅仅是为了提高计算机的跟踪性能~\cite{schrimpf2020integrative},更主要目的是实现类人的智能;
如果 BTN 获得了更好的行为相似性,就可以很好地预测眼睛注视的位置和范围。
否则会导致深度神经网络获得了完美的边界框拟合,却无法获得良好的人眼行为预测效果。

如表~\ref{fig:c2:introduction} 所示,使用的 12 个视频序列是在自然背景下有一个显著的目标在移动,每个视频序列大约 20 秒,并利用眼动仪记录下受试的注视范围来表示跟踪显著目标,
并使用 12 个图像序列中的 149 张图像的受试眼球运动和深度神经网络预测来分析和评估所提出的模型的眼动行为预测性能。
总共的 149 帧图像中,每一帧都作为深度神经网络的输入,用于预测眼睛的注视范围。
然后通过每一帧中眼睛注意力的范围来衡量这种眼动行为预测性。

目标跟踪深度神经网络的输出结果是所跟踪目标的边界框,而受试的注意力是一个包含中心坐标 $x$、$y$ 和半径的圆形范围。
因此将眼动预测性或行为评分建模为实际眼睛注意的圆形范围 $S_{roi}^a$ 与深度神经网络预测边界框 $S_{roi}^b$ 之间的交并比(Intersection over Union,IoU),
并将所有帧序列中的整体行为指标~$s_b$ 作为眼动预测分数:
\begin{equation}
s_b=\frac{area(S_{roi}^a \cap S_{roi}^b) }  { area(S_{roi}^a \cup S_{roi}^b) } \mbox{。}
\end{equation}


\subsection{皮层神经预测性}
皮层神经预测性是指在源系统中(比如深度神经网络)评估给定输入图像 $X$,预测在目标系统(比如视觉区域 V1 和 MT/MST)中的激活响应。
作为输入,该度量方法的评估过程为:刺激 $\times$ 神经网络 = 激活响应,这里的神经网络可以是深度人工神经网络,也可以是灵长类的大脑皮层。
并且源神经网络(深度人工神经网络)可以使用线性变换映射到目标神经网络(灵长类皮层网络):
\begin{equation}
y = Xw + \epsilon \mbox{,}
\end{equation}
这里 $w$ 表示线性回归的权重,$\epsilon$ 表示皮层记录的噪声。

实验中呈现给 13 个受试的 149 帧图像包括在自然场景中出现的显著跟踪对象,并且记录下 MT/MST 体素中的 2177 个神经响应。
此外还展示了 BTN 中最具预测性的网络层和特定的模型区域。

在本实验中,这些关系被拟合为从深度神经网络到大脑皮层的映射,并用这个映射关系来预测人的大脑皮层对视频帧的响应。
由于大脑的激活响应数据维度远远大于神经网络模型激活数据的维度,所以利用主成分分析(Principal Component Analysis, PCA)~\cite{2002Principal} 将大脑激活维度压缩到指定的维度来进行程序运行加速和相关性分析,
并利用来自 MT/MST 的激活来进行拟合。 
最终使用皮尔逊相关系数 $s_r$ 来衡量最终人工神经网络模型和视觉运动皮层的神经相似性分数,如下所示:
\begin{equation}
s_r=\frac{\sum_{i=1}^{n} (y_i-\bar{y}) (y_i^\prime - \bar{y}^\prime) }{\sqrt{\sum_{i=1}^{n} (y_i - \bar{y})^2 (y_i^\prime - \bar{y}^\prime)^2 }}\mbox{,}
\end{equation}
其中 $y^\prime$ 是人工神经网络模型特定层的激活,$y$ 是人类大脑皮层特定区域的激活,$n$ 是深度神经网络中对应的特征维度,$\bar{y}$ 和 $\bar{y}^\prime$ 是所有神经响应值的中位数(使用中位数是因为响应通常是非正态分布)。

\subsection{总分数}
为了全面衡量 BTN 的类脑性能,同时考虑了 IoU 行为度量和 MT/MST 皮层度量。
下面给出的 BTS $s_{t}$ 是两个分数的平均值 :
\begin{equation} \label{equ:score_btn}
s_{t} = \frac{s_b + s_r}{2}\mbox{。}
\end{equation}
考虑到归一化可能会惩罚方差较小的分数,而应该平等对待两个分数对 BTS 的重要性程度,因此 BTS 的设计没有在各个分数大小上进行归一化。

%映射步骤在多个训练-测试图像刺激集的划分中执行。
%每次运行时,使用训练图片学习的权重将源神经网络的输出映射到目标神经网络的响应,然后使用这些权重来与剩余图像的响应$y'$。
%我们分别使用MT和FEF的神经元响应来计算映射权重的拟合。
%为了获得每个神经网络的神经预测性,将预测的响应$y'$和测量的皮层神经网络响应$y$进行比较来计算皮尔逊相关系数$r$(评估两个连续变量之间的线性关系):
%\begin{eqnarray}
%r = \frac{\sum_{i=1}^{n}(y_i^\prime - \bar{y})(y_i^\prime - \bar{y})}
%{\sqrt{\sum_{i=1}^{n} (y_i - \bar{y})^2 (y_i^\prime - \bar{y}^\prime)^2}}
%\end{eqnarray}
% study: https://blog.csdn.net/qq_30081043/article/details/107154233

%使用所有单个神经网络预测值的中位数(比如在目标脑区所有测量的目标位置)来获得训练-测试集切分的预测性分数(使用中位数是因为响应通常是非正规分布的)。
%最终目标脑区的神经预测性分数为所有训练-测试集切分的平均值。
%
%我们进一步通过将相同图像的重复表示划(一个是深度人工神经网络的响应,一个是皮层神经网络的响应)分为两半,来估计神经响应之间的内部一致性,并计算每个神经网络两个划分之间的Spearman-Brown-corrected的皮尔逊相关系数。

%在实际实现中,我们发现对于源系统(深度人工神经网络)中的高维度,标准的线性回归方法相对较慢,并且鲁棒性不够。
%因此我们使用25个部件的偏最小二乘(PLS)回归\cite{performance_optimized}。
%我们通过先将源特征(深度人工神经网络的特征)使用主成分分析(PCA)投影到更低维度的空间。
%投影矩阵是图片集一个选择的特征中获得,所以在训练-测试集切分中该投影是不变的。
%然后投影矩阵用于转换源神经网络输出的特征。
%结果显示对1000个验证图片特征响应的每一层,从特征响应中获得了1000个主成分。
%该1000个主成分就捕获了源模型的最大的方差信息。

%\subsection{fMRI记录}
%目前使用的皮层响应数据记录是观察电影《阿甘正传》时的核磁共振数据\cite{gaze_forrest}。


\section{实验与结果分析} \label{sec:c2:results}
在这一节通过以下四个步骤说明所设计的 BTN 实验和其有效性。
首先,介绍了实验相关数据集和模型的具体实现。
其次,详细分析了眼动数据和对应的核磁共振数据。
然后,说明了 BTN 是一种有效的类脑跟踪模型。
最后,讨论了模型设计机制以及深度神经网络与神经科学的关系。


\vspace{0.6em}
\begin{table}[htbp]\wuhao
	\centering
	\caption{原始电影的分割和合并
	}
	\vspace{0.3em} 
	\begin{tabular} {c|cccc}
		%		{p{1.7cm}<{\centering}p{1.7cm}<{\centering}p{1.7cm}<{\centering}p{1.7cm}<{\centering}p{1.7cm}<{\centering}}
		%		\toprule[1.5pt]
		%		\hline
		\hline
		片段  & 开始时间  & 结束时间   & 开始帧 & 结束帧 \\ 
		\hline
		0   &00:00:00.00 &00:21:32:12 &0 &32,312  \\
		1   &00:24:13.24 &00:38:31.23 &36,349 &57,798   \\
		2   &00:38:58.20 &00:59:19.22 &58,470 &85,997   \\
		3   &00:59:31.17 &01:18:14.00 &89,293 &117,351   \\
		4   &01:20:24.16 &01:34:18.06 &120,616 &141,457   \\
		5   &01:37:14.19 &01:41:30.19 &145,869 &152,369   \\
		6   &01:42:49.19 &02:09:51.17 &154,244 &194,792   \\
		%		\hline
		\hline
	\end{tabular}
	\label{tab:movie_seg}
\end{table}


\subsection{数据集}
本研究所提出的 BTN 首先在具有挑战性 KITTI 数据集~\cite{kitti} 上进行训练。
该数据集包括 21 个视频帧序列,包括各种可能的干扰项。
此外,将这些序列中的 80$\%$ 作为训练数据集,20$\%$ 作为验证数据集。

为了测试和分析所提出的类脑模型,将公共数据集 Studyforrest ~\cite{gaze_forrest} 作为动态开放场景的代理。
在具体实验数据集中,使用 StudyForrest 来构建所使用的跟踪数据集~\cite{gaze_forrest}。
%\subsection{数据集}
从 StudyForrest 数据集所用的电影中抽取 12 个短视频(每个大约 20 秒)用于衡量计算机视觉跟踪和生物平滑跟踪的相似性。
为了和 fMRI 数据的采集频率保持一致,如表~\ref{tab:track_seg} 所示,所用的视频帧也采样间隔为 2 秒。
用手动的标注的方式提供人眼跟踪跟踪目标的真实边界框,用于评估计算机类脑视觉跟踪算法的精度。




\vspace{0.6em}
\begin{table}[htbp]\wuhao
	\centering
	\caption{所用到的 12 个视频片段详细信息}
	\vspace{0.3em}
	\begin{tabular} {c|cccc}
%		{p{1.7cm}<{\centering}p{1.7cm}<{\centering}p{1.7cm}<{\centering}p{1.7cm}<{\centering}p{1.7cm}<{\centering}}
%		\toprule[1.5pt]
%		\hline
		\hline
		跟踪段号  & 视频段号  & 开始帧   & 结束帧 & 图片数 \\ 
		\hline
		1     &3 &02:30.00 &02:48.00 &10  \\
		2   &4 &14:50.00 &15:14.00 &13   \\
		3   &5 &01:32.00 &01:48.00 &9   \\
		4   &5 &05:02.00 &05:20.00 &10   \\
		5   &6 &00:08.00 &00:24.00 &9   \\
		6   &6 &06:56.00 &07:10.02 &8   \\
		7   &6 &09:42.00 &09:56.01 &8   \\
		8   &6 &12:26.00 &12:44.00 &12   \\
		9   &7 &06:44.00 &07:06.00 &12   \\
		10  &7 &07:26.00 &08:28.00 &32   \\
		11  &7 &08:30.00 &08:52.00 &12   \\
		12  &7 &10:40.00 &11:10.00 &16   \\
%		\hline
		\hline
	\end{tabular}
	\label{tab:track_seg}
\end{table}

如表~\ref{tab:movie_seg} 中所述,在 StudyForrest 数据集中视频刺激总共分为八段,并对扫描仪中受试进行播放。
%简而言之,使用的数据集有 15 个观看电影的受试同时在扫描仪中记录眼动跟踪和大脑皮层激活。
%另外 15 名受试不在 fMRI 扫描仪中记录他们的眼动追踪,
%并利用不在扫描仪中记录的这些数据来提高平滑跟踪的检测性能。
%电影在带有前反射镜的投影仪上播放给扫描仪内的受试,
%而通过显示器直接显示给实验室内的受试。
%眼动追踪数据由 1000Hz 眼动仪进行采集,
%fMRI 数据是通过 3T 扫描仪获得的,重复时间为 2 秒,并且其体素大小为 $3\times 3\times 3 $ 立方毫米。
基于这些视频和相应的 fMRI 记录,如表~\ref{tab:track_seg} 所示,本实验选择了 12 个片段来测试提出的方法。

\subsection{实现细节}
每个视觉皮层区域都是由若干个深度神经网络执行经典操作来实现的,比如卷积操作和非线性激活。
这些网络模块相当于对应的大脑视觉区域,但具体建模时改变了每个皮层区域的神经元数量。
考虑到计算成本,实现时使用 Python 的 TensorFlow 软件包~\cite{abadi2015tensorflow} 来进行实现,并修改 AlexNet~\cite{imagenet} 中最重要的 3 个卷积模块对初级视觉皮层 V1 进行建模。
原始 AlexNet 的输入大小为 227 $\times$ 227,在经过 3 个卷积模块后缩小到 14 $\times$ 14。
如图~\ref{fig:structure_analysis} 所示,因为低分辨率的特征会导致跟踪效果不佳,所以将初始步长从 4 变为 1,并删除一个池化模块以保留图像的空间特征。
这样得到的特征大小为 $14\times 14\times 384$,输入的中央凹视野大小为 $56\times 56$。
此外在初级视觉皮层 V1 的尾部,有 20$\%$ 的概率随机丢弃特征。
腹侧通路包括一个卷积核大小为 $1\times 1$ 的卷积模块。
DFN 的滤波器大小是 $3\times 3$ 和 $1\times 1$。
使用 100 个 LSTM 单元并进行概率为 0.05 的 zoneout 操作~\cite{zoneout}。
训练 BTN 时除了学习率是 $2 \times 10 ^{-6}$ 外,使用的是和课程学习~\cite{curriculum} 类似的学习配置。
起始序列长度为 5,每经过 12 次训练迭代,序列长度就增加一次。


\subsection{眼动数据分析}

\subsubsection{算法定义运动内容的有效性}
使用运动估计算法定义刺激运动内容的潜在缺点是它们往往会产生噪声结果。 
因此,这里通过使用两种不同的运动估计算法\cite{barth2000the,epicflow} 验证了取得的结果。 
在这两种情况下,确定的脑部激活都是可以比较的,从而强调了所提出方法的有效性。 
在研究的第二个分析中,感兴趣的是专门确定在运动存在下人类中平滑跟踪的驱动因素。
为此,使用平均帧运动作为背景运动的近似。
但是,还有许多其他的运动建模方法,详细地研究了其中两种。 
在第一种方法中,对每个凝视位置周围的观察角为五度窗口中的运动进行建模。 
在第二种方法中,旨在通过对视网膜运动进行建模来对两个回归变量进行解相关。
为此,从同一窗口中的运动速度中减去平滑跟踪的速度(速度和方向),然后在模型中使用所得向量的大小。 
虽然两种方法在定性上都很相似,但其激活程度和强度均较弱。 
这可以部分归因于以下事实:在这两种方法中,运动和平滑跟踪的回归变量之间的相关性都比仅使用平均帧运动时高(窗口 $r=0.21$、窗口-平滑跟踪速度 $r=0.51$、平均帧 $r =0.18$)。 
相关值的变化可以归因于许多因素。
通常,由于使用较小的窗口而不是整个帧的平均值,所以运动估计算法的噪声结果可能变得更加嘈杂。 
而且,所报告的凝视可能很嘈杂,并且通常具有空间偏移,这可能导致在运动计算中完全或部分丢失运动目标。 
结果,平滑跟踪速度不成比例地影响其从窗口运动中减去的结果,因此返回较高的相关值。 
对于非常小的目标,也会出现类似的效果。 
应当注意的是,在扫描仪中从眼动仪报告的凝视位置比在实验室中嘈杂得多:对于 StudyForrest 数据集,25 毫秒凝视数据窗口的中值分散度在扫描仪中为 31 像素,而在实验室中则为 10 像素。


\begin{figure}
	\centering
	\includegraphics[width=.9\linewidth]{figures/C2Fig/structure_analysis.pdf}
	\caption{
		BTN 架构分析
	}
	\label{fig:structure_analysis}
\end{figure}

\subsubsection{眼动统计和分类的有效性}
对于有效的 fMRI 扫描仪内的受试,算法将凝视样本的 $53\%$ 标记为注视,$8.4\%$ 标记为眼跳,$14.8\%$ 标记为平滑跟踪,其他的都被标记为噪声(包括跟踪噪声、眨眼、聚类噪声等)。
由于算法更关注的是将眼跳和平滑跟踪尽可能分离干净,所以相对较高的噪声水平是可以接受的。
注视在受试之间有最大的绝对方差(标准差为 $10.1\%$),这种情况在预期之中,因为注视检测对眼跟踪噪声非常敏感,并且算法的目标并不是建模这种类型的眼动。
眼跳(标准差为 $2.5\%$)和平滑跟踪(标准差为 $3\%$ 有非常低的绝对方差,但是在受试之间有非常高的相对方差。
在第一层次的个体分析中,这种相对较高的受试之间的可变性被特定主题调制因子所捕获。

除了受试之间的可变性,还存在受试内的可变性,它对不同的眼动产生变化。
%在图\ref{fig:sp_ratio}中,我们可视化了每个受试相对相同受试全部均值在2秒时间窗口中眼跳和平滑跟踪的比例概率分布。
由于各种眼动类型之间分布范围不同,为了正则化到他们可比较的范围,需要选择特定眼动调制因子。
这里值 1 表示在给定区间中每个眼动类型的贡献等于全部受试均值。
值为 0 表示各自的眼动在区间中不会出现,大于 1 的值表示有高于平均的出现。
可以得出在受试之间眼跳有更低的可变性,并以均值比例 1 为中心。
另一方面,平滑跟踪的出现有更高的可变性,并且峰值接近于 0,这表明在电影刺激中没有移动的目标,就不会有平滑跟踪。

%\subsubsection{眼动分类的有效性}
用于自动眼动检测的 MOSP 算法已在手动标注数据集中取得了很好的性能\cite{mosp}。 
为了确保 MOSP 算法在 StudyForrest 数据集中返回高质量的输出,基于一小部分结果进行手动检查来调整其算法参数。 
鉴于要花大约 15 秒的时间来标注 1 秒的凝视数据,要对整个像 StudyForrest 这样数据集(约 30 小时)的数据进行完整的人工标注是不可行的,并且需要多个标注者才能获得最佳结果~\cite{startsev2019characterizing}。








%\vspace{0.6em}
%\begin{figure*}[ht]
%	\centering
%	\includegraphics[width=6in]{./figures/C2Fig/motion_content.jpg}
%	\vspace{0.2em}
%	\caption{
%	15个样例帧和他们对应运动内容的可视化。
%	三个主要列中的每一个(使用红线分隔)表示帧运动的不同层次(低、中、高)。
%	在每个超级单元中,我们随机显示所选取的帧(左上),电影进度5帧之后的帧(左下),随机帧的光流(右上)、绝对运动门限为2像素的光流(右下)。
%	} 
%	\label{fig:motion_content}
%\end{figure*}


% k_E表示体素的数目
%\vspace{0.6em}
%\begin{figure*}[ht]
%	\centering
%	\includegraphics[width=1.0\textwidth]{./figures/C2Fig/BOLD_mean_responses.jpg}
%	\vspace{0.2em}
%	\caption{
%	当运动加入模型中,平滑跟踪和眼跳的平均效果,以及运动本身的平均效果(使用$P_{FWE}\textless0.05$,$P\textless0.001$的初始门限。
%	(a)与平滑跟踪相关的横跨两侧视觉脑区的激活($k_E=3706$,包括MT+/V5),以及两侧中扣带皮层延伸至楔前叶($k_E=605$)。
%	(b)与眼跳相关的两侧上延伸至楔前叶的激活($k_E=7715$)。
%	(c)运动相关的横跨两侧视觉脑区(包括MT/V5)和向上延伸至中扣带皮层和楔前叶的激活($k_E=10876$)。
%	同时也存在负激活,包括颞上沟皮层内侧(左:$k_E=450$;右:$k_E=456$)、颞顶联合区(左:$k_E=242$,右:$k_E=118$)和辅助运动区($k_E=304$)的两侧。
%	} 
%	\label{fig:BOLD_mean_responses}
%\end{figure*}


%\vspace{0.6em}
%\begin{figure*}[ht]
%	\centering
%	\includegraphics[width=1.0\textwidth]{./figures/C2Fig/SP_motion_contrast.jpg}
%	\vspace{0.2em}
%	\caption{
%	平滑跟踪大于运动和EpicFlow计算得到的运动回归变量对比的激活($p_{FWE}\textless0.05$,0.001的初始门限值)。
%	激活出现在右上/中颞回(前/后颞上沟)($k_E=536$,峰值xyz:60,-55,26)、左中颞沟(后 颞上沟)($k_E=194,峰值xyz:-60,-28,11$)、双侧楔前叶($k_E=102$,峰值xyz:6,-58,35)、双侧辅助视区($k_E=177$,峰值xyz:-6,11,62)。
%	} 
%	\label{fig:SP_motion_contrast}
%\end{figure*}


\subsubsection{结果鲁棒性} \label{sec:result_robustness}
当对某些数据进行模型拟合时,就像在 fMRI 中使用广义线性模型一样,想知道呈现的结果是否是所提供数据的唯一拟合或它们是否符合基础模式。
在这里,通过运行概念验证步骤(类似于 $k = 2$ 的 $k$ 重交叉验证)来评估呈现结果的鲁棒性。
首先将 8 个《阿甘正传》视频片段分成两半,前半部分包含视频段 1 到 4,下半部分包含视频段 5 到 8。
然后对广义线性模型进行拟合,该模型包括两个集合上各自的平滑跟踪回归变量和眼跳回归变量,并比较体素值以获取每个回归变量的平均效果。 
%图\ref{fig:fit_independently}中的
结果表明,两个模型的结果高度相关,线性拟合的 $r_2$ 为 0.81。




\subsection{核磁共振眼动分析}
这个实验分析的目的是分析在动态开放自然场景中与平滑跟踪相关的脑部激活。
为了这个目的,当动态场景作为刺激时,基于现成的算法和建模技术提出方法处理运动的噪声、非结构化的信息、以及来自扫描仪记录的眼跟踪数据。
这里主要的结果和之前的研究一致,当对平滑跟踪进行建模,显示在平滑跟踪时 MT/MST 有明显的激活。
当另外表示刺激运动内容的回归变量添加到模型当中,识别出了与注意力相关的区域,然而由于受血氧水平依赖平均效应诱发的类似平滑跟踪和运动,一些其他脑区(包括 MT/MST)降低到重要门限值以下。
%这些结果表明我们方法在简单交叉验证步骤中更加鲁棒(如图\ref{fig:fit_independently}所示)。


%\vspace{0.6em}
%\begin{figure*}[ht]
%	\centering
%	\includegraphics[width=1.0\textwidth]{./figures/C2Fig/fit_independently.jpg}
%	\vspace{0.2em}
%	\caption{
%	对于两个GLM模型,眼跳和平滑跟踪平均效果的体素激活是对于“StudyForrest”数据集的前半部分和后半部分独立拟合。
%	两个模型的体素值高度相关($r_2=0.81$),这表明我们的原始模型可以可靠地拟合数据。
%	} 
%	\label{fig:fit_independently}
%\end{figure*}




\subsubsection{模型中考虑运动的优势}
将运动作为回归变量添加到模型中,使得能够确定与平滑跟踪相关的激活,这些激活本身并非由刺激的整体运动所驱动
%(图\ref{fig:BOLD_mean_responses}a)。 
有趣的是,运动本身还会导致与颞上沟和辅助视区相关的负面激活。 
因此,当直接对比平滑跟踪大于运动时,这两个区域以及楔前叶在平滑跟踪期间的激活明显强于单独的运动内容。
%(图\ref{fig:SP_motion_contrast})。
颞上沟被认为是信息处理的枢纽,包括生物运动的处理~\cite{pinar2007superior,jastorff2009human,grossman2010fmr}以及需要社交认知的情况下的面部处理~\cite{allison2000social,hoffman2000distinct,lahnakoski2012naturalistic}。 
与该模型一致,通过经颅磁刺激抑制颞上沟活性导致难以感知生物运动~\cite{grossman2010fmr}。
此外,颞上沟活动的减少~\cite{freitag2008perception,alaerts2014underconnectivity} 与难以理解自闭症谱系障碍患者的生物运动和情绪内容有关~\cite{hubert2007brief,nackaerts2012recognizing,alaerts2014underconnectivity}。 
即使在目标不可见的情况下,辅助视区的激活也与预期的眼动相关,反映出认知输入可以独立于视觉输入而进行平滑跟踪规划~\cite{lencer2004cortical,missal2004supplementary,ohlendorf2010visual}。

在解释与运动内容有关的发现时,应考虑到运动回归变量是基于像素级运动能量的低层次描述,这可能无法捕获自然场景的语义特性。 
因此,根据分析,运动内容的高价值与背景和摄像机运动有关,这两者都广泛用于专业拍摄的电影视频中~\cite{cutting2011quicker}。
%,请参见图\ref{fig:motion_content}。
相反,将中等大小目标(比如具有社会意义的目标)的移动与低运动内容价值相关联。 
因此,除非执行对有意义目标的平滑跟踪,否则由运动内容回归变量建模的无关运动可能导致观察到的双向颞上沟和辅助视区中的负激活。


% 视频只要有运动的目标产生的脑激活
\subsubsection{视频中运动的解释}
为了从运动内容的相关脑激活中识别出平滑跟踪,在第一阶段个体层次分析中添加了一个额外的运动回归变量,它和眼动调制处理过程类似,再次将其建模为时间序列。
%这里使用 EpicFlow 算法显示结果,完整的帧表示运动建模(对于基于结构张量的较小值,该算法的结果定性上是相似的,数据这里没有显示)。
%为了更好的理解EpicFlow运动估计和我们运动回归变量之间的关系,参见图\ref{fig:motion_content}。
得到的运动回归变量和眼跳回归变量没有关系(皮尔逊相关系数 $r=-0.11$),并且对于平滑跟踪回归变量仍然有效(皮尔逊相关系数 $r=0.18$)。
%和平滑跟踪、眼跳、运动相关的血氧水平依赖响应的平均效果显示在图\ref{fig:BOLD_mean_responses}中。
%从图\ref{fig:BOLD_mean_responses}a和b中可以看出,平滑跟踪和眼跳激活与图\ref{fig:sp_sac_activation}中在定量上非常接近,但是当运动加入模型(图\ref{fig:BOLD_mean_responses}a)中时,平滑跟踪有更小的大小和强度。
%平滑跟踪相关激活的减少伴随着强烈的运动(图\ref{fig:BOLD_mean_responses}c)相关激活(大致和平稳跟踪有相同的激活区域,如图\ref{fig:sp_sac_activation}a所示)。
然而,颞上沟皮层内层的激活和辅助眼区的辅助运动皮层与运动回归变量是负相关。

按照这个模型,平滑跟踪大于眼跳的对比仅仅在中扣带皮层和右侧颞顶联合区区域产生重要的激活。
%图\ref{fig:contrast}a中,
MT/MST 和楔前叶的激活没有达到 $p_{FWE} < 0.05$ 的要求门限。
%
平滑跟踪大于运动的对比揭示了在颞上沟皮层内层、楔前叶和辅助视区的辅助运动区的双侧激活。
与之相反,眼跳大于运动的对比没有揭示任何重要的激活区域。

\subsubsection{额外回归变量的考虑}
为了至少部分缓解运动能量分析的潜在混乱,引入了建模基本视频特征的其他回归变量。 
在两个控制实验中,将基于显著性和边缘密度的场景动态开放性的建模,作为吸引注意力的参数,以测试这些参数是否干扰了与平滑跟踪和运动内容有关的激活。
在这两种情况下,验证回归变量的平均效果均显示出在一些非常小的簇中的重要激活(在大脑后部总体上大约为 150-300 个体素,大部分在视觉皮层中),但并未影响涉及兴趣主要对比的激活。 
根据这些观察结果得出结论,根据对每个特征建模的方式,眼动规划过程主要受基础运动的驱动。 
但是,在将来对动态自然场景中平滑跟踪的研究中,可能需要对所有潜在参数和建模技术进行更详尽的搜索。


\subsubsection{平滑跟踪中与变化相关的脑区}
平滑跟踪大于眼跳与仅在第一层次设计矩阵中包括的平滑跟踪和眼跳回归变量的对比,揭示了中扣带皮层和楔前叶的激活,这先前与平滑跟踪眼动控制~\cite{tanabe2002brain,kimmig2008fmri}和视觉空间处理有关~\cite{berman1999cortical,e2006the}。
此外,与右颞顶联合区相关的眼跳相比,这种对比在平滑跟踪期间产生了更高的激活度,而右颞顶联合区是涉及向未注意区域提供指导的区域~\cite{corbetta2000voluntary,wu2015a,marsman2016a}。
最重要的是,这种对比揭示了与 MT/MST 区域相关的双边激活,该区域被视为核心运动处理区域~\cite{petit1999functional,lencer2008neurophysiology,nagel2006parametric},并且在以前的研究中与平滑跟踪眼动有关~\cite{kimmig2008fmri,ohlendorf2010visual,marsman2016a}。 
值得注意的是,当添加第三回归变量建模的整体刺激运动时,MT/MST 区域变得不那么显著。 
最好的解释是,这一区域的血氧水平依赖响应方差现在由两个回归变量(平滑跟踪和运动)而不是一个回归变量共享~\cite{ohlendorf2010visual},
%,这从图\ref{fig:BOLD_mean_responses}a和c中的平滑跟踪和运动的平均效果可以看出。 
这证明了在动态开放场景中找到单一激活源很困难,许多不同的因素可能会激发特定区域的激活,而这种混淆的完全解开可能难以做到。

%\subsubsection{平滑跟踪相关激活}
%\subsubsection{平滑跟踪和眼跳的对比激活}
% saccade 眼跳
经过组织和分析,平滑跟踪作为所感兴趣的眼动,识别出和眼跳相关的三个脑区。
%因此,我们关心的是在观看自然场景时平滑跟踪和眼跳相关脑区的不同之处。
%当$P_{FWE}\textless0.05$和$P\textless0.001$的初始门限值时的对比在图\ref{fig:contrast}中进行可视化。
%在图\ref{fig:contrast}a中,
第一个脑区有运动处理和平滑跟踪相关脑区 MT/MST 的双边激活(左),第二个脑区包括中扣带皮层并延伸至楔前叶,第三个脑区包括右颞顶联合处的一个激活。
%图\ref{fig:contrast}b展示了
%眼跳大于平滑跟踪的对比在 V2 有重要的激活响应(右:$k_E=91$)。
%平滑跟踪和眼跳相关的血氧水平依赖响应的平均效果如图\ref{fig:sp_sac_activation}表示,
这里使用 $p\textless0.001$ 的初始门限在 $p_{FWE}\textless0.05$ 时进行聚类。
该过程共产生三个和平滑跟踪相关的聚类。
%和两个和眼跳相关的聚类(Sac1 到 Sac2)。
%参见表\ref{tab:clusters}。
% Sac1相比于SP1颞中回有多一块橙色小区域
第一个区域最明显的是在中颞回内的最大激活,推测可能是因为 该区域包含视觉运动区域 MT/MST,这在意料之中,因为这些区域和平滑跟踪和运动处理都有关。
第二个大的聚类主要覆盖中扣带皮层和楔前叶的部分区域。
% (a)中右图为右脑,rTPJ和注意力和社会认知相关。
%同时这里也存在一个更小的和眼跳相关的聚类 Sac2,他覆盖了楔前叶的部分区域,
第三个和平滑跟踪相关的一个更小的聚类是右侧颞顶联合区。
为了检查显示的每个眼动聚类都很好的表示了对应脑区,使用了一个简单的交叉验证步骤(详细过程参考章节~\ref{sec:result_robustness})。
%两个独立模型激活之间的高度相关性($r^2=0.81$)表明回归模型不仅仅提供了数据,而且拟合了一种模式。



%表\ref{tab:anat_func_area}列出了更加详细的解剖和功能区域的描述,分别包括识别出的聚类SP1-SP3和Sac1-Sac2。
%解剖区域使用自动解剖地图集进行分割\cite{tzourio-mazoyer2002automated,rolls2015implementation}。
%为了避免相对较少体素的解剖区域弄乱表格,我们用聚类中体素数目百分比作为裁剪门限值。
%对于两个最大表中两个最大聚类的门限值设置为$-2\%$,其他的设置为$5\%$。


%\begin{figure*}[ht]
%	\centering
%	\includegraphics[width=6in]{./figures/C2Fig/sp_sac_activation.jpg}
%	\vspace{0.2em}
%	\caption{平滑跟踪和眼跳相关的激活。
%		(a)平滑跟踪相关的激活(使用$p_{FWE}<0.05$,$p<0.001$的初始化门限。
%		激活横跨大脑两边的视觉区域(SP1:$k_E=7647$,包括与平滑跟踪相关的MT+/V5),
%		(脑两边的)中扣带皮层延伸到楔前叶(后顶叶皮层的一部分,位于头顶叶内侧部分)(SP2:$k_E$=2048),
%		右颞顶连接处(SP3:$k_E=109$)。
%		(b)眼跳相关的激活(使用$p_{FWE}<0.05$,$p<0.001$的初始门限值。
%		激活横跨两边大脑的视觉区域(Sac1:$k_E=6437$)和楔前叶(Sac2:$k_E=245$)。
%		分区的详细列表参考表。
%	} 
%	\label{fig:sp_sac_activation}
%\end{figure*}


%\vspace{0.6em}
%\begin{table}[htbp]\wuhao
%	\centering
%	\caption{和平滑跟踪和眼跳眼动相关,带峰值激活T值的聚类列表和沿着聚类层次FWE校正p值的定位。
%		}
%	\vspace{0.3em}
%	\begin{tabular}{p{1.7cm}<{\centering}p{1.7cm}<{\centering}p{1.7cm}<{\centering}p{1.7cm}<{\centering}p{1.7cm}<{\centering}p{1.7cm}<{\centering}p{1.7cm}<{\centering}}
%		\toprule[1.5pt]
%		聚类名字  & X  & Y   & Z & 聚类大小 &$P_{FWE_corr}$ &t峰值\\ 
%		\midrule[1.0pt]
%		SP1     &-3 &-91 &14 &7647 &\textless0.001  &15.11 \\
%		SP2   &6 &-43 &56 &2048  &\textless0.001  &8.48  \\
%		SP3   &57 &-40 &17 &109  &0.011  &6.94  \\
%		Sac1   &-9 &-82 &17 &6437  &\textless0.001  &16.84  \\
%		Sac2   &-6 &-52 &56 &245  &\textless0.001  &6.25  \\
%		\bottomrule[1.5pt]
%	\end{tabular}
%	\label{tab:clusters}
%\end{table}


%\vspace{0.6em}
%\begin{table}[htbp]\wuhao
%	\centering
%	\caption{涉及平滑跟踪和眼跳相关聚类(着以灰色)的脑区列表,只与平滑跟踪有关标成白色。
%	可视化门限值对于大聚类选择表\ref{tab:clusters}的$2\%$,对于小聚类选择$5\%$。
%	因此在每个聚类中这些值的总和不等于体素的总数。
%	}
%	\vspace{0.3em}
%	\begin{tabular}{p{1.2cm}<{\centering}|p{1.2cm}<{\centering}|p{0.5cm}<{\centering}|p{0.5cm}<{\centering}|p{0.5cm}<{\centering}|p{1.0cm}<{\centering}|p{1.0cm}<{\centering}|p{1.0cm}<{\centering}|p{0.5cm}<{\centering}|p{0.5cm}<{\centering}|p{0.5cm}<{\centering}|p{0.5cm}<{\centering}|p{0.5cm}<{\centering}|p{0.5cm}<{\centering}}
%		\toprule[1.5pt]
%		解剖区域  & 功能区域(布罗德曼分区)  & X   & Y & Z &所属 &体素数 &T峰值 &X &Y &Z &所属 &体素数 &T峰值 \\ 
%		\midrule[1.0pt]
%		左舌回     &17,18 &-15 &-76 &2 &SP1  &528 &12.04 &-18 &-79 &2 &Sac1 &522 &14.00 \\
%		\midrule[1.0pt]
%		右舌回     &17,18 &-12 &-85 &13 &SP1  &528 &12.04 &-18 &-79 &2 &Sac1 &522 &14.00 \\
%		\midrule[1.0pt]
%		左距状皮层     &17,18,30 &-12 &-85 &13 &SP1  &528 &12.04 &-18 &-79 &2 &Sac1 &522 &14.00 \\
%		\midrule[1.0pt]
%		右距状皮层     &17,18,30 &-12 &-85 &13 &SP1  &528 &12.04 &-18 &-79 &2 &Sac1 &522 &14.00 \\
%		\midrule[1.0pt]
%		左楔叶     &18,19 &-12 &-85 &13 &SP1  &528 &12.04 &-18 &-79 &2 &Sac1 &522 &14.00 \\
%		\midrule[1.0pt]
%		右楔叶     &18,19 &-12 &-85 &13 &SP1  &528 &12.04 &-18 &-79 &2 &Sac1 &522 &14.00 \\
%		\midrule[1.0pt]
%		左枕上     &18,19 &-12 &-85 &13 &SP1  &528 &12.04 &-18 &-79 &2 &Sac1 &522 &14.00 \\
%		\midrule[1.0pt]
%		右枕上     &18,19 &-12 &-85 &13 &SP1  &528 &12.04 &-18 &-79 &2 &Sac1 &522 &14.00 \\
%		\midrule[1.0pt]
%		左枕中     &19,37(V5) &-12 &-85 &13 &SP1  &528 &12.04 &-18 &-79 &2 &Sac1 &522 &14.00 \\
%		\midrule[1.0pt]
%		右枕中     &19,37 &-12 &-85 &13 &SP1  &528 &12.04 &-18 &-79 &2 &Sac1 &522 &14.00 \\
%		\midrule[1.0pt]
%		左枕下     &18 &-12 &-85 &13 &SP1  &528 &12.04 &-18 &-79 &2 &Sac1 &522 &14.00 \\
%		\midrule[1.0pt]
%		左梭状回     &18,19 &-12 &-85 &13 &SP1  &528 &12.04 &-18 &-79 &2 &Sac1 &522 &14.00 \\
%		\midrule[1.0pt]
%		右梭状回     &18,19 &-12 &-85 &13 &SP1  &528 &12.04 &-18 &-79 &2 &Sac1 &522 &14.00 \\
%		\midrule[1.0pt]
%		左小脑 6     &- &-12 &-85 &13 &SP1  &528 &12.04 &-18 &-79 &2 &Sac1 &522 &14.00 \\
%		\midrule[1.0pt]
%		右小脑 6     &- &-12 &-85 &13 &SP1  &528 &12.04 &-18 &-79 &2 &Sac1 &522 &14.00 \\
%		\midrule[1.0pt]
%		左楔前叶     &5,7 &-12 &-85 &13 &SP1  &528 &12.04 &-18 &-79 &2 &Sac1 &522 &14.00 \\
%		\midrule[1.0pt]
%		右楔前叶     &5,7 &-12 &-85 &13 &SP1  &528 &12.04 &-18 &-79 &2 &Sac1 &522 &14.00 \\
%		\midrule[1.0pt]
%		左中颞     &19,39(V5) &-12 &-85 &13 &SP1  &528 &12.04 &-18 &-79 &2 &Sac1 &522 &14.00 \\
%		\midrule[1.0pt]
%		右中颞     &19,39(V5) &-12 &-85 &13 &SP1  &528 &12.04 &-18 &-79 &2 &Sac1 &522 &14.00 \\
%		\midrule[1.0pt]
%		右下颞     &37(V5) &-12 &-85 &13 &SP1  &528 &12.04 &-18 &-79 &2 &Sac1 &522 &14.00 \\
%		\midrule[1.0pt]
%		左中扣带皮层     &23,24,31 &-12 &-85 &13 &SP1  &528 &12.04 &-18 &-79 &2 &Sac1 &522 &14.00 \\
%		\midrule[1.0pt]
%		右中扣带皮层     &23,24,31 &-12 &-85 &13 &SP1  &528 &12.04 &-18 &-79 &2 &Sac1 &522 &14.00 \\
%		\midrule[1.0pt]
%		旁中央小叶     &5 &-12 &-85 &13 &SP1  &528 &12.04 &-18 &-79 &2 &Sac1 &522 &14.00 \\
%		\midrule[1.0pt]
%		右上颞     &40 &-12 &-85 &13 &SP1  &528 &12.04 &-18 &-79 &2 &Sac1 &522 &14.00 \\
%		\midrule[1.0pt]
%		右上缘回     &40 &-12 &-85 &13 &SP1  &528 &12.04 &-18 &-79 &2 &Sac1 &522 &14.00 \\
%		\bottomrule[1.5pt]
%	\end{tabular}
%	\label{tab:anat_func_area}
%\end{table}



%完整的解剖区域列表是这些聚类的一部分,在表\ref{tab:anat_func_area}中提供。


%\begin{figure*}[ht]
%	\centering
%	\includegraphics[width=6in]{./figures/C2Fig/contrast.jpg}
%	\vspace{0.2em}
%	\caption{
%	(a)使用0.001的初始化门限值,在$P_{FWE}\textless0.05$时,平滑跟踪大于眼跳的激活状态。
%	两侧MT/V5处的激活(右边:$k_E=169$,左边$k_E=89$),中扣带皮层延伸至楔前叶($k_E=665$),右颞顶联合区(rTPJ)($k_E=158$)。
%	(2)在0.001的初始门限且$p_{FWE}\textgreater0.05$时眼跳大于平滑跟踪的激活。
%	V2处的激活(右:$k_{E}=91$)。
%	} 
%	\label{fig:contrast}
%\end{figure*}








%\subsubsection{自然观看条件下缺乏前额叶视区的关联}
%我们没有检测到任何和前额叶视区相关的激活,该区域涉及到平滑跟踪和眼跳的规划和执行~\cite{sp_representation,berman1999cortical,gagnon2006transcranial,kimmig2008fmri}。
%一个可能的解释是在典型的实验中,受试在长久注视基线周期和点跟踪或者场景观察之间进行切换。
%相反,在我们这里使用的数据集中,受试在连续观察电影时,可能从事一些眼睛运动的规划,这对真实世界观察行为更具有代表性。
%因此,这些变化(比如在连续2秒时间窗口的眼跳)不足以识别出所有眼跳相关的激活(包括FEF)。
%另一个限制因素可能是FEF太小,在特定实验条件和操作中报告依靠激活的位置有较大的方差。
% 被平均了?


\subsection{类脑跟踪模型的有效性}
在这里比较了各种实验结果,以解释为什么所提出的 BTN 是一个类脑跟踪模型。
很容易看出,模型架构和训练过程与现有的视觉跟踪模型实现略有不同。
%
图~\ref{fig:neural_predictivity}(a)展示了进行激活对比的区域为 BTN 的动态滤波网络和大脑皮层的中颞和上颞内侧区;
图~\ref{fig:neural_predictivity}(b)展示了 BTN 在 Tracking-Gump 数据集上 BTS 的性能,获得了最好 0.365 的类脑跟踪分数。
并且获得最高跟踪分数的模型在 Tracking-Gump 数据集上也有出色的类脑跟踪分数,
表明 Tracking-Gump 数据集上的跟踪效果和皮层激活模式之间存在联系。
如图所示~\ref{fig:neural_predictivity},表明具有良好 Tracking-Gump 跟踪性能的模型与 BTS 具有很强的相关性,并且在 Tracking-Gump 数据集中存在显著的相关性($p \textless 0.05$)。
% todo

% 纵坐标:神经预测性、分数 的归一化值
%\begin{figure}[t]
%	\centering
%	\includegraphics[width=0.75\linewidth]{figures/C2Fig/comp_sim.pdf}
%	\caption{
%		BTN 捕获的 MT/MST 神经响应
%	}
%	\label{fig:neural_predictivity}
%\end{figure}

\begin{figure}[htbp]
	\centering
	
	\subfigure[BTN 和大脑进行激活对比的区域]{
		\begin{minipage}[t]{0.35\linewidth}
			\centering
			\includegraphics[width=1\textwidth]{./figures/C2Fig/similarity.pdf}
		\end{minipage}%
	}%
	\subfigure[BTN 捕获的中颞和上颞内侧区神经响应]{
		\begin{minipage}[t]{0.65\linewidth}
			\centering
			\includegraphics[width=1\textwidth]{./figures/C2Fig/comp_sim.pdf}
		\end{minipage}%
	}%
	%\n is important,(or \quad) 

	
	\centering
	\caption{BTN 与大脑激活响应的对比}
	\label{fig:neural_predictivity}
\end{figure}


% 改变跟踪模型参数(模型的深度)来看效果
% 换其他的单目标跟踪模型(IoU;激活相似性?)
% 紧致性+循环性验证
\subsubsection{BTN 与大脑皮层的结构相似性}
该研究设计了一个类脑跟踪神经网络 BTN,它比经典深度神经网络更紧密地遵循神经解剖学对齐。
此外,根据 BTS 在 Tracking-Gump 数据集上的测试性能,BTN 实现了良好的视觉跟踪性能。
因此,BTN 既满足神经科学中的神经解剖学限制,又能很好地满足计算机视觉中的工程要求。

因为 BTN 限制了模块的数量并使用了循环结构,实验发现 BTN 比当前优秀的深度神经网络跟踪模型更接近神经解剖学约束。
在神经科学中,因为没有神经学上可信的训练方法,用于训练平滑跟踪的类脑深度神经网络,它可以使用更好的神经解剖学和连接机制。
例如,使用跨层链接~\cite{he2016deep} 解决深度神经网络训练过程中梯度消失的问题并非是受大脑机制启发。
如图~\ref{fig:structure_analysis} 所示,考虑到各种网络架构,在找到合适的类脑架构 BTN 之前测试了各种架构配置。
实验分析了四个主要因素,包括 $\rm{CONV_{V1}}$ 中的初始步长、$\rm{DFN_{MT/MST}}$ 中的动态滤波器的数目、$\rm{LSTM_{FEF}}$ 和 $\rm{FC}$ 中隐藏单元的数量。
每行表示当某个超参数发生变化时,皮层相似性分数和行为相似性分数如何根据训练的 BTN 变化。

% 证明能够捕获时间序列上的信息
% 神经相似性分数 的图
\subsubsection{BTN 捕获的 MT/MST 神经响应}
\label{sec:capture_neural}

前馈神经网络无法预测时间序列的趋势,因此无法捕获跟踪模型的类脑响应~\cite{kar2019evidence, TangSchrimpfLotter2018Recurrent}。
通过使用循环连接,BTN 能够预测相应皮层中随时间变化的激活。
最新研究工作~\cite{kar2019evidence} 发现图像分类中的可解码结果是在颞下皮层神经元中产生的,
并且这些输入对于深度神经网络来说很难在颞下区域花费更多时间进行解码。
时间属性激发了一个类脑网络的假设:
它是否会随着时间的推移预测 MT/MST 皮层激活中的逐帧运动信息?
因此,当有明显的目标运动特征可用时,BTN 能够预测视频帧中的运动,
并将其与人类 MT/MST 皮层中记录的反应进行比较。
值得注意的是,BTN 从未学会预测人类皮层的响应强度。
但是却学习到了可以解码神经响应和模型响应之间的运动相似性特征。
当使用 PCA 方法进行数据压缩时,更多的组件将保留更多的神经信息。
如图~\ref{fig:neural_predictivity} 所示,最后评估了 BTN 在人类大脑皮层的 MT/MST 中捕获到了细粒度运动响应的能力,
%并给出了 0.365 ($p < 0.05$) 的 BTS。
显示了对于不同 PCA 分量所表现出的归一化神经相似性和 $p$ 值。
当 PCA 组件的数量达到 90 时,BTN 的神经预测性获得了最佳 0.365 的结果。


\subsubsection{Tracking-Gump 数据集上 BTN 的有效性}
跟踪性能在后期训练中确实提高了皮层相似度,在一系列模型中选择了具有最大 BTS 的模型。
此外,模型预测的边界框与眼睛注视的位置大致一致。
最后,在实验中使用行为和 MT/MST 相似性指标来分析 BTN 中的运动信息。
BTN 在 Tracking-Gump 数据集上实现了出色的跟踪性能,如图~\ref{fig:tracking} 所示,
对于不同的视频序列,用边界框来标识模型的预测轨迹。
此外,这些受试的平均眼睛注视位置由圆进行标识,并且其面积与跟踪时人眼瞳孔大小成正比。
该实验证明了所提出的 BTN 在动态开放环境中的有效性和鲁棒性。

\begin{figure}[t]
	\centering
	\includegraphics[width=\linewidth]{figures/C2Fig/tracking.pdf}
	\caption{
		在 Tracking-Gump 数据集上测试所提出的类脑跟踪模型的效果示例	
	}
	\label{fig:tracking}
\end{figure}

\subsection{讨论}

在该研究工作中,设计了用于预测人脑平滑跟踪的类脑跟踪模型 BTN,实现了预测机制和在线跟踪随意运动的目标。
可以看出,BTN 利用深度神经网络实现神经解剖的对齐、人眼跟踪行为和大脑激活响应的预测,
并且实验结果表明 BTN 可以产生连续的预测跟踪信号。
尽管平滑追踪研究表明,物体运动的皮层表征可能被用于跟踪推理~\cite{b21,b4,b3},但建立和保持皮层表征的皮大脑层理论是未知的。
这项研究表明,通过 BTN 在线学习和更新皮层表示可以减少跟踪滞后,在面对漂移或遮挡时调整眼球运动并产生连续跟踪。

平滑跟踪的一个有趣特征是,当遮挡发生时跟踪运动会继续进行。
BTN 无延迟地跟踪运动物体时,视网膜有运动动作。
在学习了物体运动特征后,视网膜滑移很小。
然而,小的视网膜滑动不能进行被遮挡的对象跟踪,
LSTM 却可以产生自我维持的眼动预测~\cite{kashyap2018a},
小的视网膜滑动成分持续用于调整跟踪的预测。
当发生遮挡时,这种可调节的视网膜滑动信息不存在。
因此,就像之前的研究结果一样~\cite{b4},跟踪是一步一步丢失的。
此外在 BTN 中,LSTM 中的每个神经元都有基础的自发激活。
当被跟踪目标被遮挡时,训练好的激活模型会不断产生跟踪信号。

\subsubsection{损失函数分析}
之前的实验表明使用循环注意力跟踪器能够跟踪真实世界的的目标。
除了和循环注意力跟踪器~\cite{ratm} 相似的地方外,本章的方法还使用了额外的模块,包括:边界框回归损失、背侧流损失、腹侧流损失和辅助损失,并将这些损失组合在一个统一的方法中,
现在讨论这些模块的属性。

(1)背侧流中的空间注意力损失防止梯度消失:早期的实验表明仅仅使用跟踪损失会导致梯度消失问题。
在训练的初期,改模型不能正确地估计目标的运动,导致不能抽取跟踪目标的前景,或者仅仅包含跟踪目标的一部分。
这种情况下,监督信号和模型的输入没什么联系,阻止了学习的进行。
即使当目标包含在前景中,因为任何指导信号不得不通过特征抽取阶段传向之前时间步,导致损失函数回传的梯度路径相当长。
因此,直接惩罚注意力参数可以解决这个梯度消失的问题。

(2)空间注意力始终有效:
为了使跟踪系统适应目标的外观,并且和开始位置无关,实验中将初始的边界框转化为注意力参数,在这里增加了偏置值,并从对应的视觉特征创建 LSTM 的隐藏状态。
在实验中,偏置值始终收敛到正值,相比于目标的边界框更加偏向于注意力前景。
表明对于目标跟踪丢弃不相关的特征是可取的,基于注意力模块,整个跟踪系统在空间注意力和外观注意力权衡重要性。

(3)腹侧流中外观注意力是必要的:给足够多的数据和足够大的模型容量,外观注意力能够在更新工作记忆之前过滤掉不相关的输入特征。
然而,一般情况下,如果模型使用合适的损失进行限制会加快训练的过程。
%图中展示模型使用和不使用外观注意力情况下,提取前景和对应位置映射的例子。
使用外观注意力情况下,即使跟踪的行人被其他人遮挡也能跟踪上。
而没有惩罚时,目标定位可能不会非常好,甚至可能丢失整个目标。
通过使用外观注意力损失,不仅能提升跟踪的效果,而且使模型更加具有可解释性。




\subsubsection{MT/MST 和 FEF 在目标跟踪中的作用}
许多研究都发现 FEF 拥有预测目标跟踪的能力。
当物体被遮挡或出现时,FEF 的去除或损伤会损害猴子对运动物体的跟踪~\cite{b11}。
FEF 中的神经元激活表明,在移动目标消失后,连续激活仍密集地存在~\cite{b14}。
此外,一些 fMRI 工作发现在 FEF~\cite{b33} 中存在预测眼球跟踪的运动指标。
本研究工作表明,FEF 提取了对象运动模式的皮层表示来指示预测跟踪过程。

FEF 接收来自 MT/MST 的密集映射,这些映射是背流中处理移动目标的区域。
如图~\ref{fig:c2:introduction} 所示,在后半部分,FEF 的输出被传递到脑干区域的背侧 脑桥核,
此外,这些脑干区域将信号从 FEF 传递到脑桥核,从而进行眼部调整~\cite{b36}。

神经解剖消去的结果表明,FEF 从顶叶区域获得输入并输出到脑桥核以控制眼球运动~\cite{b11}。
这些结果表明 FEF 与 LSTM 具有可靠的皮层相关性,这是基于对象跟踪预测能力的。
此外,FEF(LSTM)使用图像刺激来推断眼球运动信号以进行眼动调整。

\subsubsection{BTN 的应用和局限性}
本研究所提出的 BTN 不仅可用于传统的视觉对象跟踪任务,
此外它还是是一种更具解释性的类脑跟踪模型,可用于预测眼睛注视的位置和皮层通路中的激活。

尽管如此,独立的训练配置可能是训练深度神经网络的基本特征~\cite{Kornblith2018a}。
此外,辅助任务对分数的影响不是很大,但显著提高了迁移效果~\cite{Kornblith2018a}。
因为 BTN 是一个迁移学习问题,所以不能排除在使用不同的训练设置时 BTS 可能会发生变化。
他们通过使用优化的配置重新训练深度学习网络可以极大地提高迁移效果~\cite{Kornblith2018a}。
因此,一般认为只有特定的训练 BTN 是最优的,而不是所有的网络架构模型。
实际上可以执行网格搜索以根据验证集的性能选择最佳训练配置。

\subsubsection{深度神经网络与神经科学的关系}

构建类脑架构模型 BTN 的一个重要步骤是类脑跟踪分数 BTS,它是将深度神经网络与目标跟踪时的大脑皮层进行比较的定量指标。
尽管到目前为止还没有皮层跟踪指标,但所提出的框架是一个有启发性的想法。
首先,所提出的框架扩展了探索性研究,表明跟踪性能与皮层相似性相关。
然而,循环结构的使用改变了这种趋势,并且与大脑皮层具有极好的神经解剖学相似性。
此外,发现 Tracking-Gump 跟踪分数与 BTN 中的 BTS 之间可能存在冲突。
视觉目标跟踪有很多优秀的模型,在 BTS 中没有较高的分数,这些没有较好 BTS 的深度学习模型却可能会在 Tracking-Gump 数据集上取得较好的跟踪效果。
此外,发现有些深度神经网络不仅具有出色的 BTS 性能,而且可以轻松获得良好的跟踪结果,这支持了 BTS 是一个综合指标的假设,并且这些结果不仅仅基于所使用的行为和神经数据集。


使用 BTS 的相似性量化来比较这些深度神经网络,
并评估 BTS 上各种行为和神经数据集的指标。
对于 BTN,实验证明了根据解剖对齐的和循环结构的皮层解剖模型可以通过皮层激活预测、逐帧行为甚至神经动力学很好地学习皮层机制。
这样,BTN 可以同时获得较高的 Tracking-Gump 跟踪性能和突出的类脑效果。
总的来说,该研究表明了类脑模型是深度学习和神经科学合作的潜在机会,利用神经科学的研究成果启发深度跟踪神经网络模型的设计,同时深度跟踪模型可以对大脑皮层的激活和行为进行预测,促进脑机接口技术的发展,同时加强对人脑的理解,使机器学习和神经科学的发展相互促进。




\section{本章小结}
受人脑中视觉处理机制的启发,本研究提出了一种适用于视觉对象跟踪问题的类脑跟踪模型来解决模型的可解释性问题。
基于神经解剖学限制,本工作设计的模型具有较好的跟踪性能和可解释性。
此外,还举例说明了在动态开放的真实环境中与人类平滑跟踪特别相关的皮层模型,
并开发一种新方法来计算模型激活和皮层激活数据之间的相似性。
同时具有神经解剖对齐的模型可以更好地预测人脑的神经激活响应。
相信所提出的 BTN 可以在深度神经网络的可解释性方面激发新的灵感,甚至推动脑机接口技术的发展。

在下一步工作中,将尝试将利用单目标跟踪模型推广到动态开放场景下的多目标跟踪任务中,解决多目标跟踪任务中的所存在的目标干扰和没有利用目标运动趋势的问题。



% !Mode:: "TeX:UTF-8"

\chapter{ 基于非局部注意力机制的多目标跟踪数据关联策略} \label{chap:nonlocal}
% /data2/whd/win10/doc/paper/doctor/doctor.Data/PDF/0668102945

\section{引言}
% Introduction
视频多目标跟踪是计算机视觉方向最基本和最核心的科学问题之一,并且在智能监控、智能机器人、无人驾驶车和人机交互等方面得到了大规模的使用~\cite{autonomous_vechicle}。
视频多目标跟踪的目标是在图像帧序列中准确估计所有对象的状态(包括位置和身份),它旨在通过在整个视频帧中查找目标位置和维护目标身份来估计多个对象的轨迹。
%它是计算机视觉领域最基本和最重要的问题之一,并且在视频监控、智能机器人、无人驾驶车和人机交互等方面有着广泛的应用。
% 尽管多近年来多目标跟踪有了长足的发展,
近年来,由于深度学习~\cite{b4} 的进步,多目标跟踪取得了一定的进展,但是由于背景复杂和相互遮挡等挑战的存在使得动态开放场景下的多目标跟踪仍然是一个非常困难的问题。
一般来说,现有的视频多目标跟踪方法一般分为在线多目标跟踪算法与离线多目标跟踪算法。
离线多目标跟踪算法利用历史和将来的视频帧来产生跟踪轨迹,但是在线多目标跟踪算法仅使用当前时刻可以使用的数据。
尽管离线多目标跟踪算法可以应对一些不确定跟踪的情况,但是无法在实时场景下得到应用。

% 摘要
在线多目标跟踪作为视频分析和多媒体应用中的一个基本问题,
常用的基于检测跟踪框架的主要挑战是如何将候选检测结果与现有的轨迹段进行关联。
在这方面,本章提出了一种非局部注意关联方法,并将其应用于一个统一的在线多目标跟踪框架,该框架集成了单目标跟踪和数据关联方法各自的优点。
具体来说,该方法提出一种非局部注意关联网络(Non-local Attention Association Networks,NAAN)融合空间和时间特征来进行新目标的检测和历史轨迹的关联。
利用非局部注意力生成跨空间和时间的非局部注意力特征,使得跟踪模型能够关注整个轨迹的信息,而不是局部注意力特征,以克服噪声检测、遮挡和目标之间频繁交互等问题。


随着目标检测技术~\cite{b8} 的发展,跨帧链接检测结果的数据关联算法已经成为多目标跟踪的主流。
然而,这些方法严重依赖于不完美的检测器。
如果检测结果不准确、遗漏或错误,则跟踪对象容易丢失。
可以通过在多目标跟踪环境中使用最新的且精度较高的单目标跟踪器~\cite{b10} 来缓解这种问题。
单目标跟踪器使用第一帧中的检测结果并在线更新单目标跟踪模型,以确定后续帧中跟踪目标的位置和大小~\cite{weight_based}。
然而,当跟踪目标被遮挡~\cite{local_sparse} 时,这种方法容易发生漂移。
为此可以将单目标跟踪器和数据关联的优点结合在一个统一的框架中来解决这个问题。
在大多数视频帧中,使用单目标跟踪器来跟踪每个目标。
然后当跟踪分数低于阈值时应用数据关联方法解决漂移问题。
该方案表明被跟踪的目标可能会经历较大的外观变化或被其他物体遮挡。

通常,直接使用现有的单目标跟踪器进行多目标跟踪的主要挑战是处理跟踪目标和类内干扰物之间的频繁交互。
此外,单目标跟踪器在在线模型更新过程中通常会遇到正负样本之间数据不平衡的问题。
在单目标跟踪器的搜索区域中,被跟踪目标中心附近只有少数地方对应于正样本,而其他地方的所有样本都是负样本,所以背景区域的大多数位置将生成负样本。
这种情况可能会造成正负样本之间的不平衡,削弱单目标跟踪模型的判别能力。
如图~\ref{fig:nlaa_tracking_problem} 所示,这个问题在多目标跟踪任务的上下文中进一步加剧。
如果单目标跟踪模型被大量背景负样本淹没,那么当搜索区域中出现类似的干扰项时,跟踪器很容易发生漂移。
当跟踪过程变得不可靠时,需要使用数据关联方法将候选检测与历史行人序列联系起来。


\begin{figure*}[ht]
	\centering
	\includegraphics[width=0.8\textwidth]{figures/C3Fig/tracking_problem.pdf}
	\caption{数据关联的动机}
	\label{fig:nlaa_tracking_problem}
\end{figure*}

在执行数据关联时,需要将一系列先前跟踪的对象与当前帧检测进行比较。 
多目标跟踪任务中最常见的跟踪对象就是行人,其中数据关联问题也称为行人重新识别,该任务具有各种挑战性因素,包括相似的外观、姿势变化、频繁的遮挡等。
然而,传统的卷积操作只关注局部特征和检测区域。
在多目标跟踪的背景下,这些跟踪结果可能会带有一些未对齐错误或丢失跟踪目标部件的噪声。
因此,历史轨迹中的不准确和被遮挡的结果很可能会导致单目标跟踪模型的错误更新,从而导致轨迹特征提取模型的有效性降低。
为了解决上述问题,需要为数据关联设计更有效的轨迹特征提取方法。
本章提出了跨时空范围的非局部特征提取模型,来应对传统卷积操作中特征提取的局部性问题。
%这些因素促使为数据关联设计有效的轨迹特征提取模型,从而抑制上述问题。
%为了,确保所。
%
最后在公开的基准数据集上进行一系列研究实验说明了所提出的算法与各种基于身份保留的在线跟踪器相比表现良好。

本章的主要贡献如下:
\begin{itemize}
	\item  设计了一个嵌入在卷积神经网络中的非局部注意力层来自适应地提取跨空间和时间区域而不是局部区域的全局特征,并使用非局部注意力机制来抑制不准确和被遮挡的错误检测。
	\item  提出了一个注意力关联网络来处理多目标跟踪中的序列相关性和遮挡问题。在关联当前检测结果和历史轨迹时,所提出的网络不仅生成目标检测结果与历史轨迹之间的相似性,还生成所有行人序列的一致性,以减轻轨迹中不可靠样本的影响。
	\item  提出了一种数据关联的训练方法。在执行训练过程之前,利用多目标跟踪数据集的检测结果来生成各种行人段,并以等概率随机对轨迹段进行采样,以满足网络的输入大小要求。同时,从一系列数据增强策略中制定了一个方案,以解决模型训练过程中数据不足和模型欠拟合的问题。
	\item  通过在多目标跟踪基准数据集上进行大量的消去实验并与最先进的多目标跟踪方法进行比较来证明所提出算法的有效性。
\end{itemize}


\section{相关工作}
\subsection{多目标跟踪}
多目标跟踪任务的目标是解决数据关联问题,它通常采用检测跟踪范式。
依据多目标跟踪算法是否使用将来视频帧的信息,可以分为在线和离线跟踪方法。
离线多目标跟踪方法~\cite{b2,b17} 使用来自过去和未来帧的检测结果进行批处理。
此类方法具有利用所有视频帧全局信息的优势。
通常,离线多目标跟踪方法将多目标跟踪任务建模为各种形式的全局优化问题,例如网络流~\cite{b17} 和多割~\cite{b2}。
相比之下,在线多目标跟踪方法~\cite{b10, PHD_filter} 不能利用来自未来帧的检测结果和帧信息,并且在目标对象被严重遮挡或检测不准确时可能表现不佳。
因此,鲁棒的外观模型对于关联在线多目标跟踪的检测结果至关重要。
最近已经提出了一些使用深度学习模型的在线方法\cite{b10,b23,b24}。 
孪生网络~\cite{b1} 对来自 RGB 图像空间的外观信息和来自光流图的运动信息进行编码,然后通过基于线性规划的跟踪器处理获得的特征。
在 AMIR~\cite{b23} 中,LSTM 网络被用来对外观特征进行建模。
该方法通过逐步获取轨迹段中的图像来预测相似度分数。
在本章的工作中,引入了一种在线注意力关联多目标跟踪方案来处理不准确的检测和遮挡问题。
大量的实验表明该方法与最先进的在线多目标跟踪方法相比,所提出的在线算法可以实现良好的身份保持和跟踪性能。

\subsection{注意力机制}
许多视觉任务方法都采用了注意力机制,例如图像字幕~\cite{b25}、视觉问答~\cite{b27} 和图像分类~\cite{b29}等。
视觉注意力机制使模型能够专注于输入图像的最相关区域,以提取适合大量特定视觉任务的判别性特征。
非局部均值~\cite{b30} 是一类成熟的滤波方法,它计算视频图像内所有像素的加权平均。
该算法可以使远处的像素基于图像块的图像相似度对某个位置的响应做出贡献。
这种非局部滤波器的思想后来发展成为一种称为三维块匹配~\cite{b31} 的方案,它对一组相似但非局部的图像块进行滤波。
块匹配与神经网络一起用于图像去噪~\cite{b33}。
非局部注意力机制也成功应用于纹理合成\cite{b34}、超分辨率\cite{b35} 和修复\cite{b36} 等领域。
自注意力模块关注特征空间中的所有位置并取加权平均来进行某个位置响应的计算。
在本章的研究工作中,将利用非局部注意机制集成空间和时间特征到所提出的多目标跟踪算法中。


\section{非局部注意力关联算法}
本节中提出的解决前面提到问题的方法是利用单目标跟踪和非局部注意关联来维持多目标跟踪过程中目标的身份。
图~\ref{nonlocal_attention_network} 显示了所提出的在线多目标跟踪流程。
对每帧中的所有目标检测结果,先利用单目标跟踪器对每个检测目标进行正常的单目标跟踪过程和身份的维持。
%先是利用单目标跟踪器进行正常的目标跟踪和身份的维持。
目标刚出现时将跟踪目标的状态设置为跟踪,直到跟踪结果变得不可靠(例如,跟踪分数较低或跟踪结果与检测结果不一致),在这种情况下,跟踪目标的状态被视为漂移,
然后暂停单目标跟踪器并执行注意关联以计算历史跟踪轨迹与未被任何跟踪目标覆盖的当前检测结果之间的相似性。
一旦漂移目标通过注意力关联与检测结果相关联,则更新跟踪状态并恢复跟踪过程,
这个过程在章节~\ref{attention_association} 中进行详细的描述。
%当出现正当或相互干扰等情况让跟踪过程变得不可靠时,
%则使用注意力关联方法关联当前帧的候选检测结果与历史行人序列。



\subsection{方法框架}

\begin{figure*}[ht]
	\centering
	\includegraphics[width=0.85\textwidth]{figures/C3Fig/MOT_pipline.pdf}
	\caption{在线多目标跟踪流程}
	\label{fig:nlaa_MOT_pipline}
\end{figure*}


如图~\ref{fig:nlaa_MOT_pipline} 所示,所利用的在线多目标跟踪流程主要由三个子任务组成:单目标跟踪、检测和注意力关联。 
每个跟踪对象由五种状态组成:出生、激活、跟踪、漂移和死亡。
对于给定的行人图像序列,跟踪器的目标是使用深度卷积模块来提取行人的特征表示,从而在嵌入空间中实现基于轨迹的行人重新识别。 
学习图像序列代表性特征的关键因素是将轨迹的时空特征合并到特征中。 
为此,将非局部注意力层引入传统卷积神经网络以学习图像序列的时空依赖性。 
在章节~\ref{nonlocal_attention_network} 中提出了一个非局部注意关联网络 NAAN,以在不同的特征级别应用此操作。

\subsection{非局部注意力网络} \label{nonlocal_attention_network}

\begin{figure*}[ht]
	\centering
	\includegraphics[width=0.6\textwidth]{figures/C3Fig/association_network.pdf}
	\caption{NAAN 中的注意力关联模块概述}
	\label{fig:nlaa_association_network}
\end{figure*}

该方案如图~\ref{fig:nlaa_association_network} 所示,为了提取行人历史序列图像的特征表示,将通过统一采样策略选择的轨迹帧子集作为网络的输入。 
然后,结合非局部注意层和特征池化层的骨干卷积神经网络获得用于基于轨迹重新识别的特征,随后计算用于 $ M $ 张图像行人的合并特征与当前检测结果之间的相似性。
其中的非局部注意力网络如图~\ref{fig:nlaa_attention_network} 所示,用于提取历史行人序列的特征表示,
给定 $T$ 采样图像作为输入,并使用五个非局部注意力层和一系列 ResNet-50 网络提取行人图像序列的时空信息,然后在 3D 平均池化中将特征池化为用于注意力关联的一维向量。

\begin{figure*}[ht]
	\centering
	\includegraphics[width=0.6\textwidth]{figures/C3Fig/attention_network.pdf}
	\caption{非局部注意力网络的详细描述}
	\label{fig:nlaa_attention_network}
\end{figure*}


\begin{figure*}[ht]
	\centering
	\includegraphics[width=0.45\textwidth]{figures/C3Fig/attention_layer.pdf}
	\caption{非局部注意力层的细节}
	\label{fig:nlaa_attention_layer}
\end{figure*}

为了有效提取历史轨迹的时空特征,这里使用非局部注意机制并将非局部块~\cite{b37} 嵌入到主干卷积神经网络。
继非局部均值算法~\cite{b30} 之后,该研究工作在骨干卷积神经网络中定义了一个非局部操作,如图~\ref{fig:nlaa_attention_layer} 所示,
$\otimes$ 表示矩阵乘法,$\oplus$ 表示元素求和,并对每一行执行 softmax 操作。 
绿色框表示 $1 \times 1 \times1$ 卷积。 
在这里,采用了具有 $C$ 个通道瓶颈的嵌入式高斯版本。
\begin{equation}
y^i=\frac{1}{C\left( x \right)} \sum_{\forall j}  f{\left( x_i,x_j \right) g\left( x_j \right) }\mbox{,}
\label{nonlocal_operation}
\end{equation}
其中,$i$ 是要需要计算响应的时空输出位置下标,$j$ 为枚举历史轨迹中所有可能的时空位置下标,$x$ 为输入图像序列,并且 $y$ 是与 $ x $ 有相同尺寸的特征。 
两个输入变量的函数 $f$ 计算 $i$ 和 $j$ 之间的亲和度。
$g$ 表示处于位置 $j$ 上的输入特征。
响应由因子 $ C\left(x\right) $ 进行正则化,
公式中非局部层中的操作是一种子注意力机制,在非局部注意力网络~\cite{b37} 中也有提到,并设置:
\begin{equation}
C\left(x\right)=\sum_{\forall j}f\left(x_i,x_j\right)\mbox{,}
\end{equation}
在非局部操作中使用高斯函数的简单扩展来计算嵌入空间中的相似性。
此外在本研究中,使用点积相似度 $\theta \left(x_i\right)^T \phi \left(x_j\right)$,并将 $f$ 定义为:
\begin{equation}
f\left(x_i,x_j\right)=e^{ \theta \left(x_i\right)^T \phi \left(x_j\right) }\mbox{,}
\end{equation}
其中 $ \theta \left(x_i\right) = W_\theta x_i $ 和 $ \phi \left(x_j\right)=W_\phi x_j $ 是两个嵌入特征。
为简单起见,只考虑线性嵌入形式的 $g$,即 $g\left(x_j\right) = W_g x_j $,
其中 $W_\theta$、$W_\phi$ 和 $W_g $ 表示相应的需要要学习的权重矩阵。 
该块($\theta$、$\phi$ 和 $g$)在空间-时间中实现为 $1 \times 1 \times 1$ 卷积。

总之,从长度为 $T$ 行人图像的序列中获得输入特征张量 $ X\subseteq R^{C\times T\times H\times W} $,让 $ x_i \in R^C $ 从 $X$ 中采样,目的是从所有图像中聚合空间位置特征和时间特征。  
非局部注意力操作对应的输出 $y_i\in R^C$ 可以详细表述如下:
\begin{equation}
y^i=\frac{1}{\sum_{\forall j} e^{\theta\left(x_i\right)^T \phi \left(x_j\right)}} \sum_{\forall j} e^{\theta\left(x_i\right)^T \phi \left(x_j\right)} g\left(x_j\right)\mbox{,}
\end{equation}
其中 $i,j=\left[1,THW\right]$ 索引了二维空间特征图和所有时间序列帧中的位置。 
首先,通过使用 $ 1 \times 1 \times 1 $ 卷积核实现的线性变换函数 $\theta$、$\phi$ 和 $g$。 
随后利用嵌入的高斯实例化,通过所有坐标的加权平均 $x_j$ 来表示每个坐标处的响应 $x_i$。

最终整个非局部层最终被形式化为:
\begin{equation}
Z=W_{Z}Y+X\mbox{。}
\end{equation}
如公式~\ref{nonlocal_operation} 中所定义,非局部操作 $Y$ 的输出被添加到原始特征张量 $X$ 中,并进行了 $1 \times 1 \times 1 $ 卷积核的变换 $W_Z$,故 $Y$ 被映射到原始特征空间 $R^{C}$ 中。
直觉上可以将非局部操作归因于在给定时间内提取特定位置的特征,其中网络应通过非局部上下文来考虑序列内的时空依赖性。
如图~\ref{fig:nlaa_attention_network} 所示,在本章中的行人非局部注意力关联方案中,将五个非局部注意力层加入到骨干卷积神经网络 ResNet-50 里,以理解轨迹段中呈现的语义关系。
此外,与连续叠加卷积和循环神经网络算子相比,非局部操作能直接计算轨迹的时空位置之间的关系,达到快速捕获远程和全局依赖关系的目的。


\subsubsection{特征池化层}
如图~\ref{fig:nlaa_attention_network} 所示,将轨迹段的图像序列输入给具有非局部注意力层的骨干卷积神经网络后,使用特征池化层获得注意力关联的最终特征。
随后沿时空维度使用三维平均池化,将序列图像的输出特征聚合成一个有代表性的特征向量,然后进行批量正则化。

\subsection{注意力关联}
\label{attention_association}
在跟踪过程中,一旦单目标跟踪过程变得不可靠,就暂停单目标跟踪器并将跟踪目标的状态设置为漂移。 
然后如~\ref{nonlocal_attention_network} 节中所讨论的,利用注意力关联方法,来确定是否应该将目标状态保持为漂移或将其更改为跟踪状态。 
通常,使用目标的单目标跟踪分数(即置信度图中的最高值)来衡量单目标跟踪的可靠性。 
然而,如果只依赖于跟踪分数,那么背景上的虚警检测很容易被高置信度地持续跟踪。 
考虑到一个被跟踪的目标在几帧中都没有得到任何检测,很可能是误报检测,利用跟踪器和检测器给出的边界框之间的重叠来过滤掉误报。 
因此,可以将跟踪目标的状态定义为:
\begin{equation}
s_{tracking}=\left\{
\begin{array}{rcl}
1 & {if \ s > \tau_s \ and \ o_{m} > \tau_o}\\
0 & {otherwise}\mbox{,} 
\end{array} \right.
\end{equation}
其中,$ s_{tracking} $表示跟踪状态,1 表示跟踪,0 表示漂移,$s$、$\tau_s$ 和 $\tau_o$ 分别是跟踪目标的得分、跟踪得分的阈值和重叠率。 
历史轨迹 $o_{m}$ 的平均重叠定义为:
\begin{equation}
\label{overlap_mean}
o_{m}=\frac{\sum_{1}^{L} o\left(t_l,D_L\right)}{L}\mbox{,}
\end{equation}
考虑 $\sum_{1}^{L} o\left(t_l,D_L\right) $ 在过去 $ L $ 跟踪帧的平均值 $ o_{m} $ 作为决定跟踪状态时的另一种度量。
在公式~\ref{overlap_mean} 中,跟踪目标和检测之间的重叠率定义为:
\begin{equation}
\label{overlap_target_detection}
o \left(t_l,D_L\right) =\left\{
\begin{array}{rcl}
1& {if \ max \left(IOU \left(t_l,D_l\right) \right) > \tau_o } \\
0& {otherwise}\mbox{,} \\
\end{array} \right.
\end{equation}
其中,$t_l$ 表示第 $ l $ 帧的检测结果,$ D_l $ 为第 $ l $ 帧全部检测结果,$ T_l $ 为历史全部跟踪轨迹,如果前一个跟踪目标 $ t_1 \in T_l $ 与全部检测结果 $ D_l $ 之间的最大交并比大于 $\tau_o$,$o \left(t_l,D_l\right) $ 设置为 1。
否则,$o \left(t_l,D_l\right) $ 设置为 0。

在计算注意力关联的外观相似度之前,可以先利用运动线索来选择候选检测。 
当跟踪目标发生漂移时,将边界框的尺度保持在最后一帧 $k-1$,并利用线性运动方法来推断目标在当前视频帧 $k$ 中的坐标。 
令 $ c_{k-1}=\left[x_{k-1},y_{k-1}\right] $ 表示目标在 $ k-1 $ 帧处的中心坐标。 
计算目标在 $ k-1 $ 帧处的速度 $ v_{k-1} $ 为:
\begin{equation}
v_{k-1}=\frac{c_{k-1}-c_{k-K}}{K}\mbox{,}
\end{equation}
其中 $ K $ 表示计算速度的帧间隔。 
那么当前帧 $k$ 中跟踪目标的坐标预测为:
\begin{equation}
c_k=c_{k-1}+v_{k-1}\mbox{。}
\end{equation}

给定目标的预测位置,将预测位置周围没有被任何跟踪目标覆盖的检测(检测与预测位置之间的距离小于 $\tau_d$)作为候选检测。
同时还计算了检测结果与目标轨迹中的观察值之间的外观相似度。 
然后选择相似度最高的检测结果并设置相似度阈值 $\tau_a$ 来决定是否将漂移目标与该检测结果相关联。


\subsubsection{相似性计算}
为了计算历史轨迹和当前帧候选检测结果之间的相似性,将历史轨迹的非局部注意力特征($1 \times 2048$)和当前帧候选检测结果的嵌入特征($1\times 2048$)合并到一个单独的特征并将其输入到一个输入大小为 4096 维的全连接模块,并输出轨迹段和候选检测的相似性概率。 
这个全连接模块有三层,分别有 4096、512 和 64 个隐藏单元。 
此外,每个全连接层都与批量正则化层和整流线性单元相结合。 
全连接层的最后一层预测历史行人轨迹与候选检测结果之间的相似性概率。
然后,对相似性预测执行具有交叉熵损失的二分类器。

最后,根据当前帧检测结果和历史轨迹成对的相似性分数在候选检测和漂移目标之间进行分配。


\subsubsection{训练策略}
利用真实检测结果和身份在 MOT16 和 MOT17 训练集中提供的信息来生成一些关联的候选检测图像和轨迹检测的身份信息,用于训练关联网络。
然而,训练数据只包含有限的身份,每个身份的序列由有限的样本组成。
因此,所提出的网络容易对训练集欠拟合。
为了缓解这个问题,在训练中采用了一些数据增强策略。
首先通过随机选择轨迹来训练 NAAN。
然后,随机生成对应于轨迹段的 $T$ 个检测结果。
此外,通过随机裁剪和重新缩放输入图像来增加训练集。
为了在实验中模拟嘈杂的多目标跟踪环境并防止在训练过程中欠拟合,通过用不同于真实轨迹身份的图像随机替换轨迹段中的一些图像,将噪声样本添加到训练轨迹段序列中。
由于训练集中的一些轨迹只包含几个样本,所以以相等的概率随机采样每个轨迹,以减轻类不平衡的影响并满足 NAAN 输入大小要求。
最后,通过优化交叉熵损失来训练所提出的 NAAN。


\subsubsection{轨迹的出现和消亡}
在跟踪过程中,使用多目标跟踪基准数据集~\cite{b42} 提供的检测结果来初始化对象跟踪过程并产生新的轨迹。
如果候选检测结果与任何轨迹边界框的的交并比小于阈值,则认为它是新出现的候选轨迹。
为了避免误报,只有当新出现的候选中边界框序列在 $L$ 连续帧期间都高于阈值 $\tau_i$ 时,才将其视为新轨迹。
关于轨迹消亡,当轨迹与任何检测结果都没有重叠时认为跟踪出现漂移并从跟踪结果中去除。
如果轨迹持续漂移超过 $\tau_t$ 帧或移出视野,则将结束轨迹。
但是,如果相同的跟踪对象再次出现,那么它将以与以前相同的身份被恢复。
此外,通过对收集到的观察样本进行统一采样以减少数据冗余,具体方法是将跟踪目标的 $M$ 个最近观测值和样本长度为 $T$ 的轨迹段用于注意关联,
该变量是以固定大小输入到非局部注意力关联网络中。



\section{实验结果与分析}
在本节中,将展示所提出的方法在公共基准数据集中的性能,并将其与现有的最新方法进行比较。 
首先简要介绍本研究中使用的数据集和评估指标,然后介绍所提出方法的实施细节和消去实验。 
在与其他方法进行比较之后,将介绍所使用方法的参数并对结果进行分析。

\subsection{基准数据集和评价指标}
在 MOT16 和 MOT17~\cite{b42} 基准数据集上评估了提出的在线多目标跟踪算法。 
MOT16 数据集包含 14 个视频序列(7 个用于训练,7 个用于测试),总共 11,235 帧和 292,733 个手动注释的真实边界框。
MOT17 基准数据集包含与 MOT16 数据集相同数量的视频序列,同时还提供三种不同的检测器结果(DPM \cite{dpm}、Faster-RCNN \cite{faster-rcnn} 和 SDP \cite{sdp})以获取更多信息以综合评估跟踪算法的性能。

本章使用多目标跟踪基准数据集~\cite{b42} 的多个评估指标进行性能比较,
除了标准的多目标跟踪精度(MOTA~\cite{b44})和多目标跟踪精度(MOTP~\cite{b4})之外,性能指标还包括 ID 召回率~\cite{b45}(IDR,正确识别的真实检测)、误报数(FP)、漏报数(FN)、ID 切换数(IDS)和片段数(Frag)。 
另外 IDR~\cite{b45} 最近已被添加到多目标跟踪基准测试中,以衡量跟踪器的身份保留能力。


\subsection{实验设置}
在线多目标跟踪流程第一步中使用单目标跟踪~\cite{b46},
%使用与  跟踪器相同的功能。
当跟踪对象漂移时使用注意力关联模块,并利用在 ImageNet 数据集上预训练的 ResNet-50 卷积块作为共享基础网络。 
轨迹段的长度设置为 $ T=8 $,轨迹中收集的最大样本数设置为 $M=100$。
所有跟踪对象的输入图像尺寸都调整为 $256 \times 128$。
学习率为 $10^{-4}$ 的 Adam 优化器用于训练非局部注意力网络,批量大小设置为 32。
在 NVIDIA GeForce RTX 2080Ti 上训练过程 1.5 小时,持续 40 个迭代周期。
考虑到多目标基准数据集规模不大,阈值参数的所有值都是根据 MOT16 和 MOT17 训练集上的 MOTA 性能设置的。
$ F $ 是视频的帧率,
计算跟踪目标速度的区间设置为 $K=0.3F$。
轨迹初始化阈值设置为 $\tau_i=0.2F$,而轨迹终止的阈值设置为 $\tau_t=2F$。
跟踪分数和外观相似度的阈值分别设置为 $\tau_s=0.2$ 和 $\tau_a=0.8$。
重叠和距离的阈值分别设置为 $\tau_o=0.5$ 和 $\tau_d=2$。
此外如图~\ref{fig:nlaa_grid_search} 所示,为了避免手动随意设置超参数的局限,本实验通过网格搜索的方法选择跟踪分数和外观的阈值。
所提出的跟踪方法使用 Python 的软件库 Pytorch 0.4.1~\cite{b49} 进行实现。


\subsection{消去实验}
如图~\ref{fig:nlaa_ablation} 所示,通过每次禁用一个基础模块来进行消去研究,以验证每个模块在所提出的方法中的贡献。
与在多目标跟踪测试数据集上的完整模型(44.5$\%$) 相比,都比每个基准方法的 MOTA 分数高。 
说明提出的组件都有助于多目标跟踪性能的提升。
当直接使用跟踪分数进行注意力关联时,MOTA 指标下降了 13.1$\%$,表明所提出的 NAAN 的优势。
B2 中性能的退化证明了添加到标准卷积神经网络中的非局部注意层的有效性。 
每个基线方法可以描述如下:

\begin{itemize}
	\item  B1 表示禁用所提出的 NAAN 并使用跟踪分数来关联历史轨迹和当前检测结果。 
	具体来说,将跟踪器的卷积滤波器应用于候选检测,并直接使用置信图中的最大跟踪分数作为注意力关联的外观相似度。
	\item  B2 表示禁用非局部注意力层,并使用标准的卷积神经网络架构提取历史轨迹段的特征,将其用于轨迹的身份验证。
\end{itemize}


\begin{figure*}[ht]
	\centering
	\includegraphics[width=0.8\textwidth]{figures/C3Fig/ablation.pdf}
	\caption{消去实验结果}
	\label{fig:nlaa_ablation}
\end{figure*}


\begin{figure*}[ht]
	\centering
	\includegraphics[width=0.8\textwidth]{figures/C3Fig/grid_search.pdf}
	\caption{网格搜索超参数}
	\label{fig:nlaa_grid_search}
\end{figure*}


\subsection{在多目标跟踪基准数据集上进行评估}
表~\ref{tab:nlaa_tracking_performance} 和表~\ref{tab:nlaa_performance_MOT17} 展示了所提出的方法在 MOT16 和 MOT17 数据集的定量性能,并在 MOT16 和 MOT17 基准测试集上和常见的方法进行比较。 
所提出的方法在 MOT16 和 MOT17 数据集上获得了较好的 MOTA 分数,并且在 MOTA、MOTP、IDR、FP、FN、IDS 和 Frag 指标方面优于常见的方法。 
在 MOT16 基准测试中,与第二好的在线多目标跟踪方法相比,所提出的方法在 MOTA 上有 0.6$\%$ 的性能提升,在 MOTP 上有 0.9$\%$ 的提升。 
特别是在 MOT17 数据集上,与第二好的在线多目标跟踪方法相比,所提出的方法在 MOTA 中获得了 5.7$\%$ 的性能提升,在 MOTP 中获得了 0.6$\%$ 的性能提升。 
此外,所提出的跟踪器在 MOT16 数据集上的所有在线跟踪器中实现了最佳 FP 和 Frag 值。
所提出的跟踪器在在线多目标跟踪方法中实现了 MOTA 和 MOTP 的最佳性能,证明了所提出的方法在保持身份和跟踪方面的优势。 


\vspace{1.0em}
\renewcommand\arraystretch{1.5}
\begin{table}[htbp]\wuhao
	\centering
	\caption{在 MOT16 数据集上跟踪结果的比较}
	\vspace{0.3em}
	\begin{tabular}{c|ccccccc}
%		{p{2.5cm}<{\centering} p{1.0cm}<{\centering} p{1.0cm}<{\centering} p{1.0cm}<{\centering}p{1.0cm}<{\centering}p{1.0cm}<{\centering}p{1.0cm}<{\centering}p{1.0cm}<{\centering}}
%		\toprule[1.5pt]
%		\hline
		\hline
		方法& MOTA$\uparrow$& MOTP$\uparrow$& IDR$\uparrow$& FP$ \downarrow $& FN$ \downarrow $& IDS$ \downarrow $& Frag$ \downarrow $ \\
		\hline
%		\midrule[1.0pt]
		VOBT\cite{b50}& 38.4& 75.4& 28.7& 11,517& 99,463& 1,321& 2,140 \\
		EAMTT\cite{b51}& 38.8& 75.1& 31.5& 8,114& 102,452& 965& 1,657 \\
		oICF\cite{b52}& 43.2& 74.3& {\textbf{37.2}}& 6,651& 96,515& {\textbf{381}}& 1,404 \\
		DDAL\cite{b24}& 43.9& 74.7& 34.1& 6,450& {\textbf{95,175}}& 676& 1,795 \\
		NAAN & {\textbf{44.5}}& {\textbf{75.6}}& 32.8& {\textbf{5,346}}& 98,740& 698& {\textbf{1,252}} \\
%		\hline
		\hline
%		\bottomrule[1.5pt]		
	\end{tabular}
	\label{tab:nlaa_tracking_performance}
\end{table}


\vspace{1.0em}
\renewcommand\arraystretch{1.5}
\begin{table}[htbp]\wuhao
	\centering
	\caption{在 MOT17 数据集上跟踪结果的比较}
	\vspace{0.3em}
	\begin{tabular} {c|ccccccc}
%		{p{2.5cm}<{\centering} p{1.0cm}<{\centering} p{1.0cm}<{\centering} p{1.0cm}<{\centering}p{1.0cm}<{\centering}p{1.0cm}<{\centering}p{1.0cm}<{\centering}p{1.0cm}<{\centering}}
%		\toprule[1.5pt]
%		\hline
		\hline
		方法& MOTA$ \uparrow $& MOTP$ \uparrow $& IDR$ \uparrow $& FP$ \uparrow $& FN$ \downarrow $& IDS$ \downarrow $& Frag$ \downarrow $ \\
		\hline
		GM\_PHD\cite{b50}& 36.4& 76.2& 24.7& 23,723& 330,767& 4,607& 11,317 \\
		GMPHD\_KCF\cite{b51}& 39.6& 74.5& 29.1& 50,903& 284,228& 5,811& 7,414 \\
		GNN\cite{b53}& 45.5& 76.3& \textbf{41.8}& 25,685& 277,663& 4,091& 5,579 \\
		E2EM\cite{b52}& 47.5& 76.5& 37.9& 20,655& 272,187& 3,632 & 12,712 \\
		NAAN & {\textbf{53.2}}& {\textbf{77.1}}& 39.6& {\textbf{15,093}}& \textbf{245,802}& {\textbf{3,012}}& {\textbf{932}} \\
%		\bottomrule[1.5pt]
%		\hline
		\hline		
	\end{tabular}
	\label{tab:nlaa_performance_MOT17}
\end{table}


\subsection{参数分析}
如图~\ref{fig:nlaa_parameter} 所示,本节利用几个实验来展示不同阈值的设置对跟踪性能的影响,包括轨迹初始化阈值、轨迹终止阈值、跟踪分数、外观相似度分数、重叠率和距离。 
$\tau_s$ 和 $\tau_a$ 分别是跟踪分数和外观相似度的阈值。
$\tau_t$ 和 $\tau_i$ 是轨迹终止和初始化的正则化阈值因子。
$\tau_o$ 和 $\tau_d$ 分别是重叠率和距离的正则化阈值。 
$\tau_t$ 和 $\tau_d$ 的值从 $ [0.5, 3] $ 映射到 $ [0, 1] $。
所有超参数都在这些不同的参数设置上进行分析。 
正则化的 MOTA 随 $\tau_s$ 和 $\tau_a$ 的设置变化很大。 
因此,为了避免手动设置超参数的缺点,并减少优化超参数的工作量,基于已训练好的 NAAN ,选择 $ \tau _s $ 和 $ \tau _a $ 两个超参数进行超参数的网格搜索,以确定当前环境下合适的超参数配置。
图~\ref{fig:nlaa_grid_search} 中考虑了不同超参数值的影响,
超参数 $\tau_s$ 和 $\tau_a$ 是通过网格搜索选择的,在 MOT16 训练数据集上对于不同的 $\tau_s$ 和 $\tau_a$ 设置取得不同的 MOTA,并将其正则化到 $\left[0,1\right]$ 范围之内。 
%当 $\tau_s=0.2$ 和 $\tau_a=0.8$ 时,得到最大化的正则化 MOTA。
当 $\tau_s=0.2$ 和 $\tau_a=0.8$ 时,得到最大化正则化 MOTA,
因此,按照以上方案设置超参数。

\begin{figure*}[ht]
	\centering
	\includegraphics[width=0.8\textwidth]{figures/C3Fig/parameter.pdf}
	\caption{每个超参数对实验性能的影响}
	\label{fig:nlaa_parameter}
\end{figure*}


\subsection{讨论}
本章提出了非局部注意力关联方案联合处理轨迹级运动关联和在线多目标跟踪的相关问题。 
联合任务是通过关联历史轨迹和当前帧候选检测结果来实现的。 
在图~\ref{fig:nlaa_tracking_result} 中展示了不同环境下的跟踪结果示例。
第一排为在繁忙路口的公交车上拍摄的视频片段 MOT16-13,第二排为在夜间步行街且为高架视点拍摄的视频片段 MOT16-04,最后一排是从低角度拍摄的步行街场景的视频示例视频序列片段 MOT16-09,三个都展示了当跟踪过程在倒数第二帧中发生漂移时,所提出的方法能够很好的关联行人,解决单目标跟踪器应用到多目标跟踪环境中的漂移问题。
通常,当行人快速移动或受到其他行人的影响时,单目标跟踪器可能会发生漂移。 
使用注意力关联方法能及时纠正漂移问题。 
此外,单目标跟踪器可以有效克服遮挡的缺陷。
\begin{figure*}[ht]
	\centering
	\includegraphics[width=1.0\textwidth]{figures/C3Fig/tracking_result.pdf}
	\caption{不同环境下的跟踪结果示例}
	\label{fig:nlaa_tracking_result}
\end{figure*}

当前的实验设置有两个限制,
首先,跟踪方案的最佳性能和几个超参数的选择有关。
如图~\ref{fig:nlaa_grid_search} 所示,目前一些超参数是通过网格搜索进行设置的,这在一定程度上能搜索到一个合适但不是最优的超参数配置。
如果提供足够数量的训练数据,这些超参数可以通过上述网格搜索方法进行学习或优化,减少手动搜索参数的工作量并提高跟踪模型的泛化能力。 
其次,考虑到多目标跟踪这个巨大且困难的任务无法保证该模型是最优解决方案,在目前的工作中仅找到多目标跟踪问题的可行解。 
对上述两个问题的贡献可能会进一步改进跟踪性能。


\section{本章小结}
在本章的研究工作中,将单目标跟踪和注意关联算法的优点整合到一个统一的在线多目标跟踪框架中。 
对于轨迹段的特征提取,使用非局部注意力机制来提取轨迹段的时空特征。
对于注意力关联,利用非局部注意力模块的特征来关联候选检测和历史轨迹以抑制噪声检测和遮挡。 
在公开的多目标跟踪基准上进行了一系列消去研究和性能测试,证明了所提出的非局部注意力关联方法能较好的应对动态开放场景下多目标跟踪所存在的挑战。 
后面可以考虑将最新的且精度高的单目标跟踪器纳入到所提出的跟踪架构,并解决历史轨迹和当前检测结果特征不平衡的问题,以进一步提高多目标跟踪的精度并为实际应用做出贡献。







\section{时空互表征学习方法}
\subsection{基于时空互表征学习的鲁棒目标关联在线多目标跟踪}


\begin{frame}
	\frametitle{方法动机}
	\begin{columns}[T] % align columns
		\begin{column}<0->{.50\textwidth}
			\begin{figure}[thpb]
				\centering
				\resizebox{1\linewidth}{!}{
					\includegraphics{../figures/C4Fig/introduction.pdf}
				}
				\caption{时空相互学习和鲁棒的目标关联}
			\end{figure}
		\end{column}
		\hfill%
		\begin{column}<0->{.65\textwidth}
			\begin{itemize}
				\item<1-> 检测结果丢失、忽略或不准确
				\begin{itemize}
					\item<1-> 遵循跟踪预测范式,使用最新的高精度单目标跟踪器来缓解。
					\item<1-> 当跟踪的分数低于阈值时,将使用目标关联方法解决漂移问题
				\end{itemize}
				\item<1-> 当前检测结果的时间特征被忽略的问题,即数据关联双方特征的不平衡问题
				\begin{itemize}
					\item<1-> 通过所提出的时空相互学习方法,序列学习网络学习的时间信息被转移到检测学习网络
%					\item<1-> 由于学习到了时间信息,使得当前检测特征对各种复杂环境具有较好的鲁棒性
				\end{itemize}
			\end{itemize}
		\end{column}%
	\end{columns}
\end{frame}


\begin{frame}{数据关联中时空互学习方法的体系结构}
	\begin{figure}[!t]
		\centering
		\includegraphics[width=4.5in]{../figures/C4Fig/network.pdf}
		%		\caption{DNN 和带有BTS的神经解剖学对齐之间的协同设计}
	\end{figure}
\end{frame}


\begin{frame}
	\frametitle{目标关联流程}
	\begin{columns}[T] % align columns
		\begin{column}<0->{.50\textwidth}
			\begin{figure}[thpb]
				\centering
				\resizebox{1\linewidth}{!}{
					\includegraphics{../figures/C4Fig/i2vtesting.pdf}
				}
				\caption{当前检测结果和历史轨迹序列的数据关联流程}
			\end{figure}
		\end{column}
		\hfill%
		\begin{column}<0->{.65\textwidth}
			\begin{itemize}
				\item<1-> 相似性关联
				\begin{itemize}
					\item<1-> 当单个目标跟踪过程变得不可靠时,将跟踪目标标记为漂移状态,并根据历史目标序列与当前检测结果的相似度得分进行检测到序列的目标关联。
				\end{itemize}
				\item<1-> 目标出现和消失
				\begin{itemize}
					\item<1-> 当当前检测结果与所有跟踪目标的重叠率低于阈值时,将被视为新的潜在目标。
					\item<1-> 当单个目标保持漂移状态超过 $\tau_t$ 帧或直接移出视野时,将终止跟踪单目标跟踪过程。
				\end{itemize}
			\end{itemize}
		\end{column}%
	\end{columns}
\end{frame}



\begin{frame}
	\frametitle{实验}
	\begin{columns}[T] % align columns
		\begin{column}<0->{.50\textwidth}
			\begin{figure}[thpb]
				\centering
				\resizebox{1\linewidth}{!}{
					\includegraphics{../figures/C4Fig/ablation.pdf}
				}
				\caption{基础模块的消去实验}
			\end{figure}
		\end{column}
		\hfill%
		\begin{column}<0->{.65\textwidth}
			\begin{figure}[thpb]
				\centering
				\resizebox{1\linewidth}{!}{
					\includegraphics{../figures/C4Fig/T.pdf}
				}
				\caption{在 MOT16 数据集上使用不同 $T$ 的效果}
			\end{figure}
		\end{column}%
	\end{columns}
\end{frame}


\begin{frame}{测试效果}
	\begin{figure}[!t]
		\centering
		\includegraphics[width=3.7in]{../figures/C4Fig/tracking_result.pdf}
		\caption{在基准数据集上的跟踪结果示例}
	\end{figure}
\end{frame}

% !Mode:: "TeX:UTF-8"

\chapter{联合检测和数据关联的实时在线多目标跟踪方案}
\label{chap:jdan}

% 翻译参考:https://www.pianshen.com/article/20191669270/
\section{引言}
% 摘要
%近年来目标检测方法和数据关联方法取得了巨大的进步,这两种子任务对于一阶段在线目标跟踪必不可少。
%但是传统上这两个分离的模块是分别进行处理和优化,这导致了动态开放的模型设计,并需要冗余的模型参数需要学习。
%除此之外,这个领域中很少关注将两个子任务整合成一个端到端的模型来优化模型。
%在研究中,提出了一个端到端的检测关联网络,训练和推断都是在同一个网络模型中。
%检测关联网络的所有网络层都是可微的,并联合进行优化来学习有区分性的实体特征,同时使用网络输出的分配矩阵来进行鲁棒的多目标跟踪。
%模型直接使用检测和多目标跟踪的真实值所得到的损失来进行模型优化。
%所提出的方法在几个多目标跟踪的数据集上进行评估,与最好的方法相比取得了较好的跟踪性能。

% 引言
% 多目标跟踪简介
近年来目标检测方法和多目标跟踪方法都取得了巨大的进步~\cite{RN1002,RN1215,mahmoudi2019multi}。
在实际的许多应用中都会从多目标跟踪解决方案中受益,比如智能驾驶~\cite{auto_driving}、视频监控~\cite{deep_sort}、行人动作识别~\cite{mot16}等。
目前为了能在视频序列中进行多目标跟踪,按照处理流程可以将主流方法粗略分为两阶段方法和一阶段方法。


%\begin{figure*}[ht]
%	\centering
%	\includegraphics[width=0.7\textwidth]{./figures/C5Fig/end-to-end.pdf}
%	\vspace{0.2em}
%	\caption{端到端的目标检测和数据关联}
%	\label{fig:jdan_end-to-end}
%\end{figure*}

% 两阶阶段方法
两阶段方法~\cite{fang2018recurrent,nonlocal_association,poi} 包括两个互相分离的阶段,第一阶段的目标检测首先在当前视频帧中定位所跟踪的目标的位置和大小,然后在第二阶段的数据关联中抽取目标的再识别特征用于关联当前跟踪目标和历史的的轨迹段。
目前目标检测~\cite{faster,point,redmon2018yolov3}、再识别~\cite{k_reciprocal,expanded_re} 和数据关联~\cite{nonlocal_association}~的研究已取得了巨大的进步,同时也提高了多目标跟踪任务的性能。
然而由于两阶段模型目标特征的提取进行了两次,该方法在实际跟踪应用中并不能达到实时性的要求。

% 一阶段方法
% 两阶段
不同于两阶段方法,一阶段方法~\cite{jde,voigtlaender2019mots} 尝试将在线检测和关联两者集成到一个框架中,
如图~\ref{fig:jdan_consistency}(b)所示,两个子任务可以在目标表征提取中共享模型参数,以降低跟踪成本~\cite{jde,memory_improved}。
然而,几个明显的缺点阻碍了端到端多目标跟踪模型的实现。
首先,与图~\ref{fig:jdan_consistency}(a)所示的两阶段方法相比,目标检测和数据关联之间存在形态差异。
阶段一只涉及单张图像空间信息的处理,阶段二涉及时间序列上的数据关联。
这些差异使得端到端多目标跟踪模型的设计更加困难。
% 一阶段
其次,常见的一阶段方法采用独立的处理模式进行检测和关联,包括训练有效的检测模型,然后使用复杂的关联技巧来生成轨迹。
关联结果很大程度上取决于检测器的精度。
换句话说,检测和关联在训练过程中是相互独立的,
并且无法实现端到端的训练。
导致目标检测的误差会传播到关联阶段,从而降低了多目标跟踪的准确性。
% 端到端训练的数据问题
最后,对于如图~\ref{fig:jdan_consistency}(a)第一阶段的离线检测模块,多目标跟踪数据集中现有的检测结果或标签没有对应的检测网络模型参数用于构建端到端的检测跟踪模型,即检测网络的输出和关联网络的输入之间的边界框不一致阻止了整个端到端多目标跟踪模型中的训练过程。
因此,必须实现两个子模块之间的数据一致性。
此外,随着检测子模块的训练过程的继续,第二阶段预测的边界框没有相应的真实关联标签。

%而一阶段方法~\cite{jde,voigtlaender2019mots}~是在单个网络中执行目标检测和目标跟踪。
%因此,两个子任务可以在目标表征提取中共享模型参数,显著降低多目标跟踪的成本~\cite{jde,memory_improved}。

\begin{figure*}[ht]
	\centering
	\includegraphics[width=1.0\textwidth]{./figures/C5Fig/consistency.pdf}
	\vspace{0.2em}
	\caption{两阶段、一阶段和端到端方法的对比}
	\label{fig:jdan_consistency}
\end{figure*}


受上述分析的启发,本文提出了一个联合检测和关联网络(Joint Detection and Association Network,JDAN)的端到端训练框架来解决上述问题。
该框架主要由三部分组成:检测子模块、联合子模块和关联子模块。
具体来说,首先使用预训练的双流检测网络来提取初始目标候选及其表征。
然后,使用连接子模块来合并两帧之间所有可能的表征组合,以生成混淆张量。
最后,关联子模块将张量转换为关联矩阵,它表示来自两个帧的多个目标之间的匹配关系。
要联合训练前面的子模块,一个挑战在于不一致的目标问题,
与多目标跟踪任务的跟踪真实标签相比,检测子模块可能会在预测目标的位置和大小上和关联子模块所需要的数据不一致。
为了弥合这一差距,所提出的方法放弃了现有的真实跟踪标签,
如图~\ref{fig:jdan_consistency}(c)所示,利用传统的关联方法~\cite{welch1995introduction} 为检测结果生成伪标签,
然后将其输入到关联子模块以生成跟踪结果。
因此,所提出的模型可以以端到端的方式联合训练所有子模块,以生成稳健的一阶段模型。
此外,由于伪标签仅用于训练阶段而不是测试,因此它们对推理阶段的预测速度没有影响。


与之前的一阶段方法不同,所提出的方法可以以端到端的方式联合训练检测子模块和关联子模块,达到了缓解误差传播的目的。
在 MOT15~\cite{mot15} 和 MOT17~\cite{mot16} 数据集上评估所提出的方法,
并发现所提出的方法优于多个在线多目标跟踪器。
除了精度高之外,端到端方法简单且效率高,适合于实时场景的应用。
相信这项研究将对一阶段在线多目标跟踪有很好的启发作用。


总而言之,该工作的主要贡献如下:
\begin{itemize}
	\item 提出了一个端到端的架构来联合处理目标检测和关联,以缓解检测误差的传播问题。 
	该工作是第一次尝试为多目标跟踪任务进行端到端的模型训练。
	\item 所提出的方法使用伪标签来解决对象不一致问题,并提出了一个连接子模块并进行关联预测,并基于这些伪样本产生精确的跟踪结果。
	\item 通过消去研究在多目标跟踪基准数据集上进行了大量实验。
	结果表明,与几种流行的模型相比,所提出的方法可以实现实时在线跟踪,并实现具较好的跟踪精度。
\end{itemize}


\section{相关工作}
本章总结最近多目标跟踪中所取得的进展,将其分为两阶段方法和一阶段方法进行介绍,并分析了这些方法和之前所提方法的优缺点。
 
\subsection{两阶段方法}
传统的多目标跟踪方法~\cite{deep_sort,mahmoudi2019multi,zhou2018online} 通常将目标检测和数据关联作为两个独立的步骤。
首先,利用目标检测器~\cite{he2017mask,redmon2018yolov3} 以边界框的形式找出每一帧中所有的目标,并在原始图像帧中裁剪出检测结果。
第二阶段通常采用一般的数据关联方法,根据检测结果的交并比和外观表征计算相似度矩阵,
然后在视频帧之间进行状态估计~\cite{multi_pattern,local_sparse,dynamic_fusion} 和数据关联~\cite{kuhn1955hungarian,zhou2018online},以产生各个目标的运动轨迹。
已有许多研究~\cite{mahmoudi2019multi}~利用诸如图匹配~\cite{zhou2018online}、循环神经网络~\cite{fang2018recurrent} 等最新的数据关联方法。

两阶段方法的优势在于它能针对每一阶段分别利用最合适的方法来尽可能提高跟踪性能。
除此之外,两阶段多目标跟踪方法根据检测框裁剪视频帧,并在抽取目标特征之前将目标缩放成相同的大小。
这个缩放操作能较好的解决跟踪目标之间的尺度差异。
最终,这个方法~\cite{poi}在多目标跟踪基准数据集上取得了很好的跟踪效果。
然而,两阶段方法由于在目标检测中的特征抽取和目标跟踪中的特征抽取都非常耗时,在没有模型参数共享时该方法非常慢。
因此该方法很难达到实际场景中所需的实时性要求。

\subsection{一阶段方法}
随着深度学习中目标检测、多目标跟踪和多任务学习~\cite{ranjan2017hyperface,kokkinos2017ubernet} 的发展,多目标跟踪研究的一个趋势是将目标检测和目标跟踪组合在一个单独的处理框架中。
主要的思想是在一个单独的模型中利用参数共享减少模型的运行时间,以达到同时进行目标检测和数据关联。
例如,TrackR-CNN~\cite{voigtlaender2019mots} 在 Mask-RCNN~\cite{he2017mask} 的基础之上添加了一个再识别分支来预测边界框和目标特征。
基于 YOLOv3~\cite{redmon2018yolov3} 的 JDE~\cite{jde} 获得了接近视频帧率的跟踪速度。
FairMOT~\cite{fairmot} 发现基于基于锚框的检测器预测出的目标边界框可能会和实际的目标中心没有对齐,这将会产生严重的歧义和许多的身份切换。

然而目前的一阶段多目标跟踪方法没有实现完全端到端的模型,仍然会导致检测器的误差传播到数据关联步骤,不能进行两个任务的联合优化,
为了进一步解决这个问题,该工作提出了一个联合目标检测和目标关联的真正端到端的跟踪方法,在某种程度上提高了在线多目标跟踪的精度和速度。


\section{端到端跟踪框架}
为了介绍所提出的算法,首先介绍网络的处理流程,
然后描述检测子模块的信息,
再介绍所提出的连接子模块和保持数据一致性的策略。
最后,提出关联子模块和在线跟踪策略来进行端到端多目标跟踪。
本章使用了以下符号。

\begin{itemize}
	\item $ A $ 表示关联矩阵,它指定当前帧 $F_t$ 中所有目标与历史帧 $F_{t-n}$ 中所有目标之间的关联概率。
	\item $B_{t-n, t}$ 是作为历史轨迹真实标签和当前帧真实标签之间的二进制关联矩阵。
	\item $D$ 表示偏移头的输出。
	\item $E$ 表示表征头生成的表征图。
	\item $F$、$F_t$ 和$F_{t-n}$ 分别代表任意帧、当前帧和前 $n$ 帧。
	\item $M_t$, $M_{t-n}$, $M_{t-n,t}$ 和 $M_a$ 分别表示当前帧中的表征张量,前 $n$ 帧中的表征张量,当前帧和前 $n$ 帧之间的混淆张量,以及关联矩阵。
	\item $N_m$ 表示每个视频帧中跟踪目标的最大数量。
	\item $R_t$ 和 $R_{t-n}$ 分别是当前帧和前 $n$ 帧中的表征矩阵。
	\item $S_D$、$S_J$ 和 $S_A$ 分别代表目标检测子模块、检测跟踪连接子模块和数据关联子模块。
\end{itemize}


\subsection{方法流程}
本章所提出的联合检测和关联的多目标跟踪流程如图~\ref{fig:jdan_pipeline} 所示。
由分隔 $n$ 个时间戳的一对视频帧 $F_t$ 和 $F_{t-n}$ 被输入主干网络中。
两个输入视频帧被共享参数的双流检测网络所处理,其中每个流是一个检测子模块,通过它们来学习鲁棒且高分辨率的目标表征。
%
骨干网络中的数字表示相对于原始特征的比例,
后面附加了定位头和表征头,用于预测目标边界框和目标表征 $R_t$。
%$R_t$ 是为关联子模块学习的,如 章节~\ref{sec:association_submodule} 中所述。
%
所有目标表征都连接起来形成表征矩阵 $R_t \in \mathbb{R}^{128 \times N_m}$ 和 $R_{t-n} \in \mathbb{R}^{128 \times N_m}$(没有目标的位置占位符用零进行填充),其中 $N_m$ 是输入帧中所允许的最大目标的数量。
然后,将 $R_t$ 和 $R_{t-n}$ 分别沿垂直和水平方向复制 $N_m$ 次,形成 $M_t \in \mathbb R^{128 \times N_m \times N_m}$ 和 $M_ {t-n} \in \mathbb R^{128 \times N_m \times N_m}$。
这些表征张量 $M_t$ 和 $M_{t-n}$ 的一对一组合被连接成一个混淆张量 $M_{t-n,t} \in \mathbb R^{(128+128) \times N_m \times N_m}$。
随后,使用关联预测器将 $M_{t-n,t}$ 转换为关联矩阵 $M_a \in \mathbb R^{N_m \times N_m}$。
同时,通过使用所获得的关联矩阵来回顾历史视频帧并执行在线多目标跟踪。
此外,利用现有的目标检测和数据关联方法来设计满足要求的子模块。
如图~\ref{fig:jdan_consistency}~所示,为了利用检测子模块,对应于基准数据集的未知检测网络实现与提出的检测子模块之间存在许多不一致。
为了解决检测子模块的输出和关联子模块的输入之间边界框位置和大小不一致的问题,首先在训练的第一阶段中训练检测子模块,
然后使用经过训练的检测子模块生成所有边界框,
并利用传统的关联方法在视频中产生轨迹。
因此,在训练的第二阶段中,通过固定检测子模块的参数,可以利用上一步的输出来训练所提出的关联子模块。
然后继续阶段一和阶段二的循环训练迭代,直到损失函数收敛。
最后,在线多目标跟踪是通过使用关联子模块的输出将当前帧与几个单独的历史帧相关联来执行的。
因此,这些策略可以实现两个拆分子模块之间的数据一致性,以完成训练和端到端实时检测跟踪。


\begin{figure*}[ht]
	\centering
	\includegraphics[width=1.0\textwidth]{./figures/C5Fig/pipeline.pdf}
	\vspace{0.2em}
	\caption{联合检测关联的网络架构图}
	\label{fig:jdan_pipeline}
\end{figure*}

\subsection{目标检测子模块}
\label{sec:detection_submodule}
检测子模块将单个视频帧 $F \in \mathbb{R} ^ {W \times H \times 3}$ 作为输入并获得每个视频帧的目标边界框和相应的表征。
特别是,在主干模型中添加了两种类型的预测头。
使用定位头来定位目标边界框。
此外,如图~\ref{fig:jdan_pipeline} 所示,表征头用于计算目标表征,
将其输入到 JDAN 的后半部分以获得每对视频帧的关联矩阵 $M_a$。

\subsubsection{主干网络}
\label{sec:backbone}
主干网络对于多目标跟踪任务至关重要,因为目标表征需要同时利用低分辨率和高分辨率表征来适应各种尺度的跟踪目标。
FairMOT~\cite{fairmot} 注意到深层聚合有利于减少一阶段方法的身份切换次数,因为编解码网络可以有效处理不同的目标大小。
然而,深层聚合在两阶段方法中并不重要,因为边界框通过裁剪和缩放将具有相同的大小。

在本研究中,为了同时考虑模型复杂性和精度,采用了 ResNet-34~\cite{resnet}。
如图~\ref{fig:jdan_pipeline} 所示,利用深层聚合~\cite{point} 的变体作为检测子模块的主干来适应各种尺度的目标。
与原始深层聚合~\cite{dla} 相比,它在低层和高层表征之间有额外的旁路。
另外,上采样过程中的所有卷积块都被可变形卷积模块~\cite{deformable}所替换,因为它可以自适应地适应目标尺寸的变化,
这些修改同样有利于缓解对齐问题。


\subsubsection{定位头}
\label{sec:detection_head}
定位头的输入是骨干网络的输出表征。
每个定位头使用大小为 $3\times3$ 的卷积核和 $256$ 输出通道,然后是 $1\times1$ 卷积以产生定位输出。
具体来说,它会生成一个低分辨率的位置和大小。

首先,使用热力图头来预测目标中心。
当它与真正的中心目标位置重叠时,该头部在某个位置的输出为 $1$。
输出值随着到目标中心位置的距离增加而减小~\cite{cornernet}。
对于视频帧中的真实边界框 $b^i = (x_1^i,y_1^i,x_2^i,y_2^i)$,目标的中心位置为
$ p^i = (\frac{x_1^i+x_2^i}{2}, \frac{y_1^i+y_2^i}{2})$。
因此,通过将中心位置除以下采样因子来计算表征图上的位置:$q^i = \lfloor \frac{p^i}{G} \rfloor $,其中 $G=4$。
形式上,位置 $q \in \mathbb{R}^2$ 处的热图响应定义为:
$r_{q} = \mathop{max}\limits_{i} ( \mathrm{exp}^{-\frac{(q - q^i)^2}{2\sigma ^2}} ) $,
其中 $\sigma$ 是高斯核,它是目标大小的函数~\cite{cornernet}。
根据焦点损失~\cite{lin2017focal} 设计热图损失函数 $ \mathcal{L}_{h} $ 作为训练目标:
\begin{align}
\mathcal{L}_{h} = -\frac{1}{N} \sum _{q} \begin{cases} (1-\hat{r}_{q})^\alpha \text{log}(\hat{r}_{q}), & \text{如果 } r_{q}=1 \\ (\hat{r}_{p})^\alpha \text{log}(1-\hat{r}_{q}) (1-r_{q})^\beta, & \text{否则}
\end{cases}
\end{align}
其中 $N$ 表示当前帧中目标的数量,$\hat{r}_{q} \in [0,1]^{\frac{W}{G} \times \frac{H}{ G} \times C_h}$ 是位置 $q$ 处的预测热图响应,类别号 $C_h=1$ 和 $\alpha, \beta$ 是焦点损失的超参数。

尺寸头用于预测目标围绕其中心位置的宽度和高度。
尺寸头的输出定义为: $\hat{Z} \in \mathbb{R}^{\frac{W}{G} \times \frac{H}{G} \times C_z} $,其中类别号 $C_z=2$ 表示宽度和高度。
虽然定位精度与目标表征没有直接的关系,但它会影响检测子任务的性能。
对于视频帧中的一个真实框 $b^i$,可以根据 $z^i = (x_2^i-x_1^i, y_2^i-y_1^i)$ 得到框的大小,
并且预测的边界框大小定义为 ${\hat{z}}^i$。

此外,FairMOT~\cite{fairmot} 表明具有中心位置的细化边界框对于提高多目标跟踪精度很重要。
骨干网络中的下采样因子 $G$ 将发挥巨大的量化效果。
偏移头用于更准确地检测目标。
虽然检测精度提升的优势微乎其微,但是多目标跟踪中的目标表征是基于极其精确的边界框学习的,因此在这里引入偏移头,
将偏移头的输出表示为 $\hat{D} \in \mathbb{R}^{\frac{W}{G} \times \frac{H}{G} \times C_d} $,其中类别号 $C_d=2$。
表征图上的真实位移表示为: $d^i = \frac{p^i}{G} - \lfloor \frac{p^i}{G} \rfloor $。
将中心位置位移表示为 ${\hat{d}}^i$。
因此,表示尺寸头和偏移头的 $L_1$ 损失表示为:
\begin{equation}
\mathcal{L}_{s} = \frac{1}{N} \sum_{i=1}^{N} \|z^i - \hat{z}^i\|_1 + 
\frac{1}{N} \sum_{i=1}^{N} \|d^i - \hat{d}^i\|_1.
\end{equation}

因此,定位损失 $\mathcal{L}_{p}$ 表示为前两个损失的组合:
\begin{equation}
\mathcal{L}_{p} = \mathcal{L}_{h} + \mathcal{L}_{s}.
\end{equation}


\subsubsection{表征头}
表征头的目的是提取可以区分各种跟踪目标的外观表征。
在理想情况下,不同身份的目标之间的差异大于同一身份目标之间的差异。
为了实现这一目标,骨干网络的输出为检测目标的表征,
生成的表征图为 $E \in \mathbb{R}^{\frac{W}{S} \times \frac{H}{S} \times C_e}$,其中输出通道 $C_e=128$。
通过表征头学习在中心位置 $p$ 的目标的表征 $E_{p}\in\mathbb{R}^{C}$。
将跟踪目标识别视为分类问题。
同时训练数据集中所有相同身份的目标都被视为一个标签。
对于视频帧中的真实框 $b^i$,获得了热图上的目标中心位置 $\hat{p}^i$。
在某个位置学习一个身份表征 $E_{p^i}$ 并输出到一维分类概率向量 $v(k)$,
并将真实分类标签表示为 $u^i{(j)}$。
因此,身份分类损失被构造为:
\begin{equation}
\mathcal{L}_{c} = \frac{1}{N \times J} \sum_{i=1}^{N} \sum_{j=1}^{J} u^i{(j)} \text{log}(v(j)),
\end{equation}
其中 $J$ 是数据集中所有身份的数量。

最后,总的检测损失 $\mathcal{L}_{d}$ 表示为前两个损失的组合:
\begin{equation}
\mathcal{L}_{d} = \mathcal{L}_{p} + \mathcal{L}_{c}.
\end{equation}



\subsection{连接子模块和数据一致性}
JDAN 训练输入是没有目标边界框的当前帧 $F_t$ 和历史帧 $F_{t-n}$。
此外在关联子模块 $S_A$ 的训练中,JDAN 需要一个真实的二进制关联矩阵 $B_{t-n, t}$ 作为历史帧和当前帧之间的真实标签来计算关联损失。
在图~\ref{fig:jdan_pipeline}~的最左边显示了一对 JDAN 的输入图像帧。
下面描述连接子模块的细节和所需要解决的数据一致性问题。


\subsubsection{连接子模块}
在目标检测子模块 $S_D$ 和关联子模块 $S_A$ 之间,所提出的连接子模块 $S_J$ 将当前帧中的目标表征 $R_t$ 沿垂直方向复制到张量 $M_t \in \mathbb{R}^{128 \times N_m \times N_m}$,
并将历史帧中的目标表征 $R_{t-n}$ 沿水平方向复制到张量 $M_{t-n} \in \mathbb{R}^{128 \times N_m \times N_m}$。
随后如图~\ref{fig:jdan_pipeline} 所示,目标表征 $M_t$ 和 $M_{t-n}$ 沿着目标表征的通道方向合并到 $M_{t-n,t} \in \mathbb{R}^{(128 + 128) \times N_m \times N_m}$。
注意到,垂直和水平复制用于尽可能多地将两组目标进行排列组合,这确保了历史帧 $F_{t-n}$ 中的目标可能与当前帧 $F_t$ 中的所有目标相关联,反之亦然。
然后通过包含的关联预测器五个卷积块~\cite{inception} 将扩展的混淆矩阵 $M_{t,t-n}$ 转换为关联矩阵 $M_a \in R^{N_m \times N_m}$。
在表~\ref{tab:compression_net} 中详细描述了关联预测器的有关信息。
%I.C 是每一层的通道数目, 
%O.C 表示输出通道的数目, 
%$Stride$ 表示步长的大小, 
卷积核是 $1 \times 1$ 的卷积核来压缩维度,卷积核的大小表示感受野的大小; 
BN (Y/N) 表示是否使用批量正则化;
ReLU (Y/N) 表示是否使用 ReLU。
步长和填充在空间维度上都是相同的,卷积核的步长代表提取的精度。

\begin{table}[t]
	\centering
	\tabcolsep=3.5pt
	\caption{关联预测器压缩网络框架的详细信息}
	\label{tab:compression_net}
	\tabcolsep=0.15cm
	\begin{tabular}{c|cccccccc}
%		\hline
		\hline
%		\toprule[1.5pt]
		{子模块}	&{索引} &{输入通道数} &{输出通道数} &{卷积核} &{步长} & {填充} &{BN} &{ReLU} \\
		\hline
%		\midrule[1.5pt]
		\multirow{2}{*}{}
		&	1     & 1024  & 512  	& $1 \times 1$ 	& 1 & 0 &	Y	&	Y\\
		&	2     & 512   & 256   	& $1 \times 1$	& 1 & 0 &	Y	&	Y\\
		\multirow{1}{*}{关联预测器}
		&	3     & 256   & 128   	& $1 \times 1$ 	& 1 & 0 &	Y	&	Y\\
		\multirow{1}{*}{}
		&	4     & 128   & 64   	& $1 \times 1$ 	& 1 & 0 &	N	&	Y\\
		&	5    & 64    & 1    	& $1 \times 1$ 	& 1 & 0 &	N	&	Y\\
%		\bottomrule[1.5pt]
		\hline
%		\hline
	\end{tabular}%
\end{table}%


\subsubsection{训练数据的连接一致性}
在连接子模块中,所有目标表征 $R_t$、$R_{t-n}$ 都来自检测子模块,并且可能与多目标跟踪基准数据集上的跟踪真实值存在数据不一致。
因此,很难进行端到端的模型训练。
为了解决这个问题,在该研究中不使用数据集中跟踪的真实值,采用了一种简单而有效的传统关联方法,称为卡尔曼滤波器~\cite{welch1995introduction}来预测轨迹的位置,从而生成目标表征 $R_t$ 和 $R_{t-n}$ 之间的伪关联标签。
根据伪标签得到一个伪关联矩阵 $B_{t-n,t} \in \mathbb{R}^{(N_m+1) \times (N_m+1)}$,其中每个元素 $b_{k,l}$ 表示目标 $k$ 和 $l$ 之间的匹配关系,增加一列/行($B_{t-n,t}$记为“+1”)表示对象消失/出现在当前帧中。
为 $b_{k,l}$ 定义了三个值:$1$ 表示目标 $k$ 和 $l$ 之间的相同身份(称为“伪正对”),$0$ 表示不同的身份(称为“伪负对”) "),$0.5$ 表示不确定。
在具体的实现中,为卡尔曼滤波器中设置了高阈值以减少伪正对的错误匹配,并设置低阈值以增加伪负对的真实不匹配,余下的配对设置为不确定。

%多目标跟踪基准数据集中的真实边界框和所提出的的检测子模块的输出结果在位置和大小上不一致,
%并且在端到端多目标跟踪中所需的多目标跟踪基准数据中的相应检测器模型参数也无法获取到。
%因此如图~\ref{fig:jdan_consistency}所示,利用经过训练的检测子模块提供的边界框和身份轨迹信息以及多目标跟踪基准数据集中的传统关联方法来解决如的真实值边界框不一致的问题。
%有许多不一致的边界框,包括边界框真实值~\cite{dpm,faster,sdp}和所提出的的检测模块在多目标跟踪基准数据集中的输出之间的数量和位置。
%虽然多目标跟踪基准数据集中提供的训练数据缺乏这些检测模型的实现,但有必要实现一个阶段的多目标跟踪,如章节~\ref{sec:two_stage} 和一阶段多目标跟踪。
%
%首先,使用预训练的检测子模块和传统的关联方法在多目标跟踪数据集中生成一系列轨迹。
%然后,利用这些轨迹结果形成二进制关联矩阵 $B_{t-n,t}$ 跟随 章节~\ref{sec:similarity_loss},
%然后利用它来训练关联子模块。
%最后,使用两阶段训练策略,如章节~\ref{sec:two_stage} 中描述的那样训练整个JDAN,在检测子模块的输出和关联子模块。


\subsection{关联子模块}
\label{sec:association_submodule}
JDAN 中的目标关联子模块 $S_A$ 的目的是使用连接的混淆张量计算 $F_{t-n}$ 和 $F_t$ 这两个目标组之间的关联 $M_{t-n,t}$。

\subsubsection{关联预测器}
\label{sec:similarity_estimator}
如表~\ref{tab:compression_net}所示,关联预测器的结构是根据 $M_{t-n,t}$ 和 $M_a$ 的实际含义设计的。
该模块将目标表征的组合 $M_{t-n,t}$ 转换为关联矩阵 $M_a$,表示这些帧间跟踪目标的关联信息~\cite{dan}。
因此,它沿着目标表征的方向使用卷积核大小为 $1\times 1$ 的卷积逐步实现了从 $256$ 到 $1$ 的维度压缩,同时它不会相互影响特征图中的相邻通道。



\begin{figure*}[ht]
	\centering
	\includegraphics[width=1.0\textwidth]{./figures/C5Fig/loss.pdf}
	\vspace{0.2em}
	\caption{目标消失和出现的处理}
	\label{fig:jdan_loss}
\end{figure*}


\subsubsection{关联矩阵} \label{sec:association_matrix}
如图~\ref{fig:jdan_pipeline} 的后半部分所示,通过利用所提出的关联子模块获得帧间关联,并利用每帧中允许的最大目标数 $N_m$ 预测 $F_{t-n}$ 和 $F_t$ 之间的目标关联矩阵 $M_a$。
如章节~\ref{sec:maximum_object} 中所述,在本研究中 $N_m = 150$ 是多目标跟踪数据集的单帧中目标数目上限。
沿水平和垂直方向在目标相似性关联矩阵 $M_a$ 中插入零向量(作为目标占位符)以进行泛化。
这些零出现在 $F_{t-n}$ 和 $F_t$ 之间,因此任何视频帧最终都由 $N_m$ 个目标组成,并且 $M_a$ 的形状是 $N_m \times N_m$。

$M_a$ 中的行表示历史帧中的目标,其中的列表示当前帧中的目标。
$M_a$ 表示具有水平和垂直目标占位符的两个视频帧的关联矩阵。
在 $M_1$ 和 $M_2$ 中,在末尾矩阵附加了一个额外的水平和垂直向量,称为未识别的目标~\cite{dan}。
如图~\ref{fig:jdan_loss} 所示,最后附加的垂直向量负责建模从历史帧 $F_{t-n}$ 中消失的当前跟踪目标,最后一行中附加的水平向量负责建模在当前帧 $F_t$ 中进入视野的新目标。
对于输入的历史帧和当前帧,JDAN 预测得到相似性关联矩阵 $M_a$。
考虑到历史帧和当前视频帧之间的多目标消失和出现,通过向 $M_a$ 添加额外的列和行来设计 $M_1$ 和 $M_2$。
然后,分别对 $M_1$ 和 $M_2$ 进行水平和垂直 softmax 操作,以保证出现和消失的总概率都为 1,裁剪后的矩阵 $A_c$ 和 $A_r$ 用于与真实关联矩阵计算损失 $ L_m $。
最后,总关联损失 $\mathcal{L}_s$ 由 $\mathcal L_m$、$\mathcal L_b$ 和 $\mathcal L_d$ 所构成。
所以所提出的网络可以表示相机视野中多个目标消失和出现。
例如,可以在最后附加的垂直向量中的一行,从 1 变成 0 来表示消失,
并在最后附加的水平向量的列处,从 1 变成 0 表示出现。



\subsubsection{消失和出现} \label{sec:similarity_loss}
可以计算预测的关联矩阵 $M_a$ 和真实的二进制关联矩阵 $B_{t-n,t} \in \mathbb{R}^{(N_m+1) \times ( N_m+1)}$。
关联矩阵的标签最终用于训练所提出的的关联子模块 $S_A$。
然而,$M_a$ 忽略了历史和当前帧之间的目标消失和出现。
因此,利用历史帧和当前帧之间的相似性关联编码来考虑多个目标的消失和出现。

如图~\ref{fig:jdan_loss} 顶部所示,考虑到目标消失,在 $M_a$ 后附加一列以构建 $M_1 \in \mathbb R^{N_m \times (N_m + 1)}$。
扩展矩阵 $M_{1}$ 的第 $m^{\text{th}}$ 行将帧 $F_{t-n}$ 中的 $m^{\text{th}}$ 目标与 $F_t$ 帧中 $N_m+1$ 个的目标进行关联,其中 $+1$ 表示当前帧 $F_t$ 中未检测到的目标。
然后,通过执行 softmax 操作~\cite{train_mot} 对 $M_1$ 的水平方向上的扩展概率向量进行归一化。
因此,输出关联矩阵 $A_{1} \in \mathbb R^{N_m \times (N_m +1 )}$ 的水平向量表示视频帧 $F_{t-n}$ 中所有目标与在视频帧 $F_t$ 中的所有目标之间的关联概率,包括当前视频帧中未识别的目标。
同理如图~\ref{fig:jdan_loss}~底部所示,目标外观是通过向 $M_a$ 附加一行以构建 $ M_2 \in \mathbb R^{ (N_m + 1) \times N_m}$ 而形成的。
然后,对 $M_2$ 执行垂直 softmax 操作得到 $A_2 \in \mathbb R^{(N_m +1) \times N_m}$,其列表示来自视频帧 $F_{t-n}$ 到 $F_t$ 的关联概率~\cite{train_mot}。


此外,向目标关联矩阵 $M_a$ 添加了额外的列和行以获得可理解的损失设计。
关联矩阵 $M_a$ 添加的向量是 ${\bf u} \in \mathbb R^{N_m} = \lambda {\bf 1} $,其中 $\lambda$ 是超参数,${\bf 1}$ 是一个全为 1 的单位向量~\cite{dan}。
添加向量的这种设计意味着所有跟踪目标都有消失或出现的概率。
此外,二进制关联矩阵 $B_{t-n,t}$ 以相同的方式实现。

具体来说,使用方向性损失 $\mathcal L_{d}$ 来抑制消失和出现的错误目标关联:
\begin{align}
\mathcal L_d = \frac{\sum_{i=1}^{N_m} \sum_{j=1}^{N_m+1} \left( \left(-\log { A_1} \right) \odot {B}_1 \right)}{ \sum_{i=1}^{N_m} \sum_{j=1}^{N_m+1} B_1 }  
\notag  \\
+ \frac{ \sum_{i=1}^{N_m+1} \sum_{j=1}^{N_m} \left( \left(-\log { A_2} \right) \odot {B}_2 \right)}{\sum_{i=1}^{N_m+1} \sum_{j=1}^{N_m} B_2},
\end{align}
其中 $B_1$ 和 $B_2$ 分别通过删除 $B_{t-n,t}$ 的最后一个水平和垂直向量来进行定义,
运算符 $\odot$ 表示哈达玛积~\cite{hadamard},
\textit{log} 函数作用于参数中的每个元素。

此外,利用非极大值损失和平衡损失来训练关联子模块~\cite{dan}。
非极大值损失 $\mathcal L_{m}$ 在关联计算的消失和出现中惩罚非最大关联:
\begin{equation}
\mathcal L_m = \frac{ \sum_{i=1}^{N_m} \sum_{j=1}^{N_m} \left( \left(-\log A_m \right) \odot {B}_3 \right)}{\sum_{i=1}^{N_m} \sum_{j=1}^{N_m} B_3}.
\end{equation}
同理,$B_3$ 是同时删除 $B_{t-n,t}$ 的最后一个垂直向量和最后一个水平向量,
$A_m = max (A_c, A_r)$。
\textit{max} 函数也会作用于输入参数的每个元素。
对损失 $L_m$ 进行 $max$ 操作以获得 $A_c$ 和 $A_r$ 中的最大值,
如图~\ref{fig:jdan_loss}~所示,其中 $A_c$ 和 $A_r$ 分别表示矩阵 $A_1$ 和 $A_2$ 通过删除最后一个垂直向量和最后一个水平向量被裁剪到 $N_m \times N_m$ 的维度。
在视频过程中出现的目标数和消失的目标数应该相等,所以平衡损失 $\mathcal L_b$ 惩罚消失和出现之间的任何不平衡:
\begin{equation}
\mathcal L_b =  \sum_{i=1}^{N_m} \sum_{j=1}^{N_m} \lvert A_c^{ij} - A_r^{ij} \rvert. 
\end{equation}

最后,将总关联损失 $\mathcal L_s$ 定义为上述三项损失的总和:
\begin{equation} \label{equ:association_loss}
\mathcal L_{s} = \mathcal L_d + \mathcal L_m + \mathcal L_b.
\end{equation} 
在这里,关联子模块的训练目标是最小化关联损失 $\mathcal{L}_s$。
因此,上述三个损失是有效的,并拟合了真实目标关联。


\subsection{端到端跟踪}
这一部分描述所提出的端到端模型的训练和用法,以及在没有输入边界框的视频序列中执行多目标跟踪的详细步骤。

\subsubsection{两阶迭代段训练}
\label{sec:two_stage}
在该研究中采用迭代训练过程来训练所提出的模型,包括两个步骤:
首先,在几个检测数据集上采用预训练的目标检测模型~\cite{zhang2017citypersons,xiao2017joint,zheng2017person},并根据检测损失 $\mathcal{L}_{d}$ 对其参数进行微调;
其次,利用卡尔曼滤波器~\cite{welch1995introduction} 进行跟踪,得到目标的轨迹和身份标签,以及当前检测结果和历史轨迹之间的伪标签,
并根据方程~\ref{equ:association_loss} 中的 $\mathcal{L}_{s}$ 更新检测子模块和关联子模块。
反复重复上述两个步骤,直到损失 $\mathcal{L}_{s}$ 收敛。
与以前的方法相比,检测误差可以反向传播回来以更新检测子模块和关联子模块,
因此所提出的训练方法可以实现端到端模型训练以缓解误差传播的问题。

%为了验证基于之前的在线多目标跟踪研究所提出的模型 JDAN,它包含两个可训练的子模块,名为检测子模块 $S_D$ 和关联子模块 $S_A$。
%最后,执行端到端推理。
%首先如~\ref{sec:detection_submodule}节所示,使用边界框框和身份信息和检测损失 $\mathcal{L}_{d}$ 来训练定位头和表征头。
%其中一些数据没有身份信息只有真实边界框,但它们可以被用来训练定位头。
%
%然后,根据预训练的检测子模块生成的边界框对多目标跟踪数据集执行传统的数据关联方法,
%并利用输出生成二进制关联矩阵 $B_{t-n,t}$ 作为训练关联子模块 $S_A$ 的真实数据。
%在第二个训练阶段,通过 $S_D$ 固定检测子模块 $S_D$ 的参数与生成的边界框一致,
%并使用关联子模块 $S_J$ 产生的混淆矩阵和传统数据关联方法生成的二进制关联矩阵训练 $S_A$。
%由于使用了预训练的检测子模块和固定参数,第二个训练阶段和端到端推理阶段生成的表征和框是相同的。
%这些策略确保检测子模块生成的边界框和目标表征在第二个训练阶段在 $S_D$ 和 $S_A$ 之间是一致的。


\label{sec:dep}
\subsubsection{模型预测}
尽管 JDAN 中的目标检测子模块在训练时为两个并行分支,但在线多目标跟踪使用的是同一个网络。
这样做是合理的,因为两个并行检测子模块的权重相互共享。
如图~\ref{fig:tracking} 所示,所提出的网络的预测通过有序的方式呈现主要模块。
JDAN 的输入是一个视频帧 $F_t$,尺寸大小为 $1088 \times 608$。
基于根据章节~\ref{sec:detection_head} 预测出的热力图,由热图响应进行非极大值抑制操作以获得最强点,
选择热图最强响应超过极限值的位置。
然后,根据模型推断结果的大小和偏移量预测相应的框。

此外,检测子模块中的表征头学习当前帧和历史帧的目标表征 $R_t$ 和 $R_{t-n}$。
复制并组合这两个表征矩阵为这两个图像的混淆张量 $M_{t-n,t}$。
然后如章节~\ref{sec:similarity_estimator} 所示,张量 $M_{t-n, t}$ 通过关联预测器的前向传播转换为关联矩阵 $M_a$。

\begin{figure*}[ht]
	\centering
	\includegraphics[width=1.0\textwidth]{./figures/C5Fig/tracking.pdf}
	\vspace{0.2em}
	\caption{JDAN 执行在线多目标跟踪的过程}
	\label{fig:jdan_tracking}
\end{figure*}

\subsubsection{在线跟踪}
在初始化跟踪过程并在第一帧中生成表征 $R_0$ 后,可以预测出当前帧目标和响应历史帧目标表征之间的关联矩阵 $M_{t-n:t-1,t}$。
如图~\ref{fig:jdan_tracking} 所示,推断出的关联矩阵通过回顾历史帧信息来刷新先前的跟踪结果。
在输入视频帧 $F_t$ 中,通过 JDAN 中具有单个流检测子模块,并由定位头和表征头分别得到目标框和目标表征。
目标表征 $R_t$ 用于与最近的 $n$ 个历史表征 $R_{t-n:t-1}$ 进行配对,并且每对表征都通过关联预测器来估计相应的关联矩阵 $M_{t-n:t-1,t}$。
此外,表征 $R_t$ 保存到轨迹记录器中用于估计下 $n$ 帧中的关联矩阵。
最后,通过使用预测的关联矩阵将当前图像目标与 $n$ 个历史图像目标联系起来并更新轨迹记录器。

在这里描述了图~\ref{fig:jdan_tracking} 所示的详细在线多目标跟踪过程。
记录器 $T_0$ 具有相同数量的轨迹,因为识别的目标在第一个视频帧 $F_0$ 中被初始化。
此外,每条轨迹都是一个键值对的记录器,其中的每一项都包含视频帧索引和目标表征。
使用 Kuhn-Munkres 方法更新当前帧的轨迹记录器~\cite{Munkres1957},并通过最大化当前目标和历史轨迹的关联,来进行关联推断。
此外,将其记录到累加器 $Y_{t}$ 中。
累积器矩阵中的每个元素都是历史轨迹记录器 $T_{t-1}$ 中的目标与当前帧中的目标的相似度总和。

因此,视频中的每一帧图像仅使用检测子网络提取一次特征。
然而,目标表征被保存并重复使用多次来评估与剩余图像的相似性。
所以基于关联矩阵可能将许多轨迹分配给累积器矩阵中的特定未检测目标列。
这个问题是通过复制累加器矩阵的最后一列来解决的,直到 $T_{t-1}$ 中的每个估计都分配给唯一的一列~\cite{dan}。
因此,该策略能使每个未检测到的轨迹与未检测到的目标相关联。

总之,所提出的检测跟踪器是一种在线多目标跟踪方法,它不利用任何未来信息来预测目标轨迹。
因此,它可以应用于在线应用中。
一个潜在的问题是过长的轨迹可能会导致大量的存储和计算成本。
因此,阈值 $\mu_m$ 用于限制在现有轨迹中查看的历史帧数。
如果轨道中的图像数量超过 $\mu_m$,则最远的目标信息将被丢弃。
此外,如果跟踪目标从视野中消失超过 $\mu_r$ 帧,它将从跟踪列表中删除~\cite{train_mot}。
在提出的多目标跟踪方法中使用这些有物理含义的参数,可以根据运行时内存和计算资源的限制进行更改。

 
\section{实验}
在这一节分三步展示所提出的 JDAN 实验和结果。
首先,介绍了所使用的数据集和所提出模型的实现细节。
其次,分析了不同训练方法在同一多目标跟踪数据集上的性能。
最后,将所提出的方法与经典和最新的多目标跟踪方法进行比较。

\subsection{数据集和度量标准}
基于已有的研究~\cite{jde,fairmot},通过合并来自多个行人检测数据集的训练数据来利用一个大型训练集来训练所提出的检测子模块。
CityPersons~\cite{zhang2017citypersons} 和 ETH~\cite{ess2008mobile} 数据集仅提供框信息,仅可以使用这些数据集来训练定位头。
CUHK~\cite{xiao2017joint}、Caltech~\cite{dollar2009pedestrian}、MOT16~\cite{mot16} 和 PRW~\cite{zheng2017person} 提供了行人边界框和身份信息,便可以联合训练定位头和表征头。
最后,基于训练好的检测子模块和传统的关联方法~\cite{kernelized,fairmot},通过执行多目标跟踪过程构建训练数据来训练目标关联子模块。

在两个不同的多目标跟踪基准数据集 MOT15 和 MOT17 上对所提出方法的各个组件进行了大量的测试。
MOT17 包含七个训练视频和七个测试视频。
MOT16 包含与 MOT17 数据集相同的视频序列。
而 JDAN 的输入是没有检测信息的纯图像。
因此,该研究中只使用 MOT17 数据集。
另外,使用标准多目标跟踪准确度(MOTA)和 多目标跟踪精度(MOTP),
而 IDF1~\cite{ristani2016performance} 综合了 ID 准确率和 ID 召回率。
评估指标还包括 ID\_Sw、ID Precision(IDP)、误报总数(FP)、遗漏目标总数(FN)和碎片轨迹总数(Frag)~\cite{clear}。
在多目标跟踪基准测试中利用这些指标来衡量多目标跟踪性能。

\subsection{实现细节}
\label{sec:implementation_details}
在该研究中使用 PyTorch 1.2.0~\cite{pytorch} 实现所提出的 JDAN,在 Quadro RTX 6000 GPU 上花费 20 小时进行训练。
使用训练数据集来训练所提出的模型,并基于 MOT17 选择所提出网络的超参数。
因为基准数据集 MOT17 的数据规模不大,选择它进行超参数调整。
最后,实验中使用的超参数描述如下。

如章节~\ref{sec:backbone} 中所述,修改后的 DLA~\cite{point} 被用作检测子模块的主干。
使用在微软 COCO~\cite{lin2014microsoft} 上训练的权重初始化检测子模块。
%输入帧缩放到 $1088\times 608$。
%提议的网络的输入帧尺寸为 $1088\times608$。
在将输入帧输入到 JDAN 之前,这些训练和测试样本被重新缩放到指定的大小,表征图的大小为 $272 \times 152$。
如图~\ref{fig:jdan_consistency} 训练阶段一所示,用 SGD 优化器~\cite{sgd} 训练检测子模块 50 次迭代,初始学习率为 $1e-4$,在第 25 和 40 次迭代时乘以 0.1。
%
每帧允许的最大目标数 $N_m$ 设置为 150,最小批次大小 $B$ 设置为 4,迭代轮次设置为 160,单位向量 $\lambda$ 的乘数因子设置为 10。
将历史帧和当前帧 $N_a$ 之间的最大间隙帧数设置为 30。
然后如图~\ref{fig:jdan_consistency} 训练阶段二所示,使用 SGD 优化器~\cite{sgd} 训练关联子模块,其动量和权重衰减分别设置为 0.9 和 5e-4。
以 0.01 的学习率开始训练,并在第 60、100 和 140 次迭代时乘以 0.1。
训练检测子模块和关联子模块时,需要优化超参数 $\mu_r$ 和 $\mu_m$。
使用网格搜索技术选择两个超参数的最佳值,以在 MOT17 验证数据集上获得最佳 MOTA 性能。
利用 $[3, 30]$ 范围内以 3 为步长来构建网格。
因此,在实际跟踪过程中使用这种方法选择了 $\mu_m=15$ 和 $\mu_r=12$。


%\subsubsection{数据增强}
\label{sec:PP}
同时利用了一系列数据增强方法,例如调整图像尺寸、裁剪和像素值抖动等。
首先,使用 1.0-1.25 范围内的随机采样率增加图像帧的大小,并用多目标跟踪训练集中的平均像素值填充扩展图像中的像素。
同时,裁剪了在 0.75-1.0 随机采样范围内的视频帧。
此外,图像中的每个像素值都乘以范围 0.75-1.25 内的随机值。
输出帧转换到 HSV 空间,饱和分量乘以范围 0.75-1.25 内的随机值。
最后参照 SSD~\cite{Liu2016} 将图像变换回 RGB 空间并乘以随机因子样本。
%
注意到历史帧 $F_{t-n}$ 和当前视频帧 $F_t$ 在视频序列中不一定是连续的,
可以让它们有 $n$ 帧的分隔,其中 $n \in [0, N_a-1] $。
然而,所提出的 JDAN 用于关联连续帧中的目标。
使用跳跃的视频帧进行训练有利于在当前帧与一系列历史视频帧之间的数据关联中使用现有的多目标跟踪方法。
以 0.25 的概率对每个轨迹上的历史和当前视频帧进行采样。
然后,这些视频帧被重新调整为指定的大小 $W \times H \times 3$。
同时,使用了概率为 0.5 的水平翻转。
此外,多目标跟踪中使用的训练数据~\cite{mot16,Lyu2017} 缺乏捕捉背景变化、相机失真和许多现实效果以保持多目标跟踪鲁棒性的能力。
在所提出的跟踪方法中,至关重要的是训练数据涉及足够多的不相关跟踪属性,以增强多目标跟踪模型的鲁棒性。
%因此,对多目标跟踪训练数据集进行后续的数据增强。
%
%所有数据增强方法都在章节~\ref{sec:implementation_details} 中描述
%修改多目标跟踪训练集的方法受到~\cite{Liu2016}的启发,他修改了视频帧以增强训练过程。
%然而,使用先前报告的数据增强方法同步处理历史和当前帧~\cite{Liu2016}。



\subsection{消去实验和讨论}

\subsubsection{训练方法}

\vspace{0.5em}
\renewcommand\arraystretch{1.5}
\begin{table}[htbp]\wuhao
	\centering
	\caption{在 MOT17 基准数据上测试各种训练配置}
	\vspace{0.3em}
	\begin{tabular}
%		{p{3.0cm}<{\centering} p{1.2cm}<{\centering} p{1.0cm}<{\centering} p{1.0cm}<{\centering}
%	p{1.0cm}<{\centering}
%	p{1.0cm}<{\centering}
%	p{1.0cm}<{\centering}
%	p{1.0cm}<{\centering}}
		{c|ccccccc}
%		\toprule[1.5pt]
%		\hline
		\hline
		方法 & MOTA$\uparrow$ & MOTP$\uparrow$ & IDF1$\uparrow$ & MT$\uparrow$ & ML$\downarrow$ &  ID\_Sw$\downarrow$ & Frag$\downarrow$\\
%		\midrule[1.5pt]
		\hline
		{基准模型} & 20.7 & 38.3 & 38.2 & 17.6 & 48.8 & 24,875 & 6,731\\
		{预训练的检测模型} & 34.6 & 42.8 & 40.4 & 19.9 & 46.7 & 18,264 & 4,084\\
		{精调的检测模型} & {42.9} & {52.8} & {48.3} & {21.5} & {45.8} & {13,236} & {3,387}\\
		{精调的关联模型} & {53.0} & {65.2} & {49.7} & {23.4} & {44.3} & {11,875} & {2,845}\\
		JDAN & {\bf58.1} & {\bf79.8} & {\bf59.2} & {\bf27.7} & {\bf32.9} & {\bf6,129} & {\bf1,515}\\
%		\hline
		\hline
%		\bottomrule[1.5pt]
	\end{tabular}
	\label{tab:jdan_training_methods}
\end{table}

在这个阶段,通过使用所提出的几个组件来评估训练过程的效果。
如表~\ref{tab:jdan_training_methods} 所示,执行了五种消去实验,包括基准模型、{预训练的检测模型}、{精调的检测模型}、精调的关联模型和所提出的 JDAN。

基准模型首先采用在微软 COCO~\cite{lin2014microsoft} 上预训练的检测子模块,然后使用生成的伪标签进行关联子模块的微调。
在此基础上,预训练的检测模型表示在检测数据集上重新训练{检测子模块},然后使用在生成的伪标签上训练的{关联子模块}。
考虑到多目标跟踪和检测数据集之间的视觉差距,在多目标跟踪检测数据集上对检测子模块进行微调来提高检测性能,
而精调的关联模型不对{检测子模块}使用多目标跟踪数据集,而是在多目标跟踪数据集上微调{关联子模块}。
JDAN 是在多目标跟踪数据集上对{检测子模块}和{关联子模块}进行端到端训练的整个模型。

表~\ref{tab:jdan_training_methods} 报告了上述训练范式之间的性能比较结果,分析如下:
\begin{enumerate} 
	\item 与{基准模型}相比,{预训练的检测模型}实现了显著的性能提升,MOTA 从 $20.7\%$ 增加到 $34.6\%$,MOTP 从 $38.3\%$ 增加到 $42.8\%$。
	这种改进表明在检测数据集上重新训练{检测子模块}的有效性,因为微软 COCO 上的预训练无法为多目标跟踪任务中的拥挤场景提供准确的行人结果。
	
	\item {精调的检测模型}与{预训练的检测模型}相比进一步提升了 MOTA 性能,MOTP 从 $42.8\%$ 上升到 $52.8\%$,ID\_Sw 从 $18,264$ 降低到 $13,236$。
	这种改进源于在{检测子模块}上使用多目标跟踪数据集来弥合检测和多目标跟踪数据集之间的视觉差距。
	同时,由于使用多目标跟踪数据集更新{关联子模块},{精调的关联模型}也实现了与{精调的检测模型}相同的改进。
	
	\item 从实验结果可以看出 JDAN 以最佳的效果改进了所有指标。
	这种增强不仅归功于精调的{检测子模块},还归功于精调的{关联子模块}。
	具体来说,JDAN 在 MOTA 上达到了 $58.1\%$,这反映了跟踪的准确性。
	如前所述,JDAN 采用{端到端}训练来缓解误差传播问题,并可以获得强大的对象关联能力。
\end{enumerate}

% 
%在这个阶段,使用所提出的子模块评估两个训练阶段的效果。
%并执行各种训练配置,包括“基准模型”、“检测模型训练”、“未精调检测网络”、“未精调关联网络”和提出的 “JDAN” 训练方法。
%这些方法的组件在每种训练方法中仅更改一次。
%将 MOT17 分为 7 个训练集和 7 个验证集来训练目标关联子模块。
%这里没有使用其他更多的数据来验证所提出的训练方法的有效性。
%
%跟踪性能结果显示在表格~\ref{tab:jdan_training_methods} 中。
%在加粗字体中展示了最佳性能。
%“未训练”是基线原始模型。
%“用基础数据训练”是从除 MOT17 以外的各种数据集训练的初始模型。
%“未精调检测网络”是从 MOT17 关联数据而非 MOT17 检测数据训练的微调模型。
%“未精调关联网络” 是从除 MOT17 关联数据之外的多目标跟踪检测数据训练的微调模型。
%JDAN 是在多目标跟踪检测和关联数据上微调的整个模型。
%对“未训练” 和 “用基础数据训练”的分析表明,一定数量的数据会提升整个 MOTA 的性能。
%目标检测和关联极大地受益于更大的训练数据集。
%例如,MOTA 从 $20.7\%$ 增加到 $34.6\%$,
%和 MOTP 从 $38.3\%$ 增加到 $42.8\%$ 对于“用基础数据训练”。
%这些性能改进证明,通过使用更多的训练数据,基本数据集在提高目标检测和关联精度方面具有巨大优势。
%
%“未精调检测网络”仅在“用基础数据训练”的基础上对关联子模块进行微调,获得良好的 MOTA 性能。
%特别是,MOTP 从 $42.8\%$ 显著提高到 $65.2\%$,同时将 ID\_Sw 从 $18264$ 减少到 $11875$。
%跟踪性能表明通过使用更多的训练数据增强了联想能力。
%同时,“未精调关联网络” 与 “用基础数据训练”实现了相同的效果。


\vspace{0.5em}
\renewcommand\arraystretch{1.5}
\begin{table}[htbp]\wuhao
	\centering
	\caption{在 MOT17 数据集上评估目标表征的维度对性能的影响}
	\vspace{0.3em}
	\begin{tabular}
%		{
%			p{2.0cm}<{\centering} p{1.5cm}<{\centering} p{1.3cm}<{\centering} p{1.3cm}<{\centering}
%			p{1.3cm}<{\centering}
%			p{1.3cm}<{\centering}}
		{c|ccccc}
%		\toprule[1.5pt]
%		\hline
		\hline
		特征维度 & MOTA $\uparrow$ & MOTP $\uparrow$ & IDF1 $\uparrow$ & ID\_Sw $\downarrow$ & FPS$ \uparrow$\\
		\hline
%		\midrule[1.5pt]
		1024 & 55.3 & 76.3 & 57.3 & 8,107 & 15.3\\
		512 & 54.7 & 75.1 & 57.1 & 6,810 & 18.7\\
		256 & 56.2 & 78.9 & {\bf59.7} & 7,657 & 19.6\\
		128 & {\bf58.1} & \bf{79.8} & 59.2 & {\bf6,129} & 21.7\\
		64 & 58.1 & 71.6 & 56.7 & 11,675 & {\bf21.9}\\
%		\bottomrule[1.5pt]
		\hline
%		\hline
	\end{tabular}
	\label{tab:jdan_dimension}
\end{table}





%此外,评估了整个训练过程,在 表~\ref{tab:jdan_training_methods} 中命名为 JDAN。
%可以看出,所提出的方法的所有指标都比其他方法更好。
%这种增强不仅归功于微调的检测子模块,还归功于经过训练的关联子模块。
%例如,JDAN 的 MOTA 从 $53.0\%$ 提高到 $58.1\%$。
%特别是 JDAN 表现出最好的 MOTA 性能,因为它获得了强大的目标关联能力。
%此外,JDAN 比“未精调关联网络”具有更强的联想能力。
%因此,认为在 JDAN 中利用关联训练是性能改进的主要来源,因为它可以提高多目标跟踪中的目标关联能力。


\subsubsection{目标表征维度} \label{sec:dimension}
目前已有的行人重新识别方法通常利用高维的目标表征,例如 $1024$,并在没有探究表征维度这个超参数的情况下在数据集上获得了优异的性能。
直到 FairMOT~\cite{fairmot} 发现了表征维度在目标跟踪中起着重要作用。
因为重识别数据集中缺少原始视频帧,所以多目标跟踪方法无法利用它。
故合适的低维目标表征在多目标跟踪中有更好的性能,因为与重新识别任务相比,多目标跟踪缺乏足够的公共训练数据集。
提取低维表征减轻了较小数据集的模型欠拟合问题,并提高多目标跟踪性能。
已有的两阶段多目标跟踪方法不受数据不足的影响,因为它们可以利用丰富的只提供裁剪行人图像的重新识别数据。
而一阶段多目标跟踪方法需要原始未裁剪的视频帧,它无法利用这些重新识别数据,
解决这个数据依赖问题的一种方法是降低目标表征的维数。

在表~\ref{tab:jdan_dimension} 中测试各种维度配置,可以看出当维度从 $1024$ 减少到 $128$ 时,MOTA 不断增加,这证明了多目标跟踪训练数据中低维表征的优点。
此外,当维数减少到 64 时,MOTA 开始减少,因为过低的目标表征已开始导致表征受损。
尽管 MOTA 分数的改进很小,而 ID\_Sw 改进了很多,从原先的 $8107$ 减少到 $6129$,这在提高整体多目标跟踪性能方面起着重要作用。
通过减少目标表征维度,模型运行速度也略有提高。
然而,只有在缺乏训练数据集时,使用低维目标表征才有效。
随着训练数据集的增加,可以缓解表征维度带来的问题。


\vspace{1.0em}
\renewcommand\arraystretch{1.5}
\begin{table}[htbp]\wuhao
	\centering
	\caption{最大目标数阈值 $N_m$ 对目标关联性能的影响}
	\vspace{0.3em}
	\begin{tabular}
%		{
%			p{2.0cm}<{\centering}
%			p{1.5cm}<{\centering}
%			p{1.3cm}<{\centering}
%			p{1.3cm}<{\centering}
%			p{1.3cm}<{\centering}
%			p{1.3cm}<{\centering}
%		}
		{c|ccccc}
%		\hline
		\hline
%		\toprule[1.5pt]
		特征维度 & Det-1024 & Det-512 & Det-256 & Det-128 & Det-64 \\
		\hline
%		\midrule[1.5pt]
		MOTA$^S$ & 56.3 & 55.7 & \bf 57.8 & 58.1 & 53.6 \\
		MOTA$^M$ & \bf{57.8} & \bf 57.9 & 55.3 & \bf 58.4 & \bf 55.3 \\
		MOTA$^L$ & 52.1 & 56.0 & 52.8 & 49.8 & 47.3 \\
		\hline
		ID\_Sw$^S$ & 8,519 & \bf 7,236 & \bf 7,083 & \bf 6,129 & 8,913 \\
		ID\_Sw$^M$ & \bf 2,107 & 8,013 & 7,657 &  6,346 & \bf 6,675 \\
		ID\_Sw$^L$ & 10,397 & 9,281 & 9,597 &  7,475 & 7,286 \\
%		\bottomrule[1.5pt]
		\hline
%		\hline
	\end{tabular}
	\label{tab:jdan_max_obj}
\end{table}




\subsubsection{最大目标数} \label{sec:maximum_object}
由于各种多目标跟踪环境中的目标密度不同,利用适当的最大目标数 $N_m$ 来适应不同的多目标跟踪环境。
与章节~\ref{sec:dimension} 类似,从表~\ref{tab:jdan_max_obj} 中最大目标数的各种配置中可以发现它会影响多目标跟踪性能。
上标~{S} 表示最大目标数为 100; 上标~{M} 表示最大目标数为 150; 上标~{L} 表示最大目标数为 200。
%{Feature dim} 是目标表征维度。
%在加粗中展示了最佳的多目标跟踪性能。
发现当根据章节~\ref{sec:dimension} 将最大目标数设置为 $ 150 $ 且目标表征维度设置为 $128$ 时,可以获得更好的结果。
可以看出过大的最大目标数将导致关联子模块中的欠拟合并降低多目标跟踪性能。





%\vspace{1.0em}
%\renewcommand\arraystretch{1.5}
%\begin{table}[htbp]\wuhao
%	\centering
%	\caption{提出的方法在 MOT15 和 MOT17 基准上与最新的一阶段方法比较}
%	\vspace{0.3em}
%	\begin{tabular}{p{2.0cm}<{\centering} p{1.8cm}<{\centering} p{1.3cm}<{\centering} p{1.2cm}<{\centering}p{1.3cm}<{\centering}p{1.2cm}<{\centering}}
%		\toprule[1.5pt]
%		Benchmark & Tracker & IDF1$\uparrow$ & MOTA$\uparrow$ & ID\_Sw$\downarrow$ & Hz$\uparrow$\\
%		\midrule[1.0pt]
%		MOT15 & JDE \cite{jde} & \bf 66.7 & \bf 67.5 & \bf 218 &  22.5\\
%		& \emph{JDAN}(ours) & { 61.8} & {57.8} & { 494} & {\bf 23.5}\\
%		\hline
%		MOT17 & JDE \cite{jde} & 55.8 & {\bf 64.4} & \bf 1,544 & 18.5\\
%		& \emph{JDAN}(ours) & {\bf 59.2} & 58.1 & {6,129} & {\bf 21.7}\\
%		\bottomrule[1.5pt]		
%	\end{tabular}
%	\label{tab:jdan_onestage}
%\end{table}


\vspace{0.5em}
\renewcommand\arraystretch{1.5}
\begin{table}[htbp]\wuhao
	\centering
	\caption{与其他在线多目标跟踪方法进行比较}
	\vspace{0.3em}
	\begin{tabular}{c|c|cccccc}
%		\hline
		\hline
		数据集 & 方法 & IDF1$\uparrow$ & MOTA$\uparrow$ & MT$\uparrow$ & ML$\downarrow$ & ID\_Sw$\downarrow$ & FPS$\uparrow$\\
		\hline
		MOT15 
		& MDP\_SubCNN\cite{xiang2015learning} & 55.7 & 47.5 & 30.0\% & 18.6\% & 628 & 2.1\\
		& CDA\_DDAL\cite{bae2017confidence} & 54.1 & 51.3 & 36.3\% & 22.2\% & 544 & 1.3\\
		& EAMTT\cite{sanchez2016online} & 54.0 & 53.0 & 35.9\% & 19.6\% & 7,538 & 11.5\\ 
		& AP\_HWDPL\cite{chen2017online} & 52.2 & 53.0 & 29.1\% & 20.2\% & 708 & 6.7\\
		& RAR15\cite{fang2018recurrent} & 61.3 & 56.5 & 45.1\% & 14.6\% & {\bf 428} & 5.1\\
		& {JDE\cite{jde}\textsuperscript{*}} & {56.9} & {{\bf 62.1}} & {34.4\%} & {16.7\%} & {1,608} & {22.5}\\
		& JDAN\textsuperscript{*} & {\bf 61.8} & 57.8 & {\bf 45.3\%} & {\bf 12.9\%} & 494 & {\bf 23.5}\\
		\hline
		MOT17 
		& DMAN~\cite{dual_matching} & 55.7 & 48.2 & 19.3\% & 38.3\% & 2,193 & 0.3\\
		& MTDF~\cite{gm_phd} & 45.2 & 49.6 & 18.9\% & 33.1\% & 5,567 & 1.3\\
		& FAMNet~\cite{famnet} & 48.7 & 52.0 & 19.1\% & 33.4\% & 3,072 & 0.6\\
		& Tracktor++~\cite{tracktor} & 52.3 & 53.5 & 19.5\% & 36.6\% & 2,072 & 2.0\\
		& SST\cite{dan} & 49.5 & 52.4 & 21.4\% & 30.7\% & 8,431 & 6.3\\
		& {JDE\cite{jde}\textsuperscript{*}} & {55.8} & {{\bf 64.4}} & {{\bf 32.8\%}} & {{\bf 17.9\%}} & {{\bf 1,544}} & {18.5}\\
		& JDAN\textsuperscript{*} & {\bf 59.2} & 58.1 & 27.7\% & 32.9\% & {6,129} & {\bf 21.7}\\
		\hline
%		\hline
	\end{tabular}
	\label{tab:jdan_sota}
\end{table}


\vspace{1.0em}
\renewcommand\arraystretch{1.5}
\begin{table}[htbp]\wuhao
	\centering
	\caption{在 MOT17 上每个视频中跟踪效果的详细信息}
	\vspace{0.3em}
	\begin{tabular}{c|ccccccc}
%	\begin{tabular}{p{2.5cm}<{\centering} | p{1.2cm}<{\centering} p{1.2cm}<{\centering} p{1.2cm}<{\centering}p{1.3cm}<{\centering}p{1.2cm}<{\centering}p{1.2cm}<{\centering}p{1.2cm}<{\centering}}
%		\hline
		\hline
		视频序列 & MOTA$\uparrow$ & IDF1$\uparrow$ & MOTP$\uparrow$ & MT$\uparrow$ & ML$\downarrow$ & FP$\downarrow$ & FN$\downarrow$\\
		\hline
%		\midrule[1.0pt]
		MOT17-01 & 55.57 & 53.88 & 80.49 & 33.33\% & 20.83\% & 249 & 2,507\\
		MOT17-03 & 77.09 & 73.35 & 80.69 & 75.68\% & 7.43\% & 10,102 & 13,533\\
		MOT17-06 & 26.68 & 18.91 & 64.54 & 2.25\% & 62.16\% & 6,017 & 8,464\\
		MOT17-07 & 51.53 & 41.13 & 80.60 & 20.67\% & 18.33\% & 512 & 5,767\\
		MOT17-08 & 34.49 & 29.95 & 81.18 & 19.74\% & 38.16\% & 287 & 12,706\\
		MOT17-12 & 50.90 & 54.45 & 80.08 & 28.57\% & 28.57\% & 792 & 3,310\\
		MOT17-14 & 41.83 & 41.56 & 75.13 & 23.17\% & 21.95\% & 739 & 7,696\\
		\hline
		所有 & 58.05 & 59.22 & 79.84 & 27.73\% & 32.99\% & 18,698 & 53,983\\
%		\hline
		\hline	
	\end{tabular}
	\label{tab:jdan_mot17_detailed}
\end{table}



\subsection{性能比较}
在这一部分中,通过与现有的一阶段方法和两阶段方法在内的多目标跟踪方法进行比较来评估和分析所提出的方法。


\subsubsection{两阶段多目标跟踪方法}
本节通过与许多现有的在线多目标跟踪方法进行性能比较来测试和分析 {JDAN}。
为了证明所提出的端到端检测跟踪方法的有效性,在表~\ref{tab:jdan_sota} 中包含了一些关于 MOT15 和 MOT17 的一阶段和两阶段方法。
值得注意的是,两阶段多目标跟踪方法中的跟踪速度(Hz)只包含跟踪阶段而不包括的检测阶段的时间。
然而,在一阶段方法的测试中,所花费的时间同时包括检测和关联。
请注意,用“*”标记了一阶段多目标跟踪方法。
%最好的结果显示在加粗中。
因为不使用多目标跟踪基准数据集中提供的边界框,所以使用了私有检测器。
在 MOT15 和 MOT17 中的测试数据集上展示了多目标跟踪性能。
在评估性能之前,对训练后的模型进行了 6 个迭代轮次的微调。

在 MOT15 和 MOT17 数据上,所提出算法的性能优于其他在线多目标跟踪方法。
与之前的多目标跟踪方法相比,所提出的算法在两个多目标跟踪基准数据上获得了最好的 MOTA 分数,这显示了所提出的方法拥有出色的 MOT 性能,并且所提出的方法计算效率也非常高,所提出的方法获得了接近帧率的跟踪速度。
相比之下,许多高性能方法,例如~\cite{fang2018recurrent,poi},其预测速度低于本章所提出的方法。


\subsubsection{一阶段多目标跟踪方法}
在之前的研究中,只有 TrackR-CNN~\cite{voigtlaender2019mots}、JDE~\cite{jde} 和 FairMOT~\cite{fairmot} 都使用了行人检测和表征学习。
但是 TrackR-CNN 需要额外的图像分割真实标签,并在图像分割问题中使用不同的方法衡量跟踪性能。
%此外,由于使用传统跟踪生成关联数据并训练的关联子模块,因此跟踪精度自然会低于之前开发的方法。
因此,在研究中,将所提出的方法与 JDE 进行比较,只是为了证明提出的端到端检测跟踪方法的有效性。

为了进行合理的评估,使用了 JDE~\cite{jde} 中用于描述的类似数据集,但是没有使用 MOT16 数据集,因为它与 MOT17 具有相同的视频序列,并且它包含的原始视频帧和 MOT16 没有区别。
还利用了 IDF1~\cite{ristani2016performance} 和 CLEAR 指标~\cite{bernardin2008evaluating} 来评估跟踪性能,跟踪性能见表~\ref{tab:jdan_sota}。
容易看出在 MOT17 数据上,所提出的方法优于 JDE~\cite{jde}。
并且它将 IDF1 从 $55.8$ 上升到 $59.2$,整体提高了多目标跟踪性能,详细地跟踪效果信息如图~\ref{tab:jdan_mot17_detailed} 所示。
多目标跟踪性能证明了一阶段方法的优势。
此外,多目标跟踪速度是该算法进行实际应用的一个非常重要优势。



\section{本章小结}
在本章中设计了一个端到端的框架 JDAN 来缓解多目标跟踪任务的误差传播问题,它可以联合训练检测和关联任务,将目标检测和跟踪同时进行优化,超越现有方法。
从技术实现上讲,使用伪标签来解决目标不一致问题,并设计了一个{连接子模块}以及一个{关联预测器}来产生精确的跟踪结果。
{端到端}的架构非常简单而有效,与现有的多目标跟踪方法相比,它避免了繁琐的手动设置。
在广泛使用的多目标跟踪数基准数据集上进行的一系列实验证明了所提出的方法在精度和效率方面都有很大的优势。
相信这个工作可以启发和激励端到端多目标跟踪任务的进一步研究。

%在本章中,设计了一种端到端方法来联合解决目标检测和多目标跟踪任务。
%超越现有的 JDE 方法~\cite{jde},将目标检测和关联结合到一个单一的神经网络中。
%特别是,负责区分不同目标的目标表征嵌入被提取两次,导致两阶段多目标跟踪的重复计算。
%该工作的实现放弃了目标检测和跟踪中的显式锚点,从而提高了多目标跟踪的性能。
%此外,展示了一种明确且自然的端到端 方法,该方法直接利用目标表征将不同帧中的目标关联起来,
%解决端到端检测跟踪中{检测子模块}的输出与{关联子模块}的输入不一致的问题。
%最后,对广泛使用的多目标跟踪基准进行的大量实验证明了所提出的方法在有效性和效率方面的优越性。
%相信所提出的 {JDAN} 可以鼓励和激发一阶段多目标跟踪任务的新方法。


% !Mode:: "TeX:UTF-8"

\externaldocument{chap2}
\externaldocument{chap3}
\externaldocument{chap4}
\externaldocument{chap5}




\chapter{智能驾驶场景下的多目标跟踪分析应用验证—— 灵动慧眼系统}

% 参考:
% /data2/whd/win10/doc/paper/doctor/doctor.Data/PDF/2687609247

%6.1  应用问题提出(多目标跟踪分析是场景理解、意图分析、预测决策的基础技术,***参考申报书***)
\section{应用问题背景}
随着国内外汽车产业的发展和城镇化的推进,智能交通技术受到世界各国政府与社会越来越广泛的重视。
2017 年 7 月,国务院颁布《新一代人工智能发展规划》~\cite{new_ai_plan},提出发展智能驾驶技术,建立智能驾驶和车路协同技术体系。
2020 年 2 月 24 日,由国家发改委、工信部、科技布等部委印发的《智能汽车创新发展战略》~\cite{car_plan_11}指出推动与人工智能、通信、互联网等行业进行深度融合,全面加速汽车产业转型。
2020 年 3 月,美国发布《智能交通系统战略规划(2020-2025)明确提出了“加速应用智能交通系统,转变社会运行方式”的愿景。
而我国人口基数大、公路网密集,由于汽车作为人们常用的交通工具,其智能化运行有着巨大的市场潜力。
同时近些年,国家“新基建”战略规划的提出和 5G 通信、人工智能等前沿技术的快速发展,为实现车对外界的信息交换(Vhicle to everything,V2X)环境下“人-车-路-云”协同的智能驾驶提供了很好的政策保证和技术支撑。
同时多目标跟踪算法作为智能驾驶实现的前提和基础,为场景理解、行为预测、意图分析等高层决策提供了强有力的支撑。
%
实现实时的多目标跟踪分析系统不仅可以提高出行的安全、运行的效率,还能加速形成具有自主知识产权的智能驾驶技术新产业集群,具有重大的经济价值和社会意义。


因此,本工作除了前面章节中多目标跟踪算法机理和技术的研究之外,也将所提出的算法在智能驾驶环境进行了工程化落地,开发出一套实时在线多目标跟踪系统——灵动慧眼。
%本项目的基本技术想定包括:智慧公共交通环境具备智能交通基础设施、智能路边单元、云中心等辅助支持,智能客车包括多传感器支持下单车智能“感知-决策-控制”技术手段,也具备V2X通信、“车-路-云”协作等技术条件。


\begin{figure*}[ht]
	\centering
	\includegraphics[width=1.0\textwidth]{figures/C6Fig/introduction.pdf}
	\caption{V2X 环境下“人-车-路-云”协同的智能驾驶跟踪感知系统示意图}
	\label{fig:introduction}
\end{figure*}


%6.2 系统需求分析   (车、路、场景的图)
% 参考:https://baijiahao.baidu.com/s?id=1708958727067894390&wfr=spider&for=pc
\section{系统需求分析}

% 系统的范围
\subsection{系统范围}
开发该系统的目的不仅仅是为了收集并保存相机采集到的实时交通场景下的画面,更是为智能驾驶场景提供支撑,对动态开放的智能驾驶场景进行实时多目标跟踪和分析,最终为行为分析、动作预测等更高层的功能提供支持。
该系统主要是一个基于视频的跟踪和分析平台,提供包括实时视频监控、目标跟踪、目标统计分析等信息。
不仅可以在线浏览和处理实时视频流信息,而且可以进行历史视频和信息的回溯。
利用“人-车-路-云”协同的方案,满足智能驾驶场景下实时感知环境的需求,成为智能驾驶场景中必不可少的模块和基础,提升交通和驾驶环境下的智能化水平。


\subsection{系统功能}
% 系统的功能(需求分析)
该系统主要的功能是为智能驾驶系统提供一套高效的感知平台,
如图~\ref{fig:introduction}所示为 V2X 环境下“人-车-路-云”协同的智能驾驶跟踪感知系统的示意图,灵动慧眼有如下几个功能:
\begin{enumerate} 
	\item \textbf{多相机支持}:可以同时配置多个 IP 相机进行同时跟踪,同时可以通过视频流来模拟 IP 相机。
	
	\item \textbf{更低的误报率}:相对于传统多目标跟踪方法,该系统使用本文所提出的多目标跟踪模型进行工程化部署,通过适量采集并标注的数据进行模型微调,并通过使用低置信度跟踪进行过滤。
	
	\item \textbf{结果显示可配置}:显示的检测结果可以进行配置选择,目前主要包括行人和车辆,但是会隐藏平均置信度,并由最常见的检测类别来确定跟踪的类别。
	
	\item \textbf{统计功能}:该系统不仅可以给出每小时的每种类别目标数的统计信息,而且记录每个计数对象交叉点详细信息,比如:相交时间、相交坐标等。	
\end{enumerate}



\subsection{性能需求}
为了实现安全高效的智能驾驶系统,需要该系统提供较高的检测、跟踪和分析的准确度,为了避免危险目标的遗漏,相比于准确率,对算法的召回率提出了更高的要求。
同时为了保证系统能够为高层功能提供实时的感知信息,该系统需要满足基本的实时性需求,实际开发和测试环境中需要达到实时的处理速度。
需要解决一系列提高运行速度的的瓶颈,包括算法性能、硬件加速、网络带宽等。

% 其他需求
%\subsection{其他需求}
%\subsection{运行环境需求}
%本系统的运行环境一般包括:

% 用户和特性
\subsection{用户和特性}
本系统根据最终所具备的功能将用户分为以下几类:
\begin{enumerate} 
	\item \textbf{算法用户}:需要为行为分析、动作预测等高层算法提供支撑;
	
	\item \textbf{监控用户}:为用户提供交通场景下实时和回放的监控画面和跟踪分析信息;
	
	\item \textbf{管理用户}:管理介入到系统中的用户和相机,保证系统的正常秩序;
	
	\item \textbf{系统管理员}:维护该系统的正常运行。
\end{enumerate}

% 系统的风险
%\subsection{系统风险}

%\subsection{安全性需求}
同时该系统在建立和运行的过程中存在一定的安全性风险:可能会产生一定的恶意用户,对建立的这个平台进行不良行为的攻击,由于该实时跟踪分析系统的有效性、实时性优于功能完备性,过多的计算请求会导致合理的跟踪分析功能不能得到实时性的满足,从而对系统的安全性产生侵害。

%\subsection{软件质量属性}




%6.3  系统架构设计   (云-边-端部署场景,软硬件平台,功能模块,功能与性能测试模块等)
\section{系统架构设计}
灵动慧眼系统的核心的实时在线多目标跟踪模块都是基于本文前面章节提出的各个模型和方法进行构建,并采用“人-车-路-云”协同的方式进行设计和部署。
该系统的检测和关联算法为第~\ref{chap:jdan} 章的联合检测关联模型,此外还以第~\ref{chap:btn}~章的类脑单目标跟踪模型作为它们的基础,
使用第~\ref{chap:nonlocal}~章和第~\ref{chap:stml}~章的基于预测的跟踪算法作为对比。
随后对前面在基准数据集上训练好的模型根据自己所采集和标注的部分视频数据进行模型微调,以增强模型的环境适应性,提高系统的跟踪和分析效果。

图~\ref{fig:system_architecture}~展示了灵动慧眼系统的架构设计。
整个多目标跟踪和分析系统主要包括三个组成模块:输入和预处理模块、云端后台处理模块、展示交互模块。
输入和预处理模块是指从各个不同的场景,包括车载摄像头、边缘端的监控摄像头、手持的智能手机摄像头等,使用 IP 相机进行数据的采集并发送到到灵动慧眼系统的流媒体服务器进行初步处理,为多目标跟踪和数据分析提供合适的输入数据。
后台处理模块是该系统最核心的部件,也是本文中各个算法及对比算法实际部署和实现的地方。
最后将跟踪和分析的结果进行页面的展示。

下面对以上每个系统子模块进行详细的介绍。

\begin{figure*}[ht]
	\centering
	\includegraphics[width=1.0\textwidth]{figures/C6Fig/pipeline.pdf}
	\caption{灵动慧眼系统的架构设计}
	\label{fig:system_architecture}
\end{figure*}


\subsection{输入和预处理模块}
由于每个相机所处的空间位置都不同,本系统采用配置好地址和端口的 IP 相机进行输入视频信息的采集,包括智能汽车上的车载摄像头、路边边缘端安装的监控摄像头、行人手持的移动智能手机摄像头等设备,并将不同数据源的视频流数据同时进行处理。
每个车载端或者边缘端的相机都作为一个实时视频数据源服务,通过 RTSP 协议,利用有线或者无线网络实时地向流媒体服务端发送视频流数据。
为了解决流媒体服务器或云端后台处理模块来不及处理视频流数据,导致视频帧堆积的问题,该系统在预处理模块采用生产者消费者模式,建立一个作为临时缓存的生产队列,队列中保存从车载端和边缘端发送过来的视频帧数据。
然后启动一个消费者线程读取生产队列前端中的最新视频帧数据,这样保证后续模块每次处理的都是最新的一帧。
同时将接收到的所有视频数据都保存到数据库服务器中进行备份,以便用户可以通过信息回放功能查看历史视频数据。
服务器端再通过应用程序路由转发给响应的后台处理模块响应的函数进行视频数据的处理,并将最新的视频帧缩放成检测跟踪模型输入所需要的尺寸。


\subsection{云端后台处理模块}
图~\ref{fig:system_architecture} 中的云端后台处理模块简要说明了灵动慧眼系统的实时在线多目标跟踪过程。
对数据传输和预处理模块传输过来的输入图像帧预测其中所有目标的当前目标表征。
然后使用第~\ref{chap:jdan} 章的联合检测关联模型,将当前目标表征和历史表征传递到连接子模块和关联子模块以生成对应的关联矩阵,并通过匹配算法将结果更新到跟踪轨迹管理器中。
跟踪轨迹管理器中存储的历史帧的跟踪信息,以及历史轨迹与当前帧中检测到的目标的相似度之和。
在云端后台处理模块中一个跟踪轨迹对应于一个被跟踪目标的实体。
当接收到一个新的目标表征后,跟踪轨迹管理器将保留表征并将所有历史表征输出到关联子模块以生成每个历史帧和当前帧之间的关联矩阵。
之后更新跟踪轨迹管理器,并生成当前帧中的跟踪结果,传递给展示交互模块。
同时在跟踪过程中将原始视频和所产生的跟踪和以及分析统计信息保存到数据库服务器,用于后面对历史信息进行回放和查询。

该模块使用的联合检测关联模型属于一阶段的端到端在线多目标跟踪模型,其中每个被跟踪目标的外观特征在整个系统的处理流程中只提取了一次,同时将提取到的外观特征保存下来供将来的关联步骤使用。
为了提高系统的响应速度,在测试和部署时使用高性能的计算机硬件进行加速,特别是使用计算能力强的显卡优化联合检测关联模块的计算过程,使灵动慧眼系统能在实际运行中基本达到工程化的实时性要求。
这对于智能驾驶系统的及时感知和预测有着至关重要的作用,提升了系统的实时性和安全性。

为了保证该系统对其他检测跟踪算法的可扩展性,可以使用其他检测跟踪算法替换本文所提出的方法,以便进行不同算法的测试和对比,实现系统模块内的高内聚和模块间的低耦合。
主要可以替换和测试一些经典的检测跟踪算法和最新的且精度较高的算法,测试不同算法在不同驾驶场景下的跟踪和分析效果,便于找出算法的不足之处以便进行优化和改进。



\subsection{展示交互模块}
由于根据函数路由返回的流式响应视频流需要在网页中显示,所以展示交互模块使用客户端浏览器显示跟踪和分析结果的视频流来自动更新页面中的图像元素。
可以使用不同的客户端,比如 PC 终端、智能手机等从 Web 服务器获取跟踪和统计分析信息并进行实时地显示。
在该系统中,对于每个检测并跟踪到的目标都会对当前目标的总计数进行记录的更新。
当相机视野中出现一个新的类别,即每检测到一个新的类别还会创建一个新的类计数器。
灵动慧眼系统除了跟踪和分析常见的行人,还可以通过配置进行比如行人、车辆等多个类别的多目标跟踪和统计分析。
使该系统能很好地对变化中的需求进行适应,以达到更好的应用效果,对高层应用和用户使用提供了更高的支撑。

该模块仅仅用于用户与系统的展示和交互上,对于多目标跟踪的过程和为其场景分析功能的添加只起到调试和显示效果,在真实工程化部署时可以进行配置并省略,以减少系统的开销,提高系统的运行速度和增强实时响应能力。


\subsection{系统部署和性能评测}
灵动慧眼系统的运行环境包括操作系统及其版本,服务器端采用 Ubuntu 18.04 及以上版本,客户端可以为任意的 PC 终端、手机移动端等;
系统数据传输和预处理模块的输入端基于 RTSP 实时流协议使用 IP 相机收集车载端、边缘端以及其他视频数据,并向流媒体服务端发送视频流数据。
服务器端的云端后台处理模块使用 OpenCV、TensorFlow~\cite{abadi2016tensorflow}、Pytorch~\cite{paszke2019pytorch} 等 Python 库进行实现。
整个云端后台处理模块部署在 128G 内存的 20 核且拥有 4 块 RTX 2080 GPU 的高性能服务器上,可以为其他更高级的场景分析和用户浏览提供实时服务。
目前主要的跟踪服务主要运行在云端,并测试多路相机输入时的跟踪情况。
系统展示交互模块的网页前端~\footnote{https://flask.palletsprojects.com/en/2.0.x/}及服务端使用轻量级的 Flask~\footnote{https://github.com/LeonLok/Multi-Camera-Live-Object-Tracking} 框架进行实现。


将高分辨率的单个摄像机流在以 30 FPS 的速度进行流式传输时,在服务器上托管下平均可提供约 15 FPS 的检测跟踪速度。
由于该系统支持多个车载端和边缘端的 IP 相机进行视频流输入,但是服务器资源有限,同时处理多个视频流将会一定程度上降低多目标检测跟踪和分析的速度。
降低处理视频流的分辨率或图像质量将提高系统的运行速度,但同时也会降低检测跟踪的精度。
在基本满足实时性的要求时,需要在速度和精度之间做出权衡,以获得最佳的系统效果。
同时还有很多其他因素会影响灵动慧眼系统的整体性能,比如网络信道质量、带宽等。




%6.4  系统测试 (按照功能、性能指标,分场景(目标密集、目标稀疏;高速、低速;场景清晰,场景复杂等),多个对比算法)
\section{系统测试和验证}
为了说明该系统的有效性,分别从定量和定性两个角度展示灵动慧眼系统的实际效果。
首先进行定量化测试和验证,将本文的方法和一些经典多目标跟踪算法在同样的智能交通场景视频进行多目标跟踪并把结果保存下来,并利用手工标注的方法获得跟踪所对应的真实值,然后如图~\ref{tab:quantitative_test} 所示计算出多目标跟踪性能评价指标,可以看出本文所提出的方法在智能驾驶场景下有较大的优势。

\vspace{0.5em}
\renewcommand\arraystretch{1.5}
\begin{figure}[htbp]
	\centering
	
	\subfigure[动态开放交通场景下车辆的跟踪和统计效果]{
		\begin{minipage}[t]{0.90\linewidth}
			\centering
			\includegraphics[width=1\textwidth]{figures/C6Fig/cidi_car.pdf}
		\end{minipage}%
	}%
	%	\subfigure[With STURE]{
		%		\begin{minipage}[t]{0.48\linewidth}
			%			\centering
			%			\includegraphics[width=1\textwidth]{figures/C6Fig/dongfanghong.pdf}
			%		\end{minipage}%
		%	}%
	
	\subfigure[动态开放场景下行人的跟踪和统计效果]{
		\begin{minipage}[t]{0.90\linewidth}
			\centering
			\includegraphics[width=1\textwidth]{figures/C6Fig/cidi_person.pdf}
		\end{minipage}
	}%
	%	\subfigure[Tracking results with STURE]{
		%		\begin{minipage}[t]{0.48\linewidth}
			%			\centering
			%			\includegraphics[width=1\textwidth]{figures/C6Fig/experiment.pdf}
			%		\end{minipage}
		%	}%
	
	\centering
	\caption{灵动慧眼系统多相机多类型跟踪统计效果展示}
	\label{fig:system_present}
\end{figure}

\vspace{0.5em}
\renewcommand\arraystretch{1.5}
\begin{table}[htbp]\wuhao
	\centering
	\caption{智能驾驶场景下不同方法之间性能的比较}
	\vspace{0.3em}
	%\vspace{0.5em}\wuhao{\textwidth}
	\begin{tabular}{c|cccccccccc}
%		\hline
		\hline
		方法   & MOTA$ \uparrow $ & MOTP$ \uparrow $  & IDF1$ \uparrow $  & IDR$ \uparrow $ & FP$ \downarrow $  & FN$ \downarrow $  & MT$ \uparrow $  & ML$ \downarrow $  & IDS$ \downarrow $  & Frag$ \downarrow $\\ 
		%		\midrule[1.0pt]
		\hline
		PHD\_DAL~\cite{2019Online}    &36.9 &68.1 &29.5 &37.2 &150 &452 &6.5 &53.7 &375 &438\\
		HISP~\citep{baisa2021robust}   &37.2 &70.8 &30.1 &39.5 &150 &366 &8.7 &43.1 &247 &347\\
		GMPHD\_ReId~\citep{baisa2021occlusion}  &37.3 &73.3 &41.2 &30.8 &114 &252 &8.6 &42.8 &165 &157\\
		DASOT17~\citep{chu2020dasot}   &40.1 &71.1 &36.2 &27.9 &217 &363 &8.2 &39.3 &121 &240\\
		NAAL   &45.2 &\bfseries73.8 &\bfseries50.1 &40.9 &151 &215 &9.2 &41.3 &92 &107\\
		STURE     &47.1 &73.2 &48.2 &34.6 &\bfseries51 &\bfseries123 &\bfseries11.2 &41.0 &73 &35\\
		JDAN     &\bfseries53.7 &73.5 &41.2 &\bfseries47.2 &143 &493 &9.16 &\bfseries36.8  &\bfseries31 &\bfseries24\\
		\hline
%		\hline
		%		\bottomrule[1.5pt]		
	\end{tabular}
	\label{tab:quantitative_test}
\end{table}

如图~\ref{fig:system_present} 所示按照功能、场景、算法等方面展示系统验证的效果。
选取了两个典型的实时场景进行多目标跟踪和统计的功能验证效果展示,其中图~\ref{fig:system_present}(a) 展示了动态开放十字路口的多目标车辆跟踪和统计的效果,图~\ref{fig:system_present}(b)展示了典型室外场景下的多目标行人跟踪和统计的效果,表明该系统在典型动态开放场景下能实现同时进行不同类型目标的跟踪和分析,很好地实现了该系统的功能需求。


\vspace{0.5em}
\renewcommand\arraystretch{1.5}
\begin{figure*}[ht]
	\centering
	\includegraphics[width=0.8\textwidth]{figures/C6Fig/system_test.pdf}
	\caption{灵动慧眼系统在具有挑战性的环境下的测试}
	\label{fig:system_test}
\end{figure*}

% Faster RCNN~\cite{b8} 和 YOLOv3~\cite{redmon2018yolov3}
图~\ref{fig:system_test} 中显示了该系统在具有各种挑战条件下进行响应的验证,选取了阴天或者雨天等光照不足的天气情况作为基本背景,以测试该跟踪分析系统应对复杂极端情况的能力。
图中第一行的两图分别表示在目标稠密和稀疏的交通场景下系统的运行效果,第二行分别表示智能汽车在高速和低速运动时的跟踪和分析结果,第三行表示使用其他方法进行测试时的效果,这里使用经典的 PHD\_DAL~\cite{2019Online} 和 DASOT17~\citep{chu2020dasot} 作为测试对比方法,这种情况下存在一定的检测和跟踪错误,反过来证明本文所提出的方法在应对有挑战的现实环境下具有较强的鲁棒性。

以上测试和验证效果均表明该系统在现实动态开放场景下能取得不错的性能并产生较大的实际应用价值。



%6.5  本章小结(应用效果,如何服务于其他模块;验证前面算法效果;从应用的角度强调内脑计算分析的有点)
\section{本章小结}
基于本文研究工作所开发的灵动慧眼系统,实现 V2X 智能驾驶环境下“人-车-路-云”的协同感知,不仅能够自动进行目标的检测和跟踪,还能提供所关注环境中的统计信息,如目标交汇、速度、方向等,同时在实际测试过程中表现出较好的效果和实时性。
该系统不仅仅提供自动实时地监控服务,极大地减少了人力消耗,同时为人和其他算法进行高层次的分析和决策提供助力,增强驾驶场景下的智能化水平,帮助智慧城市的实现。
该系统以人工智能作为促进社会智能化发展的新手段,为社会的文明和进步做出贡献。







% -----------------------------------------------------------------------------
\end{document}
%文档结束