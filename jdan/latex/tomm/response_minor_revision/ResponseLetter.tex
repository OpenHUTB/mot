

\documentclass[12pt,journal,onecolumn]{IEEEtran}


% correct bad hyphenation here
\hyphenation{op-tical net-works semi-conduc-tor}
\usepackage{graphicx}
\usepackage[table,xcdraw]{xcolor}
\usepackage[TABBOTCAP]{subfigure}
\usepackage{bm}
\usepackage{upgreek} 
\usepackage{amsmath}
\usepackage{breqn}
\usepackage{color}
\usepackage{cite}
\usepackage[none]{hyphenat}
\usepackage{algorithm}
\usepackage{algorithmic}
\usepackage{caption2}
\usepackage{setspace}

\usepackage{tikz}
\newcommand*{\circled}[1]{\lower.7ex\hbox{\tikz\draw (0pt, 0pt)%
		circle (.4em) node {\makebox[1em][c]{\small #1}};}}


\linespread{1.5}

% 把这两个参数的调整加到tex文件里。hyphenpenalty值越大断字出现的就越少。tolerance越大,换行就会越少,也就是说,LaTex会把本该断开放到下一行的单词,整个儿的留在当前行。
\hyphenpenalty=5000
\tolerance=1000

\begin{document}

\begin{center}
	\textbf{\LARGE Response to Comments} 
\end{center}
\vspace{8pt} 

\noindent TOMM-Manuscript: TOMM-2021-0302.R2 \\

\noindent Title: JDAN: Joint Detection and Association Network for Real-Time Online Multi-Object Tracking \\

\noindent \textcolor{blue}{Dear Editor}:

\textcolor{blue}{
We want to thank you and the review team for the comments on our submission, and for the opportunity to reconsider the paper to Transactions on Multimedia Computing Communications and Applications.  
Since both referee \#1 and referee \#3 did not make further comments, here the correction of the manuscript is mainly made by taking into account the comments of referee \#2.
Besides, we have proofread the manuscript carefully and have made correction to meet with your approval. 
The responds to the reviewer's comments are labeled in blue, 
and the main corrections in the paper are marked in red to facilitate your reading. 
}

\vspace{8pt} 

\newpage





\textbf{To Referee \#2:}

\textcolor{blue}{Thanks for the comments for our paper, we have revised the manuscript according to your recommendations as follows:}

\textbf{Comment (1).} Thanks for the authors’ responses and their massive efforts to revise the paper. 
The current version is much easier and clearer to follow.
My only left concern is about the comparison with FairMOT. 
As FairMOT also introduces a Re-ID loss which learns to associate targets in the training, should it be classified into the end-to-end MOT as well? 
If so, is the proposed association submodule comparable to using the Re-ID loss?
%
Regarding the current inferior performance compared to FairMOT, have the authors tried to add the Re-ID loss in their end-to-end training? 
For fair comparisons, the authors could consider to apply the post-association processing used in FairMOT to their approach. 
This may help to narrow the performance gap.


% TODO: 是否可以使用再识别损失
% 使用了再识别损失,但是没有关联损失;
% A: 由于我们是端到端的,中间不好增加re-id损失
% 是否可以增加关联的后处理(或者re-id损失)来提高精度
% A: 已经包括了再识别过程,FairMOT利用的手动设计的技巧精度才比较高。
\textbf{Response:} \textcolor{blue}{Thank you for your comment.
First, FairMOT is not a complete \emph{end-to-end} approach, but an one-stage method for MOT task. 
Although it attempts to integrate both detection and association into one framework, they do not adopt deep networks to make target association.
Actually, based on learned Re-ID features, they adopt hand-designed and complex association methods with abundant manual settings to boost the performance.
Differently, our approach adopts a differential deep network to generate object association without any manual setting for association to adapt to different datasets. 
Moreover, our approach could implement an \emph{end-to-end} training without explicit Re-ID step, instead using identity classification loss in \emph{detection submodule}. 
We have considered post-association techniques to improve the performance to some extent, but since our approach is \emph{end-to-end} and potentially includes a Re-ID step, post-processing techniques for Re-ID cannot be considered.
We hope our work could attract more researches to focus on one-stage MOT method by using \emph{end-to-end} training strategy. 
We have emphasized it in Section 4.4 as follows:
}

% 2.1 第2段Recently开头
\textcolor{red}{``
%Meanwhile, we compare JDAN with the one-stage approach JDE [56], which is a representive MOT method.
%TrackRCNN [42] is not listed and compared because it requires additional image segmentation ground truth, which is not supported on the MOT datasets. 
Besides, FairMOT [55] adopts both visual features and IoU distance with complex and hand-designed tricks in its association stage to improve tracking performance.
Its engineering implementation requires to manually tune many model parameters to adapt to different datasets.
While our JDAN is simple and abandons manual parameter settings in association stage by using an \emph{association matrix}."
} \\


%\begin{figure}[H]
%	\centering 
%	\includegraphics[width=0.9\textwidth]{imgs/inconsistency.png}
%\end{figure}
%
%\begin{figure}[H]
%	\centering 
%	\includegraphics[width=0.9\textwidth]{imgs/framework.png}
%\end{figure}
%\vspace{8pt}








\vspace{8pt} 


\vspace{8pt}

\bibliographystyle{ACM-Reference-Format}
%\bibliography{test1}

\end{document}


